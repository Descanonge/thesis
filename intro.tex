
\chapter{Introduction}

\nocite{*}

Quelques pages d'introduction générale pour donner les objectifs.

\section{Le phytoplancton dans le système terre}

(intro globale sur le phytoplancton)

définition: plancton = organisme porté par les courants.
Phyto: ils pratiquent la photosynthèse, ils transforment le carbone dissous dans l'eau en matière organique.
Grande variété: de taille et de type et d'espèces.

50\% de la productivité primaire.
base de la chaîne alimentaire dans les océans.

Pour croître: ont besoin de lumière, de nutriments, de certaines conditions environmentales (température, salinité?, acidité). Dans le détail il y a des spécificités selon chaque espèce.
Sources de mortalité: vieillesse, virus, broutage par le phytoplankton.
Quelques mots sur le phytoplankton. Plus gros. Migration diurne. Ici aussi grande variété, préférence de broutage: compléxité supplémentaire.

Les limitations majeures restent la lumière et les nutriements.
La lumière n'est pas présente partout identiquement: latitude, saisonalité, et surtout profondeur. Importance de la couche euphotique, en surface.
Également limité en nutriments dans la quasi-totalité des eaux libres. Les eaux profondes (ou la lumière ne pénètre pas) contiennent des nutriments cependant. On peut définir une nutricline.

Cette dualité des sources de croissance (lumière en surface, nutriments en profondeur) rend les échanges verticaux très importants.
Or on considère généralement les courants océaniques à l'ordre 1 comme strictement horizontaux.
Les petites et moyennes échelles ((sub-)mésoéchelles) présentent des moyens de créer des échanges verticaux (comme on le verra plus tard).

Le phytoplankton est complexe à étudier.
Une observation synoptique est seulement possible par satellite.
Ces obs se font sur la couleur de l'océan. (quelques mots sur le principe, depuis quand ça existe, révélation des fines échelles dans les 1ères images).
Obs de la couleur présente des problèmes. Le lien avec le phytoplankton est indirect et imparfait (utilisation de la Chl-a comme proxy de biomasse). Limitation importante des observations par la couverture nuageuse. Et cela ne scanne que la composition de la surface (combien de m en vrai ?), pas accès à toute la colonne d'eau.

C'est pour cela que les observations in-situ toujours très utiles (pour combler les trous, ou compléter les obs sat). Vision de toute la colonne d'eau. Associé à une obs de la densité (temp + salinité qui n'est pas dispo en sat. à petite échelle).
Accès à la composition du phytoplancton (et zoo) grầce à différents outils (cytomètre, zooscan, HPLC, -omiques).
Mais aussi limitations. Toutes ces obs sont compliquées à mettre en place. Avoir une vue de toute la colonne d'eau pour tous ces paramètres (physiques, bio phyto + zoo) est compliqué techniquement. Une obs est limitée à une très petite zone spatio-temporelle. Obligé de cibler une zone d'intérêt (supposé).
Heureusement ces dernières années de telles campagnes se multiplient, devant la nécessité d'avoir tous ces paramètres dans des zones précises pour mieux comprendre ce qu'il se passe.

J'ai parlé que des obs mais des modèles numériques sont aussi dispo cependant.
Tous les modèles sont faux. Tous les modèles bio sont très faux.
La très grande variabilité des organismes est mal représentée (faute de puissance entre autre). Les petites échelles qui sont pourtant si importantes à la bio (subméso) sont très couteuses à faire tourner et on est limité à de petites régions géographiques. Impossible de quantifier les effets convenablement avec des modèles.
On a pas de paramétrisation des processus bio. Est-ce que c'est seulement possible sachant qu'on peut avoir des effets non-locaux potentiellement ?
Les modèles climatiques sont, du coup, très incertains en ce qui concerne la bio.
Grosses barres d'erreurs dans les projections pour le prochain siècle.

\section{Impact des fronts sur phyto}

\subsection{présentation physique des fronts de SMS}

Ce sont les images satellite de couleur de l'océan qui ont révélé l'ubiquité des fines échelles dans l'océan (en surface tout du moins).

Une partie des fines échelles observées correspondent à l'action de l'advection par les courants de méso-échelle. L'action passive du mélange horizontal fait apparaître de fins filaments.
Cela permet le mélange de communauté, car rapproche spatialement des parcelles d'eau de différentes origines, avec des caractéristiques physiques et des historique biologiques propres.

Mais à ces échelles (0.1-10km) émergent également des processus dynamiques nouveaux.
On est alors en dessous du rayon de déformation de Rossby (\(Ro < 1\)).
Forcage par l'atmosphère (hétérogène) et les courants méso qui génèrent des gradients de densité.
On décrit certains de ces processus et leur(s) impact(s) dans la suite.

Difficile de séparer les trois: processus dynamiques, mélange horizontal, et biologie, car ils ont les même temps caractéristiques.

\subsection{mécanismes d'impact sur phyto}

\subsubsection{upwelling des nutriments}

Comme dit plus haut, les échanges verticaux sont important pour la bio.
Or aux petites échelles on voit apparaître des vitesses verticales de grande magnitude.

advection selon les isopycnes qui peuvent être penchées.

aspect intermittent (local, petit spot)

winners and losers

\subsubsection{réduction du mélange}

explication mécanismes

exemples de détection précédentes.


\section{Détection des fronts sur images satellites}
[ou plus généralement à partir de vue synoptique de SST]

méthodes avec gradient

méthodes avec fenêtre glissant

méthodes se basant sur histogrammes (Otsu, Cayula)

méthodes utilisant variance, entropie

Quelques mots sur le suivi des fronts + reconstruction de fronts 'linéaire'
qu'on utilise pas on reste en 'vue pixelisée'

% LocalWords:  Cayula
