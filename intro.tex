
\chapter{Introduction}

\nocite{*}

\section{Le phytoplancton dans le système terre}

intro globale sur le phytoplancton

50\% de la productivité primaire.
base de la chaine alimentaire dans les océans.

présentation couche euphotique, nutricline
différent régimes en global (ou dans le Gulf-Stream)

complexité dans son étude:
accès synoptique grace satellite, seulement biomasse (proxy)
composition: réservé (jusqu'à récemment) à in-situ

inconnues dans les projections
mal représentés par les modèles climatiques

importance des courants
niche écologiques
changements des conditions environnementales de développement

\section{Impact des fronts}

\subsection{présentation physique des fronts de SMS}

différents méchanismes

\subsection{méchanisme d'impact sur phyto}

\subsubsection{upwelling des nutriments}

explication méchanismes qui engendrent circulations verticales

winners and losers

\subsubsection{réduction du mélange}

explication méchanismes

exemples de détection précédentes.


\section{Détection des fronts sur images satellites}
[ou plus généralement à partir de vue synoptique de SST]

méthodes avec gradient

méthodes avec fenêtre glissant

méthodes se basant sur histogrammes (Otsu, Cayula)

méthodes utilisant variance, entropie

Quelques mots sur le suivi des fronts + reconstruction de fronts 'linéaire'
qu'on utilise pas on reste en 'vue pixelisée'
