
\RequirePackage{silence}
\WarningFilter{latex}{Overwriting file}
\WarningFilter{minitoc}{W0024}

% Add fields to the bibliography datamodel
% Formatting and sorting is done in preamble in bib_setup.tex
\begin{filecontents*}[overwrite]{biblatex-dm.cfg}
\DeclareDatamodelFields[type=list,datatype=literal,label=true,nullok=true]{datavar}
\DeclareDatamodelEntryfields[dataset]{datavar}
\DeclareDatamodelFields[type=field,datatype=verbatim,nullok=true]{pypi}
\DeclareDatamodelEntryfields[software]{pypi}
\end{filecontents*}

\documentclass[
12pt,
a4paper,
french,
openright,
twoside,
hyperfootnotes=false, % put here otherwise option clash
draft,
]
{memoir}

\usepackage{geometry}
\usepackage[T1]{fontenc}
\usepackage[otfmath]{XCharter}

\usepackage[french]{babel}
\usepackage{csquotes}
\frenchsetup{
  ItemLabels=\textendash,
  ThinColonSpace
}

\usepackage[style=authoryear-comp]{biblatex}

\usepackage{microtype}

% \usepackage[dvipsnames]{xcolor}
\usepackage{siunitx}
\usepackage[depth=4]{bookmark}
\usepackage{graphicx}
\usepackage{listings}
\usepackage{hyperref}
\usepackage{amsmath}

\useshorthands*{"}
% Hard hyphen, no breaks (but word after is breakable)
\defineshorthand{"~}{\babelhyphen{nobreak}}
% Hard hyphen, breakable (word after also breakable)
\defineshorthand{"-}{\babelhyphen{hard}}

%%%%%%%%%%%%%%%%%%%%
%% GLOSSARIES CONFIG
\usepackage[
xindy, shortcuts=abbr,
nostyles, nonumberlist,
toc, section=section, symbols,
]{glossaries-extra}
\usepackage{glossary-long}
\usepackage{glossary-tree} % needed for \setglswidth

\makeglossaries%

\setglossarystyle{long}

\makeatletter
% Automatically set glossary width to maximum available
% minus largest 'short' from all entries
\newcommand{\setglswidth}{%
  \glsfindwidesttoplevelname%
  \newlength{\gls@widest}%
  \settowidth{\gls@widest}{\glstreenamefmt{\@glswidestname\space}}%
  \multiply\gls@widest by -1%
  \setlength{\glsdescwidth}{\textwidth}%
  \addtolength{\glsdescwidth}{\gls@widest}%
  \addtolength{\glsdescwidth}{-21pt} % We are overfull by 20.5pt otherwise, idk why
  \message{glsdescwidth: \the\glsdescwidth\newline}
}
\makeatother

%% Acronyms glossary
\setabbreviationstyle[acronym]{short-nolong}
\loadglsentries{glossaries/acronyms}


%% Symbols glossary
\newignoredglossary{symbol}

% Custom style for Math symbols
% When an abbreviation is added, we should have
%   short = \(<command>\)
%   long = description
%   symbol = <command>
% This is achieved with \newsymbol
% The style only display short (so put in math mode).
% To input the symbol when already in math mode use \am{<label>}
% If needed the full format display <long>~\(<command>\)
\newabbreviationstyle{short-symbol}%
{%
  \GlsXtrUseAbbrStyleSetup{nolong-short}%
  \renewcommand*{\CustomAbbreviationFields}{%
    name={\the\glslabeltok},%
    sort={\the\glslabeltok},%
    first={\the\glsshorttok},%
    firstplural={\the\glsshorttok},%
    text={\the\glsshorttok},%
    plural={\the\glsshorttok},%
    description={\the\glslongtok}}%
  \renewcommand*{\GlsXtrPostNewAbbreviation}{%
    \glssetattribute{\the\glslabeltok}{regular}{true}%
    \glssetattribute{\the\glslabeltok}{nohyper}{true}%
  }%
}
{%
  \GlsXtrUseAbbrStyleFmts{nolong-short}%
  \renewcommand*{\glsfirstlongfont}[1]{\glsfirstlongdefaultfont{##1}}%
  \renewcommand*{\glslongfont}[1]{\glslongdefaultfont{##1}}%
  \renewcommand*{\glsxtrinlinefullformat}[2]{%
    \protect\glsfirstlongfont{\glsaccesslong{##1}%
      \ifglsxtrinsertinside##2\fi}%
    \ifglsxtrinsertinside\else##2\fi~%
    \glsaccessshort{##1}%
  }%
  \renewcommand*{\glsxtrinlinefullplformat}[2]{%
    \protect\glsfirstlongfont{\glsaccesslongpl{##1}%
     \ifglsxtrinsertinside##2\fi}%
    \ifglsxtrinsertinside\else##2\fi~%
    \glsaccessshortpl{##1}%
  }%
  \renewcommand*{\Glsxtrinlinefullformat}[2]{%
    \protect\glsfirstlongfont{\Glsaccesslong{##1}%
      \ifglsxtrinsertinside##2\fi}%
    \ifglsxtrinsertinside\else##2\fi~%
    \glsaccessshort{##1}%
  }%
  \renewcommand*{\Glsxtrinlinefullplformat}[2]{%
    \protect\glsfirstlongfont{\Glsaccesslongpl{##1}%
       \ifglsxtrinsertinside##2\fi}%
     \ifglsxtrinsertinside\else##2\fi~%
    \glsaccessshort{##1}%
  }%
  \renewcommand*{\glsxtrfullformat}[2]{%
    \glsaccessshort{##1}\ifglsxtrinsertinside##2\fi%
    \ifglsxtrinsertinside\else##2\fi
  }%
  \renewcommand*{\glsxtrfullplformat}[2]{%
    \glsaccessshort{##1}\ifglsxtrinsertinside##2\fi%
    \ifglsxtrinsertinside\else##2\fi
  }%
  \renewcommand*{\Glsxtrfullformat}[2]{%
    \glsaccessshort{##1}\ifglsxtrinsertinside##2\fi%
    \ifglsxtrinsertinside\else##2\fi
  }%
  \renewcommand*{\Glsxtrfullplformat}[2]{%
    \glsaccessshort{##1}\ifglsxtrinsertinside##2\fi%
    \ifglsxtrinsertinside\else##2\fi
  }%
}

\setabbreviationstyle[symbol]{short-symbol}

% \newsymbol[<options>]{<label>}{<symbol>}{<long>}
\newcommand{\newsymbol}[4][]{%
  \newabbreviation[#1, category=symbol,
  sort={#2}, symbol={#3}
  ]{#2}{\(#3\)}{#4}
}
\newcommand{\am}[1]{\glsentrysymbol{#1}}

\loadglsentries[symbol]{glossaries/symbols}

%% Links glossary
% This allows to define URLs that can be used in the document
% Should be used with \glsurl, \glsentryurl et \glshref
\newignoredglossary{link}
\glssetcategoryattribute{link}{nohyper}{true}
\glssetcategoryattribute{link}{regular}{true}

% Add keys
\glsaddkey{url}{none}{\glsentryurl}{\Glsentryurl}%
{\glsurlorg}{\Glsurlorg}{\GLSurlorg}
\glsaddkey{doi}{none}{\glsentrydoi}{\Glsentrydoi}%
{\glsdoiorg}{\Glsdoiorg}{\GLSdoiorg}

\newcommand{\newlink}[4][]{%
  \newglossaryentry{#2}{type=link, category=link,%
    name={#3}, url={#4}, #1}%
}
\newcommand{\newdoi}[4][]{%
  \newglossaryentry{#2}{type=link, category=link,%
    name={#3}, url={https://doi.org/#4}, doi={#4}, #1}%
}

\newcommand{\glshref}[2][]{%
  \href{\glsentryurl{#2}}{%
    \ifstrempty{#1}{\glsentryname{#2}}{#1}%
  }%
}
\newcommand{\glsurl}[1]{%
  \ifglsfieldeq{#1}{doi}{none}%
  {%
    \url{\glsentryurl{#1}}%
  }{%
    \textsc{doi}:~\href{\glsentryurl{#1}}{\texttt{\glsentrydoi{#1}}}%
  }%
}

\loadglsentries[link]{glossaries/links}

%% Other abbreviations
\newignoredglossary{other}
\glssetcategoryattribute{other}{nohyper}{true}
\glssetcategoryattribute{initialism}{discardperiod}{true}
\setabbreviationstyle[other]{short}
\setabbreviationstyle[initialism]{short}
\setabbreviationstyle[biomes]{long}

\newcommand{\newabbreviationOther}[4][]{%
  \newglossaryentry{#2}{type=other, category=other,%
    name={#3}, short={#3}, long={#4}, #1}%
}
\newcommand{\newalias}[2]{\newglossaryentry{#1}{name={#1}, alias=#2}}

% Discard period if enddot key is defined
% Only need that to avoid double dots.
% Inter word/sentence are equal, we use frenchspacing
\glsxtrprovidestoragekey{enddot}{}{}
\renewcommand*\glsxtrifcustomdiscardperiod[2]{%
  \glsxtrifkeydefined{enddot}{#1}{#2}}

\loadglsentries[other]{glossaries/other}

\glsaddall%


\usepackage[nameinlink]{cleveref}

\usepackage{minitoc}

% Metadata
\newcommand\Author{Clément Haëck}
\newcommand\Email{clement.haeck@locean.ipsl.fr}
\newcommand\Title{Impact des fronts sur le Phytoplancton dans la région du Gulf-Stream quantifié par imagerie satellitaire}
\newcommand\Subject{biogeochemistry}
\newcommand\Keywords{}

%%%%%%%%%%%%%%
%% PAGE CONFIG
\chapterstyle{ger}
\pagestyle{ruled}
\setsecnumdepth{subsubsection}
\renewcommand{\thesection}{\arabic{section}} % No chapter number in section
\renewcommand{\thechapter}{\Roman{chapter}}

%% Page Layout
% A4 297x210mm (global a4paper option)
\settrims{0pt}{0pt} % no trim needed
% Fix width/height for comfort
\settypeblocksize{230mm}{160mm}{*}
% Upper/Lower margins, fix upper, lower has what's left (enough hopefully)
\setulmargins{3cm}{*}{*}
% Left/Right margins, set ratio so that spine is smaller than edge
\setlrmargins{*}{*}{1.3}
% Set header size and foot skip, header taller than baseline for É in sections titles
\setheadfoot{1.1\onelineskip}{2\onelineskip}
\setheaderspaces{*}{\onelineskip}{*}
% Check and apply layout
\checkandfixthelayout%

%%%%%%%%%%%%%
%% TOC CONFIG
%% Section numbering and toc options
\renewcommand{\cftchaptername}{\chaptername~}
\renewcommand{\cftappendixname}{\appendixname~}
\setcounter{lofdepth}{1}
\setcounter{tocdepth}{2}
\mtcsetdepth{minitoc}{3}
\mtcsetfont{minitoc}{section}{\normalfont\small}

\renewcommand\lofheadstart{}
\renewcommand\printloftitle[1]{\section*{#1}}
\renewcommand\afterloftitle{\thispagestyle{ruled}}
\renewcommand\lofmark{\markboth{}{\listfigurename}}
\setlength{\cftfigureindent}{2.5em}

% New chapter but add it to lof
% Lifted from memoir.cls \@chapter
\makeatletter
\newcommand\chapterlof[1]{
  \chapter{#1}
  \ifanappendix%
    \addcontentsline{lof}{appendix}{%
      \protect\chapternumberline{\thechapter}\f@rtoc}%
  \else
    \addcontentsline{lof}{chapter}{%
      \protect\chapternumberline{\thechapter}\f@rtoc}%
  \fi
}
\makeatother

\bookmarksetup{numbered}
\AtBeginDocument{%
  \bookmark[named=FirstPage, level=section]{Page de garde}%
}

% Unumbered section with an entry in the TOC
\newcommand{\unsection}[1]{%
  \phantomsection%
  \addcontentsline{toc}{section}{#1}
  \section*{#1}
}
% Unumbered subsection with an entry in the TOC
\newcommand{\unsubsection}[1]{%
  \phantomsection%
  \addcontentsline{toc}{subsection}{#1}
  \section*{#1}
}
% Reference by name/page for those sections
\makeatletter
\newcommand\refunsection[1]{%
  \cref@section@name~\frquote{\titleref{#1}} \cpageref{#1}}
\makeatother

% Cross-chapter references {<label>}{<chap. label>}
\newcommand{\chapref}[2]{\cref*{#1}, \cref*{#2}}

\crefname{equation}{éq.}{éq.}%
\Crefname{equation}{Équation}{Équations}%
\crefname{figure}{fig.}{fig.}%
\Crefname{figure}{Figure}{Figures}%
\crefname{subfigure}{fig.}{fig.}%
\Crefname{subfigure}{Figure}{Figures}%
\crefname{table}{tab.}{tab.}%
\Crefname{table}{Tableau}{Tableaux}%
\crefname{subtable}{tab.}{tab.}%
\Crefname{subtable}{Tableau}{Tableaux}%
\crefname{page}{p.}{pp.}%
\Crefname{page}{Page}{Pages}%
\crefname{part}{part.}{part.}%
\Crefname{part}{Partie}{Parties}%
\crefname{chapter}{chap.}{chap.}%
\Crefname{chapter}{Chapitre}{Chapitres}%
\crefname{section}{section}{sections}%
\Crefname{section}{Section}{Sections}%
\crefname{subsection}{section}{sections}%
\Crefname{subsection}{Section}{Sections}%
\crefname{subsubsection}{section}{sections}%
\Crefname{subsubsection}{Section}{Sections}%
\crefname{appendix}{ann.}{ann.}%
\Crefname{appendix}{Annexe}{Annexes}%
\crefname{subappendix}{ann.}{ann.}%
\Crefname{subappendix}{Annexe}{Annexes}%
\crefname{subsubappendix}{ann.}{ann.}%
\Crefname{subsubappendix}{Annexe}{Annexes}%
\crefname{subsubsubappendix}{ann.}{ann.}%
\Crefname{subsubsubappendix}{Annexe}{Annexes}%
\crefname{enumi}{pt.}{pts.}%
\Crefname{enumi}{Point}{Points}%
\crefname{enumii}{pt.}{pts.}%
\Crefname{enumii}{Point}{Points}%
\crefname{enumiii}{pt.}{pts.}%
\Crefname{enumiii}{Point}{Points}%
\crefname{enumiv}{pt.}{pts.}%
\Crefname{enumiv}{Point}{Points}%
\crefname{enumv}{pt.}{pts.}%
\Crefname{enumv}{Point}{Points}%
\crefname{footnote}{note}{notes}%
\Crefname{footnote}{Note}{Notes}%
\crefname{theorem}{th.}{th.}%
\Crefname{theorem}{Théorème}{Théorèmes}%
\crefname{definition}{déf.}{déf.}%
\Crefname{definition}{Définition}{Définitions}%
\crefname{result}{rés.}{rés.}%
\Crefname{result}{Résultat}{Résultats}%
\crefname{example}{ex.}{ex.}%
\Crefname{example}{Exemple}{Exemples}%
\crefname{remark}{rq.}{rq.}%
\Crefname{remark}{Remarque}{Remarques}%
\crefname{note}{comm.}{comm.}%
\Crefname{note}{Commentaire}{Commentaires}%
\crefname{algorithm}{algo.}{algo.}%
\Crefname{algorithm}{Algorithme}{Algorithmes}%
\crefname{listing}{liste}{listes}%
\Crefname{listing}{Liste}{Listes}%
\crefname{line}{l.}{l.}%
\Crefname{line}{Ligne}{Lignes}%


\changecaptionwidth%
\captionwidth{0.9\textwidth}

\graphicspath{{resources}}

%%%%%%%%%%%%%
%% BIB CONFIG

\addbibresource{references/software.bib}
\addbibresource{references/zotero_export.bib}
\addbibresource{references/custom.bib}
\ExecuteBibliographyOptions{
  abbreviate = true,
  dashed = false,
  doi = true,
  eprint = false,
  hyperref = true,
  indexing = bib,
  isbn = false,
  giveninits = true,
  maxnames = 100,
  maxcitenames = 1,
  mergedate = false,
  pluralothers = true,
  % refsection = chapter,
  sorting = nyt,
  sortcites = false,
  uniquename = init,
  url = false,
}

% Add filter for 'normal entries'
\defbibfilter{normal}{%
  not type=software
  and not type=dataset
}

% Sort dataset, software and rest separately
\DeclarePresort[dataset]{dd}
\DeclarePresort[software]{ss}

\newcommand\bibsoftwareTitle{Logiciels}
\newcommand\bibdataTitle{Données}

% Remove small caps in family names
\DefineBibliographyExtras{french}{\restorecommand\mkbibnamefamily}

% New bibliography format for dataset entries
\DeclareBibliographyDriver{dataset}{%
  \usebibmacro{begentry}%
  \printlist{datavar}
  \printfield{shorttitle} -- \printfield{title}% chktex 8
  \newunit\newblock%
  \printnames{author}%
  \newunit%
  \printlist{publisher}%
  \newunit%
  \printfield{year}%
  \newunit\newblock%
  \printfield{doi}%
  \usebibmacro{finentry}%
}

% New bibliography format for software entries
\DeclareBibliographyDriver{software}{%
  \usebibmacro{begentry}%
  \printfield{shorttitle} -- \printfield{title}% chktex 8
  \setunit{\addspace}%
  \printfield{type}%
  \setunit{,\addspace}%
  \printfield{version}%
  \newunit\newblock{}
  \printnames{author}%
  \newunit%
  \printlist{publisher}%
  \newunit\newblock{}
  \printfield{pypi}%
  \newunit%
  \printfield{url}%
  \usebibmacro{finentry}%
}

\DeclareSourcemap{
  \maps{
    % Adapt nyt sorting for dataset (datavar, ,shortitle)
    \map{
      \pertype{dataset}
      % Set sortname to datavar
      \step[fieldsource=datavar]
      \step[fieldset=sortname, origfieldval]
      % Set sortyear to a constant (no incidence on sorting)
      \step[fieldset=sortyear, fieldvalue=0]
      % Set sorttitle to shorttitle
      \step[fieldsource=shorttitle]
      \step[fieldset=sorttitle, origfieldval]
    }
    % Sort software by their shorttitle (ie software name)
    \map{
      \pertype{software}
      \step[fieldsource=shorttitle]
      \step[fieldset=sortname, origfieldval]
    }
  }
}

\DeclareFieldFormat[software]{type}{\mkbibparens{#1}}
\DeclareFieldFormat[software]{url}{\mkbibacro{url}\addcolon~\url{#1}}
\DeclareFieldFormat[software]{pypi}{\mkbibacro{pypi}\addcolon~\href{https://pypi.org/project/#1}{#1}}

% Add abbreviation for version
\DefineBibliographyStrings{french} {%
  version = {ver\adddot},%
}

% Citation command for software
% (too much hassle to rewrite textcite or cite)
\DeclareCiteCommand{citesoft}{}{%
  \printfield[citehyperref]{shorttitle}%
}{, }{}


% Citations commands that are fully hyperlinked

\DeclareFieldFormat{citehyperref}{%
  \DeclareFieldAlias{bibhyperref}{noformat}% Avoid nested links
  \bibhyperref{#1}}

\DeclareFieldFormat{textcitehyperref}{%
  \DeclareFieldAlias{bibhyperref}{noformat}% Avoid nested links
  \bibhyperref{%
    #1%
    \ifbool{cbx:parens}
    {\bibcloseparen\global\boolfalse{cbx:parens}}
    {}}}

\DeclareFieldFormat{citeyearhyperref}{%
  \DeclareFieldAlias{bibhyperref}{noformat}%
  \bibhyperref{#1}}

\savebibmacro{cite}
\savebibmacro{textcite}
\savebibmacro{citeyear}

\renewbibmacro*{cite}{%
  \printtext[citehyperref]{%
    \restorebibmacro{cite}%
    \usebibmacro{cite}}}

\renewbibmacro*{textcite}{%
  \ifboolexpr{%
    (not test {\iffieldundef{prenote}} and
      test {\ifnumequal{\value{citecount}}{1}})
    or
    (not test {\iffieldundef{postnote}} and
      test {\ifnumequal{\value{citecount}}{\value{citetotal}}})
  }%
    {\DeclareFieldAlias{textcitehyperref}{noformat}}
    {}%
  \printtext[textcitehyperref]{%
    \restorebibmacro{textcite}%
    \usebibmacro{textcite}}}

% \renewbibmacro*{citeyear}{%
%   \printtext[citeyearhyperref]{%
%     \restorebibmacro{citeyear}%
%     \usebibmacro{citeyear}}}


%%%%%%%%%%%%
%% SI CONFIG
\sisetup{mode=math, locale=FR, product-units=single}
\DeclareSIUnit\mgm{\milli\gram\per\cubic\metre}
\DeclareSIUnit\dC{\degreeCelsius}
\DeclareSIUnit\pixel{\text{px}}
\DeclareSIUnit\pixels{\text{pixels}}
\newcommand\resol[2]{\qty[parse-numbers=false]{\slashfrac{#1}{#2}}{\degree}}
\newcommand\pct[2][]{\qty[#1]{#2}{\percent}}
\newcommand\tC[1]{\qty{#1}{\degreeCelsius}}

%%%%%%%%%%%%%%%
%% HYPER CONFIG
\hypersetup{
  hidelinks,
  draft = false,
  unicode = true,
  linktoc = section,
}

\AtEndPreamble{
  \hypersetup{
    pdftitle={\Title},
    pdfauthor={\Author, \Email},
    pdfsubject={\Subject},
    pdfkeywords={\Keywords}
  }
}


% Some titles of articles
\newcommand{\articleTitle}{Satellite data reveal earlier and stronger phytoplankton blooms over fronts in the Gulf Stream region}
\newcommand{\articleCceTitle}{Multi-trophic planktonic responses to fronts in a coastal upwelling ecosystem}
\newcommand{\articleReviewTitle}{Impact of submesoscale processes on biogeochemical cycles in a changing ocean}

%%%%%%%%%%%%%%%%
% MATHS COMMANDS
\newcommand\moy[1]{\left[#1\right]}
\newcommand\med[1]{\overline{#1}}
\newcommand\norm[1]{\operatorname{norm}\paren{#1}}
\DeclareMathOperator{\std}{std}

\newcommand\norme[2][]{\left\lVert#2\right\rVert_{#1}}
\newcommand\paren[1]{\left(#1\right)}

\newcommand\tot{\text{total}}
\newcommand\frt{\text{front}}
\newcommand\bkg{\text{background}}

% For datasets
% Set the label for the section describing the dataset
% in the methods chapter. The (given) label is the same
% as in the bibliography
\newcommand\declareDataset[1]{\label{sec:donnees-#1}}
% Use that same label to refer to the section
\newcommand\datasect[1]{\cref{sec:donnees-#1}}
\newcommand\dataname[1]{%
  \hyperref[sec:donnees-#1]{\citefield{#1}[citefield]{shorttitle}}}

%%%%%%%%%%%%%%%%
% BEGIN DOCUMENT
\begin{document}
\dominitoc


\pagenumbering{Alph}
\newgeometry{left=1.5cm, right=1.5cm, top=2cm, bottom=3cm}

\NewDocumentCommand \addjury { m m m m }{
  \textit{#3} & #1 \textsc{#2} \\
  & \anysize{8pt}{\raisebox{2pt}{#4}} \\ \addlinespace[0.5ex]
}

\begin{titlingpage}

\begin{center}
  \def\logosheight{3.2em}
  \includegraphics[height=\logosheight]{Logos/sorbonne.pdf}
  \hfill
  \includegraphics[height=\logosheight]{Logos/locean.png}
  \hfill
  \includegraphics[height=\logosheight]{Logos/ipsl.png}
  \hfill
  \includegraphics[height=\logosheight]{Logos/chanel_enspsl.png}

  \vspace{3.5ex}

  \anysize{22pt}{\scshape Thèse de doctorat de Sorbonne Université}\\
  \vspace{3ex}
  {\normalsize École Doctorale 129 \textendash\ Sciences de l'Environnement\\
  Spécialité: Cycles Biogéochimiques et Changements Environnementaux Globaux}

  \vspace{2ex}

  {\small\textit{effecutée au}}\\
  {\normalsize Laboratoire d'Océanographie et du Climat: Expérimentations et Approches Numériques}

  \vspace{\stretch{1.5}}

  \parbox{0.9\textwidth}{
    {
      \centering\parskip=0pt
      \xhrulefill[thickness=2pt, height=-0.6ex]\par
      \anysize{19pt}{\linespread{1.5}\bfseries\Title\\
      \strut\xhrulefill[thickness=2pt, height=0.5ex]}
    }
  }

  \vspace{\stretch{1}}

  \anysize{14pt}{{\normalsize \textit{par}} \textsc{Clément Haëck},\\
    \vspace{2ex}
    {\normalsize \textit{dirigée par}} \textsc{Marina Lévy} {\normalsize \textit{et}} \textsc{Laurent Bopp}}

\end{center}

\vspace{\stretch{2}}

{
  \small
  \itshape
  \noindent Présentée et soutenue publiquement le 31 mai 2023,\\
  devant un jury composé de:
}

\vspace{1ex}

{
  \fontsize{10pt}{12pt}\selectfont
  \hspace{0.8cm}\begin{tabular}{r l}
  \addjury{Lars}{Stemmann}{Président du jury}{Prof.\ Sorbonne Université, Lab. Océanographique de Villefranche}
  \addjury{Aida}{Alvera-Azcárate}{Rapportrice}{Université de Liège, Départ.\ d'Astrophysique, Géophysique et Océanographie}
  \addjury{Hubert}{Loisel}{Rapporteur}{Université du Littoral Côte d'Opale, Lab.\ d'Océanologie et de Géosciences}
  \addjury{Élodie}{Martinez}{Examinatrice}{IRD, Lab.\ d'Océanographie Physique et Spatiale}
  \addjury{Daniele}{Iudicone}{Examinateur}{Stazione Zoologica Anton Dohrn, Napoli}
  \addjury{Marina}{Lévy}{Directrice de thèse}{DR CNRS, LOCEAN}
  \addjury{Laurent}{Bopp}{Co-dir.\ de thèse}{DR CNRS, Lab.\ de Météorologie Dynamique}
  \end{tabular}

}
\end{titlingpage}

\restoregeometry%


\frontmatter

\unsection{Résumé}
Résumé

% \clearpage
% \section*{Remerciements}
% \label{sec:thanks}
% Remerciements \dots

\clearpage
\unsection{Publications et productions}
\label{sec:productions}

\begin{itemize}
        \item Article CHL
\end{itemize}
\medskip

Lors de mon travail de thèse, j'ai été amené à écrire des outils qu'il m'a parru utile de rendre publics et accessibles.
<appuyer sur l'importance de l'open source et de faciliter la reproducibilité.>
Tous les codes utilisés sont disponibles sur un dépôt public\footnote{%
  \glsurl{gitlab}
}, et sur un répertoire Zenodo (\glsurl{zenodo}).
\medskip

Certains outils sont distribués à part:
\begin{itemize}
  \item \citesoft{filefinder}:
        un paquet python qui permet, entre autres, de trouver des fichiers grâce à la structure de leur nom de fichier.
        <Expliquer que j'utilise ça pour toutes mes bases de données.>
  \item \citesoft{xarray-histogram}:
        un paquet python qui permet de calculer des histogrammes depuins des données gérées par Xarray.
  \item \citesoft{tol-colors}:
        un paquet python qui donne accès à des jeux de couleurs adaptés aux personnes atteintes de daltonisme. Les jeux de couleurs ont été développés par \glsname{paultol}\footnote{voir~\glsurl{paultol}}, je les ai seulement rendus accessibles sur PyPi, et plus facilement modifiables.
  \item \citesoft{dateloop}:
        un script bash permettant de générer des ensembles de dates.
\end{itemize}
\medskip

J'ai également participé à un projet open-source (\citesoft{cf-xarray}~\href{https://github.com/xarray-contrib/cf-xarray/pull/354}{pull request \#354}) visant à rendre accessible au paquet python XArray les conventions de métadonnées CF.\@
J'y ai ajouté le support pour les variables drapeaux utilisant un masque binaire.

\begin{center}
  \vspace{1\baselineskip}
  \rule{0.77\textwidth}{0.5pt}
  \vspace{1\baselineskip}
\end{center}

{%
  \raggedright%
  \emergencystretch=\textwidth
  \printbibliography[heading=none, type=software, keyword=personnal]
}


% TOC
\setglswidth%
\printglossary[toctitle=Glossaire]

\clearpage
{
  \setlength{\baselineskip}{1.03\onelineskip}
  \tableofcontents*
  % phantomsection already called by \tableofcontents
  \addcontentsline{toc}{section}{\contentsname}
}

\clearpage

\listoffigures*
% phantomsection already called by \listoffigures
\addcontentsline{toc}{section}{\listfigurename}

\clearpage

\mainmatter%

% chktex-file 13

\chapter{Introduction}
\addChpLof
\label{chp:introduction}
\graphicspath{{resources/introduction}}

\minitoc%
\clearpage

\begin{figure}[!h]
  \centering
  {%
    \setlength{\fboxsep}{0pt}%
    \framebox[\figwidth]{\insertfig{gs_false_colors.jpg}}
  }%
  \captionT{Les couleurs du Gulf Stream par satellite}{%
    \small
    Image en fausses couleurs de l'océan Atlantique Nord, prises par \as{modis} \sur~Aqua le \frenchdate{2020}{02}{23}, retouchées par Norman Kuring (groupe \eng{Ocean Biology}, \abbrv{NASA}).
    Adapté de l'\eng{Image of the day} 2020-03-10 \textit{Hints of Spring in the Atlantic}, sur \eng{\glshref{ve-illustration} \& \glshref{eo-illustration}}.
  }
  \label{fig:oc-illustration}
\end{figure}

\vspace{1\baselineskip}

\section{Préambule}

L'essort de l'imagerie satellite appliquée à la couleur de l'océan \encadra{depuis la fin du \siecle{20}~siècle} a révélé la grande variabilité de la biologie aux échelles les plus fines de l'océan.
Cette variation de la couleur de l'océan, comme illustrée sur la \nref[figure]{fig:oc-illustration}, est due aux pigments du phytoplancton, une collection de micro"-organismes phytosynthétiques portés par les courants.
À la base de la chaîne trophique océanique, et partie centrale des cycles biogéochimiques océaniques \encadra{dont celui du carbone} la compréhension des phénomènes et facteurs régissant les évolutions du phytoplancton est cruciale.

Comme pour ses homologues terrestres, le phytoplancton a besoin de lumière et de nutriments.
Dans l'océan cependant, la répartition de ces deux composantes engendre un forcage particulier.
La lumière ne pénètre que dans une couche superficielle, profonde d'une centaine de mètres environ, la couche euphotique.
Les nutriments, rapidement consommés en surface, sont à l'inverse trouvés en profondeur.
Les échanges verticaux \encadra{de nutriments vers la couche euphotique, et de matière organique (ensuite reminéralisée) vers les profondeurs} sont ainsi nécessaires à la conservation d'un équilibre dans la pompe biologique ainsi décrite.

L'ensemble des courants de large échelle (\ab{cad} des basins océaniques \OM(\qty{1000}{\km})) définit une cartographie des caractéristiques biophysiques de l'environnement.
On peut distinguer par exemple les \guil*{deserts} que sont les gyres subtropicales, où les faibles échanges verticaux engendre un milieu très oligotrophe et peu productif.
À l'inverse, les zones d'\eng{upwelling} de bord est sont de véritables \guil*{forêts}, très productives, en raison des nutriments remontés avec les eaux profondes.
Une telle séparation à grande échelle entre deux biomes est par exemple bien visible sur la \nref[figure]{fig:oc-illustration}, où les eaux au sud du Gulf Stream sont peu productives (en bleu foncé), et celles au nord apparaissent beaucoup plus productives (en turquoise).
Toutefois, sur cette même image, il est également évident que des processus façonnent le paysage biologique à de plus petites échelles.
Par exemple, dans les méandres du Gulf Stream, et des tourbillons se forment.
Ici deux apparaissent en bleu foncé, ayant capturé des eaux du sud (chaudes et peu productives).

De tels tourbillons se forment aux \emph{méso"-échelles} (\qtyrange{10}{100}{\km}) et leurs effets sont multiples.
Comme on l'a vu dans l'exemple ci"-dessus, ils peuvent \guil*{capturer} des masses d'eaux et les transporter sur de larges distances (ici en transportant des masses d'eaux chaudes et oligotrophe au nord, \ab{cad} dans un milieu plus froid, productif, et riche en nutriments).
Ces tourbillons amplifient localement les échanges entre surface et profondeur, à la fois en générant des vitesses verticales, ainsi qu'en déplacant les isopycnes.
Néanmoins, ces échelles ne suffisent pas à décrire toute la variabilité biologique observée. De plus, leur contributions (estimées) à la pompe biologique ne permettent pas de boucler le budget de cette dernière <mal dit>.

En zoomant encore un peu plus sur la \nref[figure]{fig:oc-illustration}, nous pouvons distinguer des structures plus fine, dite de \emph{sub"-mésoéchelle} (\qtyrange{1}{10}{\km}).
Les champs de différentes variables (dont la densité) sont mélangés, étirés, par les courants des échelles supérieures (et les forcages atmosphériques), faisant apparaître des structures plus fines, et plus éphémères (entre le jour et la semaine).
Notamment, émergent des emplacements de fort gradient \encadra{autrement dit des fronts} de densité propices à la formations de circulations secondaires verticales, plus localisées mais aussi plus intenses que celles engendrées par les tourbillons de mésoéchelle.
Les fronts de sub-mésoéchelle, s'ils sont suffisamment marqués peuvent donc engendrer des circulations verticales suffisamment profondes pour remonter des nutriments dans la couche euphotique, et augmenter localement la productivité.

Une autre conséquence de ces gradients, due à leur tendance à applanir les isopycnes par instabilité barocline, est de restratifier les couches supérieures (toujours localement).
La restratification étant un facteur fort dans le démarrage du bloom printanier, ces structures de sub-mésoéchelles peuvent donc également conduire à des démarrages précoces (localement).

Enfin, bien que les effets de la sub-mésoéchelle soient locaux, ils peuvent néanmoins avoir une rétroaction sur les échelles supérieures, modifiant la circulation de grande échelle, ou la productivité d'un bassin océanique.

Les effets de la sub"-mésoéchelle sur les cycles biogéochimiques présentés ci"-dessus ne sont toutefois pas encore complétement élucidés, malgré les efforts des dernières décennies dans ce domaine.
En effet, l'étude des processus à ces échelles présente des difficultés.
Les diverses méthodologies usuelles peinent à donner une vue entière du problème.
Capturer les variations rapide de la biologie en travers d'un front lors d'une campagne in"-situ présente un défi technique certain.
De plus, la courte période de vie des structures d'intérêt rend la tâche d'autant plus ardue.
Il en va de même pour l'imagerie satellite, pour laquelle la couverture nuageuse rend difficile de suivre temporellement une structure.
Par ailleurs, il est compliqué pour les satellites d'accéder à toute la biodiversité du phytoplancton, ainsi qu'au delà des premiers mètres en surface.
Enfin, les simulations numériques peuvent résoudre ces fines échelles, mais à un coût calculatoire élevé et prohibitif sur de trop longues durées (des projections climatiques \ab{par-ex}).

Malgré les limites évoquées, l'imagerie satellite donne l'opportunité d'observer ces effets à une large échelle (spatiale et temporelle), et permettrait d'en fournir une quantification.
Une telle démarche a été entreprise par \textcite{liu_2016}, dans la gyre subtropicale du Pacifique Nord.
Les auteur·ices \encadra{en colocalisant les valeurs satellites de la \al{chl} avec les positions de fronts détectés à partir de la température de surface satellite} ont montré une augmentation des valeurs de \as{chl} dans les fronts par rapport au reste de la zone.
Nous étendons leur méthode, et l'appliquons à la région de l'Atlantique Nord, autour du Gulf Stream.

Cette zone comprend trois biomes, \al{cad} trois régimes biogéochimiques différents.
Au sud de notre zone, la gyre subtropicale est une zone oligotrophe, d'une productivité faible tout au long de l'année.
Au nord de celle-ci mais au sud du Gulf Stream, ce régime oligotrophe est mitigé par l'approfondissement de la couche de mélange en hiver, apportant des nutriments.
Enfin, au nord du Gulf Stream, on trouve des eaux riches et productives, berceau d'un fort bloom au printemps, et d'un second bloom en automne.

Ces trois régions sont également hétérogène en terme des structures qu'on peut y trouver.
Le front de densité (très marqué) associé au Gulf Stream est un réservoir d'énergie potentielle, qui est convertie en énergie cinétique par le biais des structures de fines échelles qui nous intéressent.
Nous nous attendons donc à trouver des fronts de densité plus stables, associés à de forts gradients, et donc à d'intenses vitesses verticales dont il est plus probable qu'elle puisse atteindre les nutriments en profondeur.
En revanche, loin de ce courant intense, dans la gyre subtropicale, nous pourrons étudier des fronts et gradients plus faibles.
Cette variété nous permet d'étudier la réponse biologique aux fronts, en fonction de leur intensité.

Afin de mieux définir nos objectifs, et avant de les présenter, nous dressons dans la section suivante un état de l'art des sujets concernés.
Nous exposerons ensuite le plan de ce manuscript.

\section{État de l'art}
\label{sec:etat-de-lart}

\subsection{Le phytoplancton dans le système terre}
\label{sec:phyto-ds-sys-terre}

La biologie marine, à sa base, repose sur les conversions d'un certain nombre d'éléments entre eux.
Schématiquement, le phytoplancton converti par photosynthèse le carbone et les nutriments dissouts dans l'eau, en dioxygène et en matière organique (pour sa croissance).
À la mort de ces organismes, ce stock de matière organique inerte plonge par gravité, et sera éventuellement reminéralisé plus en profondeur par d'autres organismes.
Le carbone et autres nutriments libérés pourront alors être remontés à la surface et réutilisés, complétant la pompe biologique marine.
Le phytoplancton joue un rôle clef dans


\subsubsection{Définitions générale}
\label{sec:phyto-def-gen}

Le phytoplancton étant au cœur de la problématique qui nous intéresse ici, je m'arrête ici brièvement sur ses caractéristiques.
Le plancton \guil*{\hbox{πλαγκτόσ}}: il \guil*{divague}.
C'est l'ensemble des organismes dont la motilité se lui permet guère plus que de se laisser dérivier au gré des courants.
Bien que certains de ces organismes puissent se déplacer de manière significative par leurs propres moyens, concentrons"-nous plutôt sur le sous"-ensemble moins mobile du phytoplancton.
Cet ensemble d'organismes est remarquablement divers, totalisant environ \num{20 000}~espèces, dont les tailles varient de la fraction de micromètre à la fraction de millimètre. Cette diversité se retrouve aussi bien dans leurs formes, au grand bonheur des zoologues et photographes, que dans leurs fonctions écosystèmiques.
Comme leur nom le suggère, les espèces de phytoplancton pratiquent la photosynthèse.
Ils transforment le dioxyde de carbone et les nutriments (Nitrate, Phosphate, Silice, Fer,\dots) dissous, en oxygène et en matière organique.
À l'instar des plantes terrestres, cette opération nécessite de capter de la lumière à l'aide de pigments, qui se trouvent être principalement la \al{chl}.
Ce sont ces pigments qui affectent la couleur de l'océan comme nous l'avons vu dans le préambule (\nref{fig:oc-illustration}), mais nous reviendrons sur cet aspect plus loin.

Cette production de matière organique à partir de carbone in"-organique (du CO2 dissous), est dite primaire.
Le phytoplancton étant prédaté par le zooplancton, cette matière organique remonte éventuellement la chaîne (ou plutôt réseau) trophique: zooplancton, petits prédateurs, grands prédateurs, oiseaux marins, mamifères marins,\dots
Le phytoplancton occupe donc une place centrale dans l'écosystème marin.
Il est à la base de la chaîne alimentaire dans les océans, et il est estimé qu'il génère~\pct{50} de la productivité primaire mondiale.

Pour croître: ont besoin de lumière, de nutriments, de certaines conditions environnementales (température, salinité?, acidité). Dans le détail il y a des spécificités selon chaque espèce.
Sources de mortalité: vieillesse, virus, broutage par le phytoplancton
Quelques mots sur le phytoplancton Plus gros. Migration diurne. Ici aussi grande variété, préférence de broutage: complexité supplémentaire.

Les limitations majeures restent la lumière et les nutriments.
La lumière n'est pas présente partout identiquement: latitude, saisonnalité, et surtout profondeur. Importance de la couche euphotique, en surface.
Également limité en nutriments dans la quasi-totalité des eaux libres. Les eaux profondes (ou la lumière ne pénètre pas) contiennent des nutriments cependant. On peut définir une nutricline.

Cette dualité des sources de croissance (lumière en surface, nutriments en profondeur) rend les échanges verticaux très importants.
Or on considère généralement les courants océaniques à l'ordre 1 comme strictement horizontaux.
Les petites et moyennes échelles ((sub-)mésoéchelles) présentent des moyens de créer des échanges verticaux (comme on le verra plus tard).

\subsubsection{Mesurer le phytoplancton}
\label{sec:teledetection}

\begin{figure}
  \centering

  \captionT{Structures fines vue dans la couleur de l'océan}{%
    prendre une image depuis visibleearth
  }
  \label{fig:oc-fine-illustration}
\end{figure}

L'observation du phytoplancton comporte un certain nombre de défis que nous allons rapidement évoquer dans cette section.

Comme pour d'autres variables géophysiques, l'utilisation de l'imagerie satellite a permis de fournir depuis ces dernières décennies une vision synoptique du phytoplancton.
En effet cette dernière offre une couverture globale, journalière, et à haute résolution spatiale.
La possibilité d'observer des micro-organismes depuis l'espace n'apparaît pourtant pas comme évidente.
Rappelons alors que ces organismes, réalisant de la phytosynthèse, contiennent différents pigments leur permettant de récupérer l'énergie lumineuse du soleil.
Toutes les espèces de phytoplancton contiennent majoritairement de la \al{chl}~(\as{chl}) et, éventuellement, des pigments secondaires.
Ces pigments absorbent plus ou moins certaines longueurs d'onde du rayonnement pénétrant dans la couche de surface de l'océan, affectant ainsi la rétrodiffusion (\engquote{backscatter}).
En d'autres termes, la couleur de l'océan apparaît plus verte\footnote{%
  la chlorophylle, pigment majoritaire, absorbe dans le rouge et le bleu}
sur les zones où la concentration en phytoplancton est élevée.

Ce phénomène, compris depuis longtemps, fut testé avec succès par des mesures en avion: \ab{par-ex} par \textcite{clarke_1970} dans la région du Gulf Stream, au large de Cape Cod; puis par satellite par John Arvesen et Dr.\ Ellen Weaver\footnote{%
  Je conseille d'ailleurs l'article suivant qui s'attarde sur sa carrière scientifique passionnante (qui inclut un passage par le projet Manhattan !): \citetitle{marshall_2010}, \cite{marshall_2010}.},
en coopération avec le capitaine Cousteau à bord du Calypso\footnote{%
  Ce dernier commentera d'ailleurs: \engquote{The space-age satellites have opened a whole new dimension to ocean resources monitoring}, comme reporté par la presse locale de l'époque:
  \href{ https://news.google.com/newspapers?nid=1454&dat=19730320&id=Nmo0AAAAIBAJ&sjid=8QkEAAAAIBAJ&pg=897,4493586&hl=en}{\citetitle{macomber_1973}}, \cite{macomber_1973}.
}.
Ces premières expériences permettront le déploiement d'un premier capteur dédié à l'étude de la couleur de l'océan: \as{czcs} à bord de Nimbus-7, lancé en 1978 qui fonctionnera jusqu'en 1986.
Ce n'est que dix plus tard, en 1997, que le lancement de \as{seawifs} marquera le début d'une observation ininterrompue de la couleur de l'océan par un ensemble de capteurs et satellites.

Attardons nous brièvement sur le processus permettant d'obtenir une concentration en \al{chl} à partir des données de ces capteurs.
Ce derniers mesurent l'intensité lumineuse renvoyée, ou réflectance (\as{rrs}), pour différentes bandes de longueurs d'ondes (entre 6 et 16 dans le visible).
Le satellite mesure les réflectances en dehors de l'atmosphère, et non pas à la surface de l'eau; il est nécessaire d'apporter plusieurs corrections.
Les contributions de l'écume et du reflet direct du soleil sont retirées.
La diffusion de Rayleigh est corrigée assez simplement, mais l'absorption et diffusion par divers aérosols sont plus complexes, leurs répartitions (horizontale et verticale) étant variables (\cite{werdell_2018}).

Nous voulons maintenant relier ces réflectances au contenu de la surface de la colonne d'eau.
En toute rigueur, il faudrait relier les réflectances aux propriétés optiques de la colonne d'eau: c'est"-à"-dire l’absorption et la rétrodiffusion causées par le phytoplancton, mais aussi les particules non-algales (\as{nap}) et les matières organiques dissoute colorées (\as{cdom}) (\cite{werdell_2018}).
Certains algorithmes dits \emph{semi"-analytiques} s'attachent à résoudre ce problèmes inverse, ce qui nécessite un certain nombre de simplifications, mais permet en théorie de remonter à tous les composants optiques en jeux (\as{chl}, \as{nap}, \as{cdom}) (\cite{werdell_2019}).

Cependant, dans la majorité de l'océan ouvert il est raisonnable de considérer que le contenu optique de la colonne d'eau est le seul fait du phytoplancton\footnotemark, et qu'il est donc possible de directement relier les réflectances à la concentration en \as{chl} (\cite{bailey_2006, brewin_2015a}).
\footnotetext{C'est-à-dire qu'on fait l'hypothèse que les autres composantes (\as{nap} et \as{cdom}) sont des produits du phytoplancton, et co"-varient avec sa concentration.}
Ainsi, un certain nombres d'algorithmes dits \emph{empiriques} infèrent la \as{chl} en s'appuyant sur une simple relation vérifiée empiriquement par des mesures in"-situ.
Par exemple, les algorithmes \guil{OC} utilisent des polynômes de ratio entre les réflectances bleu et verte (voir plus bas).

Cette hypothèse ne tient plus dans certains cas (classifiés comme les \engquote{case II waters}, où \guil{optiquement complexes}), par exemples les régimes côtiers où la colonne d'eau est riche en sédiments (\cite{bailey_2006, brewin_2015a}), mais certains algorithmes sont tout de même disponibles pour ces cas plus complexes (\cite{gohin_2002}).
On gardera néanmoins en tête ces hypothèses, et le fait que ces relations empiriques (validées sur des données passées) pourraient perdre en validité face à des écosystèmes modifiés par le changement climatique (\cite{dierssen_2010}).
Toutefois, ces modèles empiriques sont de manière générale plus performant que les modèles semi"-analytiques (\cite{brewin_2015a}), et sont donc employés pour les données opérationelles.

%%% section plus technique
Nous décrivons ici rapidement deux exemples d'algorithmes empiriques, utilisés de manière opérationelle par le groupe NASA \as{obpg}.
L'algorithme \abbrv{OC3/4} se base sur un polynôme d'ordre~4 du ratio des bandes verte~(\(V\)) et bleue~(\(B\)):
\begin{equation} \label{eq:ocx}
  \log_{10}\paren{\am{chl}} =
  \sum_{i=1}^{4} a_i \paren{\log_{10}\paren{
      \frac{ \am{rrs}^{max}\paren{\lambda_B} }
           { \am{rrs}\paren{\lambda_V} }
    }}^i ,
\end{equation}
où \(\am{rrs}^{max}\paren{\lambda_B}\) est la réflectance maximale parmis les bandes bleues, et \(a_i\) des coefficients empiriques adaptés à chaque capteur (\cite{oreilly_1998, oreilly_2000}).
Ces coefficients, régulièrement mis à jour par le groupe NASA \as{obpg}\footnote{voir l'historique \url{https://oceancolor.gsfc.nasa.gov/reprocessing/}}, sont dérivés par regression de la relation ci"-dessus avec les données in-situ \as{nomad} (\cite{werdell_2005}).
L'algorithme utilise au total 3~bandes pour \abbrv{OC3}, et 4 pour \abbrv{OC4}.

Cependant, \abbrv{OC3/4} montre des limites dans les eaux oligotrophes, où la concentration de \as{chl} est faible.
Dans ces cas là, l'algorithme CI (\eng{Color Index}) est utilisé (\cite{hu_2012}).
Il utilise la différence entre la bande verte~(\(V\)) et une combinaison des bandes rouge~(\(R\)) et bleue~(\(B\)):
\begin{equation}
  \label{eq:CI}
  \begin{split}
    CI &= \am{rrs}\paren{\lambda_V} -
         \paren{
          \am{rrs}\paren{\lambda_B}
          + \frac {\lambda_V - \lambda_B}
                  {\lambda_R - \lambda_B}
          \times \paren{\am{rrs}\paren{\lambda_R}
                      - \am{rrs}\paren{\lambda_B}}
         }
    \\
    \am{chl} &= 10^{\paren{a_0' + a_1' \times CI}},
  \end{split}
\end{equation}
avec ici aussi des coefficients empiriques \(a_i'\).

L'implémentation actuelle du groupe NASA \as{obpg} utilise \abbrv{OC3/4} pour les eaux où la \as{chl} est supérieure à~\qty{0.35}{\mgm}, et CI où elle est inférieure à~\qty{0.25}{\mgm}; entre ces deux valeurs, un mélange des résultats des deux algorithmes est utilisé assurant ainsi la continuité des valeurs de Chlorophylle (\cite{oreilly_2019}, \url{https://oceancolor.gsfc.nasa.gov/atbd/chlor_a/}).

À noter que de nombreux autres algorithmes ont été développés dans les deux dernières décennies, on pourra citer par exemple l'ajout d'une bande violette pour \abbrv{OC5/6} (\cite{oreilly_2019}), ou l'utilisation de 5~bandes et des tables de références afin d'améliorer l'estimation de \as{chl} en zone côtière (\cite{gohin_2002}).
%%% fin

Comme nous venons de le voir, la dérivation de la concentration de \as{chl} est un processus compliqué, comprenant de multiple sources d'incertitudes dans les valeurs absolues inférées.
Ce processus comporte un certain nombre de limitations.
Nous avons évoqué la difficulté à travailler dans des zones où d'autres éléments que le phytoplancton dicte le comportement optique, comme des sédiments en suspension par exemple.
C'est souvent le cas près des côtes et des décharges fluviales.
La faible profondeur en régime côtier apporte également une difficulté, puisque dans le cas où la lumière incidente pénètre suffisamment profond, la réflectance mesurée comportera la lumière réfléchie par le fond marin (et re"-traversant la colonne d'eau).
Plus généralement, quelque soit la zone sondée, la profondeur jusqu'à laquelle pénètre la lumière est un point critique.
En effet, la réflectance mesurée ne peut comporter des informations que sur cette couche éclairée.
De plus, les modèles empiriques sont ajustés par des mesures in"-situ réalisées sur les premiers mètres.
Les données satellite manquent ainsi une biomasse plus en profondeur, potentiellement importante.

\begin{note}
  c'est une variable dérivée.
  incertitude sur la réflectance déjà!
  objectif de \pct{30}, qui est vérifié, RMSE de 0.26.
\end{note}

Obs de la couleur présente des problèmes. Le lien avec le phytoplancton est indirect et imparfait (utilisation de la Chl-a comme proxy de biomasse). Limitation importante des observations par la couverture nuageuse.

C'est pour cela que les observations in-situ toujours très utiles (pour combler les trous, ou compléter les obs sat). Vision de toute la colonne d'eau. Associé à une obs de la densité (temp + salinité qui n'est pas dispo en sat. à petite échelle).
Accès à la composition du phytoplancton (et zoo) grâce à différents outils (cytomètre, zooscan, HPLC, -omiques).
Mais aussi limitations. Toutes ces obs sont compliquées à mettre en place. Avoir une vue de toute la colonne d'eau pour tous ces paramètres (physiques, bio phyto + zoo) est compliqué techniquement. Une obs est limitée à une très petite zone spatio-temporelle. Obligé de cibler une zone d'intérêt (supposé).
Heureusement ces dernières années de telles campagnes se multiplient, devant la nécessité d'avoir tous ces paramètres dans des zones précises pour mieux comprendre ce qu'il se passe.

J'ai parlé que des obs mais des modèles numériques sont aussi dispo cependant.
Tous les modèles sont faux. Tous les modèles bio sont très faux.
La très grande variabilité des organismes est mal représentée (faute de puissance entre autre). Les petites échelles qui sont pourtant si importantes à la bio (subméso) sont très coûteuses à faire tourner et on est limité à de petites régions géographiques. Impossible de quantifier les effets convenablement avec des modèles.
On a pas de paramétrisation des processus bio. Est-ce que c'est seulement possible sachant qu'on peut avoir des effets non-locaux potentiellement ?
Les modèles climatiques sont, du coup, très incertains en ce qui concerne la bio.
Grosses barres d'erreurs dans les projections pour le prochain siècle.

Difficile de séparer les trois: processus dynamiques, mélange horizontal, et biologie, car ils ont les même temps caractéristiques.

\subsection{Interactions biophysiques}
\label{sec:interactions-biophys}

Définition par l'échelle spatiale (0.1-10km).

Ce sont les images satellite de couleur de l'océan qui ont révélé l'ubiquité des fines échelles dans l'océan (en surface tout du moins).

Importance de la SMS:
1) on observe que la variance de la biophysique est importante à ces échelles.
2) les processus dynamiques crée des échanges verticaux
3) Mal représenté dans les modèles climatiques, on doit mieux comprendre (et quantifier pour savoir à quel point c'est important)


\subsubsection{Mélange horizontal}
\label{sec:melange-horizontal}

Une partie des fines échelles observées correspondent à l'action de l'advection par les courants de méso-échelle. L'action passive du mélange horizontal fait apparaître de fins filaments.
Cela permet le mélange de communauté, car rapproche spatialement des parcelles d'eau de différentes origines, avec des caractéristiques physiques et des historique biologiques propres.

Approche lagrangienne très utile dans l'étude du phytoplancton, et dans son observation (méthode de sampling Lagrangiennes, ref ).


\subsubsection{Upwelling de nutriments par les circulations agéostrophiques}
\label{sec:upwelling-nutriments}

Comme dit plus haut, les échanges verticaux sont important pour la bio.
Or aux petites échelles on voit apparaître des vitesses verticales de grande magnitude.

à ces échelles (0.1-10km) émergent également des processus dynamiques nouveaux.
On est alors en dessous du rayon de déformation de Rossby (\(Ro < 1\)).
Forçage par l'atmosphère (hétérogène) et les courants méso qui génèrent des gradients de densité.
On décrit certains de ces processus et leur(s) impact(s) dans la suite.

advection selon les isopycnes qui peuvent être penchées.

importance des vents

aspect intermittent (local, petit spot)

winners and losers


\subsubsection{Modification de la phénologie du bloom}
\label{sec:modif-phenologie}

Les gradients de densité existant (créés par mélange par courants méso ou forçages atmos) sont des réservoirs d'énergie potentielle. À un front les isopycnes sont penchées et des circulations sub-méso se créent et transforment l'énergie potentielle en énergie cinétique (tourbillons), ce faisant ramenant les isopycnes à l'horizontale.
Ces tourbillons formés par l'instabilité de Mixed-Layer (?) s'étendent sur la hauteur de la ML. Ce sont les Mixed-Layer Eddies.
À travers ces instabilités la sub-mesoéchelle contribue fortement à re-stratifier la couche  de surface, et réduire le mélange.
C'est un processus local et on s'attend donc à un soulèvement local de la couche de mélange au niveau des fronts.


Expliquer lancement du bloom printanier par réduction mélange.

On s'attend à un départ du bloom d'abord dans les fronts.

exemples de détection précédentes (mahadevan 2020).


\subsection{Région d'étude: Extension du Gulf Stream}
\label{sec:region-detude}

Notre zone d'étude: 15°N-55°N, 82°W-40°W

\subsubsection{La physique}
\label{sec:gs-physique}

Gyre subtropicale Atlantique Nord.

Courant de bord ouest chaud et salé qui remonte des caraïbes le long de la Floride
Se détache à Cape Hatteras, quitte le plateau continentale et méandre vers l'est.
Plume énergétique autour de ces méandres (surtout au sud).

Au nord du jet, courant retour (slope current) avec notamment un jet sur le shelf break.
Entre le Gulf Stream North Wall et le shelf: slope seas. Très froid, plutôt fraîche (plus salé que sur le shelf néanmoins).
Ce courant froid plonge. Fait partie de la circulation d'overturning de l'atlantique.

\subsubsection{La biologie}
\label{sec:gs-biologie}

Gyre très pauvre en nutriment et productivité faible.
Pourquoi ?
Pas de circulation méso qui permette l'apport de nutriment en surface.
Importance de la sub-méso donc pour créer des échanges verticaux.

Eaux au nord du GS très productives.
Également importances à cause de pêcheries.
Important bloom (plus de détail, ref sur la phénologie, )


\subsection{Détection des fronts sur images satellites}
\label{sec:detection-fronts}

Afin de mesurer l'impact des fronts sur le phytoplancton, nous partons de l'idée de détecter les fronts de densité à partir de la SST afin de les colocaliser aux valeurs de \al{chl}.
Nous nous concentrons dans cette section sur les méthodes existantes cherchant à détecter les fronts sur des images satellites de SST.

Une première catégorie de méthodes s'appuie la dérivation spatiale du champ de SST, avec divers opérateurs: gradient (\cite{kazmin_1996, moore_1997, kostianoy_2004}), Sobel (\cite{sauter_1994}), ou Laplacien (\cite{holyer_1989}), etc.
Ces étapes de dérivation introduisent cependant du bruit qui impacte négativement la détection.
On notera d'ailleurs que la détection des fronts de fine échelle, qui nous intéresse ici, est plus sensible au bruit du fait de gradients plus faibles.
Un filtrage préalable de l'image permet d'en limiter l'impact.
C'est le cas de la méthode de détection de contours \emph{Canny} (\cite{canny_1986}), qui applique un filtre gaussien avant de calculer le gradient, et par ailleurs ajoute des traitements supplémentaire aux contours détectés.
Initialement développée pour de la détection de contours en traitement d'image \encadra{domaine où elle demeure un standard, et où ses implémentations sont amplement disponibles}, elle a également été largement appliquée en océanographie à la détection de fronts en température, entre autres.
% (cite{93-101}).

Similairement, l'algorithme de Belkin--O'Reilly~(\cite{belkin_2009}) applique un filtre itératif contextuel capable de réduire le niveau de bruit de l'image d'entrée tout en préservant les forts gradients.
Un simple opérateur Sobel peut ensuite être utilisé.
Il a été développé afin de repérer les fronts aussi bien à partir de la SST que de la Chorophylle.

Il est également possible d'éviter l'utilisation du gradient, comme le fait la méthode de \al{cc}~(\as{cc}, \cite{cayula_1992}).
L'algorithme que nous utiliserons par la suite, décrit dans le \nref[chapitre]{chp:methodes}~(\nref{sec:HI}), en est inspiré. Nous le décrivons donc succinctement dans la suite.

Le principe de l'algorithme \as{cc} est le suivant: Un front sépare deux masses d'eau de température différentes; il est raisonnable que la température des pixels à proximité du front soit distribuées de manière bimodale, chaque mode se situant autour de la température d'une des masses d'eau.
Ainsi, afin de ne considérer les pixels que proche des fronts, la méthode utilise une fenêtre glissante dans laquelle on s'intéresse à l'histogramme des valeurs de SST.
Une température seuil est choisie pour séparer l'histogramme en deux modes, de façon à minimiser la variance intra"-mode\footnote{%
  La méthode de séparation des valeurs en deux modes est identique à la méthode d'\textcite{otsu_1979}, couramment utilisée en traitement d'image pour réaliser un \guil*{seuillage}, \as{cad} pour catégoriser les pixels d'une photo en deux.
}.
Pour cette séparation optimale, un critère évaluant la séparation des deux modes est calculé. Si ce dernier est supérieur à un certain seuil, la fenêtre est estimée contenir un front.

Une étape supplémentaire consiste à vérifier que la cohérence spatiale de chacune des masses d'eau dans la fenêtre est suffisante.
Cela permet d'éviter les faux positifs, notamment sur des images bruitée ou contaminée par des nuages.

\begin{note}
  On notera que \textcite{cayula_1992} choisissent les seuils de cohérence spatiale de manière à ce que le cas le plus cohérent soit un front en ligne droite.
  Cela signifie qu'un front sinueux se vera attribuer une cohérence spatiale moindre, et pourrait être disqualifié.
  En fonction des structures à detecter, il sera nécessaire d'adapter les divers seuils de la méthode, éventuellement au détriment la spécificité de la méthode (éviter les faux positifs).
\end{note}

La méthode \as{cc} a largement été utilisée pour détecter des fronts de SST (voir les nombreux exemples dans la review de \cite{belkin_2021}), notamment à l'échelle globale (\cite{belkin_2009a, belkin_2007}).
Elle a également été appliquée avec succès sur des données de Chlorophylle (\cite{stegmann_2004, kahru_2012, bontempi_2004}).
Diverses dérivations en ont été proposées.
\textcite{cayula_1995} proposent une version travaillant sur plusieurs images consécutives.
Plutôt que d'utiliser des images à des instants différents, \textcite{nieto_2012} applique l'algorithme \as{cc} sur quatres fenêtre glissantes décalées en longitude et latitude et fusionnent les résultats des pixels se chevauchant.
\textcite{miller_2009} combine les résultats de l'algorithme \as{cc} pour des produits SST et \as{chl} afin d'augmenter la couverture disponible.

Une méthode au fonctionnement similaire est proposée par \textcite{vazquez_1999}, qui pour quantifier la séparation des deux distributions en température (de part et d'autre d'un éventuel front) utilise une mesure entropique: la divergence de Jensen--Shannon (\cite{barranco-lopez_1995}).
\textcite{shimada_2005} reprendra cette méthode et l'adjoindra d'un algorithme de morphologie mathématique (\cite{jiang_1997}).
Cette configuration sera réutilisée à plusieurs reprises (\cite{lan_2012} \ab{par-ex}).

Enfin, dans le but de quantifier les valeurs de \as{chl} dans les fronts de sub-mesoéchelle de la gyre subtropicale du Pacifique Nord, \textcite{liu_2016} propose une méthode inspirée de l'algorithme \as{cc}.
Elles définissent un indice d'hétérogénéité (\as{hi}) du champ de SST.
La valeur de cette indice est calculé en chaque pixel à partir de la distribution en température dans une fenêtre (glissante donc) autour de ce pixel.
La valeur de l'indice est la somme (pondérée par des coefficient de normalisation) de la variance de la distribution, de sa bimodalité (calculée différement que dans les méthodes ci"-dessus), et de son asymétrie (afin de cibler les plus petits fronts).

On notera dans cette dernière étude une différence d'approche, essentiellement sémantique mais qui reste néanmoins d'intérêt.
C'est bien l'indice de l'\al{hi} qui apparaît comme au cœur de cette étude, qui finalement cherche plus à quantifier l'hétérogénéité spatiale de la SST pour chaque pixel, qu'à \guil{détecter les fronts}.
Par ailleurs, bien qu'elle reprenne l'utilisation d'une fenêtre glissante, cette méthode se démarque de celle de \as{cc} et ses variations, en donnant un résultat nuancé plutôt que purement dichotomique (front / pas-front).
De plus, bien que nous les avons omises des descriptions ci"-dessus, la plupart des méthodes incluent une étape algorithmique permettant d'obtenir en sortie uniquement la position des fronts, représentée par une ligne d'épaisseur nulle (ou 1~pixel).
En revanche, le positionement des fronts par \citeauthor{liu_2016} est réalisé par un simple\footnotemark{} seuil sur le HI.
\footnotetext{Bien que le choix du seuil soit fait de manière non"-triviale par \textcite{liu_2016}, nous montrons dans notre implémentation qu'un seuil fixe suffit à notre étude.}
Par cette construction orientée vers une mesure graduelle des fronts, cette méthode de détection apparaît pertinente pour distinguer différents types de fronts par intensité.

\begin{note}
  La démocratisation des techniques de \eng{machine learning} s'est aussi étendu au domaine de la détection des fronts.
  Bien que nous les ayons pas considérées ou décrites ici, ces dernières années, diverses méthodes ont été proposée (voir la review de \cite{liu_2022}).
\end{note}

\subsubsection{Vers des fronts de densité}

Lien entre densité et température.
La salinité intervient aussi. La salinité par satellite est très basse résolution (SMOS) donc on a pas vraiment accès.

On ne peut qu'espérer que la salinité ne joue pas un rôle trop grand dans la densité, ie que les fronts ne soient pas trop compensés.
Pour vérifier cela on doit passer par les campagnes en mer.

Dans la région Nord-Atlantique des transects sont réalisé régulièrement par un navire d'opportunité, l'Oleander.
Des résultats suggèrent que la salinité ne joue pas beaucoup (Flagg 2006, succinct).

\section{Motivation et objectifs}
\label{sec:problematique}

\begingroup
\defaultlists
\begin{itemize}
        \setlength{\topsep}{\baselineskip}
        \setlength{\itemsep}{\baselineskip}
        \renewcommand*\labelitemi{\adfrightarrowhead}
  \item question uno
  \item question dos
  \item question tres
\end{itemize}
\endgroup

\section{Plan de thèse}
\label{sec:plan-de-these}

Cette thèse en organisée en 6~chapitres, en comptant l'introduction ci"-dessus.
Dans le \nref[chapitre]{chp:methodes}, nous commencons par détailler les différents ensembles de données que nous avons considérés, et expliciter le


\chapterlof{Méthodes}
\label{chp:methodes}
\graphicspath{{resources/méthodes}}

\minitoc%
\clearpage

Maintenant que les objectifs de ce projet ont été précisés, nous détaillons dans ce chapitre l'ensemble des outils nécessaire à leur accomplissement.
Nous décrivons dans les sections qui suivent les données utilisées, puis les méthodes utilisées pour séparer notre zone d'étude en sous"-régions, pour repérer les fronts en utilisant le \engquote{\al{hi}}~(\as{hi}), et pour extraire nos résultats.

\section{Données}
\label{sec:donnees}

Dans cette section sont décrits les jeux de données utilisés pour ce projet.
Le choix de ces données s'est avéré être une question importante et difficile dès le départ.
Nous savions qu'il allait être nécessaire de combiner plusieurs jeux de données: la \ab{sst} avec la \ab{chl}, puis avec des données de composition de la communauté de phytoplancton.
Cela a donc informé le choix des données, ainsi que la création des outils nécessaires à la gestion de plusieurs sets de données.

Plusieurs options ont été considérées, notamment pour la \ab{sst}.
Elles sont reportées ci"-dessous, même si elles ne sont pas utilisées pour le reste des résultats présentés.

Au long de ce chapitre (et ailleurs dans ce manuscript) les articles publiés correspondant aux jeux de données sont bien évidemment cités, mais sont également précisés les accès par lesquels nous avons obtenus les données.
Ces derniers sont cités en note de bas de page et listés dans la section~\reftitle{bib:data} de la bibliographie~(\cpageref{bib:data}).

\subsection{Sea Surface Temperature (SST)}
\label{sec:donnees-sst}

Nous commencons par la température de surface que nous allons notamment utiliser comme proxy de la densité pour repérer les fronts.
Malgré que de nombreux produits de \ab{sst} soient disponibles, il n'en demeure pas moins difficile de trouver un produit qui satisfasse les exigences inhérentes à notre étude.
Étant donné que nous voulons repérer des structures de fine échelle la résolution est peut être l’exigence qui apparaît comme première.
Cependant, et comme nous allons le voir plus en détail, d'autres paramètres rentrent en compte et il est nécessaire de faire des compromis dans ce domaine.
Brièvement ces autres paramètres sont (entre autres) la disponibilité des autres variables à une résolution similaire, la facilité d'obtention\footnote{%
  Par facilité d'obtention j'entends le téléchargement des données mais aussi et surtout les traitements supplémentaires nécessaires, comme le repérage des pixels nuageux \ab{par-ex}},
ou encore la couverture spatio"-temporelle.

% Éventuellement à déplacer à une mention antérieure des résolutions
\begin{note}
  Bien que notre zone d'étude s'étende à de (relativement) hautes latitudes~(\latlon{\approx55N}), nous ne considérons pas les différences de distances zonales et méridionales.
  Nous faisons donc une correspondance simple entre fraction de degré et kilomètres, quelque soit la distance considérée, notamment pour la taille des pixels et des fenêtres.
\end{note}

\subsubsection{MODIS-1km}
\declareDataset{sst_modis}
\declareDataset{chl_modis}

Un des premiers jeu de données \ab{sst} à considérer est celui utilisé par \textcite{liu_2016}, dont l'étude consistait également à détecter des fronts de \ab{sst} et les colocaliser aux données de \ab{chl}.
Ces variables étant rarement distribuées sur des grilles à des résolutions kilométriques, il apparaît comme nécessaire de projeter sur grille régulière nous même des produits~L2 \encadra*{\al{cad} des données d'un seul capteur, converties en variables géophysiques, et disposées à la résolution de capture}.

% Éventuellement à déplacer à une mention antérieure des niveaux
\begin{note}
  Tout au long de ce manuscript nous utilisons les dénominations des \guil*{niveaux} de données tels que typiquement utilisés par les distributeurs, et qui correspondent de manière générale à:
  \begin{itemize}
    \item L1:,
    \item L2:,
    \item L3:,
    \item L4:.
  \end{itemize}
\end{note}

Pour le jeu de données qu'on désignera par \dataname{sst_modis} dans la suite, il s'agit de données provenant du capteur \ab{modis}, à bord du satellite Aqua~(\cite{kilpatrick_2015}).
Les données sont téléchargées au niveau~L2 depuis le \href{https://cmr.earthdata.nasa.gov/search/}{Common Metadata Repository (CMR)}\footfullcite{sst_modis} en sélectionnant les \eng{swaths} intersectant notre région d'étude, collectés de jour, et pour les courtes longueurs d'onde infrarouges~(\qty{11}{\um}).
Pour ensuite projeter les données sur une grille kilométrique globale, j'ai repris les outils utilisés par \citeauthor{liu_2016}, à savoir les programmes présents dans le paquet de l'\href{https://oceandata.sci.gsfc.nasa.gov/ocssw}{\ab{ocssw}}, version~\verb|v7.5|, qui est maintenu par l'Ocean Biology Processing Group et notamment distribué à travers l'outil \href{https://seadas.gsfc.nasa.gov/}{SeaDAS}.
Les \eng{swaths}, maintenant tous sur la même grille spatiale peuvent être moyennés par date.

Cela permet donc d'obtenir des données à la limite de la résolution des capteurs, mais présente un certain nombre de désavantages majeurs.
C'est tout d'abord un travail important à réaliser: d'abord de téléchargement, les fichiers~L2 étant particulièrement lourds et nombreux (jusqu'à une dizaine pour une journée); ensuite de projection, qui à une résolution kilométrique demande une certaine puissance de calcul.
De plus, toutes les étapes décrites jusqu'ici ne concernent qu'un seul capteur.
Des étapes et calculs additionnels seraient nécessaires pour y combiner les données d'autres capteurs, ce qui pose des problèmes supplémentaires et se ferait non sans difficultés.
Même pour un capteur identique (\ab{modis} à bord de Terra \ab{par-ex}), il devient nécessaire de considérer des différences de calibrations, ainsi que de possibles artefacts lorsque l'on superpose plusieurs \eng{swaths} (que nous avons d'ailleurs largement ignorés pour agréger les \eng{swaths} sur une seule journée\dots).

Face à ces difficultés, nous avons donc décidé dans un premier temps d'en rester à un seul capteur malgré la couverture spatiale réduite.
Il est également à noter qu'à ce niveau, peu d'actions ont été prises pour disqualifier les pixels nuageux.
Il en résulte des pixels aux valeurs visiblement erronées mais qui n'ont pas été masqués comme nuageux.
Cela se manifeste par exemple comme du bruit autour de nuages.
Il est difficile de les disqualifier correctement sans un bon algorithme de détection de nuage, ce qui nécessite un travail supplémentaire important.
Nous contournons le problème d'une manière similaire à celle de \textcite{liu_2016} en ne travaillant que sur des fenêtres de~\qtyproduct{100 x 100}{\km} avec une faible couverture nuageuse~(\pct{<70}), ce qui élimine une partie de ces cas problématiques.

Malgré le fait que l'on soit limité à un seul capteur, \ab{modis} présente l'avantage de mesurer concomitamment la \ab{sst} et la couleur de l'océan.
Cela permet d'obtenir les valeurs de \ab{chl} aux même pixels, en utilisant le même traitement pour le jeu de données, accessible au \ab{cmems}\footfullcite{chl_modis}.

Il est néanmoins possible de trouver d'autres jeux de données permettant notre étude et nécessitant moins de travail (et autant d'éventuelles erreurs), ces dernières années ayant vu apparaître des données distribuées à des résolutions élevées.

\subsubsection{MUR}
\declareDataset{sst_mur}

Le produit suivant, désigné ici \af{mur} est développé et distribué par le \ab{ghrsst} au JPL~\eng{Physical Oceanography}~\ab{daac}\footfullcite{sst_mur}.
Il présente les avantages d'être distribué à une résolution kilométrique et sans nuages grâce à une méthode d'interpolation par vaguelettes~(\cite{chin_2017}).
Il intègre de nombreuses sources provenant de plusieurs capteurs infrarouges et micro"-ondes, ainsi que de mesures in"-situ.
La grande couverture spatiale de ce produit se fait donc par l'utilisation de mesures non"-contraintes par la couverture nuageuse (\ab{cad} micro"-ondes et in"-situ), mais qui présentent une résolution bien plus faible.
Le champ de \ab{sst} se retrouve donc lissé à la fois par l'interpolation et par l'inclusion de ces mesures.
L'utilisation du produit suivant cherche à palier à ce problème.

\subsubsection{ESA SST CCI / C3S}
\declareDataset{sst_esacci}

Ce produit est distribué conjointement par l'\ab{esa} \ab{sst} \ab{cci} et le \ab{c3s}.
Il agglomère les mesures de tous les capteurs infrarouges disponibles depuis~1981, ce qui comprend 11~\ab{avhrr} à bord des satellites \ab{metop}["~A] et \ab{noaa} (entre les itérations 6 et~19) et trois \ab{atsr} à bord de \ab{ers} 1 et~2, et Envisat.
Cela donne au moins deux capteurs en fonctionnement simultané, et au moins trois depuis~1992.
Outre la bonne couverture obtenue, l'uniformité des capteurs permet d'obtenir un jeu de données stable dans le temps, pouvant servir à la détection de tendances climatiques~(\cite{merchant_2019}).

Les données sont distribuées à différents niveaux, dont nous allons préciser le passage de l'un à l'autre dans la suite:
\begin{itemize}
  \item L2P (mono-capteur),
  \item L3U (mono-capteur mais sur grille régulière),
  \item L3C (multi-capteurs regroupés par famille d'instrument),
  \item et L4 (tous les capteurs, combinés par interpolation).
\end{itemize}

Le produit~L2P est similaire aux données brutes utilisées pour le set \datasect{sst_modis}, mais avec l'avantage de disposer de données standardisées sur un grand nombre de capteurs.
Ces données sont projetées capteur par capteur sur une grille régulière, à une résolution de~\resol{1}{20}, soit~\qty{\approx5}{\km}, ce qui donne le produit~L3U.
Il est à noter que nous pourrions faire ce processus nous même, mais les outils que nous utilisons ne sont pas garantis de fonctionner directement sur ces données.

Les données sont regroupées par famille de d'instrument pour le produit~L3C. Une légère perte de précision est attendue, mais les produits L3U et~L3C peuvent néanmoins tous deux être utilisés dans l'étude de structures fines et où la présence de nuages n'est pas rédhibitoire~(\cite{merchant_2019}).

Enfin, le produit~L4 regroupe l'ensemble des capteurs en utilisant le schéma d'intégration variationnelle \verb|NEMOVAR| intégré dans le système \ab{ostia}~(\cite{good_2020}).
Ce produit donne ainsi un champ de \ab{sst} sans nuages, estimé par une combinaison des observations satellites, et de la prévision d'un modèle numérique d'un jour sur l'autre.
L'absence de couverture nuageuse et la simplicité d'usage se fait au détriment du lissage inévitable des structures les plus fines.

Une estimation quantitative du lissage due à l'interpolation est compliqué, car la paramétrisation de cette dernière dépend de la variabilité du champ de température à J"~1.
On peut néanmoins imaginer que le champ de \ab{sst} produit est plus lissé dans les zones normalement nuageuses, et de plus potentiellement plus éloigné de la réalité~(voir \cref{sec:donnees-sst_reanalyses}).
Il est tout de même possible de mitiger ces effets en ne comptabilisant dans nos statistiques seulement les pixels non"-nuageux, ce que nous faisons pour tous les résultats présentés.
Nous utilisons pour cela les données de \ab{chl} qui identifient les pixels nuageux.
% À noter qu'il serait également possible d'utiliser les erreurs d'intégration associées à chaque pixel, mais cela n'a pas été exploré.
% -> c'est pas vraiment possible, les erreurs d'inté sont données avec une résolution très faible

Sauf indication contraire, c'est le produit~L4 que nous utiliserons pour le champ de \ab{sst} dans toute la suite de cette thèse.
Il permet d'accéder à des données \ab{sst} simplement, avec une résolution convenable, et est stable sur une longue durée (plusieurs décennies).
Les données sont téléchargées depuis \ab{cmems}\footfullcite{sst_esacci}.

\subsubsection{Réanalyses}
\declareDataset{sst_reanalyses}

Une solution pour s'affranchir de la couverture nuageuse est de s'appuyer sur des produits de réanalyses.
Tôt dans ce projet, certaines pistes de travail (\ab{par-ex} le suivi temporel des fronts) nous ont poussé à tester l'utilisation de tels produits.
Les données produites par <infos et ref>, disponibles à la résolution~\resol{1}{12}~(\qty{\approx8}{\km})\footfullcite{sst_reanalyses}.
Outre la résolution assez faible de ce produit, on notera des différences importantes avec les autres produits, et qui ne sont visiblement pas due à un simple lissage du champ de \ab{sst}~(\cref{fig:comparaison-sst}).

% Problème ici: c'est très vieux (pendant stage de master), donc j'ai peu de traces.
% Mais visiblement j'utilisai \verb|GLOBAL_REANALYSIS_PHY_001_030| sur CMEMS\@.
% Il n'est plus dispo ou a changé de nom. Le successeur devrait être: \url{https://data.marine.copernicus.eu/product/GLOBAL_MULTIYEAR_PHY_001_030/description}.
% C'est une réanalyse au 1/12° (ie 8~km). C'est beaucoup.
% Mais peut-être que la comparaison peut quand même valoir le cout ?

\begin{figure}
  \includegraphics[width=\textwidth]{comparaison_sst.pdf}
  \captionT{Comparaison entre produits SST}{%
    Zoom sur une même fenêtre, le \frenchdate{2007}{04}{22}, de la \as{sst} pour quatres produits considérés: \dataname{sst_esacci}~(a), \dataname{sst_reanalyses}~(b), \dataname{sst_mur}~(c), et \dataname{sst_modis}~(d).
    Les données \dataname{sst_mur}, malgré leur haute résolution présentent des zones très lissées, par exemple dans la zone~\latlonRange{68W; 64W}; \latlonRange{40N; 42N}.
    Les données \dataname{sst_esacci} semblent en revanche capturer avec régularité les structures visibles sur les données plus brutes de \dataname{sst_modis}, excepté les plus fines évidemment.
  }
  \label{fig:comparaison-sst}
\end{figure}

\subsection{Chlorophylle-\emph{a}}
\label{sec:donnees-chl}
\declareDataset{chl_globcolour}

Pour le champ de \al{chl}, nous utilisons les données produites dans le cadre du projet GlobColour, développées, validées et distribuées par ACRI"~ST, France~(\cite{maritorena_2002}).
La version \guil{MultiYear}, au niveau L3 est utilisée.
On obtient des données à une résolution de~\resol{1}{24} (soit~\qty{\approx4}{\km}), journalière, avec des nuages.
Les données sont récupérées sur \ab{cmems}\footfullcite{chl_globcolour}.

Ce produit aggrège les données optiques de plusieurs capteurs: \ab{seawifs}, \ab{modis} Aqua et Terra, \ab{meris} à bord d'Envisat, \ab{viirs} à bord de \ab{snpp} et \ab{noaa}[-20], et enfin \ab{olci} à bord de Sentinel-3A et -3B.
Les données de reflectance de chaque capteur sont transformées en concentration de \ab{chl} avant d'être fusionnées en seul produit.
Les algorithmes pour le passage en \ab{chl} sont ajustés indépendamment pour chacun des capteurs, ce qui permet d'obtenir un produit cohérent et stable~(\cite{garnesson_2019}).


\subsection{Accord entre les produits}

Il est à noter qu'il est difficile de trouver des produits de \al{chl} à des résolutions supérieures à~\qty{4}{\km}, en tout cas au niveau global.
Cela motive à utiliser un produit \ab{sst} de résolution légèrement moindre afin d'éviter une étape supplémentaire de \engquote{downsampling} une fois les fronts repérés sur le champ de \ab{sst}.
Cela renforce notre choix pour les données \dataname{sst_esacci}, en défaveur du set \dataname{sst_mur}.

Les deux jeux choisis pour la \ab{sst} et \ab{chl} ont des résolutions spatiales similaires mais néanmoins légèrement différentes.
Les deux grilles sont plate"-carrées (régulières en latitude et longitude) mais le champ de \ab{chl} est défini sur une grille \glshref{epsg-chl} de résolution~\resol{1}{24}~(\qty{\approx4.6}{\km}); et le champ de \ab{sst} sur une grille \glshref{epsg-sst} de résolution~\resol{1}{20}~(\qty{\approx5.6}{\km}).

Sachant que les produits de composition du phytoplancton dont nous disposions (non décrits ici) sont définis sur la même grille que la Chlorophylle, nous adaptons la \ab{sst} sur la grille de la \ab{chl} par une simple interpolation bi"-linéaire.

\subsection{Bathymétrie}
\label{sec:donnees-bathymetrie}
\declareDataset{etopo1}

Nous utilisons les données de bathymétrie ETOPO1\footfullcite{etopo1}, fournies par \ab{noaa}.
Les données sont distribuées sur une grille plate"-carrée à une résolution d'une minute d'arc~(\resol{1}{60}), soit trois fois trop pour nous.
Nous sous"-échantillonnons les données en appliquant une moyenne à chaque groupe de 3\texttimes3~pixels, puis en interpolant le résultat sur la même grille que le reste des données de la même manière que pour la \ab{sst}.

\begin{figure}
  \centering
  \includegraphics[width=0.6\textwidth]{bathymétrie.pdf}
  \captionT{Bathymétrie de la zone d'étude}{%
    L'isobath~\qty{1500}{\m}, indiqué en rouge (les Bahamas sont exclues par visibilité), permet de repérer le talus continental et d'ignorer le plateau dans nos résultats.
  }
  \label{fig:bathymetrie}
\end{figure}

\section{Délimitations des biomes, ou sous-régions d'étude}
\label{sec:delimitations-regions}

Comme détaillé plus en profondeur en introduction~(\cref{sec:region-detude}, \cref*{chp:introduction}) notre région d'étude est fortement hétérogène, aussi bien concernant les propriétés physiques que biologiques.
Il est ainsi nécessaire de séparer notre zone d'étude en (sous-)régions afin d'extraire des résultats sur des zones homogènes.
Nous définissons donc trois biomes.
Le biome \ab{st-perm}, le plus au sud, correspond à un régime largement oligotrophe.
Le biome \ab{sp}, le plus au nord, comprend les eaux froides et productives au nord du Gulf Stream.
Enfin le biome \ab{st-sais}, entre les deux précédents, présente un régime intermédiaire: oligotrophe mais avec une production plus élevée permise par une couche de mélange profonde en hiver.
Dans cette section, nous décrivons plus précisement la méthode utilisée pour définir ces biomes spatialement.

\subsection{Séparation des biomes}

La séparation entre les deux biomes subtropicaux est faite par une limite zonale fixée à~\latlon{32N}~(trait noir pointillé~\cref{fig:regions}).
Cette limite correspond approximativement à un saut visible des valeurs de \ab{chl} à cette latitude (isocontour~\qty{0.1}{\mgm} de la moyenne annuelle de \ab{chl}) qui ne varie que peu au cours de l'année.
Cette séparation est en accord avec la limite entre les biomes tels que délimités par \textcite{sarmiento_2004}~(\cref{fig:sarmiento}).

\begin{figure}
  \centering
  \includegraphics[width=1\textwidth]{régions.pdf}
  \captionT{Résultat de la séparation de la zone d'étude en biomes}{%
    La zone est découpée en trois biomes: le biome \al{st-perm}~(PSB) au sud de du trait pointillé à~\latlon{32N}; le biome \al{st-sais}~(SSB) entre~\latlon{32N} et la limite (sinueuse) nord du Gulf Stream dénotée par le contour noir; et le biome \al{sp}~(SP) au nord du Gulf Stream.

    Clichés de \ab{sst}~(a) et \ab{chl}~(b) au \frenchdate{2007}{04}{22}, et les distributions de \ab{sst}~(c) et {d}~\ab{chl} pour chacuns des biomes, le même jour (\al{sp}:~bleu, \al{st-sais}:~jaune, \al{st-perm}:~rouge).
    La ligne noire verticale sur~(c) marque la température détectée comme seuil de la limite nord du Gulf Stream.
    Les axes des abscisses des distributions correspondent aux barres de couleurs.
    La ligne rouge suit l'isobath~\qty{1500}{\m}. Le plateau continental~(\qty{<1500}{\m}) n'est pas considéré et est masqué.
  }
  \label{fig:regions}
\end{figure}

\begin{figure}
  \centering
  \includegraphics[width=0.7\textwidth]{sarmiento_2004_fig2b.png}
  \captionT{Délimitations des biomes par des processus physiques}{%
    Figure tirée de \textcite[figure 2b]{sarmiento_2004}.
    Les biomes de notre zone d'étude correspondent dans la nomenclature de cet article à: \al{sp}~(``SP'', jaune); \al{st-sais}~(``ST-SS'', bleu); et \al{st-perm}~(``ST-PS'', rose).

    \foreignblockquote{english}{Biome classification scheme calculated using mixed layer depths obtained from observed density and from upwelling calculated from the wind stress divergence using observed winds.
    The equatorially influenced biome covers the area between \latlon{5S}~and~\latlon{5N}, and is colored a dirty light blue in areas where upwelling occurs (labeled~``Eq-U'' on the color bar) and dark pink in areas where downwelling occurs (labeled~``Eq-D'').
    Outside of this band, the region labeled ``Ice''~(red) is the marginal sea ice biome, the region labeled ``SP''~(yellow) is the subpolar biome, the region labeled ``LL-U''~(light blue) is the low-latitude upwelling biome , the region labeled ``ST-SS''~(dark blue) is the seasonally mixed subtropical gyre biome, and the region labeled ``ST-PS''~(pink) is the permanently stratified subtropical gyre biome.}
  }
\label{fig:sarmiento}
\end{figure}

On sépare ensuite le reste de la région au nord de~\latlon{32N} en prenant comme limite le front nord du jet du Gulf Stream (le \engquote{North wall}).
Cette délimitation est donc dynamique et déterminée chaque jour à partir de l'image de température.
En effet, il apparaît que la distribution de la \ab{sst} (au nord de~\latlon{32N}) suffit à repérer de manière fiable et robuste une température seuil permettant de séparer les deux biomes (contour sinueux noir plein~\cref{fig:regions}).
Sur cette distribution apparaît clairement un pic dans des valeurs élevées correspondant au eaux du jet.
Il est aisé de repérer le pic et l'ajuster par une gaussienne.
À partir de là, la température seuil entre les biomes est prise comme la base froide de ce pic, \ab{cad} plus précisemment en soustrayant à la température moyenne du pic deux fois son écart"-type~(\cref{fig:temp-seuil-distrib}).
La valeur journalière de ce seuil est filtrée temporellement par un filtre médian glissant (avec une fenêtre de largeur 8~jours) afin d'éviter d'éventuelles anomalies de détection.

\begin{figure}
  \centering
  \includegraphics[width=0.7\textwidth]{zone_separation.pdf}
  \captionT{Délimitation des biomes subtropical permanent et subpolaire par température seuil}{%
    Sur la distribution des valeurs en température au nord de~\latlon{32N}~(en noir) pour le \frenchdate{2007}{04}{22}, le pic \encadra{ici autour de~\tC{18}} correspond aux eaux du jet. Il est ajusté par un fit gaussien (en rouge) dans un intervalle de~\tC{5} de large.
    La température de seuil~(en bleu) entre les deux biomes est définie comme la température moyenne du pic~(trait fin pointillé) moins deux fois son écart"-type.
    Cela correspond à la limite nord du Gulf Stream.
  }
  \label{fig:temp-seuil-distrib}
\end{figure}

\begin{figure}
  \includegraphics[width=\textwidth]{separation_evol_month.pdf}
  \captionT{Variation temporelle de la délimitation entre biomes}{%
    La position limite entre les biomes \al{st-sais} et \al{sp} n'évolue que peu au cours de l'année vis à vis des larges méandres du Gulf Stream.
    Néanmoins en été, la limite est moins marquée après le détachement du jet~(\latlonRange{75W; 70W}).

    L'isotherme séparant les deux biomes est tracé le 15~de chaque mois pour l'année~2007.
    La couleur de chaque contour correspond au jour de l'année comme dénoté par la barre de couleur dans l'inset.
    Dans l'inset est tracé la température de seuil au long de l'année~2007. Les cercles marquent les jours et températures utilisées pour chaque contour.
  }
  \label{fig:var-delim}
\end{figure}

\subsection{Exclusion du plateau continental}

Par ailleurs, il est également nécessaire d'éviter de considérer les pixels trop près des côtes dans notre étude.
De manière générale la \ab{chl} y suit un régime côtier, visible par des valeurs très élevées~(\qty{>10}{\mgm}).
Ensuite, nous cherchons à éviter deux zones qui ne correspondent pas aux biomes définis plus haut.
Premièrement, le jet du Gulf Stream prend forme au sud de notre zone, le long de la côte de Floride. On trouve donc sur le plateau continental à ces latitudes~(\latlon{\approx28N}) de forts courants qu'on ne retrouve pas dans le reste du biome \ab{st-perm}.
Deuxièmement, plus au nord dans l'anse Nord-Est Américaine de l'Atlantique~(\engquote{Mid"-Atlantic Bight}), on trouve une séparation nette entre les eaux au nord du Gulf Stream (la \engquote{slope sea}), et les eaux sur le plateau continental.
Un jet marque cette séparation le long du talus continental~(\cite{flagg_2006}).

Dans les deux cas, imposer une limite haute à la bathymétrie sur notre région d'étude permet de supprimer ces zones problématiques.
Ainsi, pour calculer nos résultats, nous ne considérons que les pixels où la profondeur n'excède pas~\qty{1500}{\m}.
Nous utilisons pour cela les données de bathymétrie \dataname{etopo1}~(voir \datasect{etopo1}).

\section{Heterogeneity Index (HI)}
\label{sec:HI}

Comme précisé en introduction, pour quantifier l'effets des fronts sur le phytoplancton il est nécessaire de détecter les fronts, \al{cad} classer chaque pixel comme appartenant à un front ou non (\ab{cad} à l'arrière"-plan ou au \engquote{background}).
La méthode retenue ici suit celle présentée par \textcite{liu_2016} \encadra*{qui par son utilisation d'une fenêtre glissante s'apparente elle même à celle de \textcite{cayula_1992}}.
Cette section définit cette méthode et notre implémentation, tout en indiquant les modifications que nous y avons apportées.

La méthode de \textcite{liu_2016} consiste à quantifier plusieurs quantités statistiques du champ de \ab{sst} dont les fortes valeurs sont associées à la présence de fronts et autres structures de fine échelle.
Ces quantités sont la bimodalité, l'écart"-type (qui reflète le gradient du champ), et le coefficient d'asymétrie~(\engquote{skewness}).
Ces composantes sont ensuite réunies dans un seul index qui ainsi reflète l'hétérogénéité du champ de \ab{sst}, et donc baptisé \af{hi}.

Pour limiter la taille des structures détectées, on limite le calcul de ces composantes sur une fenêtre de taille appropriée, \ab{cad} pour nous de l'ordre de grandeur d'une dizaine de kilomètres.
Ainsi, pour chaque pixel, la valeur de chaque composante est calculée sur la distribution de \ab{sst} à l'intérieur d'une fenêtre glissante centrée sur ce pixel et dont les tailles possibles sont:
\begin{itemize}
        \item \(3 \times 3 =\qty{9}{\pixels}\) soit~\qty{17}{\km} de côté,
        \item \(5 \times 5 =\qty{25}{\pixels}\) soit~\qty{28}{\km} de côté,
        \item \(7 \times 7 =\qty{49}{\pixels}\) soit~\qty{39}{\km} de côté.
\end{itemize}
Parce qu'une fenêtre plus large entraînerait la détection de structures trop grandes pour notre étude, nous nous limiterons aux tailles présentées ci"-dessus.
Néanmoins une fenêtre trop petite limite le nombre de pixels disponibles pour calculer les valeurs statistique dont nous avons besoin convenablement.
Un compromis est à définir.

\subsection{Implémentation: Calcul des composantes}
\label{sec:calcul-composantes}

Nous définissons maintenant plus en détails le calcul des composantes du \ab{hi}.
Pour chaque position de la fenêtre, on s'intéresse aux~\(N\) valeurs de \ab{sst}~\(s_{i}\) valides (\ab{cad} des provenant des pixels sans nuages).

On commence par l'écart"-type~\ab{std}, calculé simplement par:
\begin{equation}
  \am{std} = \sqrt{\frac{1}{N-1} \sum_i \paren{s_i - \moy{s}}^2},
\end{equation}
avec~\(\moy{s}\) la moyenne des valeurs de température.

Ensuite, le coefficient d'asymétrie~\ab{skew}, est défini comme la valeur absolue du moment d'ordre trois d'une variable centrée réduite, et qui se calcule donc par:
\begin{equation}
  \am{skew} = \abs{\frac{\sum_i \paren{s_i - \moy{s}}^3} {N \sigma^3}}.
\end{equation}

Enfin, on cherche à quantifier la bimodalité~\ab{bimod} de la distribution des valeurs de \ab{sst}.
Pour ce faire on compare l'écart entre ladite distribution et une distribution gaussienne dont la moyenne et l'écart"-type sont pris identiques à ceux de la \ab{sst}~(\cref{fig:bimodality}).
Cela présuppose que lorsque les températures sont uni"-modales (\ab{cad} quand il n'y pas de fronts dans la fenêtre) leur distribution tend vers une gaussienne, ce qui ne paraît pas déraisonnable.

\begin{figure}
  \centering
  \includegraphics[width=0.6\textwidth]{bimodality.pdf}
  \captionT{Illustration du calcul de la bimodalité}{
    On calcule la bimodalité comme la norme de la différence entre l'histogramme des températures dans la fenêtre~(trait noir) et une distribution gaussienne de même moyenne et de même écart"-type~(trait rouge).
  }
  \label{fig:bimodality}
\end{figure}

De manière plus précise on commence par calculer l'histogramme~\(h_i\) des valeurs de \ab{sst} dans la fenêtre, en utilisant des intervalles de largeur fixe de~\tC{0.1} et compris entre les valeurs minimales et maximales dans la fênetre.
Les données de \ab{sst} étant stockées compressées par \engquote{linear packing}\footnotemark\ avec un facteur d'échelle de~\tC{0.01}, nos intervalles ont une largeur précisement égale à dix fois l'intervalle de discrétisation des valeurs de température.
Pour éviter que trop de valeurs tombent sur les bords des intervalles et que l'histogramme soit pollué par des erreurs numériques, on décale les intervalles de~\tC[parse-numbers=false]{0,01/2}.
Par ailleurs, le nombre d'intervalles étant dépendant de la largeur de la distribution de \ab{sst}, dans les cas où il est inférieur ou égal à quatre, la bimodalité est automatiquement assignée nulle~(\(B=0\)).
\footnotetext{%
  Cette technique de compression avec pertes \encadra{utilisée notamment par l'outil \citesoft{nco}\footnotemark{}} consiste à discrétiser des valeurs flottantes sur des entiers après une transformation linéaire.
  Par exemple, en prenant pour stockage des entiers non"-signés sur \qty{16}{\bits}~(\texttt{NC\_USHORT}) on peut évidement représenter des valeurs entières entre 0 et~\(2^{16}-1 = \num{65535}\); mais en multipliant ces valeurs entières par un facteur de, disons,~\num{0.005} on peut représenter des valeurs entre 0 et~\num{327.675}.
  On a gagné en volume par rapport à un stockage typique de 32 ou~\qty{64}{\bits}, mais en perdant évidemment en précision puisque nos valeurs sont maintenant discrétisées avec un intervalle de~\num{0.005}.}
\footnotetext{voir le guide utilisateur: \glsurl{nco-packed}}

Par ailleurs on définit une distribution gaussienne~\(g_i\) sur les même intervalles en utilisant les statistiques calculées précédemment:
\begin{equation}
  g_i = \frac{1}{\sqrt{2\pi\am{std}}} \exp\paren{-\frac{\paren{x_i-\moy{s}}^2}{2\am{std}^2}},
\end{equation}
pour ensuite calculer la norme~\(\mathbb{L}^2\) entre les deux distributions:
\begin{equation}
  \begin{split}
  \am{bimod} & = \norme[2]{h - g}\\
             & = \sum_i \paren{h_i - g_i}^2 .
  \end{split}
\end{equation}

Ce principe de calcul de bimodalité s'apparente à celle de \textcite{cayula_1992}, qui consiste à trouver une valeur seuil séparant l'histogramme des valeurs en deux classes et pour laquelle la variance intra"-classe est minimale.
Cette méthode est d'ailleurs utilisée en analyse d'image, non pas sur une fenêtre glissante mais pour toute l'image, et connue sous le nom de méthode d'\textcite{otsu_1979}.
Bien qu'elle soit appliquée avec succès pour la détection de front, il apparaît que cette méthode nécessite de construire un histogramme d'une résolution suffisante, ce qui s'avère difficile aux échelles où nous travaillons.

Notre méthode de calcul de la bimodalité est la première grande modification apportée à celle de \textcite{liu_2016}.
Ces dernier·ère·s calculent également l'histogramme de la température, mais au lieu de directement le comparer avec une distribution gaussienne, iels ajustent l'histogramme par un polynôme de degré~5 avant de comparer ce dernier avec la gaussienne.
Cet ajustement, par ailleurs difficile à mettre en place, est facilement mal conditionné et rien ne garantit sa convergence.
Nous proposons donc une méthode plus facile à implémenter, plus robuste, et au coût de calcul réduit.

\subsection{Implémentation: Coefficients de normalisation}
\label{sec:coef-normalisation}

Avant de pouvoir réunir les trois composantes il est nécessaire de leur appliquer un poids statistique équivalent à chacune. Cela est accompli par calcule de coefficients constants de normalisation.
Alors que \citeauthor{liu_2016} proposent de normaliser chaque composante~\(C^n\) par ses valeurs minimales et maximales (prises sur toutes les valeurs disponibles) selon:
\begin{equation}
  \norm{C_i^n} = \frac {C_i^n - \min(C^n)} {\max(C^n) - \min(C^n)} ;
\end{equation}
nous préférons plutôt normaliser par l'écart"-type:
\begin{equation}
  \norm{C_i^n} = \am{coef}^n C_i^n
  \text{ avec } \am{coef}^n = \frac {1} {\std(C^n)} .
\end{equation}
En effet un rapide coup d’œil aux distributions des composantes~(\cref{fig:distrib-composantes}) permet de constater qu'aucune des composante ne semble avoir des valeurs bornées.
Normaliser par les valeurs maximales donne d'une part une normalisation arbitraire, très sensible aux valeurs extrêmes, et d'autre part donne un poids disproportionné aux valeurs élevées dans la normalisation.
En revanche, la normalisation par l'écart"-type permet d'attribuer une part équivalente de la variance du \ab{hi} à chaque composante.

\begin{figure}
  \centering
  \includegraphics[width=0.7\textwidth]{distrib_composantes.pdf}
  \captionT{Distribution des valeurs de \glsentryshort{hi} et de ses composantes}{%
    Densités de probabilité des valeurs de \ab{hi}~(a) et de ses composantes: l'écart"-type~(b), le coefficient d'asymétrie~(c), et la bimodalité~(d), calculées sur l'année 2007.
    Chacune des composante s'est vue appliquée son coefficient de normalisation, mais pas le \ab{hi}, qui est donc la simple somme de ces composantes.
  }
  \label{fig:distrib-composantes}
\end{figure}

Jusqu'ici nous n'imposons aucune contrainte sur l'amplitude des valeurs du \ab{hi}.
Afin de \guil*{standardiser} quelque peu ses valeurs finales, nous définissons un quatrième coefficient de normalisation défini de sorte que~\pct{95} des valeurs du \ab{hi} soient inférieures à~\num{9.5}.

Les coefficients de normalisation~\(\am{coef}^n\) sont obtenus par analyse des distributions des composantes pour l'année~2007 uniquement. Ils sont ensuite appliqués de manière uniforme au reste des données.

On peut donc enfin calculer le \ab{hi} avec:
\begin{equation}
  \am{hi} = \am{coef}^4 \paren{
    \am{coef}^1\am{std}
    + \am{coef}^2\am{skew}
    + \am{coef}^3\am{bimod}}.
\end{equation}

On obtient donc finalement un indice capable de quantifier l'hétérogénéité du champ de température à une échelle donnée, ce qui nous permet d'identifier des structures fines dans ce même champ~(\cref{fig:exemple-composantes,fig:exemples-fronts}).
Il reste toutefois à définir une méthode pour classifier chaque pixel comme appartenant, ou non, à un front.
Nous allons même plus loin et séparons les pixels en trois catégories: les pixels appartenant à un front fort~(\(\am{hi} > 5\)), à un front faible~(\(5 < \am{hi} < 10\)), ou à aucun front~(\(\am{hi} < 5\), aussi dénotés \engquote{background}).
La pertinence des valeurs de seuils entre les catégories est discutée dans la suite du manuscript.

\begin{figure}
  \centering
  \includegraphics[width=\textwidth]{exemple_composantes.pdf}
  \captionT{Exemples de construction du \glsentryshort{hi} à partir de ses composantes}{%
    Champ de \ab{sst}~(a) pour le \frenchdate{2007}{04}{07}, à partir de laquelle on calcule les composantes du \ab{hi}: l'écart"-type~(c), l'asymétrie~(e), et la bimodalité~(f) (ici toutes représentées normalisées par leur variance).
    Le \ab{hi}~(b) est ensuite obtenu par la somme de ces composantes normalisées, avant d'être normalisé lui aussi afin que~\pct{95} de ses valeurs soient inférieures à~\num{9.5}.
    Le \ab{hi} permet de détecter les fronts de \ab{sst}, ici deux valeurs du \ab{hi} normalisé sont contourées, à 5~(trait plein) et 10~(trait pointillé).
  }
  \label{fig:exemple-composantes}
\end{figure}

\begin{figure}
  \centering
  \includegraphics[width=\textwidth]{exemples_fronts.pdf}
  \captionT{Exemples de structures fines}{%
    Champs de \ab{sst}~(colonne gauche), \ab{chl}~(colonne centre), et \ab{hi}~(colonne droite) pour trois exemples de structures: 1\ier~exemple~(1a--c) le \frenchdate{2007}{04}{07}, 2\ieme~exemple~(2a--c) le \frenchdate{2007}{02}{23}, et 3\ieme~exemple~(3a--c) le \frenchdate{2007}{02}{28}.
    Chaque fenêtre représente une surface d'environ~\qtyproduct{200x200}{\km}.
    Deux valeurs seuil de \ab{hi} sont contourées, à 5~(trait plein) et 10~(trait pointillé).
  }
  \label{fig:exemples-fronts}
\end{figure}


\subsection{Sensibilité aux paramètres}
\label{sec:HI-sensibilite}

Je pense bouger cette section vers les résultats. C'est mieux de le présenter après je crois.

Nécessité d'évaluer les incertitudes sur la méthode.
Pour voir si résultat significatif.
Taille de la fenêtre glissante. Coefs de normalisation.


\section{Extraction des résultats}
\label{sec:extraction-res}

\subsection{Utilisation d'histogrammes}
\label{sec:extraction-hist}

Une fois le \ab{hi} calculé, il devient possible de catégoriser chaque pixel par la biome auquel il appartient (subtropical permanent, subtropical saisonnier, subpolaire), ainsi que par sa valeur de \ab{hi}.
On cherche ensuite à extraire des informations de ces ensembles de pixels ainsi consistués.
Le nombre total de pixels étant conséquent, et parce que l'établissement des ensembles est compliqué et coûteux, les ensembles de pixels sont chacun réduits à des histogrammes de variables d'intérêt (\ab{sst}, \ab{chl}, ou les \abp{pft}).
Cela diminue les étapes de calcul ainsi que la quantité de données à traiter pour obtenir un diagnostic.

Prenons l'exemple d'une seule image. Pour notre région d'étude, cela représente \qtyproduct{1000 x 1000}{\pixels}.
On cherche à extraire un simple diagnostic, par exemple la moyenne de la \ab{chl} dans, et hors des fronts pour chacun des biomes.
Il nous faut donc séparer la région en biomes, ce qui nécessite rappelons-le de discriminer les pixels par leur température~(\cref{sec:delimitations-regions}).
Il faut également discriminer les pixels par leur valeur de \ab{hi} pour séparer fronts et \eng{background}.
On peut maintenant calculer nos statistiques sur chacun des ensembles de pixels constitués, ce qui représente ici de faire 6~calculs (3~biomes~\texttimes\ fronts/\eng{backgound}) sur environ~\qty{e6}{\pixels}.
Mais une fois nos histogrammes calculés, un diagnostic ne requiert que de regarder le ou les histogrammes appropriés, et ce qui représente beaucoup moins d'efforts \encadra*{et de données comme nous allons le voir}.

Les histogrammes peuvent être rendus représentatifs sans pour autant utiliser un nombre prohibitif d'intervalles, d'autant plus que les données (\ab{chl} et \ab{sst}) sont déjà stockées compressées avec pertes. Pour la température par exemple, 450~intervalles suffisent à couvrir toutes les valeurs (de \tC{-5} à~\tC{40}), avec une largeur d'intervalle de~\tC{0.1} équivalente à l'incertitude sur la mesure.
Les intervalles pour la \ab{chl} et les autres variables biologiques sont pris de largeur logarithmique afin de couvrir les plusieurs ordres de grandeur que peuvent prendre leurs valeurs.

Les histogrammes présentent également l'avantage de pouvoir facilement être combinés entre eux.
En effet, tous les histogrammes calculés sont stockés non"-normalisés, c'est-à-dire en nombre de pixel par intervalle. Ainsi plusieurs histogrammes peuvent être sommés entre eux avant d'être normalisés pour en extraire une valeur, comme la valeur médiane de la distribution résultante par example.
Ce procédé est notamment utilisé pour calculer des diagnostics sur des périodes de temps autres que journalières, sans avoir besoin de refaire un calcul coûteux impliquant les pixels.

Dans la suite on détaille le processus de normalisation des histogrammes.
On considère un histogramme qui compte~\(h_i\) valeurs d'une variable quelconque~\(x\), pour le i\ieme{}~intervalle~\(\left[x_i; x_{i+1} \right]\).

Pour un histogramme donné (\as{par-ex} pour une région à une date et un type de front donné), on extrait facilement le nombre de pixels total:
\begin{equation}
  N = \sum_i h_i,
\end{equation}
ou une approximation de la valeur moyenne:
\begin{equation}
  \moy{x} = \frac{\sum_i h_i x_i} {N}.
\end{equation}

Pour d'autres valeurs à extraire il est nécessaire de normaliser les histogrammes afin d'obtenir une densité de probabilité. On prendra soin de considérer les tailles des intervalles dans les calculs.
La largeur des intervalles est~\(w_i = x_{i+1}-x_i\). On transforme le nombre de points en probabilité par unité de valeur~\(p_i = h_i / w_i \), avant de le normaliser de sorte à obtenir une intégrale égale à 1:
\begin{equation}
  f_i = \frac{p_i} {\sum_j p_j w_j},
\end{equation}
pour ainsi obtenir une approximation de la densité de probabilité~\(f\).

On peut extraire de cette distribution notamment la médiane, ainsi que des percentiles divers en trouvant la valeur de~\(x\) pour laquelle la somme cumulée de la densité de probabilité est égale au percentile recherché (\num{0.5} dans le cas de la médiane).
On délègue ce travail au paquet SciPy dont l'objet~\glshref{rv_histogram} permet d'effectuer ces calculs de manière triviale.

Il est à noter que l'interprétation de certaines métriques n'ont de sens que si la distribution de la variable concernée est convenable (uni"-modale \ab{par-ex}).
Cette vérification est reportée dans la \cref{sec:complements-chl},~\cref*{chp:res-chl}.

\subsection{Quantification de l'effet des fronts: différences de valeurs et retard du bloom}
\label{sec:extraction-surplus-lag}

Afin de comparer les valeurs de diverse variables à l'intérieur et à l'extérieur des fronts, nous définissons deux métriques.
La première est l'excès~\ab{exces} de \ab{chl} (\engquote{excess} dans l'article \cref{sec:article-bg}, \cref*{chp:res-chl}), qui compare localement les valeurs dans les fronts et le background. On le définit simplement comme la différence relative entre la médiane de \ab{chl} dans et hors des fronts:
\begin{equation}
  \am{exces} = \frac{\med{\am{chl}}_\frt - \med{\am{chl}}_\bkg}
  {\med{\am{chl}}_\bkg} .
\end{equation}
Le calcul est fait pour les fronts faibles et forts de la même manière, le background désignant toujours les pixels de \ab{hi} faible~(\(\am{hi} < 5\)).
L'excès est calculé dans des bandes de latitudes larges de~\ang{5}, afin de minimiser l'influence des gradients de grande échelle.

Cette métrique ne tient compte que de la distribution des valeurs de \ab{chl}. Elle ignore la proportion de fronts dans une zone donnée, et ne représente pas la quantité totale de \ab{chl} présente dans un biome.
On définit donc une deuxième métrique, le surplus~\ab{surplus} de \ab{chl} dans tout un biome (\enquote{biome surplus} dans l'article \cref{sec:article-bg}, \cref*{chp:res-chl}) comme la différence relative entre la moyenne de \ab{chl} dans tout le biome et la moyenne dans le background:
\begin{equation}
  \am{surplus} = \frac{\moy{\am{chl}}_\tot - \moy{\am{chl}}_\bkg}
  {\moy{\am{chl}}_\bkg} .
\end{equation}

\subsection{Décalage du bloom}
\label{sec:decalage-bloom}

L'évolution temporelle de la médiane de \ab{chl} dans le biome subpolaire présente un bloom printanier dont nous mesurons la phénologie \encadra{plus précisement la date de démarrage et la durée du bloom} à la fois dans le \eng{background} et les fronts.
Étant donné que le bloom se propage vers le nord, ces mesures sont faites sur des bandes de latitudes de largeur~\ang{5}.

Pour extraire ces mesures, nous prenons la série temporelle de la médiane de \as{chl}, à laquelle nous appliquons un filtre Butterworth d'order~2 et de fréquence de coupure~\qty[parse-numbers=false]{1/20}{\per\jours}.
La série filtré montre de fortes variations dans leur phénologie d'une année sur l'autre, mais un bloom est toujours discernable.
Seuls les données de février à juillet sont considérées ce qui permet d'isoler le bloom printanier et d'exclure le bloom automnal.
Nous utilisons également la dérivée de \as{chl} que nous calculons à partir de la série filtrée\footnote{%
  La dérivée est également ré-échantillonnée par interpolation cubique à une résolution temporelle de 3~heures, ce qui permet de repérer plus précisement le maximum de dérivée sans avoir à passer par une régression.
}.

Premièrement, nous détectons la valeur maximale de \as{chl}~(étape\,\textcircled{1}, en orange, sur la \cref{fig:methode-timing-bloom}a).
Deuxièmement, nous prenons pour démarrage du bloom l'instant~\as{onset}, antérieur à l'instant de \as{chl} maximale, pour lequel la dérivée de \as{chl} est maximale~(étape\,\textcircled{2}, en rouge, \cref{fig:methode-timing-bloom}c).
Afin d'estimer l'incertitude de cette mesure, nous prenons l'écart"-type de l'ensemble des dates pour lesquels la dérivée de \as{chl} est supérieur à~\pct{90} de sa valeur maximale.

\begin{figure}
  \centering
  \includegraphics[width=\textwidth]{phenologie_2015_N:45-50.pdf}
  \captionT{Mesure du timing du bloom entre fronts et background}{%
    Les dates de démarrage du bloom et sa durée sont obtenues à partir des séries temporelles de la médiane de \as{chl}~(ligne du haut) et de sa dérivé~(ligne du bas). Les données proviennent de l'année~2015 et la bande de latitude~\latlonRange{40N; 45N}.

    Le processus est illustré sur la série correspondant aux fronts faibles~(colonne de gauche):
    \textcircled{1}\,Nous déterminons l'instant pour lequel la \as{chl} est maximale, puis \textcircled{2}\,prenons pour démarrage du bloom l'instant antérieur à ce dernier et pour lequel la dérivée de \as{chl} est maximale.
    Enfin, \textcircled{3}\,la date de fin de bloom est prise comme la date la plus tardive pour laquelle la \as{chl} est supérieure à~\pct{75} de sa valeur maximale.

    Les résultats pour les trois séries temporelles correspondant au \eng{background}~(rouge), fronts faibles~(bleu), et fronts forts~(vert) sont superposés~(colonne de droite).
  }
  \label{fig:methode-timing-bloom}
\end{figure}

Enfin, nous prenons comme date de fin du bloom la date la plus tardive pour laquelle la \as{chl} est supérieure à~\pct{75} de sa valeur maximale~(étape\,\textcircled{3}, en vert, \cref{fig:methode-timing-bloom}a).
Pour obtenir une incertitude nous prenons ici l'écart"-type de l'ensemble des dates pour lesquelles la \as{chl} est comprise dans un intervalle de~\pct{5} autour du seuil~(\as{cad}~\pct{75\pm2.5}).
La durée~\as{duree} du bloom est prise entre les dates de démarrage et de fin ci"-dessus.

Il est légitime de questionner le détail de ces méthodes \encadra{d'autant plus pour certaines années ou le démarrage du bloom est étalé sur une longue période, ou fragmenté en plusieurs pics}, cependant on s'intéresse ici aux différences entre les valeurs trouvées dans les fronts et dans le \eng{background}~(\cref{fig:methode-timing-bloom}\,b et~d).
Étant donné que l'on effectue ces mesures de manière identique dans les deux cas, on s'attend à trouver des différences de valeurs pertinentes.

Les variables retenues pour l'étude de la phénologie du bloom sont donc le retard au démarrage du bloom~(\(\am{lag} = \am{onset}_\frt - \am{onset}_\bkg \), \enquote{lag} dans l'article), et l'écart de durée~(\(\am{lag-duree} = \am{duree}_\frt - \am{duree}_\bkg\)).
Leurs valeurs annuelles (ainsi que les incertitudes associées) sont moyennées avec pondération.
Par exemple, pour le retard de valeur annuelle~\(\am{lag}_y\) et d'incertitude~\(\err{\am{lag}}_y\), en prenant comme pondération~\(w_y\):
\begin{equation}
  w_y = \frac 1 {1 + \err{\am{lag}}_y} ,
\end{equation}
on calcule les valeurs moyennes:
\begin{equation}
  \am{lag} = \frac {\sum_y w_y \am{lag}_y} {\sum_y w_y}
  \text{ et }
  \err{\am{lag}} = \sqrt{ \frac1{N-1} \frac {\sum_y w_y \paren{\am{lag}_y - \am{lag}}^2 } {\sum_y w_y} },
\end{equation}
avec~\(N\) le nombre d'années.
Le calcul est identique pour l'écart de durée~\as{lag-duree}.

\documentclass[index]{subfiles}
\begin{document}

\chapter{Impact des fronts sur Chlorophylle}
\label{chp:res-chl}

\section{Introduction}

motivation.

Résultat attendu sur la biomasse.
Permet de valider la méthode aussi.


\section{Éxamples de fronts}

Trouver des images examples de fronts.
Comparer visuellement avec les données MODIS.

Histogrammes individuels ?

Il est difficile de trouver des examples corrects.
D'une part à cause de la grande couverture nuageuse.
D'autre part parce que l'effet sur la Chl est difficilement visible sur des images. (effet statistique).

\section{Relation Chl vs HI}

Augmentation de la \gls{chla} avec le \gls{hi}.
Plus précisement: déplacement des valeurs vers le haut.
Valeurs les plus hautes en \gls{chla} n'apparaissent que pour des \gls{hi} forts.

\section{Estimation de l'augmentation du Chl}

figure de la climatologie
augmentation de la médiane
surplus

découpage par zones, saisonalité

\section{Timing du bloom}

Ajouter courbes de détection avec explications.
Départ plus tôt des quelques jours.
Durée plus longue de quelques jours.

\section{Discussion}

Reprendre discussion de l'article essentiellement.

\biblio
\end{document}

\include{tex/res_phénologie}
% chktex-file 13

\chapter{Conclusion}
\label{chp:conclusion}

Dans le cadre de cette thèse, grâce à 20~années de données satellites nous avons montré et \emph{quantifié} l'impact des fronts sur la \al{chl} dans l'Atlantique Nord autour du Gulf Stream, selon la saison et la biorégion concernée.
Nous réaffirmons l'augmentation \emph{locale} de la \al{chl} dans les fronts par rapport à l'environnement avoisinant.
Cet excès local est d'autant plus élevé que \emph{l'intensité des fronts} est importante.
Il est particulièrement important immédiatement autour du Gulf Stream (\pct{+60} en moyenne), ainsi que pendant le bloom printanier (où il atteint \pct{+150}).
En revanche, en calculer le surplus de \al{chl} à \emph{l'échelle de la biorégion} \encadra{en donc en comptabilisant la \emph{surface occupée} par les fronts} cette est augmentation est moins spectaculaire (inférieur à \pct{+5}).
Ce surplus est de valeur comparable pour les fronts de fortes et de faible intensité, ces derniers ayant un impact local moindre mais compensent par une plus importante surface occupée.
Dans le biome subpolaire, nous apportons des preuves observationnelles que le \emph{démarrage du bloom} se fait \emph{plus tôt dans les fronts} que dans le reste de la zone \encadra*{de deux semaines environ}.
Ce résultat constitue une première observation directe et quantification de ce phénomène.
Ces observations de la réponse du phytoplancton aux fronts de fine échelle participent à mieux cerner leur impact global sur les cycles biogéochimiques, et ainsi améliorer nos prédictions face au changement climatique.

Au delà de ces résultats, nous incrémentons et validons une méthode de détection des fronts dans une zone complexe.
En particulier, nous montrons son adéquation à quantifier l'intensité des fronts détectés.
En outre, nous reconnaissons que malgré l'importance et les applications potentielles de la détection de fronts, il n'existe pas encore d'outils ou de produits facilitant leur usage scientifique.
Sans pour autant fournir une solution définitive à cette lacune, ce travail souligne des objectifs à atteindre et fournit un exemple d'outil facilement réutilisable.
Nous avons également pu identifier un certains nombre de difficultés se dressant devant l'établissement de tels outils ou produits.
De multiples algorithmes de détection de fronts existent déjà, cependant leur usage n'est pas forcément trivial.
En cause est leur distribution dans des canaux et des languages inadaptés, ou manquant de documentation, et dont on ne peut que souhaiter qu'elle se rapproche des standards et outils aujourd'hui en place dans le milieu de l'open-source.
Ces manquements incombent un travail supplémentaire à leur application, pourtant déjà complexe du fait, entre autres, de la nécessité de s'adapter aux caractéristiques régionales (que nous montrons ici), et du large volume de données nécessaire aux évaluations de l'impact des fines échelles.

Si nous avons souligné les difficultés importantes dans la mise à disposition d'outils et produits de détection de fronts, nous notons également leur fort potentiel, et même leur nécessité à la quantification et la compréhension de phénomènes biogéochimiques importants.

\fancybreak[3\onelineskip]


\begin{appendices}
\chapter[Article: \articleCceTitle][Article]{Article: \articleCceTitle}
\label[appendix]{ax:article-cce}
\chapter[Article: \articleReviewTitle][Article]{Article: \articleReviewTitle}
\label[appendix]{ax:article-review}

\renewcommand{\thesection}{\Alph{chapter}.\arabic{section}} % No chapter number in section
\chapterlof{Autres}
\label[appendix]{ax:autres}

\section{Compression par troncation binaire}
\label{ax:compression-troncation}

Description de la technique de compression \enquote{packed}.
Je le mentionne rapidement dans le texte principal, et par ailleurs c'est une technique que j'utilise beaucoup pour stocker mes données.
Mais elle n'est pas décrite académiquement (ou j'en ignore le nom réel), elle apparait seulement dans le manuel~\citesoft{nco}.

\end{appendices}

{
  \emergencystretch=1em
  \printbibliography[heading=bibintoc, filter=normal]
  {
    \raggedright%
    \printbibliography[heading=subbibintoc, type=dataset, title=Données]
    \printbibliography[heading=subbibintoc, type=software,
    notkeyword=personnal, title=Logiciels]
  }
}

\backmatter%

\end{document}

