
\RequirePackage{silence}
\WarningFilter{latex}{Writing or overwriting file}
\WarningFilter{minitoc(hints)}{W0024}

% Add fields to the bibliography datamodel
% Formatting and sorting is done in preamble in bib_setup.tex
\begin{filecontents*}[overwrite]{biblatex-dm.cfg}
\DeclareDatamodelFields[type=list, datatype=literal, label=true, nullok=true]{datavar}
\DeclareDatamodelEntryfields[dataset]{datavar}
\DeclareDatamodelFields[type=field, datatype=verbatim, nullok=true]{pypi}
\DeclareDatamodelEntryfields[software]{pypi}
\end{filecontents*}

%%% Document class
\documentclass[
12pt,
a4paper,
french,
openright,
twoside,
hyperfootnotes=false, % put here otherwise option clash
draft,
]
{memoir}

\usepackage{src/ch-thesis}

\AtBeginDocument{%
  \bookmark[named=FirstPage, level=section]{Page de garde}%
}

\begin{document}
\dominitoc

%%% Cover page

\pagenumbering{Alph}
\newgeometry{left=1.5cm, right=1.5cm, top=2cm, bottom=3cm}

\begin{titlingpage}

\begin{center}
  \includegraphics[height=3em]{Logos/sorbonne.pdf}
  \hfill
  \includegraphics[height=3em]{Logos/locean.png}
  \hfill
  \includegraphics[height=3em]{Logos/ipsl.png}
  \hfill
  \includegraphics[height=3em]{Logos/chanel_enspsl.png}

  \vspace{1cm}

  {\LARGE Sorbonne Université}\\[2ex]
  École Doctorale 129 Sciences de l'Environnement\\
  \emph{Laboratoire d'Océanographie et du Climat: Expérimentations et Approches Numériques}

  \vspace{3cm}

  \par\noindent\rule[0.7em]{\textwidth}{2pt}
  {\bfseries\Large \Title}\\
  \par\noindent\rule{\textwidth}{2pt}

  \vspace{3cm}

  Thèse de doctorat\\
  Spécialité: Cycles biogéochimiques et changements environnementaux globaux

  \vspace{1cm}

  {\normalsize par \bsc{Clément Haëck}}

  \vspace{1cm}

  Dirigée par \bsc{Marina Lévy} et \bsc{Laurent Bopp}

  \vspace{2cm}
\end{center}

\par\noindent Présentée et soutenue publiquement le 30 avril 2023,\\
devant un jury composé de:

\makeJury{
  \jury{Marina Lévy}{DR CNRS}{Directrice de thèse}
  \jury{Laurent Bopp}{DR CNRS}{Co-directeur de thèse}
}

\end{titlingpage}

\restoregeometry%


%%% Front matter
\frontmatter

\pagenumbering{Alph}
\newgeometry{left=1.5cm, right=1.5cm, top=2cm, bottom=3cm}

\begin{titlingpage}

\begin{center}
  \includegraphics[height=3em]{Logos/sorbonne.pdf}
  \hfill
  \includegraphics[height=3em]{Logos/locean.png}
  \hfill
  \includegraphics[height=3em]{Logos/ipsl.png}
  \hfill
  \includegraphics[height=3em]{Logos/chanel_enspsl.png}

  \vspace{1cm}

  {\LARGE Sorbonne Université}\\[2ex]
  École Doctorale 129 Sciences de l'Environnement\\
  \emph{Laboratoire d'Océanographie et du Climat: Expérimentations et Approches Numériques}

  \vspace{3cm}

  \par\noindent\rule[0.7em]{\textwidth}{2pt}
  {\bfseries\Large \Title}\\
  \par\noindent\rule{\textwidth}{2pt}

  \vspace{3cm}

  Thèse de doctorat\\
  Spécialité : Cycles biogéochimiques et changements environnementaux globaux

  \vspace{1cm}

  {\normalsize par Clément Haëck}

  \vspace{1cm}

  Dirigée par Marina Lévy et Laurent Bopp

  \vspace{2cm}
\end{center}

\par\noindent Présentée et soutenue publiquement le 31 février 2023,\\
devant un jury composé de:

\begin{center}
\begin{tabular}{llr<{\raggedleft}}
  Prénom NOM & Titre & Rapporteur \\
  Prénom NOM & Titre & Rapporteur \\
  Prénom NOM & Titre & Président \\
  Marina LÉVY & DR CNRS & Directrice de thèse \\
  Laurent BOPP & DR CNRS & Co-directeur de thèse \\
\end{tabular}
\end{center}

\end{titlingpage}
\restoregeometry

\frontmatter

\section{Résumé}
Résumé

\clearpage
\section*{Remerciements}
Remerciements \dots

\clearpage
\section{Publications et productions}

\begin{itemize}
        \item Article CHL
\end{itemize}
\medskip

Lors de mon travail de thèse, j'ai été amené à écrire des outils qu'il m'a parru utile de rendre publiques et accesibles. Tous les codes utilisés sont disponibles ici: \url{https://gitlab.in2p3.fr/clement.haeck/submeso-color}.
\medskip

Certains outils sont distribués à part:
\begin{itemize}
  \item \emph{FileFinder}: un paquet python qui permet entre autres de trouver des fichiers grâce à la structure de leur nom de fichier (\url{https://github.com/Descanonge/filefinder}).
  \item \emph{XArray-Histogram}: un paquet python qui permet de calculer des histogrammes depuins des données gérées par XArray (\url{https://github.com/Descanonge/xarray-histogram}).
  \item \emph{Tol-Colors}: un paquet python qui donne accès à des jeux de couleurs adaptés aux personnes daltoniennes. Les jeux de couleurs existaient déjà, je les ai seulement rendus accessibles sur Pypi (\url{https://github.com/Descanonge/tol_colors}).
  \item \emph{Dateloop}: un script bash permettant de générer des ensembles de dates (\url{https://github.com/Descanonge/dateloop}).
\end{itemize}
\medskip

J'ai également participé à un projet open-source visant à rendre accessible au paquet python \emph{XArray} les conventions de métadonnées CF (\url{https://github.com/xarray-contrib/cf-xarray}).
J'ai ajouté le support pour les variables drapeau utilisant un masque binaire.


%%% Glossary
\setglswidth%
\printglossary[toctitle=Glossaire]

%%% TOC
\clearpage
{
  \setlength{\baselineskip}{1.03\onelineskip}
  \tableofcontents*
  % phantomsection already called by \tableofcontents
  \addcontentsline{toc}{section}{\contentsname}
}

\clearpage

%%% LOF
\listoffigures*
% phantomsection already called by \listoffigures
\addcontentsline{toc}{section}{\listfigurename}

\clearpage

%%% Mainmatter
\mainmatter%

\documentclass[index]{subfiles}
\begin{document}

\chapter{Introduction}
\label{chp:introduction}

\tocsubfile

\section{Préambule}

Quelques pages d'introduction générale pour donner les objectifs.

Le plancton objet intéressant.
Importance dans le cycle du carbone et la chaine alimentaire.
Complexe car grande variété et du fait le l'environement pas fixe.
Porté par les courants: impacté par des phénomènes physiques.

D'une part à grande échelle (basins, circulation océaniques) où sont définis les caractéristiques physique de l'environnement.
Mais plus récemment, notamment grâce à l'imagerie satellites, on se rend compte que la bio est forcée aux fines échelles (quelques jours, quelques kms). Échelles ou apparaissent non seulement de la filamentation par mélange horizontal, mais aussi des circulations agéostrophiques qui affectent les échanges verticaux (par des vitesses verticales et par modification du mélange vertical).
Expliquer l'importance des fronts dans ces 2 phénomènes.

paragraphe upwelling.

paragraphe restratification.

Pourtant il est d'importance de pouvoir mieux comprendre ces interactions biophysiques et de les quantifier.
D'une part parce que visiblement ça régit la bio à ces échelles, mais ça peut changer significativement l'image à grande échelle (preuve ?).
D'autre part parce que les modèles climatiques qui font les prévisions à l'aune du CC sont incapables (et le resteront pour un bout de temps) de résoudre ces échelles. Il faut donc arriver à donner une paramétrisation des ces effets pour obtenir des prévisions sensées (ce qui est également vrai pour la physique seule).
La quantification des interactions biophysique aux fines échelles est encore à faire (quelques exemples existent).

Il est difficile d'observer le plancton à ces échelles. Les campagnes in-situ doivent sampler précisement des structures fines, et on arrive à la limite de résolution des satellites (qui ne fournissent que la biomasse totale en surface).

Malgré ses limites, l'imagerie satellite permet une observation synoptique à large échelle (spatio-temporelle).
De plus de récentes avancés donnent également accès à la composition du phytoplancton.
Opportunité de quantifier ces effets sur une large zone (avant de passer à une étude globale).
On choisit pour cela la région Nord Atlantic autour du Gulf Stream. On a accès sur une zone relativement restrainte à différent biomes (olligotrophe, bloom important dans une zone subpolaire, bloom faible dans une zone olligotrophe mais avec apport de nutriments par ML profonde en hiver).

Objectifs:
\begin{itemize}
  \item
  \item
  \item
\end{itemize}


\section{État de l'art}
\label{sec:etat-de-lart}

\subsection{Le phytoplancton dans le système terre}
\label{sec:phyto-ds-sys-terre}

\subsubsection{Définition générale}
\label{sec:phyto-def-gen}

définition: plancton = organisme porté par les courants.
Phyto: ils pratiquent la photosynthèse, ils transforment le carbone dissous dans l'eau en matière organique.
Grande variété: de taille et de type et d'espèces.

50\% de la productivité primaire.
base de la chaîne alimentaire dans les océans.

Pour croître: ont besoin de lumière, de nutriments, de certaines conditions environnementales (température, salinité?, acidité). Dans le détail il y a des spécificités selon chaque espèce.
Sources de mortalité: vieillesse, virus, broutage par le phytoplancton
Quelques mots sur le phytoplancton Plus gros. Migration diurne. Ici aussi grande variété, préférence de broutage: complexité supplémentaire.

Les limitations majeures restent la lumière et les nutriments.
La lumière n'est pas présente partout identiquement: latitude, saisonnalité, et surtout profondeur. Importance de la couche euphotique, en surface.
Également limité en nutriments dans la quasi-totalité des eaux libres. Les eaux profondes (ou la lumière ne pénètre pas) contiennent des nutriments cependant. On peut définir une nutricline.

Cette dualité des sources de croissance (lumière en surface, nutriments en profondeur) rend les échanges verticaux très importants.
Or on considère généralement les courants océaniques à l'ordre 1 comme strictement horizontaux.
Les petites et moyennes échelles ((sub-)mésoéchelles) présentent des moyens de créer des échanges verticaux (comme on le verra plus tard).

\subsubsection{Télédétection}
\label{sec:teledetection}

Le phytoplancton est complexe à étudier.
Une observation synoptique est seulement possible par satellite.
Ces obs se font sur la couleur de l'océan. (quelques mots sur le principe, depuis quand ça existe, révélation des fines échelles dans les 1ères images).
Obs de la couleur présente des problèmes. Le lien avec le phytoplancton est indirect et imparfait (utilisation de la Chl-a comme proxy de biomasse). Limitation importante des observations par la couverture nuageuse. Et cela ne scanne que la composition de la surface (combien de m en vrai ?), pas accès à toute la colonne d'eau.

Expliquer la méthode pour obtenir Chl-a depuis satellite. Et méthode de Roy également (rapidement).

C'est pour cela que les observations in-situ toujours très utiles (pour combler les trous, ou compléter les obs sat). Vision de toute la colonne d'eau. Associé à une obs de la densité (temp + salinité qui n'est pas dispo en sat. à petite échelle).
Accès à la composition du phytoplancton (et zoo) grâce à différents outils (cytomètre, zooscan, HPLC, -omiques).
Mais aussi limitations. Toutes ces obs sont compliquées à mettre en place. Avoir une vue de toute la colonne d'eau pour tous ces paramètres (physiques, bio phyto + zoo) est compliqué techniquement. Une obs est limitée à une très petite zone spatio-temporelle. Obligé de cibler une zone d'intérêt (supposé).
Heureusement ces dernières années de telles campagnes se multiplient, devant la nécessité d'avoir tous ces paramètres dans des zones précises pour mieux comprendre ce qu'il se passe.

J'ai parlé que des obs mais des modèles numériques sont aussi dispo cependant.
Tous les modèles sont faux. Tous les modèles bio sont très faux.
La très grande variabilité des organismes est mal représentée (faute de puissance entre autre). Les petites échelles qui sont pourtant si importantes à la bio (subméso) sont très coûteuses à faire tourner et on est limité à de petites régions géographiques. Impossible de quantifier les effets convenablement avec des modèles.
On a pas de paramétrisation des processus bio. Est-ce que c'est seulement possible sachant qu'on peut avoir des effets non-locaux potentiellement ?
Les modèles climatiques sont, du coup, très incertains en ce qui concerne la bio.
Grosses barres d'erreurs dans les projections pour le prochain siècle.

Difficile de séparer les trois: processus dynamiques, mélange horizontal, et biologie, car ils ont les même temps caractéristiques.

\subsection{Interactions biophysiques}
\label{sec:interactions-biophys}

Définition par l'échelle spatiale (0.1-10km).

Ce sont les images satellite de couleur de l'océan qui ont révélé l'ubiquité des fines échelles dans l'océan (en surface tout du moins).

Importance de la SMS:
1) on observe que la variance de la biophysique est importante à ces échelles.
2) les processus dynamiques crée des échanges verticaux
3) Mal représenté dans les modèles climatiques, on doit mieux comprendre (et quantifier pour savoir à quel point c'est important)


\subsubsection{Mélange horizontal}
\label{sec:melange-horizontal}

Une partie des fines échelles observées correspondent à l'action de l'advection par les courants de méso-échelle. L'action passive du mélange horizontal fait apparaître de fins filaments.
Cela permet le mélange de communauté, car rapproche spatialement des parcelles d'eau de différentes origines, avec des caractéristiques physiques et des historique biologiques propres.

Approche lagrangienne très utile dans l'étude du phytoplancton, et dans son observation (méthode de sampling Lagrangiennes, ref ).


\subsubsection{Upwelling de nutriments par les circulations agéostrophiques}
\label{sec:upwelling-nutriments}

Comme dit plus haut, les échanges verticaux sont important pour la bio.
Or aux petites échelles on voit apparaître des vitesses verticales de grande magnitude.

à ces échelles (0.1-10km) émergent également des processus dynamiques nouveaux.
On est alors en dessous du rayon de déformation de Rossby (\(Ro < 1\)).
Forçage par l'atmosphère (hétérogène) et les courants méso qui génèrent des gradients de densité.
On décrit certains de ces processus et leur(s) impact(s) dans la suite.

advection selon les isopycnes qui peuvent être penchées.

importance des vents

aspect intermittent (local, petit spot)

winners and losers


\subsubsection{Modification de la phénologie du bloom}
\label{sec:modif-phenologie}

Les gradients de densité existant (créés par mélange par courants méso ou forçages atmos) sont des réservoirs d'énergie potentielle. À un front les isopycnes sont penchées et des circulations sub-méso se créent et transforment l'énergie potentielle en énergie cinétique (tourbillons), ce faisant ramenant les isopycnes à l'horizontale.
Ces tourbillons formés par l'instabilité de Mixed-Layer (?) s'étendent sur la hauteur de la ML. Ce sont les Mixed-Layer Eddies.
À travers ces instabilités la sub-mesoéchelle contribue fortement à re-stratifier la couche  de surface, et réduire le mélange.
C'est un processus local et on s'attend donc à un soulèvement local de la couche de mélange au niveau des fronts.


Expliquer lancement du bloom printanier par réduction mélange.

On s'attend à un départ du bloom d'abord dans les fronts.

exemples de détection précédentes (mahadevan 2020).


\subsection{Région d'étude: Extension du Gulf Stream}
\label{sec:region-detude}

Notre zone d'étude: 15°N-55°N, 82°W-40°W

\subsubsection{La physique}
\label{sec:gs-physique}

Gyre subtropicale Atlantique Nord.

Courant de bord ouest chaud et salé qui remonte des caraïbes le long de la Floride
Se détache à Cape Hatteras, quitte le plateau continentale et méandre vers l'est.
Plume énergétique autour de ces méandres (surtout au sud).

Au nord du jet, courant retour (slope current) avec notamment un jet sur le shelf break.
Entre le Gulf Stream North Wall et le shelf: slope seas. Très froid, plutôt fraîche (plus salé que sur le shelf néanmoins).
Ce courant froid plonge. Fait partie de la circulation d'overturning de l'atlantique.

\subsubsection{La biologie}
\label{sec:gs-biologie}

Gyre très pauvre en nutriment et productivité faible.
Pourquoi ?
Pas de circulation méso qui permette l'apport de nutriment en surface.
Importance de la sub-méso donc pour créer des échanges verticaux.

Eaux au nord du GS très productives.
Également importances à cause de pêcheries.
Important bloom (plus de détail, ref sur la phénologie, )


\subsection{Détection des fronts sur images satellites}
\label{sec:detection-fronts}

On se concentre sur la SST, mais on peut utiliser d'autres traceurs (Chloro par ex).

méthodes avec gradient

méthodes se basant sur histogrammes (Otsu, Cayula)
méthodes avec fenêtre glissante

méthodes utilisant variance, entropie

Quelques mots sur le suivi des fronts + reconstruction de fronts 'linéaire'
qu'on utilise pas on reste en 'vue pixelisée'

\subsubsection{Vers des fronts de densité}


Lien entre densité et température.
La salinité intervient aussi. La salinité par satellite est très basse résolution (SMOS) donc on a pas vraiment accès.

On ne peut qu'espérer que la salinité ne joue pas un rôle trop grand dans la densité, ie que les fronts ne soient pas trop compensés.
Pour vérifier cela on doit passer par les campagnes en mer.

Dans la région Nord-Atlantique des transects sont réalisé régulièrement par un navire d'opportunité, l'Oleander.
[Trouver détails dessus]
Des résultats suggèrent que la salinité ne joue pas beaucoup (Flagg 2006, succinct).

\section{Motivation et objectifs}
\label{sec:problematique}

\section{Plan de thèse}
\label{sec:plan-de-these}

\begin{itemize}
  \item \cref{M-chp:methodes}
  \item \cref{M-chp:res-chl}
  \item \cref{M-chp:res-phenologie}
  \item \cref{M-chp:conclusion}
\end{itemize}

\end{document}


\chapterlof{Méthodes}
\label{chp:methodes}
\graphicspath{{resources/méthodes}}

\minitoc%
\clearpage

Maintenant que les objectifs de ce projet ont été précisés, nous détaillons dans ce chapitre l'ensemble des outils nécessaire à leur accomplissement.
Nous décrivons dans les sections qui suivent les données utilisées, puis les méthodes utilisées pour séparer notre zone d'étude en sous"-régions, pour repérer les fronts en utilisant le \engquote{\al{hi}}~(\as{hi}), et pour extraire nos résultats.

\section{Données}
\label{sec:donnees}

Dans cette section sont décrits les jeux de données utilisés pour ce projet.
Le choix de ces données s'est avéré être une question importante et difficile dès le départ.
Nous savions qu'il allait être nécessaire de combiner plusieurs jeux de données: la \ab{sst} avec la \ab{chl}, puis avec des données de composition de la communauté de phytoplancton.
Cela a donc informé le choix des données, ainsi que la création des outils nécessaires à la gestion de plusieurs sets de données.

Plusieurs options ont été considérées, notamment pour la \ab{sst}.
Elles sont reportées ci"-dessous, même si elles ne sont pas utilisées pour le reste des résultats présentés.

Au long de ce chapitre (et ailleurs dans ce manuscript) les articles publiés correspondant aux jeux de données sont bien évidemment cités, mais sont également précisés les accès par lesquels nous avons obtenus les données.
Ces derniers sont cités en note de bas de page et listés dans la section~\reftitle{bib:data} de la bibliographie~(\cpageref{bib:data}).

\subsection{Sea Surface Temperature (SST)}
\label{sec:donnees-sst}

Nous commencons par la température de surface que nous allons notamment utiliser comme proxy de la densité pour repérer les fronts.
Malgré que de nombreux produits de \ab{sst} soient disponibles, il n'en demeure pas moins difficile de trouver un produit qui satisfasse les exigences inhérentes à notre étude.
Étant donné que nous voulons repérer des structures de fine échelle la résolution est peut être l’exigence qui apparaît comme première.
Cependant, et comme nous allons le voir plus en détail, d'autres paramètres rentrent en compte et il est nécessaire de faire des compromis dans ce domaine.
Brièvement ces autres paramètres sont (entre autres) la disponibilité des autres variables à une résolution similaire, la facilité d'obtention\footnote{%
  Par facilité d'obtention j'entends le téléchargement des données mais aussi et surtout les traitements supplémentaires nécessaires, comme le repérage des pixels nuageux \ab{par-ex}},
ou encore la couverture spatio"-temporelle.

% Éventuellement à déplacer à une mention antérieure des résolutions
\begin{note}
  Bien que notre zone d'étude s'étende à de (relativement) hautes latitudes~(\latlon{\approx55N}), nous ne considérons pas les différences de distances zonales et méridionales.
  Nous faisons donc une correspondance simple entre fraction de degré et kilomètres, quelque soit la distance considérée, notamment pour la taille des pixels et des fenêtres.
\end{note}

\subsubsection{MODIS-1km}
\declareDataset{sst_modis}
\declareDataset{chl_modis}

Un des premiers jeu de données \ab{sst} à considérer est celui utilisé par \textcite{liu_2016}, dont l'étude consistait également à détecter des fronts de \ab{sst} et les colocaliser aux données de \ab{chl}.
Ces variables étant rarement distribuées sur des grilles à des résolutions kilométriques, il apparaît comme nécessaire de projeter sur grille régulière nous même des produits~L2 \encadra*{\al{cad} des données d'un seul capteur, converties en variables géophysiques, et disposées à la résolution de capture}.

% Éventuellement à déplacer à une mention antérieure des niveaux
\begin{note}
  Tout au long de ce manuscript nous utilisons les dénominations des \guil*{niveaux} de données tels que typiquement utilisés par les distributeurs, et qui correspondent de manière générale à:
  \begin{itemize}
    \item L1:,
    \item L2:,
    \item L3:,
    \item L4:.
  \end{itemize}
\end{note}

Pour le jeu de données qu'on désignera par \dataname{sst_modis} dans la suite, il s'agit de données provenant du capteur \ab{modis}, à bord du satellite Aqua~(\cite{kilpatrick_2015}).
Les données sont téléchargées au niveau~L2 depuis le \href{https://cmr.earthdata.nasa.gov/search/}{Common Metadata Repository (CMR)}\footfullcite{sst_modis} en sélectionnant les \eng{swaths} intersectant notre région d'étude, collectés de jour, et pour les courtes longueurs d'onde infrarouges~(\qty{11}{\um}).
Pour ensuite projeter les données sur une grille kilométrique globale, j'ai repris les outils utilisés par \citeauthor{liu_2016}, à savoir les programmes présents dans le paquet de l'\href{https://oceandata.sci.gsfc.nasa.gov/ocssw}{\ab{ocssw}}, version~\verb|v7.5|, qui est maintenu par l'Ocean Biology Processing Group et notamment distribué à travers l'outil \href{https://seadas.gsfc.nasa.gov/}{SeaDAS}.
Les \eng{swaths}, maintenant tous sur la même grille spatiale peuvent être moyennés par date.

Cela permet donc d'obtenir des données à la limite de la résolution des capteurs, mais présente un certain nombre de désavantages majeurs.
C'est tout d'abord un travail important à réaliser: d'abord de téléchargement, les fichiers~L2 étant particulièrement lourds et nombreux (jusqu'à une dizaine pour une journée); ensuite de projection, qui à une résolution kilométrique demande une certaine puissance de calcul.
De plus, toutes les étapes décrites jusqu'ici ne concernent qu'un seul capteur.
Des étapes et calculs additionnels seraient nécessaires pour y combiner les données d'autres capteurs, ce qui pose des problèmes supplémentaires et se ferait non sans difficultés.
Même pour un capteur identique (\ab{modis} à bord de Terra \ab{par-ex}), il devient nécessaire de considérer des différences de calibrations, ainsi que de possibles artefacts lorsque l'on superpose plusieurs \eng{swaths} (que nous avons d'ailleurs largement ignorés pour agréger les \eng{swaths} sur une seule journée\dots).

Face à ces difficultés, nous avons donc décidé dans un premier temps d'en rester à un seul capteur malgré la couverture spatiale réduite.
Il est également à noter qu'à ce niveau, peu d'actions ont été prises pour disqualifier les pixels nuageux.
Il en résulte des pixels aux valeurs visiblement erronées mais qui n'ont pas été masqués comme nuageux.
Cela se manifeste par exemple comme du bruit autour de nuages.
Il est difficile de les disqualifier correctement sans un bon algorithme de détection de nuage, ce qui nécessite un travail supplémentaire important.
Nous contournons le problème d'une manière similaire à celle de \textcite{liu_2016} en ne travaillant que sur des fenêtres de~\qtyproduct{100 x 100}{\km} avec une faible couverture nuageuse~(\pct{<70}), ce qui élimine une partie de ces cas problématiques.

Malgré le fait que l'on soit limité à un seul capteur, \ab{modis} présente l'avantage de mesurer concomitamment la \ab{sst} et la couleur de l'océan.
Cela permet d'obtenir les valeurs de \ab{chl} aux même pixels, en utilisant le même traitement pour le jeu de données, accessible au \ab{cmems}\footfullcite{chl_modis}.

Il est néanmoins possible de trouver d'autres jeux de données permettant notre étude et nécessitant moins de travail (et autant d'éventuelles erreurs), ces dernières années ayant vu apparaître des données distribuées à des résolutions élevées.

\subsubsection{MUR}
\declareDataset{sst_mur}

Le produit suivant, désigné ici \af{mur} est développé et distribué par le \ab{ghrsst} au JPL~\eng{Physical Oceanography}~\ab{daac}\footfullcite{sst_mur}.
Il présente les avantages d'être distribué à une résolution kilométrique et sans nuages grâce à une méthode d'interpolation par vaguelettes~(\cite{chin_2017}).
Il intègre de nombreuses sources provenant de plusieurs capteurs infrarouges et micro"-ondes, ainsi que de mesures in"-situ.
La grande couverture spatiale de ce produit se fait donc par l'utilisation de mesures non"-contraintes par la couverture nuageuse (\ab{cad} micro"-ondes et in"-situ), mais qui présentent une résolution bien plus faible.
Le champ de \ab{sst} se retrouve donc lissé à la fois par l'interpolation et par l'inclusion de ces mesures.
L'utilisation du produit suivant cherche à palier à ce problème.

\subsubsection{ESA SST CCI / C3S}
\declareDataset{sst_esacci}

Ce produit est distribué conjointement par l'\ab{esa} \ab{sst} \ab{cci} et le \ab{c3s}.
Il agglomère les mesures de tous les capteurs infrarouges disponibles depuis~1981, ce qui comprend 11~\ab{avhrr} à bord des satellites \ab{metop}["~A] et \ab{noaa} (entre les itérations 6 et~19) et trois \ab{atsr} à bord de \ab{ers} 1 et~2, et Envisat.
Cela donne au moins deux capteurs en fonctionnement simultané, et au moins trois depuis~1992.
Outre la bonne couverture obtenue, l'uniformité des capteurs permet d'obtenir un jeu de données stable dans le temps, pouvant servir à la détection de tendances climatiques~(\cite{merchant_2019}).

Les données sont distribuées à différents niveaux, dont nous allons préciser le passage de l'un à l'autre dans la suite:
\begin{itemize}
  \item L2P (mono-capteur),
  \item L3U (mono-capteur mais sur grille régulière),
  \item L3C (multi-capteurs regroupés par famille d'instrument),
  \item et L4 (tous les capteurs, combinés par interpolation).
\end{itemize}

Le produit~L2P est similaire aux données brutes utilisées pour le set \datasect{sst_modis}, mais avec l'avantage de disposer de données standardisées sur un grand nombre de capteurs.
Ces données sont projetées capteur par capteur sur une grille régulière, à une résolution de~\resol{1}{20}, soit~\qty{\approx5}{\km}, ce qui donne le produit~L3U.
Il est à noter que nous pourrions faire ce processus nous même, mais les outils que nous utilisons ne sont pas garantis de fonctionner directement sur ces données.

Les données sont regroupées par famille de d'instrument pour le produit~L3C. Une légère perte de précision est attendue, mais les produits L3U et~L3C peuvent néanmoins tous deux être utilisés dans l'étude de structures fines et où la présence de nuages n'est pas rédhibitoire~(\cite{merchant_2019}).

Enfin, le produit~L4 regroupe l'ensemble des capteurs en utilisant le schéma d'intégration variationnelle \verb|NEMOVAR| intégré dans le système \ab{ostia}~(\cite{good_2020}).
Ce produit donne ainsi un champ de \ab{sst} sans nuages, estimé par une combinaison des observations satellites, et de la prévision d'un modèle numérique d'un jour sur l'autre.
L'absence de couverture nuageuse et la simplicité d'usage se fait au détriment du lissage inévitable des structures les plus fines.

Une estimation quantitative du lissage due à l'interpolation est compliqué, car la paramétrisation de cette dernière dépend de la variabilité du champ de température à J"~1.
On peut néanmoins imaginer que le champ de \ab{sst} produit est plus lissé dans les zones normalement nuageuses, et de plus potentiellement plus éloigné de la réalité~(voir \cref{sec:donnees-sst_reanalyses}).
Il est tout de même possible de mitiger ces effets en ne comptabilisant dans nos statistiques seulement les pixels non"-nuageux, ce que nous faisons pour tous les résultats présentés.
Nous utilisons pour cela les données de \ab{chl} qui identifient les pixels nuageux.
% À noter qu'il serait également possible d'utiliser les erreurs d'intégration associées à chaque pixel, mais cela n'a pas été exploré.
% -> c'est pas vraiment possible, les erreurs d'inté sont données avec une résolution très faible

Sauf indication contraire, c'est le produit~L4 que nous utiliserons pour le champ de \ab{sst} dans toute la suite de cette thèse.
Il permet d'accéder à des données \ab{sst} simplement, avec une résolution convenable, et est stable sur une longue durée (plusieurs décennies).
Les données sont téléchargées depuis \ab{cmems}\footfullcite{sst_esacci}.

\subsubsection{Réanalyses}
\declareDataset{sst_reanalyses}

Une solution pour s'affranchir de la couverture nuageuse est de s'appuyer sur des produits de réanalyses.
Tôt dans ce projet, certaines pistes de travail (\ab{par-ex} le suivi temporel des fronts) nous ont poussé à tester l'utilisation de tels produits.
Les données produites par <infos et ref>, disponibles à la résolution~\resol{1}{12}~(\qty{\approx8}{\km})\footfullcite{sst_reanalyses}.
Outre la résolution assez faible de ce produit, on notera des différences importantes avec les autres produits, et qui ne sont visiblement pas due à un simple lissage du champ de \ab{sst}~(\cref{fig:comparaison-sst}).

% Problème ici: c'est très vieux (pendant stage de master), donc j'ai peu de traces.
% Mais visiblement j'utilisai \verb|GLOBAL_REANALYSIS_PHY_001_030| sur CMEMS\@.
% Il n'est plus dispo ou a changé de nom. Le successeur devrait être: \url{https://data.marine.copernicus.eu/product/GLOBAL_MULTIYEAR_PHY_001_030/description}.
% C'est une réanalyse au 1/12° (ie 8~km). C'est beaucoup.
% Mais peut-être que la comparaison peut quand même valoir le cout ?

\begin{figure}
  \includegraphics[width=\textwidth]{comparaison_sst.pdf}
  \captionT{Comparaison entre produits SST}{%
    Zoom sur une même fenêtre, le \frenchdate{2007}{04}{22}, de la \as{sst} pour quatres produits considérés: \dataname{sst_esacci}~(a), \dataname{sst_reanalyses}~(b), \dataname{sst_mur}~(c), et \dataname{sst_modis}~(d).
    Les données \dataname{sst_mur}, malgré leur haute résolution présentent des zones très lissées, par exemple dans la zone~\latlonRange{68W; 64W}; \latlonRange{40N; 42N}.
    Les données \dataname{sst_esacci} semblent en revanche capturer avec régularité les structures visibles sur les données plus brutes de \dataname{sst_modis}, excepté les plus fines évidemment.
  }
  \label{fig:comparaison-sst}
\end{figure}

\subsection{Chlorophylle-\emph{a}}
\label{sec:donnees-chl}
\declareDataset{chl_globcolour}

Pour le champ de \al{chl}, nous utilisons les données produites dans le cadre du projet GlobColour, développées, validées et distribuées par ACRI"~ST, France~(\cite{maritorena_2002}).
La version \guil{MultiYear}, au niveau L3 est utilisée.
On obtient des données à une résolution de~\resol{1}{24} (soit~\qty{\approx4}{\km}), journalière, avec des nuages.
Les données sont récupérées sur \ab{cmems}\footfullcite{chl_globcolour}.

Ce produit aggrège les données optiques de plusieurs capteurs: \ab{seawifs}, \ab{modis} Aqua et Terra, \ab{meris} à bord d'Envisat, \ab{viirs} à bord de \ab{snpp} et \ab{noaa}[-20], et enfin \ab{olci} à bord de Sentinel-3A et -3B.
Les données de reflectance de chaque capteur sont transformées en concentration de \ab{chl} avant d'être fusionnées en seul produit.
Les algorithmes pour le passage en \ab{chl} sont ajustés indépendamment pour chacun des capteurs, ce qui permet d'obtenir un produit cohérent et stable~(\cite{garnesson_2019}).


\subsection{Accord entre les produits}

Il est à noter qu'il est difficile de trouver des produits de \al{chl} à des résolutions supérieures à~\qty{4}{\km}, en tout cas au niveau global.
Cela motive à utiliser un produit \ab{sst} de résolution légèrement moindre afin d'éviter une étape supplémentaire de \engquote{downsampling} une fois les fronts repérés sur le champ de \ab{sst}.
Cela renforce notre choix pour les données \dataname{sst_esacci}, en défaveur du set \dataname{sst_mur}.

Les deux jeux choisis pour la \ab{sst} et \ab{chl} ont des résolutions spatiales similaires mais néanmoins légèrement différentes.
Les deux grilles sont plate"-carrées (régulières en latitude et longitude) mais le champ de \ab{chl} est défini sur une grille \glshref{epsg-chl} de résolution~\resol{1}{24}~(\qty{\approx4.6}{\km}); et le champ de \ab{sst} sur une grille \glshref{epsg-sst} de résolution~\resol{1}{20}~(\qty{\approx5.6}{\km}).

Sachant que les produits de composition du phytoplancton dont nous disposions (non décrits ici) sont définis sur la même grille que la Chlorophylle, nous adaptons la \ab{sst} sur la grille de la \ab{chl} par une simple interpolation bi"-linéaire.

\subsection{Bathymétrie}
\label{sec:donnees-bathymetrie}
\declareDataset{etopo1}

Nous utilisons les données de bathymétrie ETOPO1\footfullcite{etopo1}, fournies par \ab{noaa}.
Les données sont distribuées sur une grille plate"-carrée à une résolution d'une minute d'arc~(\resol{1}{60}), soit trois fois trop pour nous.
Nous sous"-échantillonnons les données en appliquant une moyenne à chaque groupe de 3\texttimes3~pixels, puis en interpolant le résultat sur la même grille que le reste des données de la même manière que pour la \ab{sst}.

\begin{figure}
  \centering
  \includegraphics[width=0.6\textwidth]{bathymétrie.pdf}
  \captionT{Bathymétrie de la zone d'étude}{%
    L'isobath~\qty{1500}{\m}, indiqué en rouge (les Bahamas sont exclues par visibilité), permet de repérer le talus continental et d'ignorer le plateau dans nos résultats.
  }
  \label{fig:bathymetrie}
\end{figure}

\section{Délimitations des biomes, ou sous-régions d'étude}
\label{sec:delimitations-regions}

Comme détaillé plus en profondeur en introduction~(\cref{sec:region-detude}, \cref*{chp:introduction}) notre région d'étude est fortement hétérogène, aussi bien concernant les propriétés physiques que biologiques.
Il est ainsi nécessaire de séparer notre zone d'étude en (sous-)régions afin d'extraire des résultats sur des zones homogènes.
Nous définissons donc trois biomes.
Le biome \ab{st-perm}, le plus au sud, correspond à un régime largement oligotrophe.
Le biome \ab{sp}, le plus au nord, comprend les eaux froides et productives au nord du Gulf Stream.
Enfin le biome \ab{st-sais}, entre les deux précédents, présente un régime intermédiaire: oligotrophe mais avec une production plus élevée permise par une couche de mélange profonde en hiver.
Dans cette section, nous décrivons plus précisement la méthode utilisée pour définir ces biomes spatialement.

\subsection{Séparation des biomes}

La séparation entre les deux biomes subtropicaux est faite par une limite zonale fixée à~\latlon{32N}~(trait noir pointillé~\cref{fig:regions}).
Cette limite correspond approximativement à un saut visible des valeurs de \ab{chl} à cette latitude (isocontour~\qty{0.1}{\mgm} de la moyenne annuelle de \ab{chl}) qui ne varie que peu au cours de l'année.
Cette séparation est en accord avec la limite entre les biomes tels que délimités par \textcite{sarmiento_2004}~(\cref{fig:sarmiento}).

\begin{figure}
  \centering
  \includegraphics[width=1\textwidth]{régions.pdf}
  \captionT{Résultat de la séparation de la zone d'étude en biomes}{%
    La zone est découpée en trois biomes: le biome \al{st-perm}~(PSB) au sud de du trait pointillé à~\latlon{32N}; le biome \al{st-sais}~(SSB) entre~\latlon{32N} et la limite (sinueuse) nord du Gulf Stream dénotée par le contour noir; et le biome \al{sp}~(SP) au nord du Gulf Stream.

    Clichés de \ab{sst}~(a) et \ab{chl}~(b) au \frenchdate{2007}{04}{22}, et les distributions de \ab{sst}~(c) et {d}~\ab{chl} pour chacuns des biomes, le même jour (\al{sp}:~bleu, \al{st-sais}:~jaune, \al{st-perm}:~rouge).
    La ligne noire verticale sur~(c) marque la température détectée comme seuil de la limite nord du Gulf Stream.
    Les axes des abscisses des distributions correspondent aux barres de couleurs.
    La ligne rouge suit l'isobath~\qty{1500}{\m}. Le plateau continental~(\qty{<1500}{\m}) n'est pas considéré et est masqué.
  }
  \label{fig:regions}
\end{figure}

\begin{figure}
  \centering
  \includegraphics[width=0.7\textwidth]{sarmiento_2004_fig2b.png}
  \captionT{Délimitations des biomes par des processus physiques}{%
    Figure tirée de \textcite[figure 2b]{sarmiento_2004}.
    Les biomes de notre zone d'étude correspondent dans la nomenclature de cet article à: \al{sp}~(``SP'', jaune); \al{st-sais}~(``ST-SS'', bleu); et \al{st-perm}~(``ST-PS'', rose).

    \foreignblockquote{english}{Biome classification scheme calculated using mixed layer depths obtained from observed density and from upwelling calculated from the wind stress divergence using observed winds.
    The equatorially influenced biome covers the area between \latlon{5S}~and~\latlon{5N}, and is colored a dirty light blue in areas where upwelling occurs (labeled~``Eq-U'' on the color bar) and dark pink in areas where downwelling occurs (labeled~``Eq-D'').
    Outside of this band, the region labeled ``Ice''~(red) is the marginal sea ice biome, the region labeled ``SP''~(yellow) is the subpolar biome, the region labeled ``LL-U''~(light blue) is the low-latitude upwelling biome , the region labeled ``ST-SS''~(dark blue) is the seasonally mixed subtropical gyre biome, and the region labeled ``ST-PS''~(pink) is the permanently stratified subtropical gyre biome.}
  }
\label{fig:sarmiento}
\end{figure}

On sépare ensuite le reste de la région au nord de~\latlon{32N} en prenant comme limite le front nord du jet du Gulf Stream (le \engquote{North wall}).
Cette délimitation est donc dynamique et déterminée chaque jour à partir de l'image de température.
En effet, il apparaît que la distribution de la \ab{sst} (au nord de~\latlon{32N}) suffit à repérer de manière fiable et robuste une température seuil permettant de séparer les deux biomes (contour sinueux noir plein~\cref{fig:regions}).
Sur cette distribution apparaît clairement un pic dans des valeurs élevées correspondant au eaux du jet.
Il est aisé de repérer le pic et l'ajuster par une gaussienne.
À partir de là, la température seuil entre les biomes est prise comme la base froide de ce pic, \ab{cad} plus précisemment en soustrayant à la température moyenne du pic deux fois son écart"-type~(\cref{fig:temp-seuil-distrib}).
La valeur journalière de ce seuil est filtrée temporellement par un filtre médian glissant (avec une fenêtre de largeur 8~jours) afin d'éviter d'éventuelles anomalies de détection.

\begin{figure}
  \centering
  \includegraphics[width=0.7\textwidth]{zone_separation.pdf}
  \captionT{Délimitation des biomes subtropical permanent et subpolaire par température seuil}{%
    Sur la distribution des valeurs en température au nord de~\latlon{32N}~(en noir) pour le \frenchdate{2007}{04}{22}, le pic \encadra{ici autour de~\tC{18}} correspond aux eaux du jet. Il est ajusté par un fit gaussien (en rouge) dans un intervalle de~\tC{5} de large.
    La température de seuil~(en bleu) entre les deux biomes est définie comme la température moyenne du pic~(trait fin pointillé) moins deux fois son écart"-type.
    Cela correspond à la limite nord du Gulf Stream.
  }
  \label{fig:temp-seuil-distrib}
\end{figure}

\begin{figure}
  \includegraphics[width=\textwidth]{separation_evol_month.pdf}
  \captionT{Variation temporelle de la délimitation entre biomes}{%
    La position limite entre les biomes \al{st-sais} et \al{sp} n'évolue que peu au cours de l'année vis à vis des larges méandres du Gulf Stream.
    Néanmoins en été, la limite est moins marquée après le détachement du jet~(\latlonRange{75W; 70W}).

    L'isotherme séparant les deux biomes est tracé le 15~de chaque mois pour l'année~2007.
    La couleur de chaque contour correspond au jour de l'année comme dénoté par la barre de couleur dans l'inset.
    Dans l'inset est tracé la température de seuil au long de l'année~2007. Les cercles marquent les jours et températures utilisées pour chaque contour.
  }
  \label{fig:var-delim}
\end{figure}

\subsection{Exclusion du plateau continental}

Par ailleurs, il est également nécessaire d'éviter de considérer les pixels trop près des côtes dans notre étude.
De manière générale la \ab{chl} y suit un régime côtier, visible par des valeurs très élevées~(\qty{>10}{\mgm}).
Ensuite, nous cherchons à éviter deux zones qui ne correspondent pas aux biomes définis plus haut.
Premièrement, le jet du Gulf Stream prend forme au sud de notre zone, le long de la côte de Floride. On trouve donc sur le plateau continental à ces latitudes~(\latlon{\approx28N}) de forts courants qu'on ne retrouve pas dans le reste du biome \ab{st-perm}.
Deuxièmement, plus au nord dans l'anse Nord-Est Américaine de l'Atlantique~(\engquote{Mid"-Atlantic Bight}), on trouve une séparation nette entre les eaux au nord du Gulf Stream (la \engquote{slope sea}), et les eaux sur le plateau continental.
Un jet marque cette séparation le long du talus continental~(\cite{flagg_2006}).

Dans les deux cas, imposer une limite haute à la bathymétrie sur notre région d'étude permet de supprimer ces zones problématiques.
Ainsi, pour calculer nos résultats, nous ne considérons que les pixels où la profondeur n'excède pas~\qty{1500}{\m}.
Nous utilisons pour cela les données de bathymétrie \dataname{etopo1}~(voir \datasect{etopo1}).

\section{Heterogeneity Index (HI)}
\label{sec:HI}

Comme précisé en introduction, pour quantifier l'effets des fronts sur le phytoplancton il est nécessaire de détecter les fronts, \al{cad} classer chaque pixel comme appartenant à un front ou non (\ab{cad} à l'arrière"-plan ou au \engquote{background}).
La méthode retenue ici suit celle présentée par \textcite{liu_2016} \encadra*{qui par son utilisation d'une fenêtre glissante s'apparente elle même à celle de \textcite{cayula_1992}}.
Cette section définit cette méthode et notre implémentation, tout en indiquant les modifications que nous y avons apportées.

\subsection{Définition et implémentation}
\label{sec:HI-definition}

La méthode de \textcite{liu_2016} consiste à quantifier plusieurs quantités statistiques du champ de \ab{sst} dont les fortes valeurs sont associées à la présence de fronts et autres structures de fine échelle.
Ces quantités sont la bimodalité, l'écart"-type (qui reflète le gradient du champ), et le coefficient d'asymétrie~(\engquote{skewness}).
Ces composantes sont ensuite réunies dans un seul index qui ainsi reflète l'hétérogénéité du champ de \ab{sst}, et donc baptisé \af{hi}.

Pour limiter la taille des structures détectées, on limite le calcul de ces composantes sur une fenêtre de taille appropriée, \ab{cad} pour nous de l'ordre de grandeur d'une dizaine de kilomètres.
Ainsi, pour chaque pixel, la valeur de chaque composante est calculée sur la distribution de \ab{sst} à l'intérieur d'une fenêtre glissante centrée sur ce pixel et dont les tailles possibles sont:
\begin{itemize}
        \item \(3 \times 3 =\qty{9}{\pixels}\) soit~\qty{17}{\km} de côté,
        \item \(5 \times 5 =\qty{25}{\pixels}\) soit~\qty{28}{\km} de côté,
        \item \(7 \times 7 =\qty{49}{\pixels}\) soit~\qty{39}{\km} de côté.
\end{itemize}
Parce qu'une fenêtre plus large entraînerait la détection de structures trop grandes pour notre étude, nous nous limiterons aux tailles présentées ci"-dessus.
Néanmoins une fenêtre trop petite limite le nombre de pixels disponibles pour calculer les valeurs statistique dont nous avons besoin convenablement.
Un compromis est à définir.

Nous définissons maintenant plus en détails le calcul des composantes du \ab{hi}.
Pour chaque position de la fenêtre, on s'intéresse aux~\(N\) valeurs de \ab{sst}~\(s_{i}\) valides (\ab{cad} des provenant des pixels sans nuages).

On commence par l'écart"-type~\ab{std}, calculé simplement par:
\begin{equation}
  \am{std} = \sqrt{\frac{1}{N-1} \sum_i \paren{s_i - \moy{s}}^2},
\end{equation}
avec~\(\moy{s}\) la moyenne des valeurs de température.

Ensuite, le coefficient d'asymétrie~\ab{skew}, est défini comme la valeur absolue du moment d'ordre trois d'une variable centrée réduite, et qui se calcule donc par:
\begin{equation}
  \am{skew} = \abs{\frac{\sum_i \paren{s_i - \moy{s}}^3} {N \sigma^3}}.
\end{equation}

Enfin, on cherche à quantifier la bimodalité~\ab{bimod} de la distribution des valeurs de \ab{sst}.
Pour ce faire on compare l'écart entre ladite distribution et une distribution gaussienne dont la moyenne et l'écart"-type sont pris identiques à ceux de la \ab{sst}~(\cref{fig:bimodality}).
Cela présuppose que lorsque les températures sont uni"-modales (\ab{cad} quand il n'y pas de fronts dans la fenêtre) leur distribution tend vers une gaussienne, ce qui ne paraît pas déraisonnable.

\begin{figure}
  \centering
  \includegraphics[width=0.6\textwidth]{bimodality.pdf}
  \captionT{Illustration du calcul de la bimodalité}{
    On calcule la bimodalité comme la norme de la différence entre l'histogramme des températures dans la fenêtre~(trait noir) et une distribution gaussienne de même moyenne et de même écart"-type~(trait rouge).
  }
  \label{fig:bimodality}
\end{figure}

De manière plus précise on commence par calculer l'histogramme~\(h_i\) des valeurs de \ab{sst} dans la fenêtre, en utilisant des intervalles de largeur fixe de~\tC{0.1} et compris entre les valeurs minimales et maximales dans la fênetre.
Les données de \ab{sst} étant stockées compressées par \engquote{linear packing}\footnotemark\ avec un facteur d'échelle de~\tC{0.01}, nos intervalles ont une largeur précisement égale à dix fois l'intervalle de discrétisation des valeurs de température.
Pour éviter que trop de valeurs tombent sur les bords des intervalles et que l'histogramme soit pollué par des erreurs numériques, on décale les intervalles de~\tC[parse-numbers=false]{0,01/2}.
Par ailleurs, le nombre d'intervalles étant dépendant de la largeur de la distribution de \ab{sst}, dans les cas où il est inférieur ou égal à quatre, la bimodalité est automatiquement assignée nulle~(\(B=0\)).
\footnotetext{%
  Cette technique de compression avec pertes \encadra{utilisée notamment par l'outil \citesoft{nco}\footnotemark{}} consiste à discrétiser des valeurs flottantes sur des entiers après une transformation linéaire.
  Par exemple, en prenant pour stockage des entiers non"-signés sur \qty{16}{\bits}~(\texttt{NC\_USHORT}) on peut évidement représenter des valeurs entières entre 0 et~\(2^{16}-1 = \num{65535}\); mais en multipliant ces valeurs entières par un facteur de, disons,~\num{0.005} on peut représenter des valeurs entre 0 et~\num{327.675}.
  On a gagné en volume par rapport à un stockage typique de 32 ou~\qty{64}{\bits}, mais en perdant évidemment en précision puisque nos valeurs sont maintenant discrétisées avec un intervalle de~\num{0.005}.}
\footnotetext{voir le guide utilisateur: \glsurl{nco-packed}}

Par ailleurs on définit une distribution gaussienne~\(g_i\) sur les même intervalles en utilisant les statistiques calculées précédemment:
\begin{equation}
  g_i = \frac{1}{\sqrt{2\pi\am{std}}} \exp\paren{-\frac{\paren{x_i-\moy{s}}^2}{2\am{std}^2}},
\end{equation}
pour ensuite calculer la norme~\(\mathbb{L}^2\) entre les deux distributions:
\begin{equation}
  \begin{split}
  \am{bimod} & = \norme[2]{h - g}\\
             & = \sum_i \paren{h_i - g_i}^2 .
  \end{split}
\end{equation}

Ce principe de calcul de bimodalité s'apparente à celle de \textcite{cayula_1992}, qui consiste à trouver une valeur seuil séparant l'histogramme des valeurs en deux classes et pour laquelle la variance intra"-classe est minimale.
Cette méthode est d'ailleurs utilisée en analyse d'image, non pas sur une fenêtre glissante mais pour toute l'image, et connue sous le nom de méthode d'\textcite{otsu_1979}.
Bien qu'elle soit appliquée avec succès pour la détection de front, il apparaît que cette méthode nécessite de construire un histogramme d'une résolution suffisante, ce qui s'avère difficile aux échelles où nous travaillons.

Notre méthode de calcul de la bimodalité est la première grande modification apportée à celle de \textcite{liu_2016}.
Ces dernier·ère·s calculent également l'histogramme de la température, mais au lieu de directement le comparer avec une distribution gaussienne, iels ajustent l'histogramme par un polynôme de degré~5 avant de comparer ce dernier avec la gaussienne.
Cet ajustement, par ailleurs difficile à mettre en place, est facilement mal conditionné et rien ne garantit sa convergence.
Nous proposons donc une méthode plus facile à implémenter, plus robuste, et au coût de calcul réduit.

Avant de pouvoir réunir les trois composantes il est nécessaire de leur appliquer un poids statistique équivalent à chacune. Cela est accompli par calcule de coefficients constants de normalisation.
Alors que \citeauthor{liu_2016} proposent de normaliser chaque composante~\(C^n\) par ses valeurs minimales et maximales (prises sur toutes les valeurs disponibles) selon:
\begin{equation}
  \norm{C_i^n} = \frac {C_i^n - \min(C^n)} {\max(C^n) - \min(C^n)} ;
\end{equation}
nous préférons plutôt normaliser par l'écart"-type:
\begin{equation}
  \norm{C_i^n} = \am{coef}^n C_i^n
  \text{ avec } \am{coef}^n = \frac {1} {\std(C^n)} .
\end{equation}
En effet un rapide coup d’œil aux distributions des composantes~(\cref{fig:distrib-composantes}) permet de constater qu'aucune des composante ne semble avoir des valeurs bornées.
Normaliser par les valeurs maximales donne d'une part une normalisation arbitraire, très sensible aux valeurs extrêmes, et d'autre part donne un poids disproportionné aux valeurs élevées dans la normalisation.
En revanche, la normalisation par l'écart"-type permet d'attribuer une part équivalente de la variance du \ab{hi} à chaque composante.

\begin{figure}
  \centering
  \includegraphics[width=0.7\textwidth]{distrib_composantes.pdf}
  \captionT{Distribution des valeurs de \glsentryshort{hi} et de ses composantes}{%
    Densités de probabilité des valeurs de \ab{hi}~(a) et de ses composantes: l'écart"-type~(b), le coefficient d'asymétrie~(c), et la bimodalité~(d), calculées sur l'année 2007.
    Chacune des composante s'est vue appliquée son coefficient de normalisation, mais pas le \ab{hi}, qui est donc la simple somme de ces composantes.
  }
  \label{fig:distrib-composantes}
\end{figure}

Jusqu'ici nous n'imposons aucune contrainte sur l'amplitude des valeurs du \ab{hi}.
Afin de \guil*{standardiser} quelque peu ses valeurs finales, nous définissons un quatrième coefficient de normalisation défini de sorte que~\pct{95} des valeurs du \ab{hi} soient inférieures à~\num{9.5}.

Les coefficients de normalisation~\(\am{coef}^n\) sont obtenus par analyse des distributions des composantes pour l'année~2007 uniquement. Ils sont ensuite appliqués de manière uniforme au reste des données.

On peut donc enfin calculer le \ab{hi} avec:
\begin{equation}
  \am{hi} = \am{coef}^4 \paren{
    \am{coef}^1\am{std}
    + \am{coef}^2\am{skew}
    + \am{coef}^3\am{bimod}}.
\end{equation}

On obtient donc finalement un indice capable de quantifier l'hétérogénéité du champ de température à une échelle donnée, ce qui nous permet d'identifier des structures fines dans ce même champ~(\cref{fig:exemple-composantes,fig:exemples-fronts}).
Il reste toutefois à définir une méthode pour classifier chaque pixel comme appartenant, ou non, à un front.
Nous allons même plus loin et séparons les pixels en trois catégories: les pixels appartenant à un front fort~(\(\am{hi} > 5\)), à un front faible~(\(5 < \am{hi} < 10\)), ou à aucun front~(\(\am{hi} < 5\), aussi dénotés \engquote{background}).
La pertinence des valeurs de seuils entre les catégories est discutée dans la suite du manuscript.

\begin{figure}
  \centering
  \includegraphics[width=\textwidth]{exemple_composantes.pdf}
  \captionT{Exemples de construction du \glsentryshort{hi} à partir de ses composantes}{%
    Champ de \ab{sst}~(a) pour le \frenchdate{2007}{04}{07}, à partir de laquelle on calcule les composantes du \ab{hi}: l'écart"-type~(c), l'asymétrie~(e), et la bimodalité~(f) (ici toutes représentées normalisées par leur variance).
    Le \ab{hi}~(b) est ensuite obtenu par la somme de ces composantes normalisées, avant d'être normalisé lui aussi afin que~\pct{95} de ses valeurs soient inférieures à~\num{9.5}.
    Le \ab{hi} permet de détecter les fronts de \ab{sst}, ici deux valeurs du \ab{hi} normalisé sont contourées, à 5~(trait plein) et 10~(trait pointillé).
  }
  \label{fig:exemple-composantes}
\end{figure}

\begin{figure}
  \centering
  \includegraphics[width=\textwidth]{exemples_fronts.pdf}
  \captionT{Exemples de structures fines}{%
    Champs de \ab{sst}~(colonne gauche), \ab{chl}~(colonne centre), et \ab{hi}~(colonne droite) pour trois exemples de structures: 1\ier~exemple~(1a--c) le \frenchdate{2007}{04}{07}, 2\ieme~exemple~(2a--c) le \frenchdate{2007}{02}{23}, et 3\ieme~exemple~(3a--c) le \frenchdate{2007}{02}{28}.
    Chaque fenêtre représente une surface d'environ~\qtyproduct{200x200}{\km}.
    Deux valeurs seuil de \ab{hi} sont contourées, à 5~(trait plein) et 10~(trait pointillé).
  }
  \label{fig:exemples-fronts}
\end{figure}


\subsection{Sensibilité aux paramètres}
\label{sec:HI-sensibilite}

Je pense bouger cette section vers les résultats. C'est mieux de le présenter après je crois.

Nécessité d'évaluer les incertitudes sur la méthode.
Pour voir si résultat significatif.
Taille de la fenêtre glissante. Coefs de normalisation.


\section{Extraction des résultats}
\label{sec:extraction-res}

\subsection{Utilisation d'histogrammes}
\label{sec:extraction-hist}

Une fois le \ab{hi} calculé, il devient possible de catégoriser chaque pixel par la biome auquel il appartient (subtropical permanent, subtropical saisonnier, subpolaire), ainsi que par sa valeur de \ab{hi}.
On cherche ensuite à extraire des informations de ces ensembles de pixels ainsi consistués.
Le nombre total de pixels étant conséquent, et parce que l'établissement des ensembles est compliqué et coûteux, les ensembles de pixels sont chacun réduits à des histogrammes de variables d'intérêt (\ab{sst}, \ab{chl}, ou les \abp{pft}).
Cela diminue les étapes de calcul ainsi que la quantité de données à traiter pour obtenir un diagnostic.

Prenons l'exemple d'une seule image. Pour notre région d'étude, cela représente \qtyproduct{1000 x 1000}{\pixels}.
On cherche à extraire un simple diagnostic, par exemple la moyenne de la \ab{chl} dans, et hors des fronts pour chacun des biomes.
Il nous faut donc séparer la région en biomes, ce qui nécessite rappelons-le de discriminer les pixels par leur température~(\cref{sec:delimitations-regions}).
Il faut également discriminer les pixels par leur valeur de \ab{hi} pour séparer fronts et \eng{background}.
On peut maintenant calculer nos statistiques sur chacun des ensembles de pixels constitués, ce qui représente ici de faire 6~calculs (3~biomes~\texttimes\ fronts/\eng{backgound}) sur environ~\qty{e6}{\pixels}.
Mais une fois nos histogrammes calculés, un diagnostic ne requiert que de regarder le ou les histogrammes appropriés, et ce qui représente beaucoup moins d'efforts \encadra*{et de données comme nous allons le voir}.

Les histogrammes peuvent être rendus représentatifs sans pour autant utiliser un nombre prohibitif d'intervalles, d'autant plus que les données (\ab{chl} et \ab{sst}) sont déjà stockées compressées avec pertes. Pour la température par exemple, 450~intervalles suffisent à couvrir toutes les valeurs (de \tC{-5} à~\tC{40}), avec une largeur d'intervalle de~\tC{0.1} équivalente à l'incertitude sur la mesure.
Les intervalles pour la \ab{chl} et les autres variables biologiques sont pris de largeur logarithmique afin de couvrir les plusieurs ordres de grandeur que peuvent prendre leurs valeurs.

Les histogrammes présentent également l'avantage de pouvoir facilement être combinés entre eux.
En effet, tous les histogrammes calculés sont stockés non"-normalisés, c'est-à-dire en nombre de pixel par intervalle. Ainsi plusieurs histogrammes peuvent être sommés entre eux avant d'être normalisés pour en extraire une valeur, comme la valeur médiane de la distribution résultante par example.
Ce procédé est notamment utilisé pour calculer des diagnostics sur des périodes de temps autres que journalières, sans avoir besoin de refaire un calcul coûteux impliquant les pixels.

Dans la suite on détaille le processus de normalisation des histogrammes.
On considère un histogramme qui compte~\(h_i\) valeurs d'une variable quelconque~\(x\), pour le i\ieme{}~intervalle~\(\left[x_i; x_{i+1} \right]\).

Pour un histogramme donné (\as{par-ex} pour une région à une date et un type de front donné), on extrait facilement le nombre de pixels total:
\begin{equation}
  N = \sum_i h_i,
\end{equation}
ou une approximation de la valeur moyenne:
\begin{equation}
  \moy{x} = \frac{\sum_i h_i x_i} {N}.
\end{equation}

Pour d'autres valeurs à extraire il est nécessaire de normaliser les histogrammes afin d'obtenir une densité de probabilité. On prendra soin de considérer les tailles des intervalles dans les calculs.
La largeur des intervalles est~\(w_i = x_{i+1}-x_i\). On transforme le nombre de points en probabilité par unité de valeur~\(p_i = h_i / w_i \), avant de le normaliser de sorte à obtenir une intégrale égale à 1:
\begin{equation}
  f_i = \frac{p_i} {\sum_j p_j w_j},
\end{equation}
pour ainsi obtenir une approximation de la densité de probabilité~\(f\).

On peut extraire de cette distribution notamment la médiane, ainsi que des percentiles divers en trouvant la valeur de~\(x\) pour laquelle la somme cumulée de la densité de probabilité est égale au percentile recherché (\num{0.5} dans le cas de la médiane).
On délègue ce travail au paquet SciPy dont l'objet~\glshref{rv_histogram} permet d'effectuer ces calculs de manière triviale.

Il est à noter que l'interprétation de certaines métriques n'ont de sens que si la distribution de la variable concernée est convenable (uni"-modale \ab{par-ex}).
Cette vérification est reportée dans la \cref{sec:complements-chl},~\cref*{chp:res-chl}.

\subsection{Quantification de l'effet des fronts: différences de valeurs et retard du bloom}
\label{sec:extraction-surplus-lag}

Afin de comparer les valeurs de diverse variables à l'intérieur et à l'extérieur des fronts, nous définissons deux métriques.
La première est l'excès~\ab{exces} de \ab{chl} (\engquote{excess} dans l'article \cref{sec:article-bg}, \cref*{chp:res-chl}), qui compare localement les valeurs dans les fronts et le background. On le définit simplement comme la différence relative entre la médiane de \ab{chl} dans et hors des fronts:
\begin{equation}
  \am{exces} = \frac{\med{\am{chl}}_\frt - \med{\am{chl}}_\bkg}
  {\med{\am{chl}}_\bkg} .
\end{equation}
Le calcul est fait pour les fronts faibles et forts de la même manière, le background désignant toujours les pixels de \ab{hi} faible~(\(\am{hi} < 5\)).
L'excès est calculé dans des bandes de latitudes larges de~\ang{5}, afin de minimiser l'influence des gradients de grande échelle.

Cette métrique ne tient compte que de la distribution des valeurs de \ab{chl}. Elle ignore la proportion de fronts dans une zone donnée, et ne représente pas la quantité totale de \ab{chl} présente dans un biome.
On définit donc une deuxième métrique, le surplus~\ab{surplus} de \ab{chl} dans tout un biome (\enquote{biome surplus} dans l'article \cref{sec:article-bg}, \cref*{chp:res-chl}) comme la différence relative entre la moyenne de \ab{chl} dans tout le biome et la moyenne dans le background:
\begin{equation}
  \am{surplus} = \frac{\moy{\am{chl}}_\tot - \moy{\am{chl}}_\bkg}
  {\moy{\am{chl}}_\bkg} .
\end{equation}

<bloom lag>
J'ai oublié d'inclure la méthode pour détecter le bloom... Dans mon plan c'était dans les compléments de l'article, mais ça a mieux sa place ici. Je le rajouterai plus tard.


\chapterlof{Impact des fronts sur le budget de Chlorophylle-\textit{a}}
\label{chp:res-chl}
\graphicspath{{resources/res_chl}}

\minitoc%
\clearpage

préambule

motivation.
Résultat attendu sur la biomasse, la composition.
Permet de valider la méthode aussi.

\section{Résumé de l'article}
\label{sec:resume-article}

\insertArticle{}

\section{Compléments}
\label{sec:complements-chl}
\suppressfloats[t]

\subsection{Examples de fronts}
\label{sec:examples-fronts}

Trouver des images examples de fronts.
Comparer visuellement avec les données MODIS\@.

Il est difficile de trouver des examples corrects.
D'une part à cause de la grande couverture nuageuse.
D'autre part parce que l'effet sur la Chl est difficilement visible sur des images. (effet statistique).

\subsection{Vérification des histogrammes}

\subsection{Sensibilité aux paramètres}
\label{sec:sensibilite-parametres}

L'implémentation de la détection des fronts avec l'\af{hi} nécessite de faire le choix d'un certain nombre de paramètres. Nous testons ici la variabilité des résultats finaux (\ab{ie} l'impact des fronts sur la \as{chl}) vis-à-vis de deux paramètres.
D'une part, nous faisons varier la taille de la fenêtre glissante utilisée pour calculer les composantes du \as{hi}~(\cref{sec:calcul-composantes}), nous testons trois tailles: \qty{20}{\km},~\qty{30}{\km} et~\qty{40}{\km}. La taille retenue est de~\qty{30}{\km}.

Les coefficients de normalisations sont à recalculer pour chacune des tailles (tel que décrit \cref{sec:coef-normalisation}) et sont listés \cref{tab:coefs}.
D'autre part nous testons l'influence de ces coefficients. En prenant les composantes calculées avec une fenêtre de~\qty{30}{\km}, tour à tour nous doublons le coefficient d'une des composantes (et donc son poids statistique dans le \ab{hi}). Le coefficient~\(K^4\) pour le \ab{hi} doit être recalculé~(\cref{tab:coefs}).

% chktex-file 2
\begin{table}
  \centering
  \begin{siunitText}
  \begin{tabular}{$r<{\hspace{1em}} ^c *{3}{^c}} \toprule
    \multirow{2}*{Variation} & \(K^1\) & \(K^2\) & \(K^3\) & \(K^4\) \\
     & (écart-type) & (asymétrie) & (bimodalité) & (HI) \\
    \midrule
    \emph{Taille fenêtre} & \\
    \qty{20}{\km} & \num{5.1768} & \num{3.1563} & \num{4.1306} & \num{1.3890} \\
    \rowstyle{\bfseries}
    \qty{30}{\km} & \num{3.9401} & \num{2.7200} & \num{4.2917} & \num{1.3418} \\
    \qty{40}{\km} & \num{3.2692} & \num{2.5031} & \num{4.3444} & \num{1.3129} \\
    \midrule
    \emph{Poids composantes} & \\
    plus d'écart-type & \(\times 2\) & --- & --- & \num{1.0239} \\
    plus d'asymétrie & --- & \(\times 2\) & --- & \num{0.9912} \\
    plus de bimodalité & --- & --- & \(\times 2\) & \num{0.9773} \\
    \bottomrule
  \end{tabular}
  \end{siunitText}
  \caption[]{%
    Coefficients de normalisation pour les différents paramètres utilisés.
  }
  \label{tab:coefs}
\end{table}

Les résultats sont très peu sensibles aux choix des paramètres.
Les évolutions climatologiques des valeurs médianes et du surplus sont très peu affectés. La fraction de surface impactée par les fronts est plus sensible mais néanmoins les tendances restent identiques~(\cref{fig:ts-sensitivity}).
Pour quantifier plus en détail cette variation, nous calculons l'écart"-type des valeurs médiane de \as{chl} sur l'ensemble des paramètres (les trois tailles de fenêtres et trois configurations de coefficients), semaine par semaine sur les 20~années de données, puis en prenons la moyenne temporelle.
Il en ressort les mêmes résultats que sur la climatologie, avec une variabilité élevée dans les fronts forts (entre \qty{4.4}{\mugm} et~\qty{11.4}{\mugm}), moyenne dans les fronts faibles (entre \qty{2.0}{\mugm} et~\qty{3.3}{\mugm}), et plus faible dans le \eng{background} (entre \qty{0.07}{\mugm} et~\qty{3.8}{\mugm})~(\cref{tab:sensibilite-mediane}).
Ces


\begin{table}
  \centering
  \bgroup
  \newcommand*\typeunits[1]{\multicolumn{1}{c}{\small\textit{#1}}}
  \newcolumntype{y}} \\
    background      & 0.05   & 0.11  & 0.25   & 0.25  & 4.89   & 1.84  \\
    fronts faibles  & 1.50   & 2.87  & 3.52   & 3.36  & 3.21   & 1.02  \\
    fronts forts    & 5.92   & 9.64  & 12.48  & 8.05  & 4.96   & 1.37  \\
    \bottomrule
  \end{tabular}
  \egroup
  \caption[]{%
    Sensibilité aux paramètres: écart"-type de la valeur médiane de \as{chl} calculé sur l'ensemble des paramètres~\(p\) testés (trois tailles de fenêtres, et trois configuration de coefficients de normalisation); en valeur absolue~(\(\std(\am{chl}_p)\), en~\unit{\mugm}) et relative~(%
    \(\std((\am{chl}_p - \moy{\am{chl}}_p) / \moy{\am{chl}}_p)\), en~\%).
  }
  \label{tab:sensibilite-mediane}
\end{table}

Dans cette étude de sensibilité nous n'avons pas considéré l'impact du seuil en \as{hi} choisi, ayant déjà couvert l'évolution de la \as{chl} avec le \as{hi}~(\cref{fig:chl-vs-hi}).
Il en resortait que les valeurs de \as{chl} tendent à augmenter avec celles du \as{hi}, et ce sur toute l'amplitude du \as{hi}.
Le choix du seuil importe donc peu vis"-à"-vis des tendances observées; Néanmoins il est possible que la variation de la \as{chl} avec les paramètres du \as{hi} soit plus le fait du seuil choisi en rapport avec la distribution de \ab{hi} \encadra{le seuil restant fixé à~5 quelque soit les paramètres} que d'un changement dans l'emplacement des fronts détectés.
Autrement dit, il est possible que la variation en \as{chl} (dans les fronts) soit autant dûe à la \emph{quantité} de fronts détectés qu'à leur \guil*{\emph{qualité}}.

Pour étudier cet éventuel lien entre la surface des fronts et les valeurs de \as{chl} dans ces fronts, nous prenons les valeurs climatologiques mensuelles de la fraction de surface impactée par les fronts \emph{faibles}, et de la valeur médiane de \as{chl} dans ces même fronts, pour tous les paramètres testés (et les trois régions).
Nous éliminons la dépendance de ces deux variables en temps et en région, en leur soustrayant leur valeur moyennée sur l'ensemble des paramètres testés.
Nous obtenons donc la relation (relative) entre la fraction de surface de fronts détectés et la médiane de \as{chl} dans les fronts~(\cref{fig:sensibilite-surface}).
Comme attendu, la médiane de \as{chl} est bien (anti-)corrélée à la fraction de surface \encadra*{le coefficient de corrélation de Pearson entre les deux variables est de~\num{-0.73}}.
Un seuil situé trop bas augmente la surface de fronts détectés et \guil*{dilue} les valeurs de \as{chl} dans ces fronts.

\begin{figure}
  \centering
  \includegraphics[width=0.7\textwidth]{sensibilité_surface.pdf}
  \captionT{Corrélation de l'augmentation de Chl-\textit{a} avec la surface couverte par les fronts}{%
  }
  \label{fig:sensibilite-surface}
\end{figure}


\subsection{Impact sur la SST}

% chktex-file 13

\chapter{Impact des fronts sur la phénologie}
\addChpLof
\label{chp:res-phenologie}
\graphicspath{{resources/res_phénologie}}

\minitoc%
\clearpage

\section{Résumé des résultats l'article}
\label{sec:resume-res-phenologie}

\section{Compléments}
\label{sec:complements-phenologie}

\subsection{Durée du bloom}
\label{sec:duree-bloom}

Extension du résultat présenté~\cref{fig:bloom}.

\begin{figure}
  \centering
  \insertfig{durée_bloom.pdf}
  \captionT{Durée du bloom}{%
    \review{TODO}
  }
  \label{fig:duree-bloom}
\end{figure}

\subsection{Inversion du flux de chaleur}
\label{sec:flux-chaleur}

% chktex-file 13

\chapter{Conclusion}
\label{chp:conclusion}

Dans le cadre de cette thèse, grâce à 20~années de données satellites nous avons montré et \emph{quantifié} l'impact des fronts sur la \al{chl} dans l'Atlantique Nord autour du Gulf Stream, selon la saison et la biorégion concernée.
Nous réaffirmons l'augmentation \emph{locale} de la \al{chl} dans les fronts par rapport à l'environnement avoisinant.
Cet excès local est d'autant plus élevé que \emph{l'intensité des fronts} est importante.
Il est particulièrement important immédiatement autour du Gulf Stream (\pct{+60} en moyenne), ainsi que pendant le bloom printanier (où il atteint \pct{+150}).

En revanche, en calculer le surplus de \al{chl} à \emph{l'échelle de la biorégion} \encadra{en donc en comptabilisant la \emph{surface occupée} par les fronts} cette est augmentation est moins spectaculaire (inférieur à \pct{+5}).
Ce surplus est de valeur comparable pour les fronts de fortes et de faible intensité, ces derniers ayant un impact local moindre mais compensent par une plus importante surface occupée.

Dans le biome subpolaire, nous apportons des preuves observationnelles que le démarrage du bloom se fait plus tôt dans les fronts que dans le reste de la zone \encadra*{de deux semaines environ}.
Ce résultat constitue une première observation directe et quantification de ce phénomène.

peut être utilisé pour une paramétrisation dans un OGCM/ESM et des projections climatiques.

Au delà de ces résultats, nous fournissons une méthode de calcul des fronts validée.
Nous reconnaissons que malgré l'importance et les potentielles applications de la détection des fronts, il n'existe pas encore d'outils ou de produits facilitant leur utilisation scientifique.
Sans fournir y une solution définitive nous avons néanmoins pu mieux en délimiter la forme.
À la fois en termes des algorithmes potentiellement utile et de leur validation régionale, ainsi que des outils qui permettrait leur implémantation, et leur application sur des données satellites, diverses et parfois de large volume.

\fancybreakdisplay


%%% Appendices
\begin{appendices}
  \changeappendixmark%

\chapter[Article: \articleCceTitle][Article]{Article: \articleCceTitle}
\label[appendix]{ax:article-cce}
\chapter[Article: \articleReviewTitle][Article]{Article: \articleReviewTitle}
\label[appendix]{ax:article-review}

\renewcommand{\thesection}{\Alph{chapter}.\arabic{section}} % No chapter number in section
\chapterlof{Autres}
\label[appendix]{ax:autres}


\chapter{Variables drapeaux sur masque binaire}
\label{ax:cf-flags}

pourquoi : flags des variables chl
obtenir les nuages et aussi les pixels 'land'.

comment: expliquer les conventions cf (lien docu).
surtout une question de compression. une variable booléenne c'est en réalité 8bits.
première méthode: valeurs indépendantes. easy simple comparaison.
deuxième méthode: masque binaire. ajouter un bitwise xor avant la comparaison.
troisième: mélange des deux yaaas. enfait idem que 2ème. trop easy.

implémentation.

ce qu'il reste à faire (doc, cache).

\chapter{Bilan carbone de la thèse}
\label{ax:bilan-carbone}
% \suppressfloats[t]

Dans cette section, j'estime le coût carbone engendré par mon travail de thèse.
Commencons par les voyages professionels, typiquement un grand poste émetteur dans les carrières scientifiques.

Le premier est un aller"-retour Paris--Venise, en avion, pour assister à la 7\ieme~conférence \abbrv(LAPCOD)~(Lagrangian Analysis and Prediction of Coastal and Ocean Dynamics).
Réalisé pendant mon stage de master, je n'étais alors que peu sensibilisé à ces questions. Avec plus de prévoyance, il aurait pu être effectué en train.
Ce trajet comptabilise~\qty{266 \pm 27}{\kg\carbone} (sans compter les traînées) selon l'outil \abbrv{GES}~1point5, développé par le collectif \href{https://labos1point5.org/}{\textit{Labos~1point5}} (\cite{mariette_2022}).
En train, ce trajet aurait coûté~\qty{55 \pm 24}{\kg\carbone}.

Le second trajet est un aller"-retour Paris--Vienne, en train, pour assister l'assemblée générale 2022 de l'\abbrv{EGU}.
Selon le même outil, ce trajet a comptabilisé~\qty{75 \pm 33}{\kg\carbone}.
En avion il aurait coûté~\qty{230}{\kg\carbone}.
Je dois ici remercier les collègues chercheurs et le personnel administratif du laboratoire ayant rendu ce trajet quelque peu aventureux possible, et agréable.

J'estime \encadra{toujours avec \abbrv{GES}~1point5} mes trajets domicile"-travail effectués (le plus généralement) en bus et métro à~\qty{162 \pm 96}{\kg\carbone} par an.
Ce coût est en dessous de celui préconisé pour respecter les accords de Paris (\qty{0.3}{\tonne\carbone} selon \textcite{dugast_2019}).
Enfin les 5~repas journaliers (végétariens), pris au self du personnel, et les très nécessaires tasses de café et thé sont estimés à~\qty{420}{\kg\carbone} annuels par l'outil \href{https://nosgestesclimat.fr}{Nos Gestes Climat}.

Côté informatique, nous n'avons pas réalisé d'achats de matériel.
La majorité des calculs sont réalisés sur les machines du mésocentre de l'\ab{ipsl}; les données y sont également stockées (\qty{<1}{\To}).
Ces calculs sont pour leur vaste majorité de nombreux processus, courts (quelques minutes), et utilisent peu de ressources (un seul processeur et quelques~\unit{\Go} de mémoire vive) \encadra*{en contraste à une situation plus \guil*{simple}, par exemple, de quelques calculs très intensifs qui représenteraient la quasi"-totalité du bilan carbone}.
Il est donc compliqué d'estimer le coût de tous ces calculs dont l'historique n'a pas été conservé, en plus des difficultés usuelles de ce genre d'estimation sur des ressources informatiques partagé.

\begin{table}
  \centering
  \caption{Récapitulatif du bilan carbone}
  \label{tab:bilan-carbone}
  \begin{tabular}{l >{\hspace{2em}} r !{} r @{\,}w{l}{1em}@{}} \toprule
    \multirow{2}*{Objet} & \multicolumn{2}{r}{Coût (\unit{\kg\carbone})} \\
                         & par an & total                                \\
    \midrule
    LAPCOD (Venise)      &        & 266 & \rdelim\}{2}*[Voyages: \qty{341}{\kg\carbone}] \\
    EGU (Vienne)         &        & 75                                   \\
    \addlinespace

    Domicile-travail     & 162    & 648                                  \\
    Alimentation         & 420    & 1680                                 \\

    \midrule
    Total                &        & \bfseries 2669                       \\
    \bottomrule
  \end{tabular}
\end{table}

\end{appendices}

%%% Bibliography
\chapter*{\bibname}
\mtcaddchapter[\bibname]
\markboth{\bibname}{}
\label{bib}
{
  \emergencystretch=1em
  \printbibliography[heading=none, filter=normal]
}

\unsection{\bibdataTitle}
\markright{\bibdataTitle}
\label{bib:data}
{
  \emergencystretch=1em
  \printbibliography[heading=none, type=dataset]
}

\unsection{\bibsoftwareTitle}
\markright{\bibsoftwareTitle}
\label{bib:software}

J'ai été amené à utiliser durant ma thèse de nombreux outils informatiques sans lesquels ce travail n'aurait pu aboutir, ou en tout cas avec assurément beaucoup plus de difficultés.
Il me semble important d'en citer au moins une partie ici.
La reproductibilité de mon travail est déjà garantie par la mise à disposition de mes codes (accompagnés d'une courte documentation et de tout le nécessaire pour reproduire les résultats) sur un dépôt public\footnote{%
  Dépot Gitlab: \glsurl{gitlab}}
et dont une version est également déposée sur un répertoire Zenodo\footnote{\glsurl{zenodo}}.
Il s'agit ici plutôt de créditer les nombreux contributeurs qui ont participé à l'élaboration de ces outils.

Bien évidemment beaucoup des calculs reposent sur des librairies qui ne sont plus à présenter: \citesoft{numpy}, \citesoft{scipy}, et \citesoft{pandas}.
Cependant il est plus commun (en géosciences en tout cas) d'interfacer avec la librairie \citesoft{xarray}.
Les outils de \citesoft{dask} sont également avérés indispensables pour gérer les quantités importantes de données que sont les notres.

Il convient de citer plusieurs paquets permettant le calcul efficace d'histogrammes:
\citesoft{xhistogram} qui implémente ses fonctionalités à partir de fonctions NumPy et Dask \enquote*{élémentaires}, et
\citesoft{dask-histogram} qui s'appuie sur la librairie C \citesoft{boost}.

Plusieurs paquets sont utilisés pour réaliser les figures, bien évidemment \citesoft{matplotlib},
\citesoft{cartopy} pour les cartes,
mais également \citesoft{cmocean} qui fourni des palettes de couleurs, notamment pour un usage en océanographie. Ces dernières présentent les avantages (majeurs) d'être linéairement perceptibles, adaptés à une conversion en nuance de gris pour impression, et robustes à plusieurs types de daltonisme.
Similairement, les couleurs utilisées dans les figures sont le fruit du travail de Paul Tol\footnote{voir~\glsurl{paultol}}, distribué pour Python par mes soins (voir~\creftitle{sec:productions}).

Enfin il me parait approprié de citer les outils \citesoft{ipython} avec lequel j'execute tous mes scripts, \citesoft{mamba} et le projet \citesoft{conda-forge} lesquels me permettent de gérer les environments Python, et enfin le travail épatant derrière le framework de configuration \citesoft{doom} sur lequel tous les scripts et ce manuscript (entre autres) ont été écrits.

{
  \emergencystretch=1em
  \raggedright%
  \printbibliography[heading=none, type=software]
}

\backmatter%

\end{document}

