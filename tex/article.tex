
\authorArticle[1]{Clément}{Haëck},
\authorArticle[1]{Marina}{Lévy},
\authorArticle[1]{Inès}{Mangolte},
and \authorArticle[2]{Laurent}{Bopp}

{
  \small
  \affil[1]{LOCEAN-IPSL, Sorbonne Université, CNRS, IRD, MNHN, Paris, France}\\
  \affil[2]{LMD-IPSL, École Normale Supérieure / Université PSL, CNRS, École Polytechnique, Paris, France}
}

\vspace{\baselineskip}
\foreignlanguage{french}{%
  Manuscript envoyé à \glshref{biogeosciences} le \frenchdate{2022}{12}{22}.
  Pre-print publié sur EGUsphere le \frenchdate{2023}{01}{06} (\textsc{doi}:~\href{\glsentryurl{preprint-bgs}}{\texttt{\glsentrydoi{preprint-bgs-doi}}}).
}

% \begin{keypoints}
% \item Over \as{sst} fronts in the Gulf Stream region \al{chl} is enhanced by up to a factor two and the bloom onsets earlier by up to two weeks.
% \item The enhancement intensity increases significantly with the strength of the front.
% \item The net contribution of fronts to \al{chl} enhancement over the Gulf Stream region is less than 5\%.
% \end{keypoints}

\begin{articleBlock}{Abstract}
  Fronts affect phytoplankton growth and phenology by locally reducing stratification and increasing vertical nutrient supply.
  Biomass peaks at fronts have been observed in-situ and linked to local nutrient upwelling, and reduced stratification over fronts has been shown to induce earlier blooms in numerical models.
  However observation of these biophysical interactions through satellite imagery have been scarce, despite the opportunity to quantify them at synoptic scales.
  Here we used twenty years of \as{sst} and \al{chl} satellite data in a large region surrounding the Gulf Stream to quantify the impact of fronts on phytoplankton in contrasting regimes, from oligotrophy to bloom, and throughout the year.
  We computed an heterogeneity index \as{hi} from \as{sst}, and used it to sort fronts into weak and strong fronts based on \as{hi} thresholds.
  We observed that the localization of strong fronts corresponded to western boundary current fronts, and weak fronts to more ephemeral submesoscale fronts.
  We compared \al{chl} distributions over strong fronts, weak fronts and outside of fronts.
  We assessed three metrics, the local enhancement of \al{chl} over fronts, the global enhancement of \al{chl} due to fronts at the scale of the region, and the lag in spring bloom onset due to fronts.
  We found that weak fronts lead to a local enhancement of \al{chl} weaker than strong fronts, but because they are also more frequent they contribute equally to the regional \al{chl} budget.
  We also find the the local enhancement of \al{chl} was two to three times larger for the spring bloom than in the oligotrophic subtropical gyre.
  We also provide observational evidence that blooms start earlier over fronts, by one to two weeks.
  Nevertheless our results suggest that the spectacular impact of fronts at the local scale may be misleading, considering their impact on a regional scale budget remains limited.
\end{articleBlock}


\begin{articleBlock}{Plain Language Summary}
  Phytoplankton vary in abundance in the ocean over of large regions, and with the seasons, but also because of small-scale heterogeneities in surface temperature, called fronts, where phytoplankton growth can be favoured.
  Here, using satellite imagery, we found that fronts enhance phytoplankton much more where it is already growing well, but despite large local increases the  enhancement for the region is modest (\pct{5}). We also found that blooms start by one to two weeks earlier over fronts.
\end{articleBlock}


\subsection{Introduction}

Phytoplankton form the basis of marine food webs and are key players in the ocean carbon cycle.
The transport of limiting nutrients to the sunlit euphotic layer by advective and convective processes, and the amount of light received by the cells \encadra{which is closely related to the stratification of the water column} are two important factors that control their growth.
As there are marked contrasts in nutrient and light availability in the ocean, it follows that the global ocean can be divided into different regional biomes (or bioregions), characterized by different phytoplankton abundances and seasonality \parencite{longhurst_2007, vichi_2011a, bock_2022}.
The contrasts between biomes are largely explained by consistent physical forcings and environmental conditions, operating at the biome scale, which determine how the two main controlling factors, nutrient and light, limit growth.
For example, subtropical gyres are areas where wind-driven circulations induce a deepening of the thermocline and nutricline, resulting in oligotrophic biomes where productivity is relatively constant and low throughout the year; at higher latitudes, the nutricline is shallower and the strong seasonality of the vertical mixing will induce a multi-stage operation, with a time of reduced productivity and convective nutrient supply in winter when the mixing is strong, and a bloom in spring when the stratification sets in \parencite{wilson_2005, williams_2011}.

In addition to these large-scale patterns, there has been considerable evidence over past years that submesoscale motions actively influence the nutrient and light environments at the horizontal scale of the order of~\qtyrange{1}{50}{\km} \parencite[see reviews by][]{levy_2012, mahadevan_2016, levy_2018}.
Submesoscale motions arise dynamically through advective interactions involving mesoscale strain that continuously create sharp submesoscale density fronts, or at more steady wind-driven density fronts, such as western boundary currents \parencite{thomas_2008, mcwilliams_2016, mahadevan_2020}.
These fronts are characterized by an energetic secondary vertical circulation, with upwelling on the warm side of the front, and downwelling on the cold side.
Submesoscale dynamics also involve restratification and suppression of vertical mixing at the front \parencite{thomas_2008a}.
Thus submesoscale dynamics occurring at ocean fronts may affect phytoplankton in various ways.
Of interest here, the upward branch of the secondary circulation may enhance phytoplankton growth by transporting nutrients into the euphotic zone, while the downward branch may subduct biomass and excess nutrients into the subsurface \parencite{calil_2011, omand_2015, hauschildt_2021}.
In addition, in highly seasonal regimes where productivity is slowed in winter due to deep mixing, submesoscale restratification may promote localized phytoplankton blooms before the large-scale outburst associated with seasonal stratification \parencite{mahadevan_2012}.

Despite numerous local observations and a strong theoretical basis for these processes \parencite[e.g.\ recent studies by][]{marrec_2018, little_2018, verneil_2019, ruiz_2019, uchida_2020, kessouri_2020, tzortzis_2021}, their integrated contribution at the scale of regional biomes is still largely unknown.
Indeed, as most fronts form, move and dissipate continuously on time scales of days to weeks, they are particularly difficult to sample, and this limitation is reinforced by the fact that only a limited number of fronts can be observed with in situ field observations.
Thus satellite-derived estimates of \al{chl} (hereafter \as{chl}) although limited to the surface and an imperfect proxy for phytoplankton biomass, are the only data that allow to track the impact of fronts synoptically over large areas.
A first attempt to assess the contribution of fine scales to regional satellite \as{chl} budgets was based on a geostatistical analysis derived from data at 9km resolution \parencite{doney_2003}, extended later in \textcite{glover_2018}, with which they examined the change in spatial variance with distance.
This methodology was too coarse to reveal the impact of submesoscales but confirmed the important role of mesoscale eddies in stirring large scale gradients of phytoplankton abundance.
The role of submesoscales has been assessed with three different methods.
\textcite{guo_2019} combined ocean color data with altimetry and drifting floats, and estimated that, over subtropical gyres of the global ocean, the respective contributions of mesoscales and submesoscales to high \as{chl} anomalies were comparable in magnitude.
\textcite{keerthi_2022} proposed an approach based on deconvolution of local \as{chl} time series into different timescales; they observed that sub-seasonal time scales contributed roughly~\pct{30} of the total satellite \as{chl} variance and were associated with small~(\qty{<100}{\km}) spacial scales.
Finally, the most quantitative approach was proposed by \textcite{liu_2016}, which they applied to the North Pacific Subtropical Gyre.
They detected sea-surface temperature (\as{sst}) fronts by computing an index that measures the local heterogeneity of the \as{sst} field from satellite \as{sst} data.
This allowed them to compare satellite \as{chl} values over areas impacted by fronts (characterized by a large value of the heterogeneity index) with values over areas that were not impacted.
They found that the increase in \as{chl} over the fronts was negligible in summer but reached almost~\pct{40} in winter.

Here we built on this last approach, and quantify the excess \as{chl} due to the presence of fronts at the scale of biomes.
This more global quantification will depend on several factors, since the local contribution of fronts to \as{chl} depends on many factors.
First it depends on how efficient fronts are at supplying nutrients, which itself depends on how deep the fronts reach into the nutricline, and of the seasonality of this supply.
It should be noted that the overall efficiency of fronts has been questioned \parencite{levy_2018}; firstly because submesoscale fronts are more numerous in winter when convective nutrients input is also greatest; and secondly because the submesoscale vertical currents are often trapped within the mixed-layer, and may not reach the nutricline, which is often well below the base of the boundary mixed-layer.
Second, the overall contribution of fronts will differ between biomes, with submesoscale vertical advection of nutrients likely to be more important in oligotrophic biomes where other nutrient supply routes are scarce, and submesoscale restratification in blooming biomes.
Finally, the contribution of fronts will depend on their spatio-temporal footprint, which also varies seasonally \parencite{callies_2015} and regionally \parencite{mauzole_2022}.

Thus, more precisely, our intention is to explore and quantify how the contribution of fronts to biome-scale \as{chl} varies in three contrasted biomes, ranging from subtropical to subpolar, varies along the year, and varies with the occurrence and strength of fronts.
We focus our analysis on the North-Atlantic region surrounding the Gulf Stream, where multiple biomes and fronts of different strengths are found in a limited geographical area \parencite{bock_2022} with strong seasonality.
In the south, our study area encompasses part of the North Atlantic subtropical gyre, characterized by an oligotrophic regime, year-long low productivity.
In the north, north of the Gulf Stream jet, is a more productive subpolar regime characterized by a recurrent spring bloom.
In between, there is a moderately productive regime, with maximum productivity in winter.
Another feature that makes this study area particularly relevant is that it has different types of fronts.
On the one hand, there are two strong persistent fronts, the Gulf Stream and the shelf-break front, which are held in place by topography and atmospheric circulation, and which are both associated with strong and deep-reaching vertical circulations \parencite{flagg_2006, liao_2022}.
But there are also more ephemeral and weaker fronts, created by mesoscale strain that are continuously forming at more random locations \parencite{drushka_2019, sanchez-rios_2020}.
We use satellite data of \as{chl} and \as{sst}, and extend on the approach of \textcite{liu_2016}, to distinguish between persistent and ephemeral fronts.
We evaluate the impact of both types of fronts on \as{chl} on the basis of three indicators, the excess (or deficit) \as{chl} over fronts at the local scale of the front, the surplus \as{chl} attributable to fronts at the scale of regional biomes, and the change in the timing of the \as{chl} spring bloom over fronts.


\subsection{Methods and data}

Our approach combines daily satellite \as{sst} data, which are used to detect fronts and sort them by their strength, with daily satellite surface \as{chl}, from which we derive anomalies over fronts.
Our region of interest is the North Atlantic from~\latlon{15N} to~\latlon{55N}, and from~\latlon{40W} to the North American shelf break (Fig.~\ref{fig:zone-separation}).
This region covers three biomes, a more oligotrophic one in the south, a more productive one in the north where a spring bloom occurs, and an intermediate biome between the two, described below.
We will refer to them in the following as permanent subtropical biome (PSB), seasonal subtropical biome (SSB) and subpolar biome (PB), moving from south to north.
They are separated from south to north by a fixed boundary at~\latlon{32N} and by a variable boundary near~\latlon{40N} dynamically set at each time step to the instantaneous position of the Gulf Stream.

\subsubsection{Data}

For \as{chl}, we used the L3 product distributed by ACRI-ST over the period 2000--2020, generated by Copernicus-GlobColour, constructed with data from different sensors (\as{seawifs}, \as{modis} Aqua \& Terra, \as{meris}, \as{viirs}-\as{snpp} \& \as{jpss}[1], \as{olci}[-S3A] \& S3B) merged and reprocessed, available daily at 4km resolution \parencite{article_chl}\footnote{Dataset \dataname{chl_globcolour} décrit section~\ref{sec:donnees-chl_globcolour}}.

For \as{sst}, we used the European Space Agency Sea Surface Temperature Climate Change Initiative analysis product version 2.1 \parencite{merchant_2019, article_sst, good_2020}\footnote{Dataset \dataname{sst_esacci} décrit section~\ref{sec:donnees-sst_esacci}}, also available daily at~\qty{4}{\km} resolution over the period 2000--2020.
This product combines data from all available infrared sensors (\as{atsr}, \as{slstr}, and \as{avhrr} sensors), ensuring good resolution where data are available, unlike other \as{sst} products which also include microwave and in-situ measurements, resulting in considerable smoothing of the \as{sst} field. %chktex 36
Where \as{sst} data is not available, spatial interpolation is performed to obtain a cloud free product which, at the cost of resolution on finer features, provides complete synoptic coverage of our large study area.
This interpolation tends to provide an underestimate of the detection of fronts, as the \as{sst} field is smoother over cloud-covered areas \parencite{merchant_2019}.
However, the combination of several sensors allows to reduce these areas to a minimum.
Furthermore, we have only considered cloud-free pixels for our analysis, which ensures that cloudy areas are not taken into account in our quantification.


\subsubsection{Delimitation of biomes}

The three biomes in our domain correspond to well known production regimes which have been described previously, although sometimes with different names.
The region is characterized by the presence of a large-scale north-south gradient in \as{chl}.
All pixels where water depth is less than~\qty{<1500}{\m} (red isobath in Fig.~\ref{fig:zone-separation}) are masked to exclude the continental shelf.

The permanently oligotrophic regime to the south of our study area, characterized by warm waters and low \as{chl} (Fig.~\ref{fig:zone-separation}), is known as the subtropical gyre permanently stratified biome \parencite{sarmiento_2004} or permanent deep \as{chl} maximum biome \parencite{bock_2022}.
Their is no clear physical boundary to the northern limit of this permanent subtropical biome, so we have chosen the latitudinal limit of 32°N to delineate it, which roughly corresponds to the~\qty{0.1}{\mgm}~\as{chl} isocontour in annual mean \as{chl}.

North of~\latlon{32N}, the seasonal subtropical biome is also mainly oligotrophic, with intermediate levels of \as{chl} and temperature, and characterized by slightly increased productivity in winter.
This second biome is known as the subtropical gyre seasonally stratified biome \parencite{sarmiento_2004}.
It is bounded to the north by the meanders of the Gulf Stream jet.
The Gulf Stream jet conveys warm, salty waters poleward along the Florida coast up to Cape Hatteras (\latlon{35N}), where the jet separates from the continental shelf and meanders essentially zonally.
The north wall of the jet, so called because of its steep temperature gradient, marks the sharp, sinuous and unsteady northern limit of this biome.

To the north of the Gulf Stream is the Slope sea which extends to the shelf-break, with colder and fresher waters \parencite{linder_1998}.
Aligned with the shelf break, a persistent front with an intensified surface jet separates the shelf waters (excluded from this study) from the slope sea.
This highly productive subpolar biome, known as subpolar waters \parencite{sarmiento_2004} and high-chlorophyll-bloom \parencite{bock_2022}, is characterized by a strong spring bloom whose onset is tied to the spring stratification of the mixed-layer.

The position of the north wall of the Gulf stream, that delimits the subpolar and seasonal subtropical biomes (black meandering contour in Fig.~\ref{fig:zone-separation}a-b), is determined at each daily time step by thresholding the daily \as{sst} map.
The daily threshold values (black vertical line in Fig.~\ref{fig:zone-separation}c) are determined from the daily \as{sst} distributions above 32°N (blue line, continued by the yellow line in Fig.~\ref{fig:zone-separation}c).
Indeed, the Gulf Stream is easily identifiable in this distribution as it is manifested by a temperature peak (yellow line in Fig.~\ref{fig:zone-separation}c).
To detect the start of this peak, which marks our boundary, we fit the peak with a Gaussian, and define the threshold temperature as the mean minus twice the standard deviation.
Then, the threshold temperature time series is median filtered over an 8-day window to eliminate spurious detection anomalies.
This separation method ensures quasi-unimodal \as{chl} distributions within each biome (Fig.~\ref{fig:zone-separation}d).

With the above delimitation of biomes, the two atmospherically and topographically fronts, i.e.\ the Gulf stream and the shelf-break, are located within the seasonal subtropical biome and subpolar biome respectively, while the permanent subtropical biome only contains submesoscale fronts.

\begin{figure}
  \centering
  \includegraphics[width=12cm]{article/zone_separation.pdf}
  \caption[Delimitation of biomes]{
    Delimitation of the three biomes in the Gulf stream extension region: the Permanent Subtropical Biome~(PSB, south of the dashed line at~\latlon{32N}), the Seasonal Subtropical Biome~(SSB, between~\latlon{32N} and the meandering Gulf stream northern wall on that day marked with the black contour), and the Subpolar Biome~(PB, north of the Gulf stream northern wall).
    (a)~\as{sst} and (b)~\as{chl} snapshots on the 22 April 2007 (with data masked by clouds in white), (c)~\as{sst} and (d)~\as{chl}  distribution within each biome for the same day (PB:\@blue, SSB:\@yellow, PSB:\@red).
    The black line in~(c) shows the \as{sst} threshold value detected to delimit the Gulf Stream northern wall (see methods section).
    The x"-axis of the distributions correspond to the x-axis scale of the corresponding color bars.
    The red line follows the 1500m isobath.
    Data on the continental shelf~(\qty{<1500}{\m}) is not considered here and have been masked.
  }%
  \label{fig:zone-separation}
\end{figure}


\subsubsection{Front detection}

Our front detection method aims to sort the domain into pixels that are located (or not) over fronts, at each daily time step.
We follow the approach of \textcite{liu_2016} that builds upon the well established Cayula-Cornillon algorithm \parencite{cayula_1992, belkin_2009}, and identify pixels as belonging to fronts when the region in the vicinity of the pixel is characterized by high \as{sst} heterogeneity.
More precisely, on each pixel and for each time step, we compute an Heterogeneity Index~(\as{hi}) which quantifies the heterogeneity of the \as{sst} in a square window comprising a few pixels~(\numproduct{7 x 7}) centered on the pixel of interest.
Thus \as{hi} measures the \as{sst} heterogeneity at spatial scales less than~\qty{30}{\km}.
The \as{hi} is defined as the weighted sum of the skewness~\ab{skew}, standard deviation~\ab{std}, and bimodality~\ab{bimod} of \as{sst} within the window (\(\am{hi} = a \left( b \am{skew} + c \am{std} + d \am{bimod} \right)\), with a, b, c, d constant normalization coefficients).

We have adapted the original formulation of \textcite{liu_2016} with minor modification in the computation of bimodality and in the relative weighing of each component which are described below.
We computed the bimodality as the L2 norm of the difference between the \as{sst} histogram (with bins of~\tC{0.1}) and a Gaussian fit of the histogram.
We chose this method because it was simpler to implement and more robust than that of \textcite{liu_2016} which computed the normalized absolute difference between a polynomial fit of order~5 of the \as{sst} histogram and a Gaussian fit of the histogram, the polynomial fit often being poorly constrained.
We normalized each component by its variance (b, c and d being defined as the inverse of the standard deviation of each component computed over one year), as opposed to \textcite{liu_2016} who normalized the coefficients by their annual maximum.
This avoided putting too much weight on extrema.
Finally we normalized \as{hi}~(coefficient a) such that~\pct{95} of values are below an arbitrary value of~\num{9.5}.
For simplicity, the normalization coefficients were computed for year 2007 and used over the entire time series.
We performed sensitivity tests on the different parameters used to compute \as{hi}, namely the number of pixels in the rolling window and the choice of normalization coefficients.
An example of the resulting \as{hi} over a single front is given on Fig.~\ref{fig:zoom}; elevated \as{hi} values are located along the curved \as{sst} gradient to the north east of the window.

At each time step, we sorted the pixels into those belonging to strong fronts (defined as \(\am{hi} > 10\)), those belonging to weak fronts (defined as \(5 < \am{hi} < 10\)), and those that do not belong to fronts (when \(\am{hi} < 5\), and called them ``background'' in the following).
Sorting fronts by range of \as{hi} enabled us to roughly separate the quasi-permanent fronts, which are associated with the strongest \as{sst} gradients~(strong fronts), from the more ephemeral ones, associated with weaker gradients~(weak fronts).
The choice of the two \as{hi} threshold is somehow arbitrary, but it is supported by the \as{hi} distributions presented below.


\begin{figure}
  \centering
  \includegraphics[width=12cm]{article/front_example.pdf}
  \caption[Example of front]{
    The \as{sst}, \as{chl}, and heterogeneity index~(\as{hi}) of a front on the 7 July 2007.
    The plain and dashed contours correspond to \as{hi} values of~5 and~10.
    This front is categorized as weak.
    \as{chl} are elevated inside the front.
  }%
  \label{fig:zoom}
\end{figure}

\subsubsection{Quantification of the impact of fronts on Chl-\emph{a}}

We used three metrics to quantify the impact of fronts on \as{chl}, the excess \as{chl}~\as{exces} for the local scale (scale of fronts), the surplus \as{chl}~\as{surplus} for the global scale (scale of biomes) and the lag in onset day~\as{lag} for the bloom timing.

In order to quantify how \as{chl} is affected by fronts at the local scale, the metrics E is based on the comparison between the distributions of \as{chl} over fronts and the distributions of \as{chl} outside of fronts.
Distributions were computed over 8~days windows to limit the influence of particularly cloudy days.
They were computed within each biome, and over latitudinal bands of width \ang{5}, to minimize the influence of the large-scale north south gradient in \as{chl}.
Distributions were compared in terms of their median value, but using mean values yielded similar results (not shown).
We defined and computed the local excess \as{chl}, \as{exces} (expressed in per cent), as the median value over fronts minus the median value in the background, divided by the median value in the background, for each distribution.
We repeated this for weak and strong fronts.

To quantify the large-scale impact of fronts on \as{chl} at the scale of biomes, we computed the biome surplus of \as{chl} which we defined and computed as the relative difference (expressed in per cent) between the mean \as{chl} over the entire biome (\(\mathrm{MT}\)), and the mean \as{chl} over the background (\(\mathrm{MB}\)), \(\am{surplus} = (\mathrm{MT}-\mathrm{MB}) / \mathrm{MB}\).
Thus the surplus~\as{surplus} measures the extra quantity of \as{chl} at the scale of the biome (\(\mathrm{MT} - \mathrm{MB}\)), relative to what would be the situation in the absence of fronts (\(\mathrm{MB}\)), and thus accounts for the local increase over fronts~\as{exces}, but also for the proportion of fronts in a given biome.
To better understand the meaning of the surplus, let us consider the simplified case where \as{chl} is homogeneously doubled over fronts compared to the background value, i.e.\ when \(\am{exces} =\pct{100}\); in that case, the surplus \as{chl} is~\pct{50} if there are~\pct{50} of fronts, and is~\pct{1} if there are~\pct{1} of fronts.
Note that the computation of the local excess~\as{exces} is based on median values because it relies on the comparison of distributions, while for the computation of the biome surplus~\as{surplus}, we used mean values in order to be conservative.

Finally, the time series of the \as{chl} median in the subpolar biome is characterized by a spring bloom, of which we measured the timing (onset date) both in the fronts and in the background.
Because the spring bloom onsets propagates from south to north, these timings were inferred over latitudinal band of width~\ang{5}.
To extract the onset date, we filtered the \as{chl} median time series with a low-pass Butterworth filter of order~2 and cutting frequency~\(1/20\,\textrm{days}^{-1}\).
The filtered time-series displayed strong variations in their phenology from year to year, but a bloom was always discernible.
We considered data from February to July, which allowed us to isolate the spring bloom and exclude the autumn bloom.
First, we detected the maximum value of \as{chl} in this time window, then defined the bloom onset as the time of maximum \as{chl} derivative prior to the time of maximum \as{chl}.
To estimate the uncertainty in this evaluation, we computed the standard deviation of all days for which the \as{chl} derivative was above~\pct{90} of its maximum value.
Finally, we averaged the yearly values of lag~\as{lag} in onset days over the 20~years range by computing a weighted average and standard deviation of the difference between values in fronts and in background for each latitudinal band, with the weights equal to the inverse of the standard deviations for each year.

\subsection{Results}

\subsubsection{Distribution of fronts}

Our definition of weak fronts and strong fronts is primarily based on thresholds, derived from the \as{hi} distributions (Fig.~\ref{fig:chl-vs-hi}, red dashed line).
In each biome, the majority of \as{hi} values are below~5, and as mentioned before, we used this threshold to distinguish pixels in the background from those over fronts (when \(\am{hi} > 5\)).
Secondly, the number of points in the \as{hi} distribution decreases sharply as the \as{hi} value increases above~5, reflecting the fact that fronts with stronger \as{sst} gradients are much less frequent than fronts with weaker gradients, as one would expect.
We used the \as{hi} threshold of~10 to distinguish weak fronts from strong fronts.
This separation is imperfect, as seen for instance in Fig.~\ref{fig:zoom} where a few pixels with \as{hi} values larger than~10 appear at the core of an otherwise weak front.

\begin{figure}
  \centering
  \includegraphics[width=12cm]{article/chl_vs_hi.pdf}
  \caption[Distribution of \glsentryshort{chl} against \glsentryshort{hi}]{%
    Normalized distribution of the Heterogeneity Index~(\as{hi}, red dashed line) within each biome, and distribution of \as{chl} as a function of \as{hi} (representing front strength), over the full period 2000--2020.
    Shown are the median value of the \as{chl} distributions (solid black line), and 1st and 3rd quartiles (dashed lines).
    Note that~\pct{0.5} of pixels have outstanding large \as{hi} values and are not included here.
  }%
  \label{fig:chl-vs-hi}
\end{figure}

However, the choice of these two \as{hi} thresholds is also guided by the resulting global spatial climatology of weak and strong fronts (Fig.~\ref{fig:frt-occurrence}).
Weak fronts are abundant, and more or less evenly distributed, over a broad band around and north of the Gulf Stream jet (Fig.~\ref{fig:frt-occurrence}a).
To the south of the Gulf Stream jet, weak fronts are less present, with nevertheless more fronts on the edges of the subtropical gyre (around~\latlon{28N}) than in its center.
This distribution of weak fronts is consistent with the predominance of mesoscale variability observed along the Gulf Stream system from satellite altimetry \parencite{zhai_2008}, and the injection of eddy kinetic energy north of the Gulf Stream jet by the Gulf Stream extension.
It is thus consistent with the generation process of submesoscale fronts through mesoscale strain.

\begin{figure}
  \centering
  \includegraphics[width=12cm]{article/fronts_occurrence.pdf}
  \caption[Occurrence map of strong and weak fronts]{%
    Occurrence of (a)~weak fronts and (b)~strong fronts expressed as the percentage of time over the entire time series~(2000--2020) that a given pixel is occupied by a front.
  }%
  \label{fig:frt-occurrence}
\end{figure}

The climatological distribution of strong fronts shows that they coincide mainly with the Western Boundary Current system, which consists of two main permanent fronts, the first following the Gulf Stream jet going northeast from Cape Hatteras, and a second more northerly and extending eastward to~\latlon{50W}, following the northeastern U.S.
continental shelf break (Fig.~\ref{fig:frt-occurrence}b).
Thus these contrasted coverage of localized strong fronts and more widespread weak fronts are consistent with the hypothesis that weak fronts are representative of ephemeral submesoscale fronts, whereas strong fronts are representative of the more permanent fronts associated with the Western Boundary Current system that covers part of the seasonal subtropical and subpolar biomes.
This separation allows us to provide a first order sorting of the impact of each type of front.

% <check list of percent correctly typeset>
With the chosen thresholds, the areal proportion of weak fronts tends to increase from South to North, with~\qtylist{7;19;42}{\percent} of \as{hi} values comprised between~5 and~10 in the permanent subtropical, seasonal subtropical, and subpolar biome, respectively.
Regarding strong fronts, there are only present in the seasonal subtropical and subpolar biome (as expected from the fact that the Gulf Stream and shelf-break fronts are located within these biomes), where \as{hi} values above 10 account for~\pct{6} and~\pct{17}, respectively.

The fraction of the area occupied by fronts also varies with the seasons, with generally less fronts in summer (Fig.~\ref{fig:ts-climato}d-f).
In the permanent subtropical biome, weak fronts cover on average up to~\pct{12} in spring and drop to~\pct{2} in summer.
In the seasonal subtropical biome, the variation is from~\pct{27} to~\pct{13} for weak fronts, and~\pct{8} to~\pct{4} for strong front.
In the subpolar biome, strong fronts cover between~\pct{11} in summer and~\pct{26} at their peak.
This seasonality is consistent with other estimates that submesoscale activity is greater in winter due to greater mixed-layer thickness compared to summer \parencite{callies_2015}, and also consistent with the most recent modelling results by \textcite{dong_2020} in another Western Boundary system (the Kuroshio Extension) where they show that the strongest submesoscales occur with a lag of about a month after the mixed layer thickness maximum is reached.
An exception to this seasonal pattern is the greater presence of weak fronts in summer in the subpolar biome (\pct{50} in summer versus~\pct{35} in winter), which may be partly explained by the reduction of the \as{hi} over Western Boundary Current fronts in summer, which results in them being counted as weak fronts in summer.

\subsubsection{Local impact of fronts on Chl-\emph{a}}

The local impact of fronts on \as{chl} is illustrated by the example shown in Fig.~\ref{fig:zoom}, where the highest values of \as{chl} are found within the \as{hi} contour delimiting fronts.
It should be noted, however, that in this example as in other similar examples, the localization of the highest \as{chl} do not perfectly coincide with elevated values of \as{hi};\ they are places where the \as{hi} is large and \as{chl} is small, and there are also places with elevated patches of \as{chl} outside of \as{hi} contours.

Nevertheless, it appears that on average, \as{chl} is affected by the presence of fronts, and depends on their strength.
In order to measure that effect, we computed for each biome the \as{chl} distribution sorted by bins of \as{hi} (of width~\num{0.1}) for the whole time series~(Fig.~\ref{fig:chl-vs-hi}).
For low values of the heterogeneity index, these distributions are representative of background conditions and reflect the expected differences between biomes: the median \as{chl} is lowest~(\qty{0.05}{\mgm}) in the permanent subtropical biome, intermediate in the seasonal subtropical biome~(\qty{0.1}{\mgm}), and highest in the subpolar biome~(\qty{0.2}{\mg1}).
The \as{chl} variability along the year and within the biome (i.e.\ the width of the distribution, highlighted by grey shading in Fig.~\ref{fig:chl-vs-hi}) is larger moving northwards, because of the higher seasonal variability.

Importantly, in all 3 biomes, \as{chl} values increase with \as{hi}, and are also more dispersed as \as{hi} increases~(Fig.~\ref{fig:chl-vs-hi}).
This suggests that the distinction between background, weak and strong fronts is somehow artificial and that the observed changes in \as{chl} are overall rather continuously dependent on the value of \as{hi}.
Nevertheless, our partitioning into background, weak and strong fronts remains meaningful because it enables us to isolate the most permanent fronts from the ephemeral ones, and we will therefore retain it in the following.


\begin{figure}
  \centering
  \includegraphics[width=12cm]{article/ts_climatology.pdf}
  \caption[Seasonal impact of fronts on \glsentryshort{chl}]{
    Seasonal impact of fronts on \as{chl} in the permanent subtropical biome (1st column), in the seasonal subtropical biome (2nd column) and in the subpolar biome (3rd column).
    (a-b-c)~\as{chl} median values (top row) over weak fronts (blue), strong fronts (green) and background (red).
    The differences between the curves show the local impact at the scale of fronts.
    (d-e-f)~Surface fraction occupied by weak fronts and strong fronts.
    (g-h-i)~Global \as{chl} surplus due to weak fronts and strong fronts at the scale of the biome.
    The surplus accounts for the local excess and for the number of fronts (see method).
    The plain lines represent the climatological mean, and the envelopes mark the standard deviation over the period~2000--2020.
    \as{chl} is more strongly enhanced over strong fronts than over weak fronts, but weak fronts are more numerous than strong fronts, resulting in a \as{chl} surplus that can be reversed.
  }%
  \label{fig:ts-climato}
\end{figure}

Now we examine how the impact of weak and strong fronts varies seasonally in the three biomes, which are characterized by different seasonal variations of \as{chl}~(Fig.~\ref{fig:ts-climato}).
We first describe these seasonal variations, then examine the impact of fronts over each biome as a whole, and then by latitudinal bands within each biome.

In the permanent subtropical biome (Fig.~\ref{fig:ts-climato}a), the seasonal variations of \as{chl} are very modest with a weak peak in winter.
In the seasonal subtropical biome (Fig.~\ref{fig:ts-climato}b), the seasonal variations are well marked, with a minimum in summer, and an increase that starts in fall and peaks in late winter.
In the subpolar biome (Fig.~\ref{fig:ts-climato}c), there is a marked bloom in spring with a peak in \as{chl} in April, followed by an autumn bloom albeit with a smaller magnitude.
These different phenologies are well documented and largely explained by the differences in the seasonal cycle of the mixed-layer, and the relative depths of the winter mixed-layer and the nutricline (see for instance \textcite{levy_2005a} for a description of the drivers of these three production regimes in similar biomes of the Northeast Atlantic).
In the permanent subtropical biome, the low productivity is due to the fact that winter mixing is not sufficient to provide a substantial convective supply of nutrients; in contrast, in the seasonal subtropical regime, the increase in production in fall starts when the mixed-layer deepens and reaches the nutricline, leading to a fall-winter bloom.
In the subpolar biome, this fall bloom is interrupted in winter when the mixed-layer significantly deepens, diluting cells vertically, and a spring bloom is initiated when the mixed-layer stratifies at the end of winter.

In both subtropical biomes, \as{chl} over fronts is systematically larger than in the background, and this throughout the year (Fig.~\ref{fig:ts-climato}a-b).
This increase is very modest over the permanent subtropical biome.
The local increase over weak fronts also remains modest in the seasonal subtropical gyre, but is much larger over strong fronts.
Finally in the subpolar biome (Fig.~\ref{fig:ts-climato}c), \as{chl} is significantly higher over fronts from the fall until the peak of the bloom but this difference diminishes as the spring bloom decays, and throughout summer.
These results are very weakly sensitive to the choice of parameters used to compute \as{hi} (supplementary Fig.~\ref{fig:ts-sensitivity}).
We can also note that the standard deviation range of the yearly median values (Fig.~\ref{fig:ts-climato}a-c) is smaller than the differences between fronts and background, which further confirms that these differences are robust over 20 years of data, for the three biomes and the two types of fronts.

The strength of the local excess in \as{chl} over fronts also depends on latitude.
In the permanent subtropical biome (Fig.~\ref{fig:latbands-s}), the excess is only detectable north of~\latlon{25N} and remains modest (\(\am{exces}<\pct{10}\)); it reaches its maximum of~\pct{10} at the beginning and end of the production season.
In the seasonal subtropical biome (Fig.~\ref{fig:latbands-i}), the excess is also small in the southern part of the biome (south of~\latlon{35N}) but strongly increases further north (between~\latlon{35N} and~\latlon{45N}) and decreases again going even further North (between~\latlon{40N} and~\latlon{45N}).
moreover, in this biome, the increase is stronger in summer (July"-August) compared to winter (December"-January).
Finally in the subpolar biome, the magnitude of the local excess diminishes going northward.
It tends to be stronger during winter and during the bloom, and weaker in summer.
In fact in summer, the differences are hardly discernible when averaged over the entire biome (Fig.~\ref{fig:ts-climato}c), but there is an excess \as{chl} in the southern part of the biome~(\latlon{<45N}), and a slightly deficit in the northern part of the biome~(\latlon{>45N}) (Fig.~\ref{fig:latbands-n}).

It ensues that the annual mean local excess in \as{chl} over fronts has a distinct latitudinal pattern (Fig.\ref{fig:recap}a).
The strongest local increase due to fronts is located in the latitudinal band~\qtyrange{35}{45}{\degree}N, where the seasonal subtropical and subpolar biomes meet.
The excess is two to three times larger over fronts North of the Gulf stream (i.e.\ in the subpolar biome) than South of it (i.e.\ in the seasonal subtropical biome).
Overall, the annual mean local excess over weak fronts varies between 0 and~\pct{+30}\%, and between~\pct{15} and~\pct{+60} over strong fronts.



\begin{figure}
  \centering
  \includegraphics[width=12cm]{article/ts_latbands_S.pdf}
  \caption[Local impact of front on \glsentryshort{chl} in the permanent subtropical biome]{%
    Permanent subtropical biome: local impact of fronts on \as{chl} by range of latitudes in the biome.
    (a-c-e)~\as{chl} median values over weak fronts (blue) and background (red), (b-d-f)~corresponding local excess of \as{chl} in weak fronts computed as the relative difference of \as{chl} in fronts and in the background.
    The plain lines represent the climatological mean, and the envelopes mark the standard deviation over the period~2000--2020.
    The excess increases from south to north.
  }%
  \label{fig:latbands-s}
\end{figure}

\begin{figure}
  \centering
  \includegraphics[width=12cm]{article/ts_latbands_I.pdf}
  \caption[Local impact of front on \glsentryshort{chl} in the seasonal subtropical biome]{%
    Seasonal subtropical biome: local impact of fronts on \as{chl}  by range of latitudes in the biome.
    (a-c-e)~\as{chl} median values over weak fronts (blue), strong fronts (green) and background (red), (b-d-f)~corresponding local excess of \as{chl} in weak and strong fronts computed as the relative difference of \as{chl} in fronts and in the background.
    The plain lines represent the climatological mean, and the envelopes mark the standard deviation over the period~2000--2020.
    The excess is maximum at mid-latitudes.
  }%
  \label{fig:latbands-i}
\end{figure}

\begin{figure}
  \centering
  \includegraphics[width=12cm]{article/ts_latbands_N.pdf}
  \caption[Local impact of front on \glsentryshort{chl} in the subpolar biome]{
    Subpolar biome: local impact of fronts on \as{chl}  by range of latitudes in the biome.
    (a-c-e)~\as{chl} median values over weak fronts (blue), strong fronts (green) and background (red), (b-d-f)~corresponding local excess of \as{chl} in weak and strong fronts computed as the relative difference of \as{chl} in fronts and in the background.
    The plain lines represent the climatological mean, and the envelopes mark the standard deviation over the period~2000--2020.
    The excess decreases from south to north.
  }%
  \label{fig:latbands-n}
\end{figure}


\subsubsection{Biome-scale impact of fronts on Chl-\emph{a}}

Within each biome, the surplus \as{chl} due to the presence of fronts accounts both for the relative increase of \as{chl} over fronts and for the relative area covered by fronts.
In the permanent subtropical biome, the surplus due to weak fronts oscillates between a maximum of~\pct{3}, and a minimum of zero in summer when the coverage of fronts is the lowest~(Fig.~\ref{fig:ts-climato}g).
The magnitude of the surplus is larger in the seasonal subtropical biome where it reaches~\pct{7} in May for weak fronts (Fig.~\ref{fig:ts-climato}h).
We can note that the surplus due to strong fronts is much weaker despite a much stronger local impact of strong fronts (Fig.~\ref{fig:ts-climato}b) due to the small surface area covered by strong fronts (Fig.~\ref{fig:ts-climato}e).
The largest surplus is found in the subpolar biome with maximum values of~\pct{12} and due to strong fronts in March.
In contrast to the seasonal subtropical biome, in the subpolar biome the surplus associated to strong fronts is larger than the surplus associated to weak fronts.
We can also note a deficit of \as{chl} due to fronts in summer (negative surplus, Fig.~\ref{fig:ts-climato}i) due to the negative excess in the northern part of the subpolar biome (Fig.~\ref{fig:latbands-n}b-d).

Overall, the annual mean \as{chl} surplus due to fronts is very modest (Fig.\ref{fig:recap}b).
All fronts added, the surplus is of the order of~\pct{1} for the permanent subtropical biome and~\pct{6} for the seasonal subtropical and subpolar biomes, with contributions from weak and strong fronts which have similar magnitudes.
Therefore, there is roughly~\pct{5} more \as{chl} in the Gulf Stream region than there would be in the absence of fronts.

\begin{figure}
  \centering
  \includegraphics[width=12cm]{article/recap.pdf}
  \caption[Recap of local and global impact of fronts on \glsentryshort{chl}]{%
    (a)~Annual mean local \as{chl}  excess over fronts (in \%), sorted by latitudinal band (x-axis),  by biome (shape of symbol) and by front type (weak fronts in blue, strong fronts in green).
    (b)~Annual mean global surplus of \as{chl} (in \%) for each biome, sorted by front type.
  }%
  \label{fig:recap}
\end{figure}


\subsubsection{Bloom timing in the subpolar biome}

Another strong impact of fronts is the earlier onset of the bloom over fronts in the subpolar biome (Fig.~\ref{fig:ts-climato}c).
The spring bloom onset propagates from South to North in the biome, starting in early April at~\latlon{35N} and in late June at~\latlon{55N} (Fig.~\ref{fig:latbands-n}).
The bloom onsets earlier over fronts in each latitudinal band (Fig.~\ref{fig:latbands-n}).
To quantify the lag in bloom onset day over fronts, we estimated bloom onset days for each year and within each latitudinal band, over the background and over fronts (Fig.~\ref{fig:bloom}).
Our estimates are based on averaged statistics over Eulerian time series, assuming that the bloom evolves coherently in the background (resp.\ over fronts) within each latitudinal bands in the subpolar biome.
Thus we are assuming that front and non front pixels can be pooled apart to follow the bloom evolution over the two contrasting environments.
This assumption is suggested by high-resolution models of the bloom \parencite[e.g.][]{levy_2005a, karleskind_2011}, but it is imperfect as the bloom evolves along Lagrangian trajectories.
Nevertheless we found very consistent results with this approach.
In all latitudinal bands, we found that the bloom onset occurs one week earlier in weak fronts than in the background (by~\num{-6.7 \pm 1.1}~days) and two weeks earlier in strong fronts (by~\num{-13.5 \pm 1.5}~days).
There is a large spread in bloom onset dates in the 20~years of data, due to the very intermittent nature of the bloom onset \parencite{keerthi_2021} which makes it difficult to detect with precision during certain years.
Furthermore, in many cases no difference in bloom onset or duration could be detected between the fronts and the background (dots aligned on the diagonal in Fig.~\ref{fig:bloom}).
Nevertheless, for individual years, delays larger than one months could occur.

\begin{figure}
  \centering
  \includegraphics[width=8.3cm]{article/bloom.pdf}
  \caption[Comparison of bloom onset dates in the background and over fronts]{%
    Subpolar biome. Comparison of bloom onset dates (in day of year) in the background (x-axis) and over fronts (y-axis), sorted by strength of fronts (shape of symbol), and latitudinal band (color).
    The line \(y=x\) is plotted in black.
    The distance between the black line and the dotted (resp.\ dashed) grey line is the measure of the average difference between weak (resp.\ strong) fronts and background.
    The bloom onset day propagates from south to north and starts earlier over fronts at all latitudes in the subpolar biome.
  }%
  \label{fig:bloom}
\end{figure}


\subsection{Discussion}

Our analyses of surface satellite data over the Gulf Stream extension region, based on the computation of an heterogeneity index \as{hi}, allowed us to show a substantial local increase of \as{chl} concentrations over \as{sst} fronts compared to background levels, to detect earlier blooms by one to two weeks over fronts, and to quantify that the global effect of fronts on \as{chl} concentration at the scale of the region was less than~\pct{5}.
The background levels in this region are very contrasted seasonally and geographically, with a productive and highly seasonal subpolar biome North of the Gulf Stream, a more steady oligotrophic permanent subtropical biome to the South, and an intermediate situation in between the two, where a seasonal subtropical biome prevails.
The main results above hold for these three contrasted biomes, although with different intensities, which also depend on the strength of \as{hi}.

\subsubsection{Caveats}

The level~4 \as{sst} product used in this study has the advantage to be readily available on download platforms (here\as{cmems}) at a reasonably high spatial resolution~(\qty{4}{\km}), avoiding the need to regrid as was done in \textcite{liu_2016} who used~\qty{1}{\km} resolution L2 MODIS-Aqua data.
It also has an excellent spatial coverage (e.g.\ Fig.~\ref{fig:zone-separation}) as it includes merged data from several sensors, but there is a trade-off between coverage and resolution.
We have performed initial tests to investigate the differences in using 1km and 4km resolution \as{sst} data, which convinced us that the use of 4km resolution data was appropriate (not shown), particularly since the heterogeneity index \as{hi} is computed here over boxes of~\qtyproduct{30 x 30}{\km}.
Nevertheless a more in-depth study of the sensitivity of our results to the resolution of the satellite products could be carried out in the future.

Another caveat of this level~4 product is that the spatial interpolation performed to merge data from several sensors smooths out the finer features, particularly when some of the data are obstructed by clouds.
Here we somehow avoided these smoothed areas by using only cloud-free \as{chl} pixels.
Nevertheless, a bias remains in that there may be a positive correlation between areas with fronts and the presence of clouds.
This is the case over the Gulf Stream jet, where dramatic surface temperature gradients are found, and constant clouds are detected over the front.
Similar effects can be expected over smaller, short-lived fronts, but probably on a smaller scale.

Moreover, our assessment of the effect of fronts on phytoplankton, based on surface \as{chl}, is probably a lower estimate given that the biological signal due to the upwelling of nutrients at fronts often does not reach the surface and is more intense at sub-surface \parencite{mourino_2004, ruiz_2019}.

Finally, the ratio of \as{chl} to total phytoplankton biomass in carbon, \as{chl}:C, changes under varying environmental conditions and community changes \parencite{behrenfeld_2015, halsey_2015, inomura_2022}.
Diatoms exhibit higher \as{chl}:C ratios and are more prevalent in fronts and thus would tend to make our biomass surplus estimation overestimated \parencite{treguer_2018}.
This uncertainty could be restricted by taking advantage of recent advances in synoptic estimations of the phytoplanktonic functional types concentrations \parencite{elhourany_2019}.

\subsubsection{Local impact of fronts on Chl-\emph{a}}

With these caveats in mind, we find that the degree of local \as{chl} increase at fronts varied seasonally, but mostly varied from one biome to another, with an intensity which was weaker in the more oligotrophic region, stronger between~\qtyrange{35}{45}{\degree}N, and intermediate in the subpolar biome (Fig.~\ref{fig:recap}a).
Moreover, the local excess of \as{chl} was always significantly larger over strong fronts than over weak fronts.
The increase in \as{chl} with increasing front strength (i.e.\ with increasing \as{hi}, Fig.~\ref{fig:chl-vs-hi}) is consistent with the hypothesis that phytoplankton production is enhanced at fronts by a submesoscale vertical flux of nutrients, and that this flux is stronger the stronger the front is.
On the other hand, the larger dispersion in the \as{chl} distribution with increasing \as{hi} reflects the fact that not all fronts are equally efficient.

Co-occurrence between frontal vertical velocities (or divergence) and enhanced \as{chl} has been observed over specific fronts in the North Atlantic \parencite{mourino_2004, allen_2005, lehahn_2007}.
The only study that has statistically connected enhanced \as{chl} with the presence of temperature front was conducted in the North Pacific subtropical gyre \parencite{liu_2016}, which shares characteristics with the permanent subtropical biome examined here.
Our results thus extend those of \textcite{liu_2016} to a region with stronger biological contrasts and phenologies.

A key factor determining the magnitude of the local \as{chl} response to frontal dynamics is the magnitude of the nutrient fluxes, which itself depends on the magnitude of the vertical velocities, of their depth penetration, and of the depth of the nutricline.
The nutricline depth shows a sharp latitudinal gradient within this region, from~\qty{150}{\m} depth at~\latlon{25N} to~\qty{50}{\m} at~\latlon{50N} \parencite{romera-castillo_2016}.
This explains the maximum magnitude of the \as{chl} response at the northern edge of the subtropical gyre, where the lack of nutrients is more severely controlling phytoplankton abundance than further north, and where the nutricline is closer to the surface than further south.

Moreover, vertical velocities associated with ephemeral fronts, often confined to the mixed layer, are likely to be a less efficient nutrient flux pathway to the euphotic zone from the interior than deep, dynamic, persistent fronts extending well below the mixed layer \parencite{levy_2018}.
The contrasting impacts of deep and shallow fronts are striking in models \parencite{levy_2012}, but are difficult to quantify from a small number of in situ observations.
Here we observed that the magnitude of the \as{chl} response over fronts increased with the strength of the heterogeneity index \as{hi} (Fig.~\ref{fig:chl-vs-hi}).
In other words, strong fronts, characterized by high values of \as{hi} (\(\am{hi} > 10\)), led to a stronger increase in \as{chl} values than weak fronts, characterized by intermediate values of \as{hi} (\(5 < \am{hi} < 10\)).

Finally, an important outcome of this study is that, contrary to what is generally thought, the impact of fronts is stronger in bloom regimes than in oligotrophic regimes.
In the subpolar biome, the \as{chl}  excess over fronts reaches~\pct{150} during the bloom and~\pct{50} in summer (Fig.~\ref{fig:latbands-n}), while in the permanent subtropical biome it never exceeds~\pct{10} (Fig.~\ref{fig:latbands-s}).
We should note that this last number is lower than that find by \textcite{liu_2016} in the permanent subtropical biome of the North Pacific, possibly because the higher resolution of the product that they have used, as discussed earlier.
The enhancement of the spring bloom by submesoscale processes observed here was recently also put forward in a modelling study by \textcite{simoes-sousa_2022}.

\subsubsection{Permanent and ephemeral fronts}

The above results are suggestive that the heterogeneity index could be used as a way to discriminate between permanent (and deep) fronts, and ephemeral (and shallower) fronts.
The localization and frequency of strong and weak fronts is consistent with this hypothesis.
Weak fronts are much more frequent than strong fronts, as we expect from ephemeral sub-mesoscale fronts compared with permanent fronts (Fig.~\ref{fig:frt-occurrence}).
In addition, the localization of strong fronts coincide with the position of the Gulf Stream.
Another element that supports this hypothesis is the scale over which the \as{hi} is computed (\qty{30}{\km}) which gives a strong weight to \as{sst} heterogeneities associated with large contrasts which is the case across the Gulf Stream.
Of course, more direct evidence linking the penetration of fronts with the intensity of the heterogeneity index would be needed to confirm the association.

\subsubsection{Biome-scale impact of fronts on Chl-\emph{a}}

The categorization of fronts based on \as{hi} has allowed us to quantify the respective contribution of two types of fronts on the regional \as{chl} budget (Fig.~\ref{fig:recap}b).
Weak fronts lead to a local \as{chl} enhancement which is weaker than strong front, in general, but because they are also more frequent than strong fronts, depending on the biome and seasons, they contribute equally to the regional \as{chl} budget as strong fronts.
There is also some degree of seasonality in this small surplus of \as{chl} attributed to fronts, which heavily depends on the region of interest (Fig.~\ref{fig:ts-climato}).
As predicted by theory and noted by previous studies, sub-mesoscale fronts \encadra{which are confined to the mixed-layer} are less abundant in summer when mixed-layers are shallower.
In the south zone, this leads to an overall weaker effect of fronts in summer (near~\pct{0}) relative to the rest of the year (less than~\pct{3} average), but in the jet area, it is compensated by a larger intensity of the increase in \as{chl} in summer leading to a \as{chl} surplus in summer (\pct{7} for weak fronts) which is much larger than in winter (\pct{1}).
In the north, the situation is quite different with an impact of fronts close to zero during the spring bloom, negative in summer as vertical velocities at fronts are also capable of sinking the surface bloom \parencite{levy_2018}, and maximal in autumn and winter.

Besides these small spatial and temporal variations in amplitude, a key result of this study is that despite strong local impact of fronts, their overall contribution at large-scale remains small, a few percent at most, and of the order of~\pct{5} for the entire region.
Nevertheless, this result should be considered as a lower bound, first because increases in \as{chl} at fronts are often stronger at subsurface than at the surface, and second because in a region characterized by strong gradients like this one, additional nutrient fluxes due to frontal activity might not necessarily lead to local anomalies in \as{chl}, but could also be hidden by the large-scale gradient.


\subsubsection{Earlier blooms over fronts}

Another key result of this study is the detection of earlier blooms over fronts than over background conditions in the north of the Gulf Stream jet.
Several field and modeling studies have shown that frontal dynamics, by tilting existing horizontal density gradients, increase the vertical stratification of surface mixed layers \parencite{taylor_2011} and the residence time of phytoplankton in the euphotic zone, leading to early local phytoplankton blooms compared to surrounding areas.
However, while the increased stratification over fronts can be directly observed in situ \parencite{karleskind_2011, mahadevan_2012}, how it affects the timing of the bloom has so far been quantified with numerical models, due to the difficulty in tracking the bloom evolution over fronts which themselves evolve over time \parencite{levy_2000, karleskind_2011, mahadevan_2012}.
We provide here the first observational evidence of the early onset of blooms over fronts.
Moreover, our estimate leads to smaller values (earlier blooms by one to two weeks) than previously estimated from models (20--30~days by \textcite{mahadevan_2012}).
This effect alone is unlikely to affect productivity budget, but may impact phytoplankton competition at the onset of the bloom season.
The method that we used to quantify differences in bloom timing over fronts and background is based on the time evolution of an eulerian quantity, the \as{chl} median over latitudinal bands, whereas the bloom evolves along lagrangian trajectories.
Considering a rather small area, as we have done here, is a way of overcoming the difficulty of following the temporal evolution on fronts whose life history is too complex to be captured and shorter than the bloom itself.
It also limits the impact that the northward propagation of the bloom could have on the temporal assessment.
It should also be noted that it is inherently difficult to pinpoint the precise onset and end days of a bloom, as the spring bloom shows large intraseasonal variability in its characteristics; its beginning can be more or less sudden, and is often made of multiple peaks \parencite{keerthi_2020}.


\subsection{Conclusions}

The Gulf stream extension region is a region of strong biological contrasts and particularly strong frontal activity of the world's ocean, undergoing rapid warming which strongly affect fisheries \parencite{pershing_2015, neto_2021}.
Quantifying the impact of fronts on phytoplankton there is thus particularly relevant, and we expected to detect a large impact.
The use of 20~years of satellite data of \as{sst} to detect fronts and of surface \as{chl} to compute anomalies over the front allowed us to provide a robust assessment of this impact.
We found three main results.
First, that the regional increase in surface phytoplankton due to fronts much weaker than we expected, \pct{5} at most; second that nutrient supplies at fronts enhanced the spring bloom two to three three times more than they enhanced oligotrophic regions; and third, that the spring bloom onset was earlier over fronts by one to two weeks, which we already knew from models \parencite{karleskind_2011, mahadevan_2012} but for which we had no direct evidence nor sound quantification.

Although limited to the Gulf Stream region, this study provides a well-tested methodology that could enable the study of the links between small-scale ocean physics and phytoplankton response in other regions of the global ocean.
In addition, these results on the importance of fronts for phytoplankton biomass and phenology could also be used to evaluate models coupling ocean physics and phytoplankton at high spatial resolution, or to test parameterizations representing the effect of small scales on phytoplankton production in coarser resolution models.
Finally, the combination of these observation-based results with theoretical arguments and well-assessed models should also allow us to better constrain the response of phytoplankton production to climate change \parencite{couespel_2021}, which still has very large uncertainties as shown by the latest set of Earth system models \parencite{kwiatkowski_2020}.


\begin{articleSubBlock}{Code availability}
  All the scripts needed to reproduce our results, as well as the data necessary to generate the figures in this manuscript are available at~\textcite{haeck_2022_zenodo}.
\end{articleSubBlock}


% \appendixfigures{}

\begin{figure}
  \centering
  \includegraphics[width=12cm]{article/ts_sensitivity.pdf}
  \caption[Sensitivity of the seasonal impact of fronts on \glsentryshort{hi} parameters]{%
    Climatological mean of \as{chl} median values (top row) over weak fronts (blue), strong fronts (green) and background (red), surface fraction occupied by weak fronts and strong fronts (middle row), and global \as{chl} excess due to weak and strong fronts (bottom row).
    Each line represent a set of parameter with the bolder line indicating the retained set of parameters.
    The tested rolling window sizes are~\qtylist{20;30;40}{\km}.
    Different normalization coefficients are tested for a~\qty{30}{\km} window size: double the variance, double the bimodality, and double the skewness.
  }%
  \label{fig:ts-sensitivity}
\end{figure}


\begin{articleSubBlock}{Author contributions}
  ML, LB and CH conceived the study. CH conceived the methodology and performed the analysis. ML and CH wrote the paper. All authors contributed to the analysis and discussion of the results.
\end{articleSubBlock}


\begin{articleSubBlock}{Competing interests}
  The authors declare that they have no conflict of interest.
\end{articleSubBlock}

\begin{articleSubBlock}{Acknowledgements}
  CH benefited from a PhD scholarship by ENS.\@ The project was supported by TOSCA CNES and by the ENS CHANEL chair.
  We thank Daniele Iudicone, Francesco d'Ovidio, Sakina-Dorothée Ayata, and Amala Mahadevan for the useful discussions which helped to refine our methodology. We thank Xioa Liu for helping us reproduce their work.

  This study has been conducted using E.U. Copernicus Marine Service Information \parencite[datasets used:][]{article_chl, article_sst}.

  GlobColour data (\url{https://globcolour.info}) used in this study has been developed, validated, and distributed by ACRI-ST, France.
\end{articleSubBlock}

% \bibliography{resources/references}
