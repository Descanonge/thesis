% chktex-file 13

\chapter{Conclusion}
\label{chp:conclusion}

Dans le cadre de cette thèse, grâce à 20~années de données satellites nous avons montré et \emph{quantifié} l'impact des fronts sur la \al{chl} dans l'Atlantique Nord autour du Gulf Stream, selon la saison et la biorégion concernée.
Nous réaffirmons l'augmentation \emph{locale} de la \al{chl} dans les fronts par rapport à l'environnement avoisinant.
Cet excès local est d'autant plus élevé que \emph{l'intensité des fronts} est importante.
Il est particulièrement important immédiatement autour du Gulf Stream (\pct{+60} en moyenne), ainsi que pendant le bloom printanier (où il atteint \pct{+150}).
En revanche, en calculant le surplus de \al{chl} à \emph{l'échelle de la biorégion} \encadra{en donc en comptabilisant la \emph{surface occupée} par les fronts} cette augmentation est moins spectaculaire (inférieure à \pct{+5}).
Ce surplus est de valeur comparable pour les fronts de fortes et de faible intensité, ces derniers ayant un impact local moindre mais compensent par une plus importante surface occupée.
Dans le biome subpolaire, nous apportons des preuves observationnelles que le \emph{démarrage du bloom} se fait \emph{plus tôt dans les fronts} que dans le reste de la zone \encadra*{de deux semaines environ}.
Ce résultat constitue une première observation directe et quantification de ce phénomène.
Ces observations de la réponse du phytoplancton aux fronts de fine échelle participent à mieux cerner leur impact global sur les cycles biogéochimiques, et ainsi améliorer nos prédictions face au changement climatique.

Au delà de ces résultats, nous incrémentons et validons une méthode de détection des fronts dans une zone complexe.
En particulier, nous montrons son adéquation à quantifier l'intensité des fronts détectés.
En outre, nous reconnaissons que malgré l'importance et les applications potentielles de la détection de fronts, il n'existe pas encore d'outils ou de produits facilitant leur usage scientifique.
Sans pour autant fournir une solution définitive à cette lacune, ce travail souligne des objectifs à atteindre et fournit un exemple d'outil facilement réutilisable.
Nous avons également pu identifier un certains nombre de difficultés se dressant devant l'établissement de tels outils ou produits.
De multiples algorithmes de détection de fronts existent déjà, cependant leur usage n'est pas forcément trivial.
En cause est leur distribution dans des canaux et des languages inadaptés, ou manquant de documentation, et dont on ne peut que souhaiter qu'elle se rapproche des standards et outils aujourd'hui en place dans le milieu de l'open-source.
Ces manquements incombent un travail supplémentaire à leur application, pourtant déjà complexe du fait, entre autres, de la nécessité de s'adapter aux caractéristiques régionales (que nous montrons ici), et du large volume de données nécessaire aux évaluations de l'impact des fines échelles.

Si nous avons souligné les difficultés importantes dans la mise à disposition d'outils et produits de détection de fronts, nous notons également leur fort potentiel, et même leur nécessité à la quantification et la compréhension de phénomènes biogéochimiques importants.

\fancybreak[3\onelineskip]
