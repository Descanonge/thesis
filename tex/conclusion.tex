\documentclass[index]{subfiles}
\begin{document}

\chapter{Perspectives}
\label{chp:perspectives}

\tocsubfile%

\section{Applications à d'autres régions}
\label{sec:appl-autres-regions}

\subsection{À l'échelle globale}
\label{sec:global}

\subsection{Dans le courant de Californie}
\label{sec:CCE}

Voir \cref{M-ax:article-cce}.

\section{Étendre à la composition du phytoplankton}
\label{sec:persp-pft}

Est-ce que la somme des PFT en sortie de l'algo de Roy est contrainte à être égale à 1?

Motivation: mieux comprendre les réponses spécifiques de différents types de phyto aux fronts.
Il est connu que certains types répondent plus favorablement aux inputs de nutriments locaux et ponctuels (les diatomées, quelle taille?).

On essaye de vérifier ça.

\section{Article review}
Voir \cref{M-ax:article-review}.

\chapter{Conclusion}
\label{chp:conclusion}
% plan à préciser

\end{document}
