% chktex-file 13

\chapter{Conclusion}
\label{chp:conclusion}

Dans le cadre de cette thèse, grâce à 20~années de données satellites nous avons montré et \emph{quantifié} l'impact des fronts sur la \al{chl} dans l'Atlantique Nord autour du Gulf Stream, selon la saison et la biorégion concernée.
Nous réaffirmons l'augmentation \emph{locale} de la \al{chl} dans les fronts par rapport à l'environnement avoisinant.
Cet excès local est d'autant plus élevé que \emph{l'intensité des fronts} est importante.
Il est particulièrement important immédiatement autour du Gulf Stream (\pct{+60} en moyenne), ainsi que pendant le bloom printanier (où il atteint \pct{+150}).

En revanche, en calculer le surplus de \al{chl} à \emph{l'échelle de la biorégion} \encadra{en donc en comptabilisant la \emph{surface occupée} par les fronts} cette est augmentation est moins spectaculaire (inférieur à \pct{+5}).
Ce surplus est de valeur comparable pour les fronts de fortes et de faible intensité, ces derniers ayant un impact local moindre mais compensent par une plus importante surface occupée.

Dans le biome subpolaire, nous apportons des preuves observationnelles que le démarrage du bloom se fait plus tôt dans les fronts que dans le reste de la zone \encadra*{de deux semaines environ}.
Ce résultat constitue une première observation directe et quantification de ce phénomène.

peut être utilisé pour une paramétrisation dans un OGCM/ESM et des projections climatiques.

Au delà de ces résultats, nous fournissons une méthode de calcul des fronts validée.
Nous reconnaissons que malgré l'importance et les potentielles applications de la détection des fronts, il n'existe pas encore d'outils ou de produits facilitant leur utilisation scientifique.
Sans fournir y une solution définitive nous avons néanmoins pu mieux en délimiter la forme.
À la fois en termes des algorithmes potentiellement utile et de leur validation régionale, ainsi que des outils qui permettrait leur implémantation, et leur application sur des données satellites, diverses et parfois de large volume.

\fancybreakdisplay
