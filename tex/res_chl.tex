
\chapterlof{Impact des fronts sur le budget de Chlorophylle-\textit{a}}
\label{chp:res-chl}
\graphicspath{{resources/res_chl}}

\minitoc%
\clearpage

préambule

motivation.
Résultat attendu sur la biomasse, la composition.
Permet de valider la méthode aussi.

\section{Résumé de l'article}
\label{sec:resume-article}

\insertArticle{}

\section{Compléments}
\label{sec:complements-chl}

\subsection{Examples de fronts}
\label{sec:examples-fronts}

Trouver des images examples de fronts.
Comparer visuellement avec les données MODIS\@.

Il est difficile de trouver des examples corrects.
D'une part à cause de la grande couverture nuageuse.
D'autre part parce que l'effet sur la Chl est difficilement visible sur des images. (effet statistique).

\subsection{Vérification des histogrammes}

\subsection{Durée du bloom}
\label{sec:duree-bloom}

Extension du résultat présenté~\cref{fig:bloom}.

\begin{figure}
  \centering
  \includegraphics[width=\textwidth]{durée_bloom.pdf}
  \captionT{Durée du bloom}{%
    à écrire
  }
  \label{fig:duree-bloom}
\end{figure}

\subsection{Sensibilité aux paramètres}
\label{sec:sensibilite-parametres}

meilleure explication des paramètres qu'on teste

plots individuels (histogramme sur 20 ans ?)

\begin{figure}
  \centering
  \includegraphics[width=\textwidth]{sensibilité_hist.pdf}
  \captionT{Sensibilité de la distribution de Chl-\emph{a} aux paramètres}{%
    Distribution de la \as{chl} sur les 20~années de données, pour les trois régions et un ensemble de paramètres.
    (a-b-c):~trois tailles de fênetre glissante pour le calcul du \as{hi}, et (d-e-f):~quatres jeux de coefficients de normalisations qui soit, donnent un poids statistique équilibré aux trois composantes, soit donnent 2~fois plus de poids à une des composantes.

    Les distributions dans le \eng{background}~(trait pointillé) ne sont quasiment pas affectées. Les distributions dans les fronts~(\(\am{hi}>5\), trait plein) varient légèrement selon les paramètres.
    Cela se reflète dans les valeurs médianes des distributions~(indiquées par des traits verticaux) dont l'écart"-type associé à la variation d'un paramètre est typiquement dix fois plus élevé pour les fronts~(valeur indiquée dans chacun des sous"-graphes) que le \eng{background}.

    L'écart de la valeur médiane de \as{chl} dû à la variation d'un paramètre du \as{hi} ne dépasse pas~\qty{5}{\mugm}.
  }
  \label{fig:sensibilite-hist}
\end{figure}

% chktex-file 2
\begin{table}
  \centering
  \begin{siunitText}
  \begin{tabular}{$r ^c *{3}{^c}} \toprule
    \multirow{2}*{Variation} & \(K^1\) & \(K^2\) & \(K^3\) & \(K^4\) \\
     & (écart-type) & (asymétrie) & (bimodalité) & (HI) \\
    \midrule
    \emph{Taille fenêtre} & \\
    \qty{20}{\km} & \num{5.1768} & \num{3.1563} & \num{4.1306} & \num{1.3890} \\
    \rowstyle{\bfseries}
    \qty{30}{\km} & \num{3.9401} & \num{2.7200} & \num{4.2917} & \num{1.3418} \\
    \qty{40}{\km} & \num{3.2692} & \num{2.5031} & \num{4.3444} & \num{1.3129} \\
    \midrule
    \emph{Poids composantes} & \\
    plus d'écart-type & \(\times 2\) & --- & --- & \num{1.0239} \\
    plus d'asymétrie & --- & \(\times 2\) & --- & \num{0.9912} \\
    plus de bimodalité & --- & --- & \(\times 2\) & \num{0.9773} \\
    \bottomrule
  \end{tabular}
  \end{siunitText}
  \caption{%
    Coefficients de normalisation pour les différents paramètres utilisés.
  }
  \label{tab:coefs}
\end{table}

ref annexe (\cref{fig:ts-sensitivity})
