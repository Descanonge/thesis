% chktex-file 13

\chapter{Impact des fronts sur la quantité de Chlorophylle-\textit{a}}
\addChpLof
\label{chp:res-chl}
\graphicspath{{resources/res_chl}}

{
  \hypersetup{hidelinks}
  \minitoc%
  % \clearpage
}

% \review{%
%   préambule

%   motivation.
%   Résultat attendu sur la biomasse, la composition.
%   Permet de valider la méthode aussi.
% }

\section{Résumé de l'article}
\label{sec:resume-article}

Les fronts affectent la croissance du phytoplancton en augmentant localement les apports en nutriments.
Des hausses de la biomasse au niveau des fronts ont été observées et associées à cette remontée locale de nutriments et/ou au transport latéral.
D'autre part il a été démontré dans des modèles numériques que la réduction de la stratification au-dessus des fronts induisait des blooms plus précoces, affectant donc la phénologie du phytoplancton.

L'imagerie satellitaire permet de quantifier ces impacts sur phytoplancton pour un grand nombre de fronts à des échelles synoptiques.
Ici, nous avons utilisé vingt ans de données satellitaires de température de surface (SST) et de \al{chl} (\as{chl}) dans une vaste région entourant le Gulf Stream afin de quantifier l'impact des fronts sur la \as{chl} de surface (utilisée comme indicateur de biomasse du phytoplancton) dans trois biomes contrastés, allant de la gyre subtropicale oligotrophe à une biorégion subpolaire présentant un bloom.

Nous avons calculé un indice d'hétérogénéité (HI) à partir de la SST pour détecter les fronts, et l'avons utilisé pour discriminer fronts faibles et fronts forts sur la base de seuils de HI.
Nous avons observé que l'emplacement des fronts forts correspondait aux fronts permanents associés au courant de bord ouest, et les fronts faibles aux fronts de submésoéchelle, plus éphémères.
Nous avons comparé les distributions de \as{chl} tout au long de l'année, dans les fronts forts, dans les fronts faibles, et à l'extérieur des fronts dans les trois biomes.
Nous avons évalué trois paramètres: l'excès de \as{chl} dans les fronts à l'échelle locale, le surplus de \as{chl} induit à l'échelle du biome, et le retard du démarrage du bloom dans les fronts.

Nous avons constaté que les fronts faibles sont associés à un excès local de \as{chl} plus faible que les fronts forts, mais du fait de leur plus grande ubiquité, ils contribuent de façon semblable au surplus régional de \as{chl}.
Nous avons également constaté que l'excès local de \as{chl} était deux à trois fois plus important dans biome présentant un bloom printanier que dans la région oligotrophe, ce qui peut s'expliquer en partie par le transport de nutriments par le Gulf Stream.
Nous avons trouvé de fortes variations saisonnières dans l'amplitude de l'excès de \as{chl} dans les fronts, ainsi que des périodes de déficit de \as{chl} dans les fronts au nord de \latlon{45N} que nous attribuons à la subduction.

Enfin, nous fournissons des indications par l'observation que les blooms printaniers commencent plus tôt des fronts, d'une à deux semaines.
Nos résultats suggèrent que l'impact spectaculaire des fronts à l'échelle locale (jusqu'à \pct{+60}) est plus limité lorsqu'il est considéré à l'échelle régionale des biomes (moins de \pct{+5}), mais qu'il peut néanmoins avoir des implications pour l'ensemble de l'écosystème de la région.

\insertArticle{}

\section{Résultats complémentaires}
\label{sec:complements-chl}
\suppressfloats[t]

\subsection{Vérification des distributions}
\label{sec:verif-hist}

Un certain nombre de diagnostiques présentés dans ce manuscrit reposent sur les valeurs médianes de la \as{chl}.
Cet indicateur permet de représenter la distribution d'une variable en une valeur unique, néanmoins pour que cette représentation soit pertinente il est nécessaire de vérifier que la distribution de laquelle elle est extraite soit adaptée.
Les distributions de valeurs de la \as{chl} s'avèrent être invariablement unimodales, avec des queues de valeurs élevées plus ou moins prononcées (\cref{fig:hist-chl,fig:verif-hist}).
Il semble donc pertinent de qualifier ces distributions par leur valeur médiane.

Les distributions de SST sont également unimodale, à l'exception du biome subpolaire, qui présente des distributions largement bimodales (\nref{fig:verif-hist}).
Une séparation en bande de latitude de~\ang{5} de large (\nref{fig:hists-sst-latbands}) ne suffit pas; les valeurs de SST étant encore distribuées de manière bimodale entre décembre et mai dans les bandes de latitude \latlonRange{40}{45N}.
Les résultats portant sur les valeurs de SST dans les fronts, présentés \nref[section]{sec:res-sst}, omettent donc le biome subpolaire par soucis de simplicité.

\begin{figure}
  \centering
  \insertfig{verif_hist.pdf}
  \captionT{Distributions saisonnières de Chl-\text{a} et SST dans, et hors des fronts}{%
    Les distributions de valeurs sont indiquées pour l'année~2007 par tranches de trois mois~(lignes), par zones~(colonnes), et pour les \eng{background}~(rouge), fronts faibles~(bleu), et fronts forts~(vert).

    La \as{chl}~(à gauche) présente une distribution unimodale avec une queue de valeurs élevées plus ou moins prononcée.
    La SST~(à droite) en revanche affiche des distributions largement bimodales, comme dans le biome subpolaire notamment.
  }
  \label{fig:verif-hist}
\end{figure}

\begin{figure}
  \centering
  \insertfig{hists_sst_latbands.pdf}
  \captionT{Histogrammes de SST par bande de latitude dans le biome subpolaire}{%
    Les distributions de valeurs sont indiquées pour l'année~2007 par tranches de trois mois~(colonnes), par bande de latitude~(lignes), et pour les \eng{background}~(rouge), fronts faibles~(bleu), et fronts forts~(vert).
  }
  \label{fig:hists-sst-latbands}
\end{figure}

\subsection{Sensibilité aux paramètres}
\label{sec:sensibilite-parametres}

L'implémentation de la détection des fronts avec l'\af{hi} nécessite de faire le choix d'un certain nombre de paramètres. Nous testons ici la variabilité des résultats finaux (\ab{ie} l'impact des fronts sur la \as{chl}) vis-à-vis de deux paramètres.

D'une part, nous faisons varier la taille de la fenêtre glissante utilisée pour calculer les composantes du HI (\nref{sec:calcul-composantes}).
Nous testons trois tailles: \qty{20}{\km},~\qty{30}{\km} et~\qty{40}{\km}. La taille retenue est de~\qty{30}{\km}.
Les coefficients de normalisations sont à recalculer pour chacune des tailles (comme décrit \nref{sec:coef-normalisation}, \nref{chp:methodes}), et sont listés dans la \nref[table]{tab:coefs}.

D'autre part nous testons l'influence de ces coefficients. En prenant les composantes calculées avec une fenêtre de~\qty{30}{\km}, tour à tour nous doublons le coefficient d'une des composantes (et donc son poids statistique dans le HI). Le coefficient~\(K^4\) pour le HI doit être recalculé~(\nref{tab:coefs}).

% chktex-file 2
\begin{table}
  \centering
  \caption[]{%
    Coefficients de normalisation pour les différents paramètres utilisés.
  }
  \label{tab:coefs}
  \begin{tabular}{$r<{\hspace{1em}} ^c *{3}{^c}} \toprule
    \multirow{2}*{Variation} & \(K^1\)      & \(K^2\)     & \(K^3\)      & \(K^4\) \\
                             & (écart-type) & (asymétrie) & (bimodalité) & (HI)    \\
    \midrule
    \emph{Taille fenêtre} & \\
    \qty{20}{\km} & \num{5.1768} & \num{3.1563} & \num{4.1306} & \num{1.3890} \\
    \rowstyle{\bfseries}
    \qty{30}{\km} & \num{3.9401} & \num{2.7200} & \num{4.2917} & \num{1.3418} \\
    \qty{40}{\km} & \num{3.2692} & \num{2.5031} & \num{4.3444} & \num{1.3129} \\
    \midrule
    \emph{Poids composantes} & \\
    plus d'écart-type  & \(\times 2\) & ---          & ---          & \num{1.0239} \\
    plus d'asymétrie   & ---          & \(\times 2\) & ---          & \num{0.9912} \\
    plus de bimodalité & ---          & ---          & \(\times 2\) & \num{0.9773} \\
    \bottomrule
  \end{tabular}
\end{table}

Les résultats sont très peu sensibles aux choix des paramètres.
Les évolutions climatologiques des valeurs médianes et du surplus sont très peu affectés. La fraction de surface impactée par les fronts est plus sensible mais néanmoins les tendances restent identiques (\nref{fig:ts-sensitivity}).
Pour quantifier plus en détail cette variation, nous calculons l'écart-type des valeurs médiane de \as{chl} sur l'ensemble des paramètres (les trois tailles de fenêtres et trois configurations de coefficients), semaine par semaine sur les 20~années de données, puis en prenons la moyenne temporelle.
Il en ressort les mêmes résultats que sur la climatologie, avec une variabilité élevée dans les fronts forts (entre~\qty{4.4}{\mugm} et~\qty{11.4}{\mugm}), moyenne dans les fronts faibles (entre~\qty{2.0}{\mugm} et~\qty{3.3}{\mugm}), et plus faible dans le \eng{background} (entre~\qty{0.07}{\mugm} et~\qty{3.8}{\mugm}) (\nref{tab:sensibilite-mediane}).
% \review{Rajouter discussion des valeurs relatives.}

\begin{table}
  \centering
  \caption[]{%
    Sensibilité aux paramètres: écart-type de la valeur médiane de \as{chl} calculé sur l'ensemble des paramètres~\(p\) testés (trois tailles de fenêtres, et trois configuration de coefficients de normalisation); en valeur absolue~(\(\std[\am{chl}_p]\), en~\unit{\mugm}) et relative~(%
    \(\std[(\am{chl}_p - \moy{\am{chl}}_p) / \moy{\am{chl}}_p]\), en~\%).
  }
  \label{tab:sensibilite-mediane}
  \begingroup
  \newcommand*\typeunits[1]{\multicolumn{1}{c}{\small\textit{#1}}}
  \newcolumntype{y}} \\
    background      & 0.05   & 0.11  & 0.25   & 0.25  & 4.89   & 1.84  \\
    fronts faibles  & 1.50   & 2.87  & 3.52   & 3.36  & 3.21   & 1.02  \\
    fronts forts    & 5.92   & 9.64  & 12.48  & 8.05  & 4.96   & 1.37  \\
    \bottomrule
  \end{tabular}
  \endgroup
\end{table}

Dans cette étude de sensibilité nous n'avons pas considéré l'impact du seuil en HI choisi, ayant déjà couvert l'évolution de la \as{chl} avec le HI (\nref{fig:chl-vs-hi}).
Il en resortait que les valeurs de \as{chl} tendent à augmenter avec celles du HI, et ce sur toute l'amplitude du HI.
Le choix du seuil importe donc peu vis-à-vis des tendances observées.
Néanmoins il est possible que la variation de la \as{chl} avec les paramètres du HI soit plus le fait du seuil choisi en rapport avec la distribution de HI \encadra{les seuils restant fixés à 5 et 10 quelque soit les paramètres} que d'un changement dans l'emplacement des fronts détectés.
Autrement dit, il est possible que la variation en \as{chl} (dans les fronts) soit autant dûe à la \emph{quantité} de fronts détectés qu'à leur \guil*{\emph{qualité}}.

Pour étudier cet éventuel lien entre la surface des fronts et les valeurs de \as{chl} dans ces fronts, nous prenons les valeurs climatologiques mensuelles de la fraction de surface impactée par les fronts faibles, et de la valeur médiane de \as{chl} dans ces même fronts, pour tous les paramètres testés (et les trois régions). Nous remarquons ainsi qu'en faisant varier les paramètres, la \as{chl} diminue lorsque la surface augmente.
À cela s'ajoutent des variations dues à la région et à la saisonnalité.
Nous éliminons la dépendance de la \as{chl} et de la surface en temps et en région, en leur soustrayant leur valeur moyennée sur l'ensemble des paramètres testés.
Pour chaque mois dans chaque région, nous obtenons autant de valeurs de \as{chl} autour de \num{0} que de paramètres: \(\Delta\am{chl}_p(\text{région, mois}) = \am{chl}_p(\text{région, mois}) - \moy{\am{chl}_p(\text{région, mois})}_p\) (\emph{idem} pour la surface).
Nous obtenons ainsi une relation (relative) entre la fraction de surface de fronts détectés et la médiane de \as{chl} dans ces fronts (\nref{fig:sensibilite-surface}).

\begin{figure}
  \centering
  \insertfig{sensibilité_surface.pdf}
  \captionT{Corrélation de l'augmentation de Chl-\textit{a} avec la surface couverte par les fronts pour différents choix de paramètres dans le calcul du HI}{%
    Les valeurs médianes de \as{chl} dans les fronts (en climatologie mensuelle) sont négativement corrélées à la surface occupée par les fronts.
    On soustrait à ces variables leur moyenne sur l'ensemble des configurations testées (taille de fenêtre et coefficients de normalisation) afin passer outre leur dépendances en temps et région. Chaque point correspond à un mois, une biorégion et un choix de paramètres.
  }
  \label{fig:sensibilite-surface}
\end{figure}

La médiane de \as{chl} est bien négativement corrélée à la fraction de surface (le coefficient de corrélation de Pearson entre les deux variables est de~\num{-0.73}), ce qui peut se comprendre.
Les fronts entraînent la présence d'un pic (local donc) de \as{chl}.
Une variation dans les paramètres du HI génère des valeurs plus élevée pour un seuil fixe, donc augmente la surface de fronts détectés, et ce-faisant \guil*{dilue} les valeurs du pic de \as{chl}.

Cet effet est consistent pour les différents biomes et types de fronts, à l'exception des fronts forts dans le biome subpolaire, pour lesquels on observe la tendance inverse (\ab{ie} une augmentation de la \as{chl} avec la surface).

\begin{note}
  Cette expérience est très similaire à l'exploration de la dépendance des valeurs de \as{chl} à celle du HI.
  Cependant il ne s'agit pas ici d'une simple variation du seuil, qui bien entendu aurait une influence sur la surface des fronts détectée et sur la valeur de \as{chl} à l'intérieur, ce qui est déjà documenté \nref[figure]{fig:chl-vs-hi}.
  Il est question ici de regarder le lien entre surface et \as{chl} pour \emph{différentes version du HI}, obtenues en faisant varier les paramètres du HI.
\end{note}

\subsection{Impact sur la SST}
\label{sec:res-sst}

Nous nous penchons ici sur l'impact local des fronts sur la SST.
Le biome subpolaire n'est pas analysé car les distributions de SST n'y sont pas unimodales.
Les nuages ne sont pas pris en compte ici, seules les données \abbrv{L4} (interpolées) sont utilisées.
Dans le biome suptropical permanent, les fronts n'ont pas d'impact significatif sur la valeur locale de SST (\nref{fig:ts-cli-sst}).
Une différence entre fronts et background est notable pour la valeur médiane sur tout le biome, mais qui est due au gradient en température et à la répartition des fronts à grande échelle.

\begin{figure}
  \centering
  \insertfig{ts_cli_sst.pdf}
  \captionT{Climatologie de la température dans, et hors des fronts}{%
    Valeurs médianes de la températures dans les fronts faibles (bleu), les fronts forts (vert), et le background (rouge), pour les biomes suptropicaux permanent (à gauche) et saisonnier (à droite).
    Les valeurs sont calculés pour le biome entier (a et b), puis par bandes de latitude dans chacun des biomes (c-h).
    Les lignes pleines représentent la moyenne climatologique, et les enveloppes la déviation standard sur la période 2000--2020.
  }
  \label{fig:ts-cli-sst}
\end{figure}

Dans le biome suptropical saisonnier, les températures sont légèrement plus chaudes dans les fronts que le background.
Cette différence est à son maximum au printemps et atteint alors \tC{1.8} dans la bande \latlonRange{32}{35N} pour les fronts forts, et \tC{1.0} dans la bande \latlonRange{35}{40N} pour les fronts faibles.
On notera que c'est également autour de ces latitudes que l'impact des fronts faibles sur la \as{chl} est maximum (\nref{fig:recap}).

Ce résultat est contraire à ce qui est observé par \textcite{liu_2016} dans la gyre subtropicale du Pacifique Nord, à savoir une température plus faible dans les fronts, à son maximum au printemps de \tC{-1.7}.
Cependant, le biais d'un gradient de grande échelle de la température et de la répartition des fronts n'y est pas discuté et pourrait être à l'origine des valeurs observées.
