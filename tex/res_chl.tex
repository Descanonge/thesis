\documentclass[index]{subfiles}
\begin{document}

\chapter{Impact des fronts sur Chlorophylle}
\label{chp:res-chl}

\section{Introduction}

motivation.

Résultat attendu sur la biomasse.
Permet de valider la méthode aussi.


\section{Éxamples de fronts}

Trouver des images examples de fronts.
Comparer visuellement avec les données MODIS.

Histogrammes individuels ?

Il est difficile de trouver des examples corrects.
D'une part à cause de la grande couverture nuageuse.
D'autre part parce que l'effet sur la Chl est difficilement visible sur des images. (effet statistique).

\section{Relation Chl vs HI}

Augmentation de la \gls{chla} avec le \gls{hi}.
Plus précisement: déplacement des valeurs vers le haut.
Valeurs les plus hautes en \gls{chla} n'apparaissent que pour des \gls{hi} forts.

\section{Estimation de l'augmentation du Chl}

figure de la climatologie
augmentation de la médiane
surplus

découpage par zones, saisonalité

\section{Timing du bloom}

Ajouter courbes de détection avec explications.
Départ plus tôt des quelques jours.
Durée plus longue de quelques jours.

\section{Discussion}

Reprendre discussion de l'article essentiellement.

\biblio
\end{document}
