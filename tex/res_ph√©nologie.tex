% chktex-file 13

\chapter{Impact des fronts sur la phénologie}
\addChpLof
\label{chp:res-phenologie}
\graphicspath{{resources/res_phénologie}}

{
  \hypersetup{hidelinks}
  \minitoc%
  \clearpage
}

Au delà de provoquer des pics de productivité en remontant des nutriments des eaux profondes, les fronts tendent à restratifier la couche de surface du fait des instabilités de couche de mélange (\nref{sec:modif-phenologie}).

\section{Résumé des résultats présentés dans l'article}
\label{sec:resume-res-phenologie}

Nous regardons donc la phénologie du biome subpolaire à travers les médianes de \al{chl} par bande de latitude (\nref{fig:latbands-n}).
Le bloom printanier se propage à travers le biome, du sud vers le nord.
Ainsi il s'amorce début avril autour de \latlon{35N} contre fin juin vers \latlon{55N}.

Pour chacune des bande latitudinale, nous observons que le démarrage du bloom survient une semaine plus tôt dans les fronts faibles que dans le background (précisement \qty{6.4 \pm 1.1}{\jours} plus tôt), et de deux semaines plus tôt dans les fronts forts (\qty{13.5 \pm 1.5}{\jours} plus tôt) (\nref{fig:bloom}).

Les dates de démarrage du bloom sur les 20~années de données sont très dispersés, en raison de la nature très intermittente du début du bloom (\cite{keerthi_2021}), ce qui la rend difficile à détecter avec précision pour certaines années.
En outre, pour de nombreuses années, aucune différence dans le démarrage (ou la durée) du bloom n'a pu être détectée entre les fronts et le background (points alignés sur la diagonale \nref{fig:bloom}).
Néanmoins, des retards de plus d'un mois peuvent survenir.

\section{Résultats complémentaires}
\label{sec:complements-phenologie}

\subsection{Durée du bloom}
\label{sec:duree-bloom}

Nous avons également mesuré la durée du bloom printanier dans les fronts et dans le background (\nref{fig:duree-bloom}).
Statistiquement, le bloom dure plus longtemps d'une à deux semaines dans les fronts (de \qty{6.6 \pm 1.4}{\jours} dans les fronts faibles et \qty{11.7 \pm 2.2}{\jours} dans les fronts forts).
Néanmoins le résultat est moins clair que pour la date de départ.
En effet, si dans la bande \latlonRange{35}{40N} 14~blooms sur 21 (\pct{67}) durent plus longtemps dans les fronts (faibles) que dans le background; dans la bande \latlonRange{50}{55N} ce nombre tombe à seulement 8~blooms (\pct{38}).

\begin{figure}
  \centering
  \insertfig[0.75]{durée_bloom.pdf}
  \captionT{Durée du bloom}{%
    Comparaison de la durée du bloom dans le background (abscisses) et dans les fronts (ordonnées), par intensité des fronts (symboles) et bande de latitude (couleur). La droite \(y=x\) est tracée en noir.
    La distance entre cette dernière et la droite grise pointillée (respectivement tiretée) mesure la différence moyenne entre les fronts faibles (respectivement forts) et le background.
  }
  \label{fig:duree-bloom}
\end{figure}

Une augmentation de la durée du bloom dans les fronts peut s'expliquer par l'apport supplémentaire en nutriments par les fronts (\cite{simoes-sousa_2022}).

\subsection{Inversion du flux de chaleur}
\label{sec:flux-chaleur}

Dans une première tentative d'observation du démarrage du bloom, nous avons tenté d'utiliser l'inversion du flux de chaleur en surface comme référence.
\Textcite{ferrari_2015} avait déjà montré que le démarrage du bloom dans l'Atlantique Nord \encadra{repéré à partir de la \al{chl} satellite (données \as{modis})} coïncide avec l'inversion du flux de chaleur (de négatif vers positif).
Le flux de chaleur, comme pour \textcite{ferrari_2015}, est extrait de la réanalyse \as{ecmwf} \abbrv{ERA-Interim}.
Similairement à cette étude, nous cherchions à regarder la distribution de la dérivée de \as{chl} par rapport au jour d'inversion du flux de chaleur, et ensuite voir si il y avait une différence dans ces distribution dans les fronts et dans le background.
Afin d'éviter des erreurs dues à la propagation du bloom, nous réalisions ces diagnostiques dans des boîtes de \qtyproduct{5x5}{\degree}.
Les résultats obtenus ne présentaient pas de synchronicité significative entre l'inversion du flux de chaleur et la dérivé de \as{chl} (\nref{fig:inv-hf}).

\begin{figure}
  \centering
  \insertfig{inversion_boxes.png}
  \captionT{Démarrage du bloom à l'inversion du flux de chaleur}{%
    La médiane de \as{chl} (en haut) et sa dérivée (en bas), à partir du jour d'inversion du flux de chaleur, pour deux boîtes à \latlonRange{50}{55N}, \latlonRange{52}{47W} (à gauche) et \latlonRange{50}{55N}, \latlonRange{47}{42W} (à droite).
    Chaque ligne de couleur représente une année (entre 2006 et 2014), et la ligne épaisse noire la moyenne de toutes ces années.
  }
  \label{fig:inv-hf}
\end{figure}

Dans la méthode finalement retenue, nous nous sommes concentrés sur la différence entre front et background, plutôt que de chercher un point de référence supplémentaire.
Nous avons utilisé des bandes de latitudes plutôt que des boîtes. Comme nous travaillons sur des données journalières il est possible de manquer de données sur une boîte ce qui affecte négativement les séries temporelles de \as{chl}.
Enfin, nous avons également limité cette étude au biome subpolaire.

Dans l'ensemble, obtenir ces résultats a nécessité de l'exploration, notamment dans le découpage des données.
Nous réitérons ici notre remarque sur la difficulté à observer un bloom se propageant de manière lagrangienne, à partir de quantités eulériennes.
