% chktex-file 13

\chapter{Impact des fronts sur la phénologie}
\addChpLof
\label{chp:res-phenologie}
\graphicspath{{resources/res_phénologie}}

\minitoc%
\clearpage

Au delà de provoquer des pics de productivité en remontant des nutriments des eaux profondes, les fronts tendent à restratifier la couche de surface du fait des instabilités de couche de mélange (\nref{sec:modif-phenologie}).

\section{Résumé de l'article}
\label{sec:resume-res-phenologie}

Nous regardons donc la phénologie du biome subpolaire à travers les médianes de \al{chl} par bande de latitude (\nref{fig:latbands-n}).
Le bloom printanier se propage à travers le biome, du sud vers le nord.
Ainsi il s'amorce début avril autour de \latlon{35N} contre fin juin vers \latlon{55N}.

Pour chacune des bande latitudinale, nous observons que le démarrage du bloom survient une semaine plus tôt dans les fronts faibles que dans le background (précisement \qty{6.4 \pm 1.1}{\jours} plus tôt), et de deux semaines plus tôt dans les fronts forts (\qty{13.5 \pm 1.5}{\jours} plus tôt) (\nref{fig:bloom}).

Les dates de démarrage du bloom sur les 20~années de données sont très dispersés, en raison de la nature très intermittente du début du bloom (\cite{keerthi_2021}), ce qui la rend difficile à détecter avec précision pour certaines années.
En outre, pour de nombreuses années, aucune différence dans le démarrage (ou la durée) du bloom n'a pu être détectée entre les fronts et le background (points alignés sur la diagonale \nref{fig:bloom}).
Néanmoins, des retards de plus d'un mois peuvent survenir.

\section{Compléments}
\label{sec:complements-phenologie}

\subsection{Durée du bloom}
\label{sec:duree-bloom}

Nous avons également mesuré la durée du bloom printanier dans les fronts et dans le background (\nref{fig:duree-bloom}).
Statistiquement, le bloom dure plus longtemps d'une à deux semaines dans les fronts (de \qty{6.6 \pm 1.4}{\jours} dans les fronts faibles et \qty{11.7}{2.2}{\jours} dans les fronts forts).
Néanmoins le résultat est moins clair que pour la date de départ.
En effet, si dans la bande \latlonRange{35}{40N} 14~blooms sur 21 (\pct{67}) durent plus longtemps dans les fronts (faibles) que dans le background; dans la bande \latlonRange{50}{55N} seuls 8~blooms durent plus longtemps dans les fronts (\pct{38}).

\begin{figure}
  \centering
  \insertfig[0.75]{durée_bloom.pdf}
  \captionT{Durée du bloom}{%
    \review{TODO}
  }
  \label{fig:duree-bloom}
\end{figure}

Une augmentation de la durée du bloom dans les fronts peut s'expliquer par l'apport supplémentaire en nutriments par les fronts (\review{ref}).

\subsection{Inversion du flux de chaleur}
\label{sec:flux-chaleur}
