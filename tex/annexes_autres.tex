

\chapter{Variables drapeaux sur masque binaire}
\label{ax:cf-flags}

Dans cette section, je décris succinctement le principe de stockage des variables drapeau (\engquote{flags}), tel que défini par les conventions \as{cf}\footnote{%
  \raggedleft%
  section 3.5: {\scriptsize\glsurl{cf-flags}}
}.
Ayant besoin d'extraire des flags des données de Chlorophylle, et voyant qu'aucune implémentation n'était disponible pour \citesoft{xarray}, j'ai écrit une simple implémentation.
Je l'ai ensuite adaptée et soumise au projet \citesoft{cf-xarray} pour qu'elle puisse être réutilisée.

Il est souvent utile voire nécessaire d'accompagner certaines quantités par des informations supplémentaires: pour des données satellites par exemple, la présence de nuages sur le pixel, la qualité de la mesure, ou encore le capteur qui a réalisé la mesure (pour un produit composite).
Ces informations sont différentes pour chaque point de mesure, et doivent donc être stockées comme des variables (\as{ie} des tableaux multi"-dimensionnels).
Par chance, ce type de variables \guil*{drapeaux} peut souvent être représenté de manière très simple: la présence de nuage peut être indiqué par un booléen (un seul bit), la qualité de la mesure est typiquement un petit entier (2 ou 3 bits), etc.

Cependant, les outils informatiques ne permettent que rarement de manipuler des variables aussi petites. En raison de l'architecture des processeurs, la plus petite unité est aujourd'hui l'octet.
Chaque variable drapeau est donc traitée, au minimum, comme un octet.
On gâche alors de l'espace de stockage puisque seuls quelques bits sont utilisés sur les huits.

Plutôt que d'utiliser plusieurs variables drapeaux, l'idée de cette technique est de réunir tous ces drapeaux sur une seule et même variable, en leur associant certains bits de cette variable.
Prenons un exemple, nous voulons stocker les informations suivantes: la qualité de la mesure (un entier entre 1 et 3), l'algorithme utilisé pour calculer la Chlorophylle (\abbrv{OC4}, \abbrv{OC5}, ou CI), la présence de nuages (oui ou non), et la présence de glace de mer (oui ou non).
On notera que ces informations sont indépendantes entre elles.

On peut stocker chacune de ces informations de la manière suivante:
\begin{itemize}
  \item qualité, sur 2 bits entre \binary{01}~(bonne) et \binary{11}~(mauvaise),
  \item algorithme, sur 2 bits: \binary{01}=\abbrv{OC4}, \binary{10}=\abbrv{OC5}, \binary{11}=\abbrv{CI},
  \item nuage, sur 1 bit: \binary{0}=clair, \binary{1}=nuageux,
  \item glace, sur 1 bit: \binary{0}=libre, \binary{1}=glace.
\end{itemize}

On les combine ensuite en une seule variable stockée sur 8~bits (le minimum nécessaire), en commençant par le bit le moins signifiant (par la droite ici): glace, nuage, algorithme, qualité.
La représentation binaire de cette variable est donc:
\begin{equation}
  \overline{\textlf{00} \ \overline{qq} \ \overline{aa} \ \overline{n} \ \overline{g}},
\end{equation}
avec \(\overline{qq}\) les bits codant la qualité, \(\overline{aa}\) codant pour l'algorithme, \(\overline{n}\) les nuages, et \(\overline{g}\) la glace.
Par exemple un pixel dont la valeur serait~44, dont la représentation binaire est \(\overline{\texttlf{00101100}}\), indique que la mesure est de qualité moyenne (2), effectuée avec l'algorithme CI, et ne comporte ni nuages, ni glace.
Comparé à une approche naïve avec 4~variables séparées, nous avons gagné un facteur~4 en stockage.

comment extraire ?

Pour décrire ce stockage en suivant les conventions \as{cf}, il nous reste à définir les attributs suivants pour cette variable:

\begin{center}
  {%
    \newcommand*\smolb{{\footnotesize b}}%
    \lfstyle%
    \begin{tabular}{>{\ttfamily\small}l *{10}{r}} \toprule
      flag\_masks  & 1\smolb & 2\smolb \Repeat{3}{& 12\smolb} \Repeat{3}{& 48\smolb} \\
      flag\_values & 1\smolb & 2\smolb & 4\smolb & 8\smolb & 12\smolb
                             & 16\smolb & 32\smolb & 48\smolb \\
      flag\_meanings & \multirow{1}*{\rotatebox[origin=r]{65}{nuage}}
                             & \multirow{1}*{\rotatebox[origin=r]{65}{glace}}
                                                  & \abbrv{OC4} & \abbrv{OC5} & CI
                             & 1 & 2 & 3 \\
                   & & & \multicolumn{3}{c}{\upbracketfill} & \multicolumn{3}{c}{\upbracketfill} \\
                   & & & \multicolumn{3}{c}{algorithme} & \multicolumn{3}{c}{qualité} \\
      \bottomrule
    \end{tabular}
  }%
\end{center}

implémentation.

ce qu'il reste à faire (doc, cache).

\chapter{Bilan carbone de la thèse}
\label{ax:bilan-carbone}
% \suppressfloats[t]

Dans cette section, j'estime le coût carbone engendré par mon travail de thèse.
Commencons par les voyages professionels, typiquement un grand poste émetteur dans les carrières scientifiques.

Le premier est un aller"-retour Paris--Venise, en avion, pour assister à la 7\ieme~conférence \abbrv(LAPCOD)~(Lagrangian Analysis and Prediction of Coastal and Ocean Dynamics).
Réalisé pendant mon stage de master, je n'étais alors que peu sensibilisé à ces questions. Avec plus de prévoyance, il aurait pu être effectué en train.
Ce trajet comptabilise~\qty{266 \pm 27}{\kg\carbone} (sans compter les traînées) selon l'outil \abbrv{GES}~1point5, développé par le collectif \href{https://labos1point5.org/}{\textit{Labos~1point5}} (\cite{mariette_2022}).
En train, ce trajet aurait coûté~\qty{55 \pm 24}{\kg\carbone}.

Le second trajet est un aller"-retour Paris--Vienne, en train, pour assister l'assemblée générale 2022 de l'\abbrv{EGU}.
Selon le même outil, ce trajet a comptabilisé~\qty{75 \pm 33}{\kg\carbone}.
En avion il aurait coûté~\qty{230}{\kg\carbone}.
Je dois ici remercier les collègues chercheurs et le personnel administratif du laboratoire ayant rendu ce trajet quelque peu aventureux possible, et agréable.

J'estime \encadra{toujours avec \abbrv{GES}~1point5} mes trajets domicile"-travail effectués (le plus généralement) en bus et métro à~\qty{162 \pm 96}{\kg\carbone} par an.
Ce coût est en dessous de celui préconisé pour respecter les accords de Paris (\qty{0.3}{\tonne\carbone} selon \textcite{dugast_2019}).
Enfin les 5~repas journaliers (végétariens), pris au self du personnel, et les très nécessaires tasses de café et thé sont estimés à~\qty{420}{\kg\carbone} annuels par l'outil \href{https://nosgestesclimat.fr}{Nos Gestes Climat}.

Côté informatique, nous n'avons pas réalisé d'achats de matériel.
La majorité des calculs sont réalisés sur les machines du mésocentre de l'\ab{ipsl}; les données y sont également stockées (\qty{<1}{\To}).
Ces calculs sont pour leur vaste majorité de nombreux processus, courts (quelques minutes), et utilisent peu de ressources (un seul processeur et quelques~\unit{\Go} de mémoire vive) \encadra*{en contraste à une situation plus \guil*{simple}, par exemple, de quelques calculs très intensifs qui représenteraient la quasi"-totalité du bilan carbone}.
Il est donc compliqué d'estimer le coût de tous ces calculs dont l'historique n'a pas été conservé, en plus des difficultés usuelles de ce genre d'estimation sur des ressources informatiques partagé.

\begin{table}
  \centering
  \caption{Récapitulatif du bilan carbone}
  \label{tab:bilan-carbone}
  \begin{tabular}{l >{\hspace{2em}} r !{} r @{\,}w{l}{1em}@{}} \toprule
    \multirow{2}*{Objet} & \multicolumn{2}{r}{Coût (\unit{\kg\carbone})} \\
                         & par an & total                                \\
    \midrule
    LAPCOD (Venise)      &        & 266 & \rdelim\}{2}*[Voyages: \qty{341}{\kg\carbone}] \\
    EGU (Vienne)         &        & 75                                   \\
    \addlinespace

    Domicile-travail     & 162    & 648                                  \\
    Alimentation         & 420    & 1680                                 \\

    \midrule
    Total                &        & \bfseries 2669                       \\
    \bottomrule
  \end{tabular}
\end{table}
