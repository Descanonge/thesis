

\section{Variables drapeaux sur masque binaire}
\label{sec:cf-flags}


\section{Bilan carbone de la thèse}
\label{sec:bilan-carbone}

Dans cette section, j'estime le coût carbone engendré par mon travail de thèse.
Commencons par les voyages professionels, typiquement un grand poste émetteur dans les carrières scientifiques.

Le premier est un aller"-retour Paris--Venise, en avion, pour assister à la 7\ieme~conférence \abbrv(LAPCOD)~(Lagrangian Analysis and Prediction of Coastal and Ocean Dynamics).
Réalisé pendant mon stage de master, je n'étais alors que peu sensibilisé à ces questions. Avec plus de prévoyance, il aurait pu être effectué en train.
Ce trajet comptabilise~\qty{266 \pm 27}{\kg\carbone} (sans compter les traînées) selon l'outil \abbrv{GES}~1point5, développé par le collectif \href{https://labos1point5.org/}{\textit{Labos~1point5}} (\cite{mariette_2022}).
En train, ce trajet aurait coûté~\qty{55 \pm 24}{\kg\carbone}.

Le second trajet est un aller"-retour Paris--Vienne, en train, pour assister l'assemblée générale 2022 de l'\abbrv{EGU}.
Selon le même outil, ce trajet a comptabilisé~\qty{75 \pm 33}{\kg\carbone}.
En avion il aurait coûté~\qty{230}{\kg\carbone}.
Je dois ici remercier les collègues chercheurs et le personnel administratif du laboratoire ayant rendu ce trajet quelque peu aventureux possible, et agréable.

J'estime \encadra{toujours avec \abbrv{GES}~1point5} mes trajets domicile"-travail effectués (le plus généralement) en bus et métro à~\qty{162 \pm 96}{\kg\carbone} par an.
Ce coût est en dessous de celui préconisé pour respecter les accords de Paris (\qty{0.3}{\tonne\carbone} selon \textcite{dugast_2019}).
Enfin les 5~repas journaliers (végétariens), pris au self du personnel, et les très nécessaires tasses de café et thé sont estimés à~\qty{420}{\kg\carbone} annuels par l'outil \href{https://nosgestesclimat.fr}{Nos Gestes Climat}.

Côté informatique, nous n'avons pas réalisé d'achats de matériel.
La majorité des calculs sont réalisés sur les machines du mésocentre de l'\ab{ipsl}; les données y sont également stockées (\qty{<1}{\To}).
Ces calculs sont pour leur vaste majorité de nombreux processus, courts (quelques minutes), et utilisent peu de ressources (un seul processeur et quelques~\unit{\Go} de mémoire vive) \encadra*{en contraste à une situation plus \guil*{simple}, par exemple, de quelques calculs très intensifs qui représenteraient la quasi"-totalité du bilan carbone}.
Il est donc compliqué d'estimer le coût de tous ces calculs dont l'historique n'a pas été conservé, en plus des difficultés usuelles de ce genre d'estimation sur des ressources informatiques partagé.
