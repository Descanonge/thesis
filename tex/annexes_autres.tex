

\chapter{Variables drapeaux sur masque binaire}
\label[appendix]{ax:cf-flags}

Dans cette section, je décris succinctement le principe de stockage des variables drapeau (\engquote{flags}), tel que défini par les conventions \as{cf}\footnote{%
  section 3.5:\par\noindent\glsurl{cf-flags}
}.
Ayant besoin d'extraire des flags des données de Chlorophylle, et voyant qu'aucune implémentation n'était disponible pour \citesoft{xarray}, j'ai écrit une simple implémentation.
Je l'ai ensuite adaptée et soumise au projet \citesoft{cf-xarray} pour qu'elle puisse être réutilisée\footnote{%
  \url{https://github.com/xarray-contrib/cf-xarray/pull/354}}.

Il est souvent utile voire nécessaire d'accompagner certaines quantités par des informations supplémentaires: pour des données satellites par exemple, la présence de nuages sur le pixel, la qualité de la mesure, ou encore le capteur qui a réalisé la mesure (pour un produit composite).
Ces informations sont différentes pour chaque point de mesure, et doivent donc être stockées comme des variables (\as{ie} des tableaux multi-dimensionnels).
Ce type d'information (que nous appellerons drapeaux) peut souvent être représenté de manière très simple: la présence de nuage peut être indiqué par un booléen (un seul bit), la qualité de la mesure est typiquement un petit entier (2 ou 3 bits), etc.

Cependant, du fait de l'architecture des processeurs, les outils informatiques ne permettent généralement pas de manipuler des variables aussi petites.
Chaque variable drapeau ne peut être traitée et stockée que sur un octet, au minimum.
On gâche alors de l'espace de stockage puisque seuls quelques bits sont utilisés sur les huit.

Plutôt que d'utiliser plusieurs variables drapeaux, l'idée de cette technique est de réunir tous ces drapeaux sur une seule et même variable, en associant à chaque drapeau certains bits de cette variable.
Prenons un exemple, nous voulons stocker les informations suivantes: la qualité de la mesure (un entier entre 1 et 3), l'algorithme utilisé pour calculer la Chlorophylle (\abbrv{OC4}, \abbrv{OC5}, ou CI), la présence de nuages (oui ou non), et la présence de glace de mer (oui ou non).
On notera que ces informations sont indépendantes entre elles.

On peut stocker chacune de ces informations de la manière suivante:
\begin{itemize}
  \item qualité, sur 2 bits entre \binary{01}~(bonne) et \binary{11}~(mauvaise),
  \item algorithme, sur 2 bits: \binary{01}=\abbrv{OC4}, \binary{10}=\abbrv{OC5}, \binary{11}=\abbrv{CI},
  \item nuage, sur 1 bit: \binary{0}=clair, \binary{1}=nuageux,
  \item glace, sur 1 bit: \binary{0}=libre, \binary{1}=glace.
\end{itemize}

On les combine ensuite en une seule variable \(v\) stockée sur 8~bits (le minimum nécessaire), en commençant par le bit le moins signifiant (le plus à droite ici): glace, nuage, algorithme, qualité.
La représentation binaire de cette variable est donc:
\begin{equation}
  v = \overline{\texttosf{00} \ \overline{qq} \ \overline{aa} \ \overline{n} \ \overline{g}},
\end{equation}
avec \(\overline{qq}\) les bits codant la qualité, \(\overline{aa}\) codant pour l'algorithme, \(\overline{n}\) les nuages, et \(\overline{g}\) la glace.
Par exemple un pixel de valeur~44, dont la représentation binaire est \(\overline{\textosf{00101100}}\), indique que la mesure est de qualité moyenne (2), effectuée avec l'algorithme CI, et ne comporte ni nuages, ni glace.
Comparé à une approche naïve avec 4~variables séparées, nous avons gagné un facteur~4 en stockage.

Cette variable est facile à calculer à partir des valeurs des drapeaux que l'on veut encoder.
Il suffit d'additionner les drapeaux entre eux, dont on aura décalé les bits vers la gauche un nombre de fois approprié.
Une fois cette variable créée, les drapeaux individuels sont récupérés simplement en appliquant un masque binaire, ce qui correspond à une opération \guil*{et} bit à bit (\engquote{bitwise AND}, notée~\textsf{\&} pour \citesoft{numpy}).
Ce sont toutes des opérations élémentaires facilement applicables à tout un tableau.
Par exemple pour récupérer la valeur de qualité d'un pixel on fera l'opération: \(v \mathbin{\&} 48 (\overline{\textosf{00110000}})\).
La valeur récupérée est encore décalée sur la gauche, ici de 4~bits (\as{ie} multipliée par 8), mais dans la convention \as{cf} la signification des valeurs est stockée décalée, on aura donc directement 16: qualité~1, 32: qualité~2, 48: qualité~3.
La valeur du masque binaire est aussi stockée.
Ainsi, pour l'exemple ci-dessus, il nous faudra définir les attributs suivants pour la variable~\(v\):
\begin{center}
  {%
    \newcommand*\smolb{{\footnotesize b}}%
    \lfstyle%
    \begin{tabular}{>{\ttfamily\small}l *{10}{r}} \toprule
      flag\_masks  & 1\smolb & 2\smolb \Repeat{3}{& 12\smolb} \Repeat{3}{& 48\smolb} \\
      flag\_values & 1\smolb & 2\smolb & 4\smolb & 8\smolb & 12\smolb
                             & 16\smolb & 32\smolb & 48\smolb \\
      flag\_meanings & \multirow{1}*{\rotatebox[origin=r]{65}{nuage}}
                             & \multirow{1}*{\rotatebox[origin=r]{65}{glace}}
                                                  & \abbrv{OC4} & \abbrv{OC5} & CI
                             & 1 & 2 & 3 \\
                   & & & \multicolumn{3}{c}{\upbracketfill} & \multicolumn{3}{c}{\upbracketfill} \\
                   & & & \multicolumn{3}{c}{algorithme} & \multicolumn{3}{c}{qualité} \\
      \bottomrule
    \end{tabular}
  }%
\end{center}

\begin{note}
  Si tous les drapeaux sont des booléens (et donc encodés sur un seul bit), l'attribut \textsf{flag\_values} peut être omis, à l'inverse si tous les drapeaux sont mutuellement exclusifs (\ab{par-ex} on ne stocke \emph{que} la qualité des pixels) l'attribut \textsf{flag\_masks} peut être omis.
\end{note}
\begin{note}
  Pour les drapeaux autres que booléens, on fait ici commencer les valeurs à~1, comme dans les exemples donnés par la convention \as{cf}. Cela réserve la valeur~0 comme, par exemple, erreur on valeur non-définie. Ce n'est pas une obligation.
\end{note}

Suit ici l'implémentation de l'extraction des drapeaux pour \citesoft{cf-xarray} plus en détail\footnote{Il s'agit ici d'une version un peu plus claire que celle soumise au projet. Elle est fonctionnellement identique, et sera incorporée plus tard dans le projet, d'autres corrections étant de toute façon requises.}.
J'ai adapté une fonction déjà présente dans le module qui ne fonctionnait que pour des drapeaux tous mutuellement exclusifs.
\begin{python}
def create_flag_dict(da) -> Mapping[Hashable, Sequence]:
    """Return possible flag meanings and associated bitmask/values.

    The mapping values are a tuple containing a bitmask and a value.
    Either can be None.
    If only a bitmask: Independent flags.
    If only a value: Mutually exclusive flags.
    If both: Mix of independent and mutually exclusive flags.
    """
    if not da.cf.is_flag_variable:
        raise ValueError(
            "Comparisons are only supported for DataArrays that represent "
            "CF flag variables .attrs must contain 'flag_meanings' and "
            "'flag_values' or 'flag_masks'."
        )

    flag_meanings = da.attrs["flag_meanings"].split(" ")
    n_flag = len(flag_meanings)

    flag_values = da.attrs.get("flag_values", [None for _ in range(n_flag)])
    flag_masks = da.attrs.get("flag_masks", [None for _ in range(n_flag)])

    if not (n_flag == len(flag_values) == len(flag_masks)):
        raise IndexError("Not as many flag meanings as values or masks.")

    return dict(zip(flag_meanings, zip(flag_masks, flag_values)))
\end{python}

Je défini quelques méthodes supplémentaires à une classe d'\emph{Accessor}, ce qui les rendra accessibles aux objets Xarray concernés (\as{ie} des variables \py{xarray.DataArray} contenant les drapeaux à extraire):
\begin{python}
@xr.register_dataarray_accessor("cf")
class CFDataArrayAccessor(CFAccessor):
    ...
\end{python}

D'abord, je défini une propriété qui servira de point d'accès publique pour les drapeaux:
\begin{python}
@property
def flags(self) -> Dataset:
    """Dataset containing boolean masks of available flags."""
    return self._extract_flags()
\end{python}
Ainsi, pour une variable \py{da} et en reprenant l'exemple ci-dessus, on peut accéder par exemple aux pixels nuageux avec \py{da.cf.flags.nuage}, ou ceux obtenus avec l'algorithme CI \py{da.cf.flags.CI}.
Il s'agit d'un simple \eng{wrapper} autour d'une fonction plus compliquée:
\begin{pythonFirst}
def _extract_flags(
        self, flags: Sequence[str] | None = None
) -> DataArray | Dataset:
    """Return dataset of boolean mask(s) corresponding to `flags`.

    Parameters
    ----------
    flags: Sequence[str]
        Flags to extract. If empty (string or list),
        return all flags in `flag_meanings`.
    """
    flag_dict = create_flag_dict(self._obj)
\end{pythonFirst}
On récupère un dictionnaire qui contient pour chacune des signification de drapeau, la valeur associée et la valeur du masque binaire, ce qui correspond en fait à une colonne du tableau plus haut.

On ne garde que les drapeaux qui sont à extraire (par défaut, tous):
\begin{pythonMiddle}[]
    if flags is None or len(flags) == 0:
        flags = flag_dict.keys()
    else:
        for f in flags:
            if f not in flag_dict:
                raise ValueError(
                    f"Flag meaning {f!r} is not "
                    f"in known meanings {list(flag_dict.keys())!r}"
                )
        flag_dict = {f: flag_dict[f] for f in flags}
\end{pythonMiddle}

On vérifie si l'on est dans un cas simplifié. Si tous les drapeaux sont mutuellement exclusifs, pas besoin d'appliquer de masque binaire; si tous les drapeaux sont indépendants, pas besoin de faire une comparaison après l'application du masque.
\begin{pythonMiddle}[]
    all_mutually_excl = False
    all_indep = False
    for flag, (mask, value) in flag_dict.items():
        if mask is None:
            all_mutually_excl = True
            break
        if value is None:
            all_indep = True
            break
\end{pythonMiddle}

Si tous sont mutuellement exclusifs, il suffit d'appliquer une comparaison:
\begin{pythonMiddle}[]
    out = {}  # Output arrays

    if all_mutually_excl:
        for f, (_, value) in flag_dict.items():
            out[f] = self._obj == value
\end{pythonMiddle}

Sinon, on doit appliquer un masque binaire. On le fait en une seule opération pour tous les drapeaux.
Ensuite, on converti en booléen soit par comparaison avec la valeur appropriée soit par simple conversion:
\begin{pythonMiddle}[]
    else:
        # We cast both masks and flag variable as integers to make the
        # bitwise comparison. We could probably restrict the integer size
        # but it's difficult to make it safely for mixed type flags.
        masks = [m for m, _ in flag_dict.values()]
        values = [v for _, v in flag_dict.values()]
        bit_mask = DataArray(masks, dims=["_mask"]).astype("i")
        x = self._obj.astype("i")
        bit_comp = x & bit_mask  # Bitwise AND

        for i, (flag, value) in enumerate(zip(flags, values)):
            b = bit_comp.isel(_mask=i)
            if all_indep:
                out[f] = b.astype(bool)
            else:  # ie value should not be None
                out[f] = b == value
\end{pythonMiddle}

Il ne reste plus qu'à retourner l'ensemble des drapeaux:
\begin{pythonLast}[]
    return Dataset(out)
\end{pythonLast}

Cette implémentation, bien que fonctionnelle (une série de tests est définie) laisse encore de la place pour des améliorations.
Elle redéfinit un dataset à chaque appel, sans cache. Le nombre d'opération est relativement faible, mais il pourrait être souhaitable que ces dernières ne soient néanmoins appelées qu'une seule fois.
À l'inverse, l'utilisateur pourrait vouloir contrôler les objets en mémoire, notamment au vu de leur importance dans des calculs de grands volumes de données avec \citesoft{dask}. Une fonction de type \engquote{getter} serait appropriée.

Par ailleurs, on ne récupère que des significations individuelles. Il pourrait être intéressant de pouvoir extraire un ensemble de drapeaux mutuellement exclusifs.
Par exemple, on voudrait pouvoir récupérer toutes les valeurs de qualité, ou d'algorithme utilisé (plutôt que seulement les pixels correspondant à \emph{un} algorithme, ou \emph{une} valeur de qualité).

Enfin, la documentation de cette nouvelle fonctionnalité n'a pas encore été rédigée.


\chapter{Conversion des pigments diagnostiques en proportion de PFT}
\label[appendix]{ax:pig2pft}

\newcommand*\sumdp{ΣDP}
Conversion des concentrations de pigments diagnostiques~(DP) en proportion de groupes fonctionnels et groupes de tailles, d'après \textcite{hirata_2011,brewin_2010,uitz_2006}.

\hspace{\baselineskip}
\begin{center}
  \renewcommand\arraystretch{1.3}
  \begin{tabular}{r!{\hspace{2em}} l}
    \toprule
    \multicolumn{1}{c}{Groupes} & \multicolumn{1}{c}{Formule} \\
    \midrule
    \textbf{Microplancton}
                                & 1.41 (Fuco + Perid) / \sumdp \\
    Diatomées & 1.41 Fuco / \sumdp \\
    Dinoflagellées & 1.41 Perid / \sumdp \\

    \midrule
    \multirow[t]{2}*{\textbf{Nanoplancton}}
                                & (1.27\,\(X\) Hex + 1.01 Chl-\textit{b} \\
                                & \hspace{3em}       + 0.35 But + 0.60 Allo) / \sumdp \\
    Algues vertes & 1.01 Chl-\textit{b} / \sumdp \\
    Prymnésiophytes & Nanoplancton \textminus{} Algues vertes \\

    \midrule
    \textbf{Picoplancton}
                                & (0.86 Zea + 1.27\,\(Y\) Hex) / \sumdp \\
    Procaryotes & 0.86 Zea / \sumdp \\
    Pico-eucaryotes & Picoplancton \textminus{} Procaryotes \\
    Prochlorococcus & 0.74 Dvchl-\textit{a} / Chl-\textit{a} \\
    \bottomrule
  \end{tabular}
\end{center}
\hspace{\baselineskip}

{
\small
\begin{itemize}
  \item Fuco:~Fucoxanthine, Perid:~Péridinine, Hex:~19\babelhyphen{hard}Hexanoyloxyfucoxanthine, But:~19\babelhyphen{hard}Butanoyloxyfucoxanthine, Allo:~Alloxanthine, Zea:~Zeaxanthine,\par\noindent Dvchl-\textit{a}:~Divinyl Chlorophylle"~\textit{a}.
  \item \sumdp{} = 1.41Fuco + 1.41Perid + 1.27Hex + 0.35But + 0.60Allo + 1.01Chl"~\textit{b} +~0.86Zea.
  \item \(X\) évolue avec la concentration de \as{chl}, il vaut 1 quand \(\am{chl}=\qty[mode=text]{0.08}{\mgm}\) et diminue linéairement jusqu'à 0 pour une concentration de \qty{0.01}{\mgm}. \(X\)~est constant en dehors de ces bornes. \(Y = 1-X\).
\end{itemize}
}

\chapter{Bilan carbone de la thèse}
\label[appendix]{ax:bilan-carbone}
% \suppressfloats[t]

Dans cette section, j'estime le coût carbone engendré par mon travail de thèse.
Commençons par les voyages professionnels, typiquement un grand poste émetteur personnel dans les carrières scientifiques.

Le premier est un aller-retour Paris--Venise, en avion, pour assister à la 7\ieme~conférence \abbrv{LAPCOD}~(Lagrangian Analysis and Prediction of Coastal and Ocean Dynamics).
Réalisé pendant mon stage de master, je n'étais alors que peu sensibilisé à ces questions. Avec plus de prévoyance, il aurait pu être effectué en train.
Ce trajet comptabilise~\qty{266 \pm 27}{\kg\carbone} (sans compter les traînées) selon l'outil \abbrv{GES}~1point5, développé par le collectif \href{https://labos1point5.org/}{\textit{Labos~1point5}} (\cite{mariette_2022}).
En train, ce trajet aurait coûté~\qty{55 \pm 24}{\kg\carbone}.

Le second trajet est un aller-retour Paris--Vienne, en train, pour assister l'assemblée générale 2022 de l'\abbrv{EGU}.
Selon le même outil, ce trajet a comptabilisé~\qty{75 \pm 33}{\kg\carbone}.
En avion il aurait coûté~\qty{230}{\kg\carbone}.
Je dois ici remercier les collègues chercheurs et le personnel administratif du laboratoire ayant rendu ce trajet quelque peu aventureux en train, possible et agréable.

J'estime \encadra{toujours avec \abbrv{GES}~1point5} mes trajets domicile-travail effectués (le plus généralement) en bus et métro à~\qty{162 \pm 96}{\kg\carbone} par an.
Ce coût est en dessous de celui préconisé pour respecter les accords de Paris (\qty{0.3}{\tonne\carbone} selon \cite{dugast_2019}).
Enfin les 5~repas journaliers (végétariens), pris au self du personnel, et les très nécessaires tasses de café et thé sont estimés à~\qty{420}{\kg\carbone} annuels par l'outil \href{https://nosgestesclimat.fr}{Nos Gestes Climat}.

Je comptabilise également les coûts communs liés au fonctionnement du laboratoire (consommation énergétique, entretien bâtiment, mails,\dots) qui ont été estimés par le groupe Climaction-LOCEAN pour l'année 2018\footnote{\url{https://climactions.ipsl.fr/groupes-de-travail/locean-climactions/}}.
Individuellement, cela représente environ \qty{1}{\tcarbone} annuelle.

Côté informatique, nous n'avons pas réalisé d'achats de matériel.
La majorité des calculs sont réalisés sur les machines du mésocentre de l'\ab{ipsl}; les données y sont également stockées (\(\lesssim \qty{1}{\To}\)).
Ces calculs sont pour leur vaste majorité de nombreux processus, courts (quelques minutes), et utilisent peu de ressources (un seul processeur et quelques~\unit{\Go} de mémoire vive) \encadra*{en contraste à une situation plus \guil*{simple}, par exemple, de quelques calculs très intensifs qui représenteraient la quasi-totalité du bilan carbone}.
Il est donc compliqué d'estimer le coût de tous ces calculs dont l'historique n'a pas été conservé, en plus des difficultés usuelles de ce genre d'estimation sur des ressources informatiques partagées.


\begin{table}
  \centering
  \caption{Récapitulatif du bilan carbone}
  \label{tab:bilan-carbone}
  \begin{tabular}{l >{\hspace{2em}} r !{} r @{\,}w{l}{1em}@{}} \toprule
    \multirow{2}*{Objet} & \multicolumn{2}{r}{Coût (\unit{\kg\carbone})} \\
                         & par an & total                                \\
    \midrule
    LAPCOD (Venise)      &        & 266 & \rdelim\}{2}*[Voyages: \qty{341}{\kg\carbone}] \\
    EGU (Vienne)         &        & 75                                   \\
    \addlinespace

    Domicile-travail     & 162    & 648                                  \\
    Alimentation         & 420    & 1680                                 \\
    Fonctionnement       & 1000   & 4000                                 \\

    \midrule
    Total                &        & \bfseries 6669                       \\
    \bottomrule
  \end{tabular}
\end{table}


\chapter{Implémentation Belkin--O'Reilly}
\label[appendix]{ax:boa}

J'ai collaboré succinctement pendant ma thèse avec un étudiant en visite au laboratoire, qui cherchait à implémenter l'algorithme de \textcite{belkin_2009} en Python.
Nous étions parvenu au code ci-dessous pour effectuer le filtrage (avant de calculer la norme du gradient ou d'appliquer l'opérateur de Sobel).
L'utilisation de \citesoft{xarray} permet d'éviter quasiment toutes les boucles sur les pixels, si coûteuses quand faites directement en Python.

\begin{python}
import numpy as np
from xarray import DataArray


def flag_n(da: DataArray, n: int):
    window_size = {name: n for name in ['lat', 'lon']}
    roll = da.rolling(window_size, center=True)

    peak_min = roll.min()
    peak_max = roll.max()

    flag = (peak_min == da) | (peak_max == da)

    return flag


def boa(da: DataArray):
    f5 = flag_n(da, 5)
    f3 = flag_n(da, 3)
    to_filter = f3 * ~f5
    filtered = da.copy()
    idx = np.where(to_filter)

    for it, ix, iy in zip(*idx):
        window = da[it, ix-1:ix+2, iy-1:iy+2]
        filtered[it, ix, iy] = np.nanmedian(window)

    return filtered
\end{python}
