
\unsection{Résumé}
Résumé

\clearpage
\section*{Remerciements}
\label{sec:thanks}
Remerciements \dots

\clearpage
\section{Publications et productions}
\label{sec:productions}

\begin{itemize}
        \item Article CHL
\end{itemize}
\medskip

Lors de mon travail de thèse, j'ai été amené à écrire des outils qu'il m'a parru utile de rendre publics et accessibles.
<appuyer sur l'importance de l'open source et de faciliter la reproducibilité.>
Tous les codes utilisés sont disponibles sur un dépôt public\footnote{
  \url{\glsentryurl{gitlab}}
}, et sur un répertoire Zenodo (\glsDOI{zenodo}).
\medskip

Certains outils sont distribués à part:
\begin{itemize}
  \item \package{filefinder}:
        un paquet python qui permet entre autres de trouver des fichiers grâce à la structure de leur nom de fichier.
  \item \package{xarray-histogram}:
        un paquet python qui permet de calculer des histogrammes depuins des données gérées par Xarray.
  \item \package{tol-colors}:
        un paquet python qui donne accès à des jeux de couleurs adaptés aux personnes atteintes de daltonisme. Les jeux de couleurs existaient déjà, je les ai seulement rendus accessibles sur PyPi.
  \item \package{dateloop}:
        un script bash permettant de générer des ensembles de dates.
\end{itemize}
\medskip

J'ai également participé à un projet open-source (\glshref{cf-xarray}) visant à rendre accessible au paquet python XArray les conventions de métadonnées CF.\@
J'y ai ajouté le support pour les variables drapeaux utilisant un masque binaire.
