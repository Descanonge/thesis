% chktex-file 13

\thispagestyle{thesis-empty}
\unsection{Résumé}

À la base de la chaîne trophique dans les océans se trouve le phytoplancton: un ensemble de micro-organismes créant de la matière organique à partir du carbone dissous, de nutriments, et de lumière.
Les courants océaniques affectent la croissance du phytoplancton de différentes manières.
En particulier, les structures (dites de submésoéchelle) à des échelles comprises entre \qtyrange[range-phrase={ et }]{1}{10}{\km}, et évoluant sur des temps caractéristiques allant du jour à la semaine, jouent un rôle important pour le paysage biogéochimique des océans.
Du fait de leur taille et leur temps de vie caractéristiques réduits, elles sont cependant difficiles à observer \encadra{à la fois par mesures in situ et par satellite} et à représenter correctement dans les modèles numériques.

Nous nous penchons sur deux effets des fronts de densité à ces échelles.
Ces derniers génèrent des circulations secondaires verticales, dont la branche ascendante peut remonter des nutriments depuis les eaux profondes vers les eaux ensoleillées de surface.
Cet apport local de nutriment, et l'augmentation de productivité qui s'ensuit, a largement été observé, mais la quantification de son impact global n'est pas encore bien établi.
Ces fronts tendent également à favoriser, toujours localement, la restratification des eaux de surface.
Des modèles numériques ont montré que ce processus pouvait occasionner des blooms printaniers plus précoces.

Dans cette thèse, nous utilisons 20~années d'images satellitaires afin de mieux quantifier ces effets dans la région entourant le Gulf Stream.
À partir de la température de surface satellite (SST) nous calculons un indice d'hétérogénéité (HI) afin de détecter les fronts, et les séparer en fronts faibles (ceux de submésoéchelle, éphémères) et fronts forts.
Nous séparons la région en trois biomes: subtropicaux permanent et saisonnier, et subpolaire.

Nous comparons les valeurs de \al{chl} (\as{chl}) à l'extérieur et dans les fronts (faibles et forts).
Nous constatons que les fronts faibles sont associés à un excès local de \as{chl} moins important que les fronts forts, mais du fait de leur ubiquité le surplus de \as{chl} qu'ils induisent à l'échelle du biome est comparable.
L'excès local de \as{chl} dans les fronts est 2 à 3 fois plus important dans le biome subpolaire que dans la région oligotrophe.
Malgré des impacts spectaculaires à l'échelle locale, avec des augmentations jusqu'à de \pct{+60}, l'effet des fronts à l'échelle régionale (\as{ie} des biomes) est plus modeste, atteignant au maximum \pct{+5}.

Enfin, nous mesurons les différences dans les jours de démarrage du bloom et dans sa durée, entre les fronts et l'arrière-plan, et fournissons ainsi la première quantification par observation de cet effet.
Nous constatons que la durée du bloom n'est pas significativement affectée, mais que le bloom démarre plus tôt dans les fronts, d'une à deux semaines.

Ces résultats approfondissent la quantification des impacts des courants de fines échelles sur la biogéochimie, aidant ainsi à cerner les processus déterminants et à mieux contraindre les modèles climatiques encore sujets à de fortes incertitudes en cette matière.
Ce travail soulève plusieurs points clés dans l'établissement de cette quantification.
Nous nous sommes uniquement penchés sur la \as{chl} totale, sans examiner les réponses des différents groupes de phytoplancton, dont l'obtention des concentrations individuelles par satellite est possible.
Par ailleurs, nous avons montré les contrastes régionaux existants dans une zone relativement réduite, et les difficultés liées à la découpe en biorégions homogènes.
L'extension de ces résultats à d'autres régions et à l'échelle globale constitue cependant une étape future dans cet effort de quantification globale.
Nous soulignons donc les défis techniques que ces perspectives impliquent, certains que nous avons rencontré dans cette étude, mais aussi les solutions que nous y avons opposées, et celles que nous pensons importantes à implémenter dans un futur proche.

\unsection{Summary}
\selectlanguage{english}

At the base of the food chain in the oceans is phytoplankton: a collection of microorganisms that create organic matter from dissolved carbon, nutrients, and light.
Ocean currents affect the growth of phytoplankton in different ways.
In particular, submesoscale structures of scales of \qtyrange{1}{10}{\km} and evolving over characteristic times ranging from day to week, play an important role in the biogeochemical landscape of the oceans.
Due to their small size and short lifetimes, they are difficult to observe \encadra{both by in situ and satellite measurements} and to represent correctly in numerical models.

Here we focus on two effects of density fronts of such scales.
They generate vertical secondary circulations, whose ascending branch can potentially bring nutrients from the deep waters to the lit surface layers.
This local input of nutrients, and the resulting increase in productivity, has been widely observed, but the quantification of its global impact is not yet well established.
These fronts also tend, still locally, to restratify surface layers.
Numerical models have shown that this process can lead to earlier spring blooms.

In this thesis, we use 20~years of satellite images to better quantify these effects in the region surrounding the Gulf Stream.
From the satellite surface temperature (SST) we compute a heterogeneity index (HI) in order to detect fronts, and separate them into weak fronts (submesoscale and ephemeral) and strong fronts.
We separate the region into three biomes: permanent and seasonal subtropical, and subpolar.

We compare \al{chl} (\as{chl}) values outside and inside the fronts (weak and strong).
We find that weak fronts are associated with a smaller local excess of \as{chl} than strong fronts, but due to their ubiquity the excess of \as{chl} they induce at the scale of the biome is similar.
The local excess of \as{chl} in fronts is 2 to 3 times greater in the subpolar biome than in the oligotrophic region.
Despite dramatic impacts at the local scale, with increases of up to \pct{+60}, the effect of fronts at the regional (\as{ie} biome) scale is more modest, reaching at most \pct{+5}.
Finally, we measure differences in the bloom onset and duration between the fronts and the background, and thus provide the first observational quantification of this effect.
We find that the duration of the bloom is not significantly affected, but that the bloom triggers earlier in the fronts, by one to two weeks.

These results further the quantification of the impacts of fine-scale currents on biogeochemistry, thus helping to identify the significant processes and to better constrain climate models, still subject to high uncertainties in this area.
This work raises several key points in the establishment of this quantification.
We have focused only on total \as{chl}, without examining the responses of the different phytoplankton groups, for which individual concentrations can be obtained by satellite.
The extension of these results to other regions and to the global scale is however a future step in this global quantification effort.
We therefore highlight the technical challenges that these perspectives imply, some of which we have encountered in this work, but also the solutions that we have used and others that we think are important to implement in the near future.

\selectlanguage{french}

\clearpage
\thispagestyle{thesis-empty}

\section*{Remerciements}
\label{sec:thanks}

Ce travail de thèse n'aurait pu être accompli sans l'aide et le soutien d'une myriade de personnes que j'aurai du mal à remercier de manière exhaustive.

À commencer par Marina et Laurent: je ne saurai suffisamment vous remercier.
Vous savez les difficultés que j'ai rencontrées en chemin mais qui n'ont pu ébrécher votre confiance et votre soutien indéfectibles.
Merci également de m'avoir donné la liberté d'explorer le côté plus \guil*{technique} de la recherche, tout en continuant à m'aiguiller dans cette voie.
Je n'aurai pu demander une direction plus idéale que la vôtre, et je suis ravi de pouvoir continuer à collaborer avec vous.

Un chaleureux merci également à Sakina, Daniele et Francesco, pour vos retours scientifiques sur mon travail, ainsi que pour m'avoir accompagné et conseillé avec recul et justesse dans ma méthodologie et mon orientation de carrière.

Impossible de ne pas remercier l'équipe administrative du laboratoire qui \encadra{malgré toutes les difficultés logistiques qu'elle continue d'affronter} s'est systématiquement démenée pour mener à bien mes diverses démarches (fussent elles entamées au dernier moment\ldots{}!).

Je tiens ensuite à remercier Théo, Arthur et Thomas, je garderai des souvenirs inoubliables de cette vie en collocation. Je n'aurai pu demander de meilleurs compagnons de vie et de thèse.
Il en va de même pour les joyeux lurons qui ont été de passage chez nous, pour les soirées déguisées, les soirées films \encadra{bons, mauvais, ou les deux} ou pour un confinement ou deux\ldots

Bien sûr, je n'aurai pu en arriver là sans le concours de ma famille.
Merci à mes parents qui se sont toujours démenés pour me donner les meilleures chances de succès.
Merci aussi à Virgile et Claire de m'avoir supporté toutes ces années (dans les deux sens du terme).
Merci enfin les cousines, toujours présentes et à l'écoute, promptes à prêter main forte; vous continuez d'être des repères quand je suis un peu perdu.

Pour finir, un merci \emph{astronomique} aux très nombreux et nombreuses doctorant·es, post-docs et stagiaires rencontré·es pendant ces années passées au LOCEAN, dont le soutien fut inestimable et sans qui la vie au laboratoire serait bien moins ensoleillée.


\clearpage
\thispagestyle{thesis-empty}

\vspace*{5ex}
\par\addvspace{\beforesecskip}\addvspace{\baselineskip}
\par\noindent\textit{À l'intention du lecteur\dots}\par
\vspace{\aftersecskip}

Ce manuscrit compile la majeure partie des travaux que j'ai réalisés durant ces 4~dernières années.
J'ai eu la chance d'avoir bénéficié d'un encadrement de thèse exceptionnel: soutenu, et pourtant me permettant une grande liberté dans mon approche, à la fois face aux problématiques scientifiques, et aux difficultés techniques.
Ces années m'ont ainsi permis d'approfondir mes compétences méthodologiques et ma compréhension des questions scientifiques abordés dans ce manuscrit, mais aussi et surtout d'explorer ce qui m'animait vraiment \encadra*{au gré de ma curiosité}.

Force est de constater que, bien que les questions scientifiques posées dans ce manuscrit m'intéressent, m'entretenir à y répondre en tant que \guil*{chercheur} ne me correspond pas.
À l'inverse, j'ai pu trouver une réelle vocation dans le développement des outils requis par la résolution de ces questions.
Si vous me permettez cette analogie: si cette thèse n'aura pas pu me mettre sur les rails du jeune chercheur, elle m'a cependant montré la voie de l'ingénieur de recherche, qui y court parallèle et dans laquelle je me plais bien plus.
En conséquence, ce manuscrit reflète cette préférence et s'attardera peut-être un peu plus qu'à l'accoutumée sur des considérations techniques.

En vous souhaitant une agréable lecture.

\clearpage
\thispagestyle{thesis-empty}

\unsection{Publications et productions}
\label{sec:productions}

\begingroup
\defaultlists
Le travail effectué durant ma thèse et présenté dans ce manuscrit a donné lieu à plusieurs publications:
\begin{itemize}
  \setlength{\itemsep}{2\baselineskip}
  \renewcommand*\labelitemi{\adfrightarrowhead}
  \item un premier article présentant la méthode développée, et certains des résultats obtenus.
        Cet article est intégré à la structure de ce manuscrit, et inclu dans sa totalité au \hyperref[sec:article-bg]{chapitre~\ref*{chp:res-chl}}.
        \textnote[title={}]{\fullcite{haeck_2023}}

  \item un second article décrivant l'étude effectuée par I.\ Mangolte dans la région du courant de Californie, et qui exploite la méthode présentée dans cette thèse. Cet article est discuté dans le \hyperref[sec:cce]{chapitre~\ref*{chp:perspectives}} et un extrait est reporté en \nref[annexe]{ax:article-cce}.
        \textnote[title={}]{\fullcite{mangolte_2023}}

  \item et un article de \eng{review} dans lequel s’inscrit ce travail,
        dont un extrait est reporté en \nref[annexe]{ax:article-review}.
        % discuté dans le \hyperref[sec:review]{chapitre~\ref{chp:perspectives}} et
        \textnote[title={}]{\fullcite{levy_2024}}

\end{itemize}
\endgroup

\bigskip

Participation à des conférences:
\begin{itemize}
  \item \glshref{conf-filachange}: présentation courte
  \item \glshref{conf-egu22}: Présentation courte dans \abbrv{OS4.5}: Ocean Remote Sensing
  \item \glshref{conf-carbon}
  \item \glshref{conf-egu21}: Présentation vPico dans \abbrv{OS4.3}: Ocean Remote Sensing
  \item \glshref{conf-lapcod} (Lagrangian Analysis and Prediction of Coastal and Ocean Dynamics)
\end{itemize}

\newpage

Pendant cette thèse une grande partie de mon effort s'est tournée sur l'élaboration d'outils informatiques cohérents \encadra{j'entends allant au-delà d'une simple collection de scripts} permettant les analyses présentées ci-après.
Une attention toute particulière a été portée à assurer leur réutilisation et extension avec facilité, d'une part pour garantir la reproductibilité effective de cette étude, et d'autre part à ce que ce travail puisse bénéficier à d'autres.
Bien entendu, d'autres considérations entrent également en compte, par exemple de performances et de libre accès.
Il en résulte environ \num{11000} lignes de codes réparties en une centaine de fichiers (sans compter les scripts pour créer des graphes).
Tous les codes utilisés sont disponibles sur un dépôt Git public\footnote{\glsurl{gitlab}} hébergé par l'\abbrv{IN2P3}~(Institut National de Physique Nucléaire et de Physique des Particules, CNRS).
Une version a également été déposée sur un répertoire Zenodo (\hbox{\glsurl{zenodo}}).
Ces codes sont accompagnés d'une documentation disponible en format \href{https://clementhaeck.pages.in2p3.fr/submeso-color/}{\textsf{html}} ou \href{https://gitlab.in2p3.fr/clementhaeck/submeso-color/-/blob/develop/docs/index.org}{\textsf{org-mode}}, et dont un extrait a été retranscrit en \nref[annexe]{ax:sms-doc}.

Une partie du code a des objectifs plutôt généraux: gérer un ensemble de données provenant de différentes sources, et offrir un cadre de travail flexible qui puisse être étendu à de larges volumes de données.
Ces parties potentiellement réutilisables se prêtent à être distribuées indépendamment pour être réutilisées dans d'autres projets.
Ainsi, durant ma thèse j'ai publié publiquement les outils suivant:
\begin{description}
  \setlength{\itemsep}{2\baselineskip}
  \item[FileFinder]
        est un paquet Python qui permet (entre autres) de trouver des fichiers grâce à la structure de leur nom de fichier, ce qui est particulièrement utile par exemple pour des fichiers dépendant de la date, ou de la valeur d'un paramètre, ou des deux.
        L'entièreté des fichiers de données utilisés pour cette thèse sont gérées grâce à cet outil.
        \textnote[title={}]{\citesoftfullPerso{filefinder}}

  \item[tol-colors]
        est un paquet Python qui donne accès à des assortiments de couleurs adaptés aux personnes atteintes de daltonisme.
        Les jeux de couleurs étaient pré-existants et ont été développés par \glsname{paultol}\footnote{voir~\glsurl{paultol}}; je les ai seulement rendus plus facilement accessibles via PyPI.
        \textnote[title={}]{\citesoftfullPerso{tol-colors}}

  \item[Xarray-histogram]
        est un paquet Python\footnote{%
        Il est à noter que ce projet est plus proche du stade de preuve de concept que d'outil exploitable.
        Son utilisation sur de large quantités de données avec Dask présente des problèmes non résolus.
        Néanmoins les bénéfices en performances sont très prometteurs (voir le dépôt Github pour des comparaisons).
        } qui permet de calculer des histogrammes de données gérées par \citesoft{xarray}.
        C'est une opération centrale dans les calculs effectués pour cette thèse (\nref{sec:extraction-hist}).
        Des solutions existent déjà, mais présentent certains désavantages (notamment de performances et de base de code complexe) que j'ai cherché à palier.
        \textnote[title={}]{\citesoftfullPerso{xarray-histogram}}

  \item[dateloop]
        est une simple commande bash permettant de générer des ensembles de dates.
        Elle a surtout été utilisée au début de ce projet afin de faciliter la gestion et le téléchargement des fichiers de données par des scripts shell. Ces scripts ont progressivement été remplacés par des équivalents Python.
        \textnote[title={}]{\citesoftfullPerso{dateloop}}

\end{description}

Les parties du code qui gèrent les paramètres d'entrée et les bases de données (dont les documentations sont reportées \nref[annexe]{ax:sms-doc}) pourraient être adaptées pour être distribuées indépendamment, dans un futur proche.

De manière plus secondaire mais que je tiens néanmoins à noter, j'ai participé au projet \citesoft{cf-xarray} (visant à rendre accessible à \citesoft{xarray} les conventions de métadonnées \as{cf}) en y implémentant l'extraction de variables drapeaux sur masque binaire (\href{https://github.com/xarray-contrib/cf-xarray/pull/354}{\eng{pull request}~\textlf{\#354}}).
Le principe est détaillé en \nref[annexe]{ax:cf-flags}.
Cette technique sera d'ailleurs ré-utilisée pour extraire certains résultats (voir note~\cpageref{note:flags-hist}, \nref{chp:methodes} \nref{sec:extraction-hist}).

\bigskip

J'ai également réalisé des missions d'enseignement en physique numérique pendant deux ans (64~heures par an).
La première année, j'ai donné un cours d'introduction à \hbox{C\raisebox{0.5ex}{\scriptsize\textbf{++}}} appliqué à la physique numérique, au niveau~\abbrv{L3}, ainsi qu'encadré les travaux dirigés et travaux pratiques correspondants. Pour la deuxième moitié de cette unité d'enseignement (UE), j'ai encadré et noté les projets des étudiants.
La deuxième année, l'UE ayant été remaniée, j'ai encadré et noté les projets des étudiants, cette fois-ci réalisés en Python et durant le semestre entier.

Cette expérience aura été l'occasion de revoir les bases de \hbox{C\raisebox{0.5ex}{\scriptsize\textbf{++}}}, ainsi que d'expérimenter avec quelques problèmes de physique intéressants.
Surtout, cela m'aura permis de me confronter à des défis de pédagogie et d'enseignement, en particulier face à des étudiants au niveau en informatique très hétérogène.
Malgré une première année entachée par de l'enseignement à distance forcé, cela aura globalement été une expérience très positive, autant grâce à l'équipe pédagogique qu'aux élèves.

\bigskip

Enfin, pour conclure cette section et en guise d'indication, l'accomplissement de cette thèse aura généré~\qty{6.7}{\tcarbone}.
Je n'ai réalisé que peu de trajets en avion, aucune campagne en mer, et je n'ai pas comptabilisé le coût carbone informatique (qui sont typiquement les trois postes les plus consommateurs pour un travail académique); ainsi la majeure partie de cette estimation correspond à la vie en laboratoire.
L'estimation (sommaire) de ce bilan carbone est détaillée en \nref[annexe]{ax:bilan-carbone}.
