% chktex-file 13

\thispagestyle{thesis-empty}
\unsection{Résumé}

Résumé

\begin{comment}

\clearpage
\thispagestyle{thesis-empty}

\section*{Remerciements}
\label{sec:thanks}

la colloc étendu: théo, thomas, arthur, rémi, rémy, clément, sarah
parler du confinement ?
les potos: benjamin, marie, charlie, paul, jessie, piwi

la mifa

mon comité de thèse ? Sakina, Daniele

le 426 le meilleur bureau, tous les autres doctorants et post-doctorants et stagiaires
c'est quand même vachement plus fun avec elleux.

l'équipe admin du labo, bigup
\end{comment}

\clearpage
\thispagestyle{thesis-empty}

\vspace*{5ex}
\par\addvspace{\beforesecskip}\addvspace{\baselineskip}
\par\noindent\textit{À l'intention du lecteur\dots}\par
\vspace{\aftersecskip}

Ce manuscrit compile la majeure partie des travaux que j'ai réalisé durant ces 4~dernières années.
J'ai eu la chance d'avoir bénéficié d'un encadrement de thèse exceptionnel: soutenu, et pourtant me permettant une grande liberté dans mon approche, à la fois face aux problématiques scientifiques et aux difficultés techniques.
Ces années m'ont ainsi permis d'approfondir mes compétences méthodologiques et ma compréhension des questions scientifiques abordés dans ce manuscrit, mais aussi et surtout d'explorer ce qui m'animait vraiment \encadra*{au gré de ma curiosité}.

Force est de constater que, bien que les questions scientifiques posées dans ce manuscrit m'intéressent, m'entretenir à y répondre en tant que \guil*{chercheur} ne me correspond pas.
À l'inverse, j'ai pu trouver une réelle vocation dans le développement des outils requis par la résolution de ces questions.
Si vous me permettez cette analogie: si cette thèse n'aura pas pu me mettre sur les rails du jeune chercheur, elle m'a cependant montré la voie de l'ingénieur de recherche, qui y court parallèle et dans laquelle je me plais bien plus.
En conséquence, ce manuscrit reflète cette préférence et s'attardera peut-être un peu plus qu'à l'accoutumée sur des considérations techniques.

En vous souhaitant une agréable lecture.

\clearpage
\thispagestyle{thesis-empty}

\unsection{Publications et productions}
\label{sec:productions}

\begingroup
\defaultlists
Le travail effectué durant ma thèse et présenté dans ce manuscrit a donné lieu à plusieurs publications:
\begin{itemize}
  \setlength{\itemsep}{2\baselineskip}
  \renewcommand*\labelitemi{\adfrightarrowhead}
  \item un premier article présentant la méthode développée, et certains des résultats obtenus.
        Cet article est intégré à la structure de ce manuscrit, et inclu dans sa totalité au \hyperref[sec:article-bg]{chapitre~\ref*{chp:res-chl}}.
        \textnote[title={}]{\fullcite{haeck_2023}}

  \item un second article décrivant l'étude effectuée par I.\ Mangolte dans la région du courant de Californie, et qui exploite la méthode présentée dans cette thèse. Cet article est discuté dans le \hyperref[sec:cce]{chapitre~\ref*{chp:perspectives}} et un extrait est reporté en \nref[annexe]{ax:article-cce}.
        \textnote[title={}]{\fullcite{mangolte_2023}}

  \item et un article de \eng{review} dans lequel s’inscrit ce travail,
        dont un extrait est reporté en \nref[annexe]{ax:article-review}.
        % discuté dans le \hyperref[sec:review]{chapitre~\ref{chp:perspectives}} et
        \textnote[title={}]{\fullcite{levy_2023}}

\end{itemize}
\endgroup

\bigskip

Participation à des conférences:
\begin{itemize}
  \item \glshref{conf-filachange}: présentation courte
  \item \glshref{conf-egu22}: Présentation courte dans \abbrv{OS4.5}: Ocean Remote Sensing
  \item \glshref{conf-carbon}
  \item \glshref{conf-egu21}: Présentation vPico dans \abbrv{OS4.3}: Ocean Remote Sensing
  \item \glshref{conf-lapcod} (Lagrangian Analysis and Prediction of Coastal and Ocean Dynamics)
\end{itemize}

\bigskip

Pendant cette thèse une grande partie de mon effort s'est tournée sur l'élaboration d'outils informatiques cohérents \encadra{j'entends allant au-delà d'une simple collection de scripts} permettant les analyses présentées ci-après.
Une attention toute particulière a été portée à assurer leur réutilisation et extension avec facilité, d'une part pour garantir la reproductibilité effective de cette étude, et d'autre part à ce que ce travail puisse bénéficier à d'autres.
Bien entendu, d'autres considérations entrent également en compte, par exemple de performances et de libre accès.
Il en résulte environ \num{11000} lignes de codes réparties en une centaine de fichiers (sans compter les scripts pour créer des graphes).
Tous les codes utilisés sont disponibles sur un dépôt Git public\footnote{\glsurl{gitlab}} hébergé par l'\abbrv{IN2P3}~(Institut National de Physique Nucléaire et de Physique des Particules, CNRS).
Une version a également été déposée sur un répertoire Zenodo (\hbox{\glsurl{zenodo}}).
Ces codes sont accompagnés d'une documentation disponible en format \href{https://clementhaeck.pages.in2p3.fr/submeso-color/}{\textsf{html}} ou \href{https://gitlab.in2p3.fr/clementhaeck/submeso-color/-/blob/develop/docs/index.org}{\textsf{org-mode}}, et dont un extrait a été retranscrit en \nref[annexe]{ax:sms-doc}.

Une partie du code a des objectifs plutôt généraux: gérer un ensemble de données provenant de différentes sources, et offrir un cadre de travail flexible qui puisse être étendu à de larges volumes de données.
Ces parties potentiellement réutilisables se prêtent à être distribuées indépendamment pour être réutilisées dans d'autres projets.
J'ai ainsi durant ma thèse publié publiquement les outils suivant:
\begin{description}
  \setlength{\itemsep}{2\baselineskip}
  \item[FileFinder]
        est un paquet Python qui permet (entre autres) de trouver des fichiers grâce à la structure de leur nom de fichier, ce qui est particulièrement utile par exemple pour des fichiers dépendant de la date, ou de la valeur d'un paramètre, ou des deux.
        L'entièreté des fichiers de données utilisés pour cette thèse sont gérées grâce à cet outil.
        \textnote[title={}]{\citesoftfullPerso{filefinder}}

  \item[tol-colors]
        est un paquet Python qui donne accès à des assortiments de couleurs adaptés aux personnes atteintes de daltonisme.
        Les jeux de couleurs étaient pré-existants et ont été développés par \glsname{paultol}\footnote{voir~\glsurl{paultol}}; je les ai seulement rendus plus facilement accessibles via PyPI.
        \textnote[title={}]{\citesoftfullPerso{tol-colors}}

  \item[Xarray-histogram]
        est un paquet Python\footnote{%
        Il est à noter que ce projet est plus proche du stade de preuve de concept que d'outil exploitable.
        Son utilisation sur de large quantités de données avec Dask présente des problèmes non résolus.
        Néanmoins les bénéfices en performances sont très prometteurs (voir le dépôt Github pour des comparaisons).
        } qui permet de calculer des histogrammes de données gérées par \citesoft{xarray}.
        C'est une opération centrale dans les calculs effectués pour cette thèse (\nref{sec:extraction-hist}).
        Des solutions existent déjà, mais présentent certains désavantages (notamment de performances et de base de code complexe) que j'ai cherché à palier.
        \textnote[title={}]{\citesoftfullPerso{xarray-histogram}}

  \item[dateloop]
        est une simple commande bash permettant de générer des ensembles de dates.
        Elle a surtout été utilisée au début de ce projet afin de faciliter la gestion et le téléchargement des fichiers de données par des scripts shell. Ces scripts ont progressivement été remplacés par des équivalents Python.
        \textnote[title={}]{\citesoftfullPerso{dateloop}}

\end{description}

Les parties du code qui gèrent les paramètres d'entrée et les bases de données (dont les documentations sont reportées \nref[annexe]{ax:sms-doc}) pourraient être adaptées pour être distribuées indépendamment, dans un futur proche.

De manière plus secondaire mais que je tiens néanmoins à noter, j'ai participé au projet \citesoft{cf-xarray} (visant à rendre accessible à \citesoft{xarray} les conventions de métadonnées \as{cf}) en y implémentant l'extraction de variables drapeaux sur masque binaire (\href{https://github.com/xarray-contrib/cf-xarray/pull/354}{\eng{pull request}~\textlf{\#354}}).
Le principe est détaillé en \nref[annexe]{ax:cf-flags}.
Cette technique sera d'ailleurs ré-utilisée pour extraire certains résultats (voir note~\cpageref{note:flags-hist}, \nref{chp:methodes} \nref{sec:extraction-hist}).

\bigskip

Enfin, pour conclure cette section et en guise d'indication, l'accomplissement de cette thèse aura généré~\qty{6.7}{\tcarbone}.
Je n'ai réalisé que peu de trajets en avion, aucune campagne en mer, et je n'ai pas comptabilisé le coût carbone informatique (qui sont typiquement les trois postes les plus consommateurs pour un travail académique); ainsi la majeure partie de cette estimation correspond à la vie en laboratoire.
L'estimation (sommaire) de ce bilan carbone est détaillée en \nref[annexe]{ax:bilan-carbone}.
