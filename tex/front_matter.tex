
\unsection{Résumé}
Résumé

% \clearpage
% \section*{Remerciements}
% \label{sec:thanks}
% Remerciements \dots

\clearpage
\unsection{Publications et productions}
\label{sec:productions}

\begin{itemize}
        \item Article CHL
\end{itemize}
\medskip

Lors de mon travail de thèse, j'ai été amené à écrire des outils qu'il m'a parru utile de rendre publics et accessibles.
<appuyer sur l'importance de l'open source et de faciliter la reproducibilité.>
Tous les codes utilisés sont disponibles sur un dépôt public\footnote{%
  \glsurl{gitlab}
}, et sur un répertoire Zenodo (\glsurl{zenodo}).
\medskip

Certains outils sont distribués à part:
\begin{itemize}
  \item \citesoft{filefinder}:
        un paquet python qui permet, entre autres, de trouver des fichiers grâce à la structure de leur nom de fichier.
        <Expliquer que j'utilise ça pour toutes mes bases de données.>
  \item \citesoft{xarray-histogram}:
        un paquet python qui permet de calculer des histogrammes depuins des données gérées par Xarray.
  \item \citesoft{tol-colors}:
        un paquet python qui donne accès à des jeux de couleurs adaptés aux personnes atteintes de daltonisme. Les jeux de couleurs ont été développés par \glsname{paultol}\footnote{voir~\glsurl{paultol}}, je les ai seulement rendus accessibles sur PyPi, et plus facilement modifiables.
  \item \citesoft{dateloop}:
        un script bash permettant de générer des ensembles de dates.
\end{itemize}
\medskip

J'ai également participé à un projet open-source (\citesoft{cf-xarray}~\href{https://github.com/xarray-contrib/cf-xarray/pull/354}{\#354}) visant à rendre accessible au paquet python XArray les conventions de métadonnées CF.\@
J'y ai ajouté le support pour les variables drapeaux utilisant un masque binaire.

\begin{center}
  \vspace{1\baselineskip}
  \rule{0.77\textwidth}{0.5pt}
  \vspace{1\baselineskip}
\end{center}

{%
  \raggedright%
  \emergencystretch=\textwidth
  \printbibliography[heading=none, type=software, keyword=personnal]
}
