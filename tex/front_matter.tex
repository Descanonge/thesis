% chktex-file 13

\unsection{Résumé}
Résumé

% \clearpage
% \section*{Remerciements}
% \label{sec:thanks}
% Remerciements \dots

\clearpage

\vspace{5\onelineskip}
\par\addvspace{\beforesecskip}\addvspace{\baselineskip}
\textit{À l'intention du lecteur\dots}\par
\vspace{\aftersecskip}

à quoi s'attendre vis à vis du contenu, plus orienté méthodo / technique.
parce que c'est ce qui m'intéresse le plus.

En vous souhaitant une agréable lecture.

\clearpage
\unsection{Publications et productions}
\label{sec:productions}

\begingroup
\resetbegentry
Le travail effectué durant ma thèse et présenté dans ce manuscrit a donné lieu à plusieurs articles:
\begin{itemize}
  \item un premier article présentant la méthode développée, et certains des résultats obtenus.
        Cet article est intégré à la structure de ce manuscrit, et inclu dans sa totalité au \nref[chapitre]{chp:res-chl}.
        \begin{note}[before upper={}]\fullcite{haeck_2023}\end{note}

  \item un second article décrivant l'étude effectuée par I.\ Mangolte dans la région du courant de Californie, et qui exploite en partie la méthode présentée dans cette thèse. Cet article est discuté dans le \hyperref[sec:CCE]{chapitre~\ref{chp:perspectives}} et resumé en \nref[annexe]{ax:article-cce}.
        \begin{note}[before upper={}]\review{cite: \articleCceTitle}\end{note}

  \item et un article de \eng{review} dans lequel s’inscrit ce travail, discuté dans le \hyperref[sec:review]{chapitre~\ref{chp:perspectives}} et résumé en \nref[annexe]{ax:article-review}.
        \begin{note}[before upper={}]\review{cite: \articleReviewTitle}\end{note}

\end{itemize}
\medskip
\endgroup

Par ailleurs, une majeure partie de mon attention et effort a été diri
Lors de mon travail de thèse, j'ai été amené à écrire des outils qu'il m'a parru utile de rendre publics et accessibles.
<appuyer sur l'importance de l'open source et de faciliter la reproducibilité.>
Tous les codes utilisés sont disponibles sur un dépôt public\footnote{%
  \glsurl{gitlab}
}, et sur un répertoire Zenodo (\glsurl{zenodo}).
\medskip

Certains outils sont distribués à part:
\begin{itemize}
  \item \citesoft{filefinder}:
        un paquet python qui permet, entre autres, de trouver des fichiers grâce à la structure de leur nom de fichier.
        <Expliquer que j'utilise ça pour toutes mes bases de données.>
  \item \citesoft{xarray-histogram}:
        un paquet python qui permet de calculer des histogrammes depuins des données gérées par Xarray.
  \item \citesoft{tol-colors}:
        un paquet python qui donne accès à des jeux de couleurs adaptés aux personnes atteintes de daltonisme. Les jeux de couleurs ont été développés par \glsname{paultol}\footnote{voir~\glsurl{paultol}}, je les ai seulement rendus accessibles sur PyPi, et plus facilement modifiables.
  \item \citesoft{dateloop}:
        un script bash permettant de générer des ensembles de dates.
\end{itemize}
\medskip

J'ai également participé à un projet open-source (\citesoft{cf-xarray}~\href{https://github.com/xarray-contrib/cf-xarray/pull/354}{pull request~\textlf{\#354}}) visant à rendre accessible au paquet python XArray les conventions de métadonnées~\as{cf}.
J'y ai ajouté le support pour les variables drapeaux utilisant un masque binaire.

\begin{center}
  \vspace{1\baselineskip}
  \rule{0.77\textwidth}{0.5pt}
  \vspace{1\baselineskip}
\end{center}

{%
  \raggedright%
  \emergencystretch=\textwidth
  \printbibliography[heading=none, type=software, keyword=personnal]
}


Bilan carbone.
AR EGU train.
AR bus + métro 4 ans.
Repas végé cantine 4 ans.
Pas d'estimation du traitement de données sur cluster.
