% chktex-file 13

\chapter{Introduction}
\addChpLof
\label{chp:introduction}
\graphicspath{{resources/introduction}}

{
  \hypersetup{hidelinks}
  \minitoc%
  \clearpage
}

\begin{figure}[!h]
  \centering
  {%
    \setlength{\fboxsep}{0pt}%
    \framebox[\figwidth]{\insertfig{illustration/illus_preamble_smol.jpg}}
  }%
  \captionT{Les couleurs du Gulf Stream par satellite}{%
    Image en fausses couleurs de l'océan Atlantique Nord, prises par \as{modis} \sur~Terra le \frenchdate{2020}{02}{23}, inspiré par le travail de Norman Kuring pour \glshref{eo-illustration}.
  }
  \label{fig:oc-illustration}
\end{figure}

\vspace{1\baselineskip}

\section{Préambule}

L'essort de l'imagerie satellite appliquée à la couleur de l'océan \encadra{depuis la fin du \siecle{20}~siècle} a révélé la grande variabilité de la biologie aux échelles les plus fines de l'océan.
Cette variation de la couleur de l'océan, comme illustrée sur la \nref[figure]{fig:oc-illustration}, est due aux pigments du phytoplancton, une collection de micro-organismes phytosynthétiques portés par les courants.
À la base de la chaîne trophique océanique, et partie centrale des cycles biogéochimiques océaniques \encadra{dont celui du carbone} la compréhension des phénomènes et facteurs régissant les évolutions du phytoplancton est cruciale.

Comme pour les plantes terrestres, le phytoplancton a besoin de lumière et de nutriments.
La lumière ne pénètre que dans la couche dite euphotique, profonde d'une centaine de mètres environ, mais les nutriments se trouvent être plus abondants en profondeur.
Le phytoplancton consomme rapidement les nutriments inorganiques en surface par photosynthèse, crééant ainsi de la matière organique.
Une partie de cette dernière est exportée en profondeur à la mort de ces organismes, principalement par gravité.
En profondeur elle est lentement reminéralisée en nutriments inorganiques par des organismes anaérobies (dont le métabolisme peut d'ailleurs mener à des zones très pauvres en oxygène).
Les échanges verticaux \encadra{de nutriments vers la couche euphotique, et de matière organique (ensuite reminéralisée) vers les profondeurs} sont ainsi nécessaires à la conservation d'un équilibre dans cette pompe biologique.

L'ensemble des courants de grande échelle (\al{cad} des bassins océaniques \OM(\qty{1000}{\km})) définit une cartographie des caractéristiques biophysiques de l'environnement (\cite{omand_2013,omand_2015a}).
On peut distinguer par exemple les \guil*{deserts} que sont les gyres subtropicales, où les faibles échanges verticaux engendrent un milieu très oligotrophe et peu productif.
À l'inverse, les zones d'upwelling de bord est sont de véritables \guil*{forêts}, très productives, en raison d'apports en nutriments depuis l'océan profond.
Une telle séparation à grande échelle entre deux biomes est par exemple bien visible sur la \nref[figure]{fig:oc-illustration}, où les eaux au sud du Gulf Stream sont peu productives (en bleu foncé), et celles au nord apparaissent beaucoup plus productives (en turquoise).
Toutefois, sur cette même image, il est également évident que des processus façonnent le paysage biologique à de plus petites échelles.
Par exemple, deux tourbillons formés dans les méandres du Gulf Stream apparaissent en bleu foncé, ayant capturé des eaux du sud.

De tels tourbillons, aux tailles de \emph{mésoéchelles} \OM(\qtyrange{10}{100}{\km}), se forment partout dans l'océan, et leurs effets sont multiples.
Comme on l'a vu dans l'exemple ci-dessus, ils peuvent \guil*{capturer} des masses d'eau et les transporter sur de larges distances (ici en transportant des masses d'eaux chaudes et oligotrophes vers le nord, c'est-à-dire dans un milieu plus froid, productif, et riche en nutriments) (\cite{lehahn_2011}).
Ces tourbillons amplifient localement les échanges entre surface et profondeur, à la fois en générant des vitesses verticales, et en déplacant les isopycnes (\cite{mcgillicuddy_1998}).
Néanmoins, ces échelles ne suffisent pas à décrire toute la variabilité biologique observée. De plus, leur contributions (estimées) à la pompe biologique ne permettent pas de clôturer le budget de cette dernière.

En zoomant encore un peu plus sur la \nref[figure]{fig:oc-illustration}, nous pouvons distinguer des structures plus fines, dites de \emph{submésoéchelle} \OM(\qtyrange{1}{10}{\km}).
Les champs de différentes variables (comme la température, la densité ou la chlorophylle) sont mélangés, étirés, par les courants des échelles supérieures et les forcages atmosphériques.
Apparaîssent ainsi des structures plus fines et plus éphémères (de temps caractéristique entre le jour et la semaine) (\cite{thomas_2008,mcwilliams_2016}).
Parmis ces structures on trouve notamment des fronts de densité propices à la formations de circulations secondaires verticales, plus localisées mais aussi plus intenses que celles engendrées par les tourbillons de mésoéchelle.
Ces circulations, pour des fronts suffisament intenses, peuvent s'étendre en profondeur et alors  remonter des nutriments dans la couche euphotique, augmentant localement la productivité (\cite{mahadevan_2016,levy_2018}).

Une autre conséquence de ces gradients, due à leur tendance à applanir les isopycnes par instabilité barocline, est de restratifier localement les couches supérieures (\cite{fox-kemper_2008}).
La restratification étant un facteur fort dans le démarrage du bloom printanier, ces structures de submésoéchelles peuvent donc également conduire à des démarrages précoces (\cite{mahadevan_2020}).

Enfin, bien que les effets de la submésoéchelle soient locaux, ils peuvent néanmoins avoir une rétroaction sur les échelles supérieures, modifiant la circulation de grande échelle, ou la productivité d'un bassin océanique (\cite{levy_2012a,balwada_2022}).

Les effets de la submésoéchelle sur les cycles biogéochimiques présentés ci-dessus ne sont toutefois pas encore complétement élucidés, malgré les efforts des dernières décennies dans ce domaine.
En effet, à ces échelles, les diverses méthodologies usuelles peinent à donner une vue entière du problème.
Les structures qui nous intéressent sont fines et ont une courte période de vie, ce qui rend leur échantillonage  un défi technique certain.
Les campagnes in-situ doivent déployer des moyens importants pour échantilloner les variables physiques et biologiques, ainsi que des techniques nouvelles (comme l'échantillonage lagrangien ou un cytomètre de flux automatisé \cite{thyssen_2015,marrec_2018,tzortzis_2021}).
Il en va de même pour l'imagerie satellite, pour laquelle la couverture nuageuse rend difficile de suivre temporellement une structure.
Par ailleurs, il est compliqué pour les satellites d'accéder à toute la biodiversité du phytoplancton, ainsi qu'au delà des premiers mètres en surface.
Enfin, les simulations numériques peuvent résoudre ces fines échelles, mais à un coût calculatoire élevé et prohibitif sur de trop longues durées (des projections climatiques \ab{par-ex}) (\cite{fox-kemper_2019}).

Malgré les limites évoquées, l'imagerie satellite donne l'opportunité d'observer ces effets de fine échelle, sur de larges régions et de longs temps, et permettrait ainsi de quantifier leurs effets globaux.
Une telle démarche a été entreprise par \textcite{liu_2016} dans la gyre subtropicale du Pacifique Nord.
Les autrices \encadra{en colocalisant les valeurs satellites de la \al{chl} avec les positions de fronts détectés à partir de la SST (température en surface)} ont montré une augmentation des valeurs de \as{chl} dans les fronts par rapport au reste de la zone.
Nous reprenons leur méthode de détection des fronts, consistant à calculer un indice d'hétérogénéité de la SST, et l'appliquons à la région de l'Atlantique Nord, autour du Gulf Stream.

Cette zone comprend trois régimes biogéochimiques différents (\cite{bock_2022}):
Au sud, notre zone d'étude englobe une partie de la gyre subtropicale de l'Atlantique Nord, caractérisée par un régime oligotrophe et une faible productivité tout au long de l'année.
Au nord du jet du Gulf Stream se trouve un régime subpolaire plus productif, caractérisé par une bloom printanier.
Entre les deux est présent un régime modérément productif, avec une productivité maximale en hiver.

Par ailleurs cette zone présente deux fronts permanents intenses, un front le long du talus continental, et le Gulf Stream, tous deux associés à des circulations verticales fortes et profondes (\cite{flagg_2006,liao_2022}).
Mais l'on peut également trouver dans le reste de la zone des fronts moins intenses, d'une durée de vie de l'ordre de la semaine, et en constante évolution (\cite{drushka_2019,sanchez-rios_2020}).

Afin de mieux définir nos objectifs, et avant de les présenter, nous dressons dans la section suivante un état de l'art des sujets concernés.
Nous exposerons ensuite le plan de ce manuscrit.

\section{Le phytoplancton dans le système terre}
\label{sec:phyto-ds-sys-terre}

% La biologie marine, à sa base, repose sur les conversions d'un certain nombre d'éléments entre eux.
% Schématiquement, le phytoplancton converti par photosynthèse le carbone et les nutriments dissouts dans l'eau, en dioxygène et en matière organique (pour sa croissance).
% À la mort de ces organismes, ce stock de matière organique inerte plonge par gravité, et sera éventuellement reminéralisé plus en profondeur par d'autres organismes.
% Le carbone et autres nutriments libérés pourront alors être remontés à la surface et réutilisés, complétant la pompe biologique marine.

Commençons par définir notre objet d'étude principal: le phytoplancton.
Conforme à sa construction étymologique, ce terme désigne un ensemble d'organismes pratiquant la photosynthèse c'est-à-dire des plantes (\emph{phyto-}), et dérivant au gré des courants (de \emph{plankton}: errer).
Comme pour les plantes terrestres, ces organismes sont influencés par leur environment.
Cependant cet environment est en perpétuel changement du fait des mouvements des masses d'eau dans l'océan.
Dans cette section, on décrira d'abord succinctement les caractéristiques biologiques générales du phytoplancton et les moyens de l'observer.
Dans la section suivante, on s'intéressera plus en détail aux intéractions biophysiques entre le phytoplancton et les courants océaniques.

\subsection{Définition générale}
\label{sec:phyto-def-gen}

Le plancton est donc l'ensemble des organismes dont la motilité ne leur permet guère plus que de se laisser dériver au gré des courants.
Nous nous concentrons ici sur une partie d'entre eux: le phytoplancton.
Ce sous-ensemble est remarquablement divers, totalisant environ \num{20 000}~espèces, dont les tailles varient de la fraction de micromètre pour les plus petites cyanobactéries, jusqu'au millimètre pour les plus grands dinoflagellés.
Cette diversité se retrouve aussi bien dans leurs formes, au grand bonheur des zoologues et photographes, que dans leurs fonctions écosystémiques.
Les organismes phytoplanctoniques sont des producteurs primaires, c'est-à-dire qu'ils obtiennent leur énergie par photosynthèse à partir de carbone inorganique.
Ils transforment le dioxyde de carbone et les nutriments dissous (Nitrate, Phosphate, Silice, Fer,\dots), en matière organique et en oxygène.
À l'instar des plantes terrestres, cette opération nécessite de capter de la lumière à l'aide de pigments.
Le pigment principalement utilisé est la \al{chl}, mais certains organismes possèdent également des pigments auxiliaires comme la fucoxanthine chez les diatomées ou la péridinine chez les dinoflagellés.
Ce sont ces pigments qui affectent la couleur de l'océan comme nous l'avons vu dans le préambule (\nref{fig:oc-illustration}), point sur lequel nous reviendrons plus tard.

Le phytoplancton est consommé par le zooplancton. La matière organique qu'ils ont créée circule ainsi dans le réseau trophique: zooplancton, petits prédateurs, grands prédateurs, oiseaux marins, mamifères marins,\dots
Le phytoplancton occupe donc une place centrale dans l'écosystème marin.
Il est à la base de la chaîne alimentaire dans les océans, et il est estimé qu'il génère~\pct{50} de la production primaire mondiale.

La croissance du phytoplancton est favorisée, selon les espèces, par des conditions environnementales comme la température, la salinité ou l'acidité (ref).
Cependant les limitations principales à cette croissance restent la lumière et les nutriments.
La lumière n'est pas distribuée identiquement dans les océans: l'ensoleillement dépend de la saison, de la latitude, de la présence de glace de mer, et surtout: de la profondeur.
Elle ne pénètre en effet qu'en surface, dans la \emph{couche euphotique} profonde en général d'une centaine de mètres.
Les nutriments, à l'inverse, sont rapidement consommés par le phytoplancton en surface et sont donc trouvés plus en profondeur, sous la \emph{nutricline}.
Cette répartition des ressources induit dès lors une grande importance des échanges verticaux entre la surface ensoleillée, et les eaux profondes riches en nutriments.

\review{définir pompe biologique}

\begin{figure}
  \centering
  \insertfig[0.8]{levy_2023_fig1.pdf}
  \captionT{Schéma de la pompe biologique}{%
    Tiré de \cite{levy_2023}, fig.~1.

    \engquote{%
      Illustration of the influence of finescales on marine biogeochemical cycles~(red colors).
      (Top, surface view)~Sea surface chlorophyll on 22/04/2007 from \as{modis}-aqua ocean color satellite \abbrv{L2} data binned on \qty{1}{\km} grid.
      Stirring by mesoscale eddies and submesoscale fronts creates finescale patterns in the phytoplankton distribution.
      Overlaid are the model grids used in high-resolution Ocean General Circulation Models (OGCMs), and in coarse resolution Earth System Models (ESMs).
      The scales resolved by monitoring plateforms (in yellow) is also shown.
      (Bottom, cutaway view)~Schematic representation of ocean biogeochemical cycles driven by the large scale circulation~(large grey loop).
      Finescales impact these cycles through local eddy-fluxes and through upcale feedback which modifies the large-scale transport.}
  }
  \label{fig:pompe-bio}
\end{figure}

\subsection{Télédétection}
\label{sec:teledetection}

L'observation du phytoplancton comporte un certain nombre de défis que nous allons rapidement évoquer dans les sections suivantes.

Comme pour d'autres variables géophysiques, l'utilisation de l'imagerie satellite a permis de fournir depuis ces dernières décennies une vision synoptique du phytoplancton.
En effet cette dernière offre une couverture globale, journalière, et à haute résolution spatiale.
La possibilité d'observer des micro-organismes depuis l'espace n'apparaît pourtant pas comme évidente.
Rappelons alors que ces organismes, réalisant de la phytosynthèse, contiennent différents pigments leur permettant de récupérer l'énergie lumineuse du soleil.
Toutes les espèces de phytoplancton contiennent majoritairement de la \al{chl}~(\as{chl}) et, éventuellement, des pigments secondaires.
Ces pigments absorbent plus ou moins certaines longueurs d'onde du rayonnement pénétrant dans la couche de surface de l'océan, affectant ainsi la rétrodiffusion (\engquote{backscatter}).
En d'autres termes, la couleur de l'océan apparaît plus verte\footnote{%
  la chlorophylle, pigment majoritaire, absorbe dans le rouge et le bleu}
sur les zones où la concentration en phytoplancton est élevée.

Ce phénomène, compris depuis longtemps, fut testé avec succès par des mesures en avion: \ab{par-ex} par \textcite{clarke_1970} dans la région du Gulf Stream, au large de Cape Cod; puis par satellite par John Arvesen et Dr.\ Ellen Weaver\footnote{%
  Je conseille d'ailleurs l'article suivant qui s'attarde sur sa carrière scientifique passionnante (qui inclut un passage par le projet Manhattan !): \citetitle{marshall_2010}, \cite{marshall_2010}.},
en coopération avec le capitaine Cousteau à bord du Calypso\footnote{%
  Ce dernier commentera d'ailleurs: \engquote{The space-age satellites have opened a whole new dimension to ocean resources monitoring}, comme reporté par la presse locale de l'époque:
  \citetitleurl{macomber_1973}, \cite{macomber_1973}.
}.
Ces premières expériences permettront le déploiement d'un premier capteur dédié à l'étude de la couleur de l'océan: \as{czcs} à bord de Nimbus-7, lancé en 1978 qui fonctionnera jusqu'en 1986.
Ce n'est que dix plus tard, en 1997, que le lancement de \as{seawifs} marquera le début d'une observation ininterrompue de la couleur de l'océan par un ensemble de capteurs et satellites.

Attardons nous brièvement sur le processus permettant d'obtenir une concentration en \al{chl} à partir des données de ces capteurs.
Ce derniers mesurent l'intensité lumineuse renvoyée, ou réflectance (\as{rrs}), pour différentes bandes de longueurs d'ondes (entre 6 et 16 dans le visible).
Le satellite mesure les réflectances en dehors de l'atmosphère, et non pas à la surface de l'eau; il est nécessaire d'apporter plusieurs corrections.
Les contributions de l'écume et du reflet direct du soleil sont retirées.
La diffusion de Rayleigh est corrigée assez simplement, mais l'absorption et diffusion par divers aérosols sont plus complexes, leurs répartitions (horizontale et verticale) étant variables (\cite{werdell_2018}).

Nous voulons maintenant relier ces réflectances au contenu de la surface de la colonne d'eau.
En toute rigueur, il faudrait relier les réflectances aux propriétés optiques de la colonne d'eau: c'est-à-dire l’absorption et la rétrodiffusion causées par le phytoplancton, mais aussi les particules non-algales (\as{nap}) et les matières organiques dissoute colorées (\as{cdom}) (\cite{werdell_2018}).
Certains algorithmes dits \emph{semi-analytiques} s'attachent à résoudre ce problèmes inverse, ce qui nécessite un certain nombre de simplifications, mais permet en théorie de remonter à tous les composants optiques en jeux (\as{chl}, \as{nap}, \as{cdom}) (\cite{werdell_2019}).

Cependant, dans la majorité de l'océan ouvert il est raisonnable de considérer que le contenu optique de la colonne d'eau est le seul fait du phytoplancton\footnotemark, et qu'il est donc possible de directement relier les réflectances à la concentration en \as{chl} (\cite{bailey_2006,brewin_2015a}).
\footnotetext{C'est-à-dire qu'on fait l'hypothèse que les autres composantes (\as{nap} et \as{cdom}) sont des produits du phytoplancton, et co-varient avec sa concentration.}
Ainsi, un certain nombres d'algorithmes dits \emph{empiriques} infèrent la \as{chl} en s'appuyant sur une simple relation vérifiée empiriquement par des mesures in-situ.
Par exemple, les algorithmes \guil{OC} utilisent des polynômes de ratio entre les réflectances bleu et verte (voir plus bas).

Cette hypothèse ne tient plus dans certains cas (classifiés comme les \engquote{case~II waters}, où \guil{optiquement complexes}), par exemples les régimes côtiers où la colonne d'eau est riche en sédiments (\cite{bailey_2006,brewin_2015a}), mais certains algorithmes sont tout de même disponibles pour ces cas plus complexes (\cite{gohin_2002}).
On gardera néanmoins en tête ces hypothèses, et le fait que ces relations empiriques (validées sur des données passées) pourraient perdre en validité face à des écosystèmes modifiés par le changement climatique (\cite{dierssen_2010}).
Toutefois, ces modèles empiriques sont de manière générale plus performant que les modèles semi-analytiques (\cite{brewin_2015a}), et sont donc employés pour les données opérationelles.

\begin{technique}
Nous décrivons ici rapidement deux exemples d'algorithmes empiriques, utilisés de manière opérationelle par le groupe NASA \as{obpg}.
L'algorithme \abbrv{OC3/4} se base sur un polynôme d'ordre~4 du ratio des bandes verte~(\(V\)) et bleue~(\(B\)):
\begin{equation} \label{eq:ocx}
  \log_{10}\paren{\am{chl}} =
  \sum_{i=1}^{4} a_i \paren{\log_{10}\paren{
      \frac{ \am{rrs}^{max}\paren{\lambda_B} }
           { \am{rrs}\paren{\lambda_V} }
    }}^i ,
\end{equation}
où \(\am{rrs}^{max}\paren{\lambda_B}\) est la réflectance maximale parmis les bandes bleues, et \(a_i\) des coefficients empiriques adaptés à chaque capteur (\cite{oreilly_1998,oreilly_2000}).
Ces coefficients, régulièrement mis à jour par le groupe NASA \as{obpg}\footnote{voir l'historique \url{https://oceancolor.gsfc.nasa.gov/reprocessing/}}, sont dérivés par regression de la relation ci-dessus avec les données in-situ \as{nomad} (\cite{werdell_2005}).
L'algorithme utilise au total 3~bandes pour \abbrv{OC3}, et 4 pour \abbrv{OC4}.

Cependant, \abbrv{OC3/4} montre des limites dans les eaux oligotrophes, où la concentration de \as{chl} est faible.
Dans ces cas là, l'algorithme CI (\eng{Color Index}) est utilisé (\cite{hu_2012}).
Il utilise la différence entre la bande verte~(\(V\)) et une combinaison des bandes rouge~(\(R\)) et bleue~(\(B\)):
\begin{equation}
  \label{eq:CI}
  \begin{split}
    CI &= \am{rrs}\paren{\lambda_V} -
         \paren{
          \am{rrs}\paren{\lambda_B}
          + \frac {\lambda_V - \lambda_B}
                  {\lambda_R - \lambda_B}
          \times \paren{\am{rrs}\paren{\lambda_R}
                      - \am{rrs}\paren{\lambda_B}}
         }
    \\
    \am{chl} &= 10^{\paren{a_0' + a_1' \times CI}},
  \end{split}
\end{equation}
avec ici aussi des coefficients empiriques \(a_i'\).

L'implémentation actuelle du groupe NASA \as{obpg} utilise \abbrv{OC3/4} pour les eaux où la \as{chl} est supérieure à~\qty{0.35}{\mgm}, et CI où elle est inférieure à~\qty{0.25}{\mgm}; entre ces deux valeurs, un mélange des résultats des deux algorithmes est utilisé assurant ainsi la continuité des valeurs de chlorophylle (\cite{oreilly_2019}, \url{https://oceancolor.gsfc.nasa.gov/atbd/chlor_a/}).

À noter que de nombreux autres algorithmes ont été développés dans les deux dernières décennies, on pourra citer par exemple l'ajout d'une bande violette pour \abbrv{OC5/6} (\cite{oreilly_2019}), ou l'utilisation de 5~bandes et des tables de références afin d'améliorer l'estimation de \as{chl} en zone côtière (\cite{gohin_2002}).
\end{technique}

Comme nous venons de le voir, la dérivation de la concentration de \as{chl} est un processus compliqué, comprenant de multiple sources d'incertitudes dans les valeurs absolues inférées.
Ce processus comporte un certain nombre de limitations.
Bien sûr en premier lieu, la couverture nuageuse bloque la télédétection dans le domaine visible.
Ensuite, nous avons évoqué la difficulté à travailler dans des zones où d'autres éléments que le phytoplancton dicte le comportement optique, comme des sédiments en suspension par exemple.
C'est souvent le cas près des côtes et des décharges fluviales.
La faible profondeur en régime côtier apporte également une difficulté, puisque dans le cas où la lumière incidente pénètre suffisamment profond, la réflectance mesurée comportera la lumière réfléchie par le fond marin (et re-traversant la colonne d'eau).

Plus généralement, quelque soit la zone sondée, la profondeur jusqu'à laquelle pénètre la lumière est un point critique.
En effet, la réflectance mesurée ne peut comporter des informations que sur cette couche éclairée.
De plus, les modèles empiriques sont ajustés par des mesures in-situ réalisées sur les premiers mètres.
Les données satellite manquent ainsi la biomasse plus en profondeur, potentiellement importante (\as{ie} le \engquote{Deep Chlorophyll Maximum}) (\cite{morel_1989,uitz_2006}).

Par ailleurs, nous nous sommes penchés sur la détection de la \al{chl}. Bien que ce soit le pigment majoritaire, ce n'est pas nécessairement un proxy parfait pour la biomasse présente dans la colonne d'eau.
La relation entre la quantité de \al{chl} et la biomasse, exprimée par le ratio \hbox{\as{chl}:C}, peut varier selon les espèces et les conditions environnementales (\cite{behrenfeld_2015,halsey_2015,inomura_2022}).

% Des efforts sont conduits pour extraire la composition de la communauté phytoplanctonique des données de télédétection.

\subsection{Observations in-situ et modèles numériques}

Pour toutes ces raisons, les mesures in-situ constituent un outil nécessaire et complémentaire à la télédétection, à la fois pour l'observation opérationelle mais aussi pour mieux comprendre les processus en jeu.
En effet, au-delà de pouvoir accéder directement aux variables que les satellites doivent inférer, parfois à travers des proxys, ces mesures peuvent explorer des dimensions invisibles aux satellites.

D'abord, les mesures in-situ ont accès à toute la colonne d'eau et non pas seulement la surface optiquement active.
D'autre part, ces mesures offrent une vue beaucoup plus large et précise de la biodiversité présente.
Les mesures \glsmargin{hplc} permettent par exemple d'accéder à la concentration exacte de la \al{chl} mais également d'autres pigments auxiliaires, donnant ainsi une meilleure idée de la composition de la communauté phytoplanctonique.
D'autres techniques permettent d'explorer d'autres communauté, comme le cytomètre de flux pour les plus petits organismes (cyanobactéries et bactéries hétérotrophes), ou le zooscan pour les plus gros organismes (essentiellement le zooplancton).
L'application plus récente des sciences -omiques (génomique, métagénomique, transcriptomique,\dots) fourni également de nouvelles perspectives sur les écosystèmes (\cite{bork_2015,richter_2019,sunagawa_2020}).

Par ailleurs, les mesures in-situ offrent l'avantage de fournir une vision précise de la densité des masses d'eau par mesure de la température et de la salinité.
Les satellites actuels n'ayant accès à la salinité qu'à relativement basse résolution (\qtyrange{35}{50}{\km} pour \as{smos} \ab{par-ex}), ils ne peuvent estimer les variations de densité aux fines échelles qu'avec la température.

Néanmoins, ces mesures présentent également un certain nombre de limitations, et de difficultés technique de mise en place.
Certaines mesures, physique ou biologiques \encadra{notamment les profils en profondeur} prennent du temps, ce qui se répercute sur la distance d'échantillonage, pourtant essentielle dans les fronts, où les communautés peuvent varier sur de courtes distances (\cite{chekalyuk_2012,mangolte_2023}).
Des techniques sont mises en place pour contrer ce problème, comme le cytomètre de flux automatique (\cite{thyssen_2015}).
Quand bien même, il leur est impossible de couvrir une large zone spatio-temporelle.
Il est usuel que les campagnes en mer ciblent une zone d'intérêt en particulier, par exemple un front ou un tourbillon, identifiée pendant la mission par images satellites (un échantillonage dit \guil{Lagrangien}, \cite{marrec_2018,tzortzis_2021}).
Heureusement ces dernières années de telles campagnes se multiplient, déployant un éventail de méthodes de mesures, devant la nécessité d'obtenir une large gamme de paramètres (physiques et biologiques) à de fines échelles (\cite{shulman_2015,barcelo-llull_2021,marrec_2018,freilich_2021,tzortzis_2021,wilson_2021}).

Parallèment aux observations in-situ, un effort est mené pour inclure les phénomènes biogéochimiques dans les modèles numériques d'océans.
Cela comprend un certains nombre d'obstacles.
Nous avons souligné la grande diversité des organismes qui nous intéressent, laquelle est difficile à représenter numériquement.
Chaque compartiment biologique ajoute un traceur à prendre en compte (\ab{ie} à advecter, diffuser,\dots). Le nombre de compartiment est ainsi limité par la puissance de calcul.
Par ailleurs, il existe un grand nombre de paramétrisation des processus purement biologiques, liés à de fortes incertitudes.

Ensuite, et c'est un point que nous allons développer par la suite, la biologie présente une grande variabilité aux fines échelles.
Ces dernières sont cependant coûteuses à intégrer, ce qui limite leur application à de petits régions géographiques ou à des temps d'intégration courts (\cite{kessouri_2020,hewitt_2022}).
En particulier, ces échelles sont absentes des modèles utilisés pour les projections biogéochimiques à long termes (\cite{bopp_2013}), et sont représentées par des paramétrisations encore en évolution (\cite{fox-kemper_2019}).

% Les modèles climatiques sont, du coup, très incertains en ce qui concerne la bio.
% Grosses barres d'erreurs dans les projections pour le prochain siècle.
% Est-ce que c'est seulement possible sachant qu'on peut avoir des effets non-locaux potentiellement ?

% Difficile de séparer les trois: processus dynamiques, mélange horizontal, et biologie, car ils ont les même temps caractéristiques.

\section{Interactions biophysiques}
\label{sec:interactions-biophys}

\subsection{Importance des fines échelles}

Comme nous l'avons évoqué précédemment, les cycles biogéochimiques marins sont particuliers en ce que la production organique et la reminéralisation de la matière organique détritique y sont fortement découplées.
Horizontalement d'une part, un découplage résulte du transport des matières organiques et inorganiques par les courants océaniques (\cite{chabert_2021}).

Verticalement ensuite, un découplage se fait entre la production dans la couche éclairée en surface, et la lente reminéralisation de la matière organique exportée en profondeur principalement par gravité.
Il en découle que des échanges verticaux sont nécessaires pour équilibrer la pompe biologique, notamment en réapprovisionnant la couche euphotique en carbone reminéralisé et en nutriments (ce qui nous intéresse ici), mais aussi en fournissant de l'oyxgène à la zone crépusculaire (consommé lors de la reminéralisation).

\textnote{Un découplage temporel existe également, de quelques mois à quelques années, entre la photosynthèse et la reminéralisation.}

Ces cruciaux échanges verticaux sont fortement dictés par la circulation générale des océans, et donc particulièrement importants dans les régions de convection, de subduction, et d'upwelling.
Ces contrastes dans les approvisionnements façonnent le paysage biogéochimique global, fait de provinces biogéochimiques à l'échelle des bassins (\nref{fig:nutricline-globale}, \cite{longhurst_2007,vichi_2011a,williams_2011,bock_2022}).

Par exemple, le rotationnel négatif des vents au-dessus des gyres subtropicales engendre un approfondissement de la thermocline et la nutricline, donnant ainsi un biome où la productivité est faible tout au long de l'année.
À l'inverse, les plus hautes latitudes \encadra{où la nutricline est moins profonde et le rotationnel des vents positif} sont soumises à une forte évolution saisonnière.
En hiver le mélange vertical est important, des nutriments sont remontés en surface par convection mais la productivité reste relativement faible du fait du faible ensoleillement et par dilution, la couche de mélange étant alors profonde.
Au printemps, la réduction du mélange vertical maintient du phytoplancton dans la zone éclairée en surface occasionnant un bloom: un forte croissance s'appuyant les nutriments remontés en hiver (\cite{wilson_2005,siegel_2002,taylor_2011a,williams_2011}).

\begin{figure}
  \centering
  \insertfig[0.9]{nutricline.pdf}
  \captionT{Répartition du Nitrate à l'échelle globale}{%
    Profondeur à partir de laquelle la concentration de Nitrate dépasse \qty{2}{\micro\mole\per\liter}.
    Données climatologiques annuelles du \eng{\emph{World Ocean Atlas}} (\cite{garcia_2019}).
  }
  \label{fig:nutricline-globale}
\end{figure}

Cependant il apparaît que les échanges à grande échelle seuls ne peuvent expliquer entièrement les flux des éléments entre les différents blocs de la pompe biologique.
L'observation d'écarts entre l'apport de nutriments et l'export de matière \encadra{observé par exemple dans la gyre subtropical oligotrophe de l'Atlantique Nord} suggère que des injections locales de nutriments à des échelles non résolues sont nécessaires pour clôturer le budget (\cite{mcgillicuddy_1998,oschlies_2002}).
L'hypothèse d'abord considérée fut celle de nutriments entraînés par des tourbillons de mésoéchelle (\qtyrange{10}{100}{\km}) (\cite{mcgillicuddy_2016}), qui sont omniprésents dans les océans, en particulier à proximité des courants de bord ouest comme le Gulf Stream, où ils sont générés.

A néanmoins émergé la nécessité de considérer les vitesses verticales associées aux courants de submésoéchelle, dont les valeurs sont bien plus importantes qu'aux mésoéchelles (\cite{klein_2009}).
Ces courants de submésoéchelle, forcés par ceux de mésoéchelle, se présentent sous forme de fronts et filaments de densité, ou de tourbillons, de taille comprise entre \qty{1}{\km} et \qty{10}{\km} (\nref{fig:finescales-photos}).
La dynamique de ces échelles est particulièrement liée aux fronts, comme nous allons le voir plus bas.

\begin{figure}
  \centering
  \makebox[\textwidth][c]{%
    \includegraphics[height=8.5cm]{sun_glint.pdf}%
    \hspace{1pt}%
    \includegraphics[height=8.5cm]{baltic_bloom.pdf}%
  }
  \captionT{La submésoéchelle en photos}{%
    Les tourbillons de submésoéchelle sont découverts dans les photos prises par les premiers astronautes américains. À gauche, un exemple d'image prise à bord de la navette spatiale Endeavour (\abbrv{STS47-94-86}, \frenchdate{1982}{9}{16}), en mer du Japon. Les reflets du soleil (\eng{sun glint}) font apparaître des structures cohérentes où des surfactants produits par phytoplancton sont concentrés (tiré de \cite{munk_2000}).

    À droite, une image en fausses couleurs prise par \textlf{Sentinel-2} le \frenchdate{2019}{7}{20}, lors d'un bloom d'été en mer Baltique, probablement de cyanobactéries.
    Version (très) haute résolution interactive sur le site de l'\glshref[ESA (article \eng{\textit{Baltic blooms}})]{esa-baltic-blooms}.
  }
  \label{fig:finescales-photos}
\end{figure}

Les courants de mésoéchelle forcent ceux de submésoéchelle, mais cette interaction fonctionne également dans le sens inverse, des petites vers les grandes échelles.
Les phénomènes de submésoéchelle, bien que locaux, influencent donc les circulations océaniques de plus grande échelle (\cite{sasaki_2020,balwada_2022,naveiragarabato_2022,taylor_2023}).

\textnote{
  Les méso- et submésoéchelles  étant fortement couplées, il est parfois difficile d'isoler leurs impacts respectifs (\cite{uchida_2019,freilich_2019,balwada_2021}), et elles sont ainsi parfois groupées sous l’appellation de \guil*{fines échelles}.}

Même si les phénomènes aux fines échelles sont relativement bien compris sur le plan dynamique (\cite{mcwilliams_2016,mcwilliams_2019,gula_2022,taylor_2023}), l'évaluation de leurs impacts sur le paysage biogéochimique reste cependant encore un défi.
En effet, nous l'avons vu, mesurer l'impact local de ces fines échelles est difficile techniquement.
En outre, il est nécessaire \encadra{à la fois pour quantifier leur impact sur tout un biome, ainsi que pour tenir compte de leur influence sur la grande échelle} de résoudre ces fines échelles sur un large domaine, ce qui s'avère difficile en utilisant des observations ou des modèles numériques (\cite{fox-kemper_2019,levy_2023}).

Nous nous penchons dans la suite sur quelques uns de ces phénomènes et leurs impacts sur les cycles biogéochimiques.
En particulier l'apport de nutriments par upwelling aux fronts, et le décalage du bloom du fait de la restratification aux fronts.

\subsection{Mélange horizontal}

Abordons d'abord rapidement le mélange horizontal du phytoplancton par advection.
Les masses  d'eau, et donc les traceurs biogéochimiques qu'elles contiennent, sont entrainées par les courants, largement horizontaux.
Ce simple principe (ré"~)organise ces traceurs dans l'espace, sans directement en altérer le contenu\footnote{%
  Dans la plupart des cas, cependant la dispersion d'une parcelle d'eau entraîne néanmoins une dilution pouvant affecter les processus biologiques (\cite{lehahn_2017}.)
}, et génère des structures de taille plus petite que celle l'écoulement lui-même \encadra{comme de fins filaments par exemple} (\cite{abraham_1998,lehahn_2007,dovidio_2010,levy_2018,lehahn_2018}), qui peuvent néanmoins s'étendre sur de larges distances (\cite{sergi_2020}).

De récentes études montrent qu'une large part de la variabilité (spatiale et temporelle) du phytoplancton observée est expliquée par le mélange horizontal (\cite{glover_2018,keerthi_2022,jonsson_2023}).

% \review{isopycnal mixing ? \cite{abernathey_2022}}


\subsection{Les fronts de submésoéchelle}

Les fronts (de submésoéchelle) sont des gradients de densité forcés par les courants de mésoéchelle, et des processus atmosphériques (flux de chaleur et d'eau douce) spatialement hétérogènes.
Les fronts sont renforcés par frontogenèse (\cite{thomas_2008,mcwilliams_2016}).
En simplifiant, un gradient de densité \(\textder{\rho}{y}\) (\nref{fig:frontogenesis}), entraîne la formation d'un jet (\(\vect{u}\)) le long du front, du fait d'un vent thermal (\(\textder{u}{z} = -\textder{\rho}{y} / f\)).
L'intensification du jet augmente le cisaillement latéral et la vorticité verticale \(\zeta\) (en valeur absolue).
Le jet/front se déstabilise et développe des méandres par instabilité barocline, et le nombre de Rossby devient important (\(\am{rossby} = \zeta / f \simeq 1\)).
La perte de géostrophie est contrée par une circulation agéostrophique en travers du front, dans le sens contribuant à applanir les isopycnes.
Ainsi des vitesses ascendantes apparaissent du côté moins dense (plus chaud et/ou moins salé) du front, et à l'inverse descendantes du côté plus dense.
Ces vitesses peuvent atteindre des valeurs de l'ordre de \OM(\qty{100}{\meter\per\jr}) (soit \OM(\qty{1}{\mm\per\s})).

\begin{note}
  Du fait de divers phénomène dynamiques, la vorticité générée tend à être disproportionnément positive, \al{cad} plus cyclonique que anticyclonique.
\end{note}

\begin{figure}
  \centering
  \insertfig[0.6]{front_circulation.pdf}
  \captionT{Frontogénèse et circulation de retournement}{%
    Schéma 3D d'un front avec circulation secondaire
  }
  \label{fig:frontogenesis}
\end{figure}

Les fronts de submésoéchelle sont généralement confinés à la couche de mélange.
En été, lorsque la couche de mélange est peu profonde, l'énergie potentielle qui leur est disponible est faible.
En revanche en hiver les forts gradients de densité se prolongent sur toute la hauteur de la couche de mélange, alors plus profonde, et l'énergie potentielle de ces structure en est d'autant plus grande (\cite{mensa_2013,callies_2015,buckingham_2016,sasaki_2020}).
Cette dernière est convertie en énergie cinétique par le biais de l'instabilité de couche de mélange (\engquote{Mixed Layer Instability} ou parfois \engquote{Mixed Layer Eddies}) (\cite{boccaletti_2007,fox-kemper_2008}).
Cette instabilité de type barocline tend à aplanir les isopycnes sur toute la hauteur de la couche de mélange en créant des tourbillons de submésoéchelles, réduisant ainsi le mélange vertical et favorisant la restratification.

\begin{note}
  Le mode de croissance maximale de cette instabilité est de taille \(L = \frac{NH}{f}\), avec \(N \simeq \qty{e-3}{s^{-1}}\) la fréquence de Brunt-Väisälä relative à la couche de mélange, \(H \simeq \qty{200}{\meter}\) la hauteur de la couche de mélange, et \(f \simeq \qty{e-4}{s^{-1}}\) le paramètres de Coriolis aux moyennes latitudes.
  Ce qui donne donc typiquement \(L \simeq \OM(\qty{1}{\km})\).
  Le temps caractéristique de croissance est celui de l'advection horizontale \(T = U/L\). Étant aux échelles où le nombre de Rossby \(\am{rossby} = \frac{U}{Lf}\) est d'ordre \OM(1), ces tourbillons se développent donc en quelques jours.
\end{note}

Les fronts de submésoéchelles causent donc des vitesses verticales importantes, et tendent à restratifier localement la couche de mélange en créant des tourbillons.

\subsection{Apport de nutriments aux fronts}
\label{sec:upwelling-nutriments}

La dynamique de submésoéchelle affectent également les cycles biogéochimiques par les circulations verticales qui se développent aux fronts.
La branche ascendante de ces circulation peut potentiellement remonter des nutriments depuis les eaux profondes vers la couche euphotique où ils sont rapidement consommé par le phytoplancton, augmentant localement la productivité (\nref{fig:nutrient-upwell}, \cite{klein_2009,calil_2011,mahadevan_2000,mahadevan_2016,mcwilliams_2016,levy_2001,levy_2012,levy_2018}).

\begin{figure}
  \centering
  \insertfig[0.8]{levy_2018_fig5.pdf}
  \captionT{Apport de nutriment par les vitesses verticales}{%
    Tiré de \textcite{levy_2018}, figure~5.

    \foreignblockquote{english}{ \small
      (a)~Sea surface temperature (SST), (b)~sea surface chlorophyll (Chl), and vertical velocities (red = upwelling, blue = downwelling) (c)~in the surface mixed-layer (at \qty{25}{\m} depth), and (d)~below (at \qty{200}{\m} depth) in a \qtyproduct{500 x 500}{\km} box from a larger ocean general circulation model representing the Gulf Stream region.
      (e)~A vertical section taken along the black line of (a–d) showing vertical velocity (blue-red), with nutrient contours (black lines) and mixed-layer depth (heavy blue lines) overlaid.
      % Vertical velocities associated with the persistent front (meandering Gulf Stream, large black arrows) extend from the surface and penetrate below the nutricline (delinated as the first nutrient contour).
      % The near-surface vertical velocity field also exhibits thin, elongated submesoscale structures, located at submesoscale temperature fronts (small black arrows).
      % Unlike the vertical velocities associated with the persistent front, the vertical velocities at submesoscale fronts do not extend much below the surface mixed layer, and do not always reach the nutricline.
      % This results in a response of surface phytoplankton at the deep persistent front (yellow color show maximum phytoplankton concentrations at the front, large black arrows) but not at the shallow submesoscale front (small black arrows)
      }
  }
  \label{fig:nutrient-upwell}
\end{figure}

L'impact local de ces vitesses verticales a largement été observé in-situ, grâce à des mesures de nutriments, ou de productivité (\cite{mourino_2004,allen_2005,johnson_2010,li_2012,shulman_2015,marrec_2018,little_2018,verneil_2019,ruiz_2019,uchida_2020,kessouri_2020,tzortzis_2021,wilson_2021}).
Ces campagnes sont difficiles à réaliser, étant donné qu'il est nécessaire d'échantilloner à haute résolution spatiale et temporelle pour capturer la variabilité de submésoéchelle.
Il est également nécessaire de cibler précisemment les structures d'intérêt (ici les fronts) qui sont, par essence, éphémères et en constante évolution.
La structure doit donc être repérée et suivie, par le biais \encadra{en temps réel} d'observations satellites, de leur analyses (en appliquant une détection de fronts \ab{par-ex}), ou de simulations numériques (stratégie du \engquote{Lagrangian sampling} \cite{marrec_2018,tzortzis_2021}).

La mesure des vitesses verticales en jeu est également compliquée.
Ces courants ascendants n'occupent pas une surface importante et sont clairsemés, même le long d'un front.
En outre, les vitesses verticales dans ces sites d'upwelling sont \encadra{bien qu'importantes relativement au reste de l'océan} trop faibles pour être directement mesurées facilement.
Ces vitesses sont généralement plutôt inférées à partir d'autres mesures sur le terrain (\ab{par-ex} \cite{dasaro_2018,tarry_2021,comby_2022,cutolo_2022}).
Notamment, elles peuvent être déterminée à partir du champ de vitesses horizontales en utilisant l'équation \guil{omega} (\cite{hoskins_1978,pietri_2021}).
La mission \as{swot}, lancée en décembre 2022, promet d'ailleurs de pouvoir accéder à ce champ de vitesses horizontales à une résolution sans précédent (\cite{dovidio_2019,morrow_2019,barcelo-llull_2021}).

Au-delà de la magnitude des vitesses verticales, cet apport de nutriment est également conditionné par l'extension en profondeur des circulations agéostrophiques.
Les fronts les plus intenses s'étendent jusque sous la nutricline, permettant le transport de nutriments (\cite{levy_2001,thomas_2013,pasquerondefommervault_2015,capet_2016}).
C'est par exemple le cas des fronts persistents tels que le Gulf Stream (\cite{levy_2012a}).
À l'inverse, les fronts plus faibles, éphémères, et leurs circulations sont typiquement confinés à la couche de mélange, n'affectant pas directement la productivité (\cite{ramachandran_2014,levy_2018}).

Si nous avons considéré jusqu'ici la branche ascendante des circulations agéostrophiques, les vitesses verticales descendantes du côté froid des fronts peuvent exporter de la matière dans l'océan intérieur.
Ces courants descendants peuvent ainsi limiter la productivité en transportant organismes et nutriments en profondeur (\cite{mcgillicuddy_2003,lathuiliere_2010,gruber_2011,levy_2012a,resplandy_2019}).
Cela a également pour effet d'exporter du carbone organique (\cite{levy_2001,omand_2015}) et d'apporter de l'oxygène à l'océan intérieur (\cite{resplandy_2012}).

Nuançons cependant ces effets en commençant par rappeler que les zones d'upwelling et de downwelling sont séparées spatialement, les nutriments remontés ne sont donc pas directement renvoyés en profondeur.
Par ailleurs, le gradient vertical en nutriment \encadra{maintenu par la consommation rapide des nutriments en surface} assure que les eaux remontées sont riches en nutriments, et que les eaux exportées en profondeur sont pauvres en nutriments (\cite{mahadevan_2016}).

\review{Effets sur la communauté.}

L'effet \emph{local} de l'apport de nutriments aux fronts a été largement observé dans les études ci-dessus.
Cependant il est difficile de quantifier précisemment cet effet sur les cycles biogéochimiques à plus grande échelle.
Les observations in-situ ne peuvent observer qu'un nombre limité de structures, et ont d'ailleurs tendances à cibler les plus intenses.
Les modèles numériques peuvent résoudre les circulations de submésoéchelles mais contre un fort coût calculatoire qui limite leur étude avec des modèles régionaux ou globaux\footnote{%
  En particulier, la modification de la circulation grande échelle par rétroaction des dynamiques de fines échelles nécessite un long temps de spin-up.
} (\cite{kessouri_2020,hewitt_2022}).
Les données satellites offrent une vue globale synoptique avec une résolution temporelle journalière, et une résolution spatiale qui s'approche du kilomètre.
Malgré certaines lacunes discutées précédemment (\nref{sec:teledetection}), elles offrent l'opportunité de mesurer les effets de fine échelle sur de larges zone et de longues périodes.

Une première tentative d'analyse géostatistique à partir de données de \al{chl} à une résolution de \qty{9}{\km}, par \textcite{doney_2003} puis étendue par \textcite{glover_2018}, a visé à évaluer l'évolution de la variance spatiale en fonction de la distance.
L'impact des processus frontaux n'apparaît pas à cette résolution, mais la méthodologie a confirmé le rôle important des tourbillons de mésoéchelle dans le mélange des gradients grande échelle.

\Textcite{jonsson_2011} ont proposé un cadre lagrangien dans lequel les données satellitaires de couleur de l'océan\footnote{%
  La \al{chl} est combinée avec l'atténuation (\(K_{490}\)) pour estimer le carbone organique et son évolution (\ab{ie} la production nette).
} sont projetées sur des trajectoires calculées à partir d'un modèle numérique opérationel haute résolution.
Ils ont pu évaluer le changement de biomasse le long des trajectoire et le relier aux apports de nutriments.
Un inconvénient majeur est que la méthode est fortement limitée par la quantité de données satellites disponibles le long des trajectoires (sans nuages).
\Textcite{zhang_2019} contourne ce problème moyennant sur un grille de \ang{2}, en utilisant par ailleurs des données globales de dériveurs de surface, d'altimétrie et de couleur de l'océan.
Leur analyse révélent une corrélation positive entre le taux de cisaillement de l'écoulement et la croissance du phytoplancton.

D'autre part, le rôle des fronts a été évalué à l'aide de trois méthodes différentes.
\Textcite{guo_2019} ont combiné des données de couleur de l'océan avec des données d'altimétrie et de flotteurs dérivants, et ont estimé que, dans les gyres subtropicales, les dynamiques de mésoéchelle et les dynamiques frontales de submésoéchelle contribuaient de manière comparable aux anomalies positives de \as{chl}.
\Textcite{keerthi_2022} ont proposé une approche basée sur la déconvolution des séries temporelles de \as{chl} (obtenue par satellite), en différentes échelles temporelles; ils ont observé que les échelles temporelles sub-saisonnières contribuaient à environ \pct{30} de la variance totale de \as{chl} et étaient associées à de petites échelles spatiales (\qty{<100}{\km}) \encadra*{qui comprenennent à la fois la mésoéchelle et la submésoéchelle}.

Enfin, l'approche la plus quantitative, et la seule s'appuyant à une colocalisation directe avec les fronts, a été proposée par \textcite{liu_2016}, qui l'ont appliquée à la gyre subtropicale du Pacifique Nord.
Ils ont détecté les fronts de température en calculant un indice mesurant l'hétérogénéité locale du champ de SST à partir de données satellitaires.
Ils ont ensuite pu comparer les valeurs de \as{chl} (satellite) dans les zones impactées par les fronts (caractérisées par une valeur élevée de l'indice d'hétérogénéité) avec les valeurs dans les zones non touchées.
Ils ont constaté une augmentation de la \as{chl} dans les fronts, négligeable en été mais qui atteignait près de \pct{40} en hiver.

\subsection{Modification de la phénologie du bloom}
\label{sec:modif-phenologie}

Nous avons abordé plus haut le déclenchement d'un bloom au printemps, aux hautes latitudes, du fait de la réduction du mélange vertical.
Cette hausse de productivité a précédemment été reliée à l'abattement des vents et au réchauffement de la surface par échange avec l'atmosphère (\cite{henson_2006,taylor_2011a}).
Les fronts de submésoéchelle favorisent également la restratification, et peuvent ainsi provoquer des blooms, localement, avant la restratification saisonnière à l'échelle du biome (\cite{taylor_2011,karleskind_2011,mahadevan_2012}).

Cet aspect de l'impact des fronts a été quantifié par \textcite{mahadevan_2012} en comparant l'évolution de la \al{chl} dans deux catégories de simulations numériques forcées de manière à correspondre à des observations récoltés dans l'Atlantique Nord. Les simulations de l'une de ces catégorie sont initialisées avec un gradient horizontal de densité de grande échelle, ce qui entraîne une restratification par instabilité de couche de mélange.
Les simulations avec gradient (donc restratifiées par la dynamique de submésoéchelle) présentent un bloom en accord avec les observations, et qui survient entre 20 et 30~jours avant les simulations sans gradient (\nref{fig:early-bloom-mahadevan}).

\begin{figure}
  \centering
  \insertfig[0.6]{mahadevan_2012_fig3.pdf}
  \captionT{Estimation de l'avance du bloom dans un modèle}{%
    Tiré de \textcite{mahadevan_2012}, figure~3G.

    \foreignblockquote{english}{%
      Model chlorophyll concentration in the upper \qty{100}{\m} (average) [from runs with lateral density gradient (shaded gray) and without (purple)].
      Chlorophyll from the combination of observations [\dots] is overlaid.
    }
  }
  \label{fig:early-bloom-mahadevan}
\end{figure}

\section{Région d'étude: Extension du Gulf Stream}
\label{sec:region-detude}

Dans la section précédente, nous avons relié les dynamiques de submésoéchelle à des surplus de productivité dans les fronts, et à la modification de la phénologie du bloom.
Ce surplus de productivité dû à un front est dépendant de l'intensité de ce front et de la distribution verticale locale en nutriments.
Nous pouvons nous attendre à des variations saisonnières et régionales de cet effet.
En particulier il est attendu que ce dernier soit maximal dans les biomes oligotrophes où les apports de nutriments (autres que par advection verticale dans les fronts) sont faibles.
En outre la dynamique du submésoéchelle est sujette à des variations saisonnières (\cite{callies_2015}) et régionales (\cite{mauzole_2022}).

Nous visons à explorer et quantifier les impacts des fines échelles sur la \al{chl} \emph{à l'échelle des biomes}, et leurs fluctuations \emph{au cours de l'année}, \emph{pour différents biomes}, et \emph{pour différents intensité de fronts}.
À cette fin nous nous concentrons sur la région de l'Atlantique Nord autour du Gulf Stream, qui comporte dans une zone restreinte plusieurs biomes contrastés, et des fronts d'intensité variée (\cite{bock_2022}).
Elle s'étend de \latlon{15N} à \latlon{55N}, et de la côte américaine (\latlon{82W}) jusqu'à \latlon{40W} (\nref{fig:region}).

\begin{figure}
  \centering
  \insertfig{zone.pdf}
  \captionT{Région d'étude}{%
    Climatologies annuelles de la température de surface (SST satellite)~(à gauche), et de la \al{chl} satellite~(à droite).
    Les isobathes sont surimposés en gris.
  }
  \label{fig:region}
\end{figure}

\subsection{La physique}
\label{sec:gs-physique}

Cette région d'étude est caractérisée par la présence du Gulf Stream, un courant de bord ouest.
Ce courant de surface intense, forcé par les vents dominants et la topographie (\cite{fieux_2010}), remonte des eaux chaudes et salées vers le pôle le long de la côte est américaine avant de s'en décrocher à Cap Hatteras (\latlon{35N}, \latlon{75W}, cercle bleu \nref{fig:region}).
Le jet continue alors vers l'est. Instable, il forme de larges méandres qui se détachent parfois en formant des tourbillons de mésoéchelle.
Le jet est encadré au nord et au sud par deux contre-courants.

Les tourbillons froids (cycloniques) formées par le jet dérivent vers le sud puis l'ouest avant de rejoindre le Gulf Stream, formant ainsi une recirculation.
Ces tourbillons transportent une importante énergie cinétique dans la mer des Sargasses (\nref{fig:biomes}a, \cite{wunsch_1998,zhai_2008}).
Nous pouvons donc nous attendre à la production de fronts de submésoéchelle éphèmeres mais intenses, forcés par la circulation énergétique de mésoéchelle.

Au nord du jet, les eaux chaudes rencontrent les eaux de pentes, bien plus froides et fraîches.
La limite entre les deux, le \engquote{North wall}, est marquée par une variation dramatique de la température (jusqu'à une douzaine de degrés Celsius en quelques dizaines de kilomètres à peine), mais aussi de la salinité et du niveau de la mer.
Les eaux au nord, les \engquote{slope waters}, font partie d'une circulation orientée vers l'équateur et qui s'étire le long de la côte nord-est américaine et rejoint le courant du Labrador (\cite{fieux_2010,townsend_2004}).
En particulier cette circulation présente un jet et front permanent le long du talus continental (\cite{linder_1998,flagg_2006}).
Comme nous nous intéressons à l'océan ouvert nous ne détaillons pas plus en profondeur la dynamique sur le plateau continental.

Enfin cette région englobe une partie de la gyre subtropicale.
Cette dernière, forcée par les Alizées au sud et les vents d'ouest au nord, ne présente pas de limite très marquée.
Du fait du rotationnel négatif des vents dominants, le cœur de la gyre, loin de l'influence du Gulf Stream, est caractérisé une circulation anticyclonique et un approfondissement de la thermocline et de la nutricline.

\subsection{Productivité primaire}
\label{sec:gs-biologie}

Comme expliqué ci-dessus, la gyre subtropicale présente une nutricline profonde, sans dynamique particulière permettant un apport important de nutriment.
La concentration en nutriments, la productivité et l'abondance en phytoplancton y sont basses tout au long de l'année.
Nous désignons ainsi cette région comme le biome subtropical permanent (PSB) (\nref{fig:biomes}), aussi désigné par le passé comme le \engquote{subtropical gyre permanently stratified biome} (\cite{sarmiento_2004}).

À l'inverse, les eaux de pentes au nord du Gulf Stream sont riches en nutriments et très productives.
Cette production est accrue lors d'un bloom printanier lié à la restratification de la couche de mélange.
Nous qualifions cette région entre la limite nord du Gulf Stream et le talus continental comme le biome subpolaire (PB), aussi désigné par \engquote{subpolar waters} (\cite{sarmiento_2004}) ou \engquote{high chlorophyll bloom} (\cite{bock_2022}).

Entre les deux précédents se trouve le biome subtropical saisonnier (SSB), ou \engquote{subtropical gyre seasonally stratified biome} (\cite{sarmiento_2004}).
Il présente des concentration de \al{chl} et des températures intermédiaires.
La productivité y est légèrement accrue en hiver, du fait d'une convection saisonnière plus profonde que dans la gyre subtropicale.

\begin{figure}
  \centering
  \insertfig[0.7]{zhai_2008_fig1.png}
  \insertfig[0.7]{sarmiento_2004_fig2b.png}
  \captionT{Délimitations des biomes par des processus physiques}{%
    (a)~Tiré de \textcite{zhai_2008}, figure 1b. Climatologie de l'énergie cinétique de tourbillon (EKE) en hiver, en échelle logarithmique, et calculée à partir de données d'altimétrie TOPEX/Poseidon et \as{ers}[\textlf{-1/2}].

    (b)~Tiré de \textcite{sarmiento_2004}, figure~2b.
    \engquote{Biome classification scheme calculated using mixed layer depths obtained from observed density and from upwelling calculated from the wind stress divergence using observed winds.}
    Les biomes de notre zone d'étude correspondent dans la nomenclature de cet article à: subpolaire~(\eng{subpolar}, ``SP'', jaune); subtropical saisonnier~(\eng{seasonally mixed subtropical gyre}, ``ST-SS'', bleu); et subtropical permanent~(\eng{permanently stratified subtropical gyre biome}, ``ST-PS'', rose).

    % \foreignblockquote{english}{
    %   The equatorially influenced biome covers the area between \latlon{5S} and \latlon{5N}, and is colored a dirty light blue in areas where upwelling occurs (labeled~``Eq-U'' on the color bar) and dark pink in areas where downwelling occurs (labeled~``Eq-D'').
    %   Outside of this band, the region labeled ``Ice''~(red) is the marginal sea ice biome, the region labeled ``SP''~(yellow) is the subpolar biome, the region labeled ``LL-U''~(light blue) is the low-latitude upwelling biome , the region labeled ``ST-SS''~(dark blue) is the seasonally mixed subtropical gyre biome, and the region labeled ``ST-PS''~(pink) is the permanently stratified subtropical gyre biome.}
  }
\label{fig:biomes}
\end{figure}


\section{Détection des fronts sur images satellites}

\subsection{Historique des méthodes existantes}
\label{sec:detection-fronts}

Afin de mesurer l'impact des fronts sur le phytoplancton, nous partons de l'idée de détecter les fronts de densité à partir de la SST afin de les colocaliser aux valeurs de \al{chl}.
Nous nous concentrons dans cette section sur les méthodes existantes cherchant à détecter les fronts sur des images satellites de SST.

Une première catégorie de méthodes s'appuie la dérivation spatiale du champ de SST, avec divers opérateurs: gradient (\cite{kazmin_1996,moore_1997,kostianoy_2004}), Sobel (\cite{sauter_1994}), ou Laplacien (\cite{holyer_1989}), etc.
Ces étapes de dérivation introduisent cependant du bruit qui impacte négativement la détection.
On notera d'ailleurs que la détection des fronts de fine échelle, qui nous intéresse ici, est plus sensible au bruit du fait de gradients plus faibles.
Un filtrage préalable de l'image permet d'en limiter l'impact.
C'est le cas de la méthode de détection de contours \emph{Canny} (\cite{canny_1986}), qui applique un filtre gaussien avant de calculer le gradient, et par ailleurs ajoute des traitements supplémentaire aux contours détectés.
Initialement développée pour de la détection de contours en traitement d'image \encadra{domaine où elle demeure un standard, et où ses implémentations sont amplement disponibles} elle a également été largement appliquée en océanographie à la détection de fronts en température, entre autres.
% (cite{93-101}).

Similairement, l'algorithme de Belkin--O'Reilly~(\cite{belkin_2009}) applique un filtre itératif contextuel capable de réduire le niveau de bruit de l'image d'entrée tout en préservant les forts gradients.
Un simple opérateur Sobel peut ensuite être utilisé.
Il a été développé afin de repérer les fronts aussi bien à partir de la SST que de la Chorophylle.

Il est également possible d'éviter l'utilisation du gradient, comme le fait la méthode de \al*{cc}~(\as*{cc}, \cite{cayula_1992}).

\begin{technique}
L'algorithme que nous utiliserons par la suite, décrit dans le \nref[chapitre]{chp:methodes} (\nref{sec:HI}), en est inspiré. Nous le décrivons donc succinctement ici.

Le principe de l'algorithme \as{cc} est le suivant: Un front sépare deux masses d'eau de température différentes; il est raisonnable que la température des pixels à proximité du front soit distribuées de manière bimodale, chaque mode se situant autour de la température d'une des masses d'eau.
Ainsi, afin de ne considérer les pixels que proche des fronts, la méthode utilise une fenêtre glissante dans laquelle on s'intéresse à l'histogramme des valeurs de SST.
Une température seuil est choisie pour séparer l'histogramme en deux modes, de façon à minimiser la variance intra-mode\footnote{%
  La méthode de séparation des valeurs en deux modes est identique à la méthode d'\textcite{otsu_1979}, couramment utilisée en traitement d'image pour réaliser un \guil*{seuillage}, \as{cad} pour catégoriser les pixels d'une photo en deux.
}.
Pour cette séparation optimale, un critère évaluant la séparation des deux modes est calculé. Si ce dernier est supérieur à un certain seuil, la fenêtre est estimée contenir un front.

Une étape supplémentaire consiste à vérifier que la cohérence spatiale de chacune des masses d'eau dans la fenêtre est suffisante.
Cela permet d'éviter les faux positifs, notamment sur des images bruitée ou contaminée par des nuages.

% \begin{note}
  On notera que \textcite{cayula_1992} choisissent les seuils de cohérence spatiale de manière à ce que le cas le plus cohérent soit un front en ligne droite.
  Cela signifie qu'un front sinueux se vera attribuer une cohérence spatiale moindre, et pourrait être disqualifié.
  En fonction des structures à detecter, il sera nécessaire d'adapter les divers seuils de la méthode, éventuellement au détriment la spécificité de la méthode (éviter les faux positifs).
% \end{note}

\end{technique}

La méthode \as{cc} a largement été utilisée pour détecter des fronts de SST (voir les nombreux exemples dans la review de \cite{belkin_2021}), notamment à l'échelle globale (\cite{belkin_2009a,belkin_2007}).
Elle a également été appliquée avec succès sur des données de chlorophylle (\cite{stegmann_2004,kahru_2012,bontempi_2004}).
Diverses dérivations en ont été proposées.
\textcite{cayula_1995} proposent une version travaillant sur plusieurs images consécutives.
Plutôt que d'utiliser des images à des instants différents, \textcite{nieto_2012} appliquent l'algorithme \as{cc} sur quatres fenêtre glissantes décalées en longitude et latitude et fusionnent les résultats des pixels se chevauchant.
\textcite{miller_2009} combine les résultats de l'algorithme \as{cc} pour des produits SST et \as{chl} afin d'augmenter la couverture disponible.

Une méthode au fonctionnement similaire est proposée par \textcite{vazquez_1999}, qui pour quantifier la séparation des deux distributions en température (de part et d'autre d'un éventuel front) utilisent une mesure entropique: la divergence de Jensen--Shannon (\cite{barranco-lopez_1995}).
\textcite{shimada_2005} reprendront cette méthode et l'adjoindront d'un algorithme de morphologie mathématique (\cite{jiang_1997}).
Cette configuration sera réutilisée à plusieurs reprises (\cite{lan_2012} \ab{par-ex}).

Enfin, dans le but de quantifier les valeurs de \as{chl} dans les fronts de submésoéchelle de la gyre subtropicale du Pacifique Nord, \textcite{liu_2016} proposent une méthode inspirée de l'algorithme \as{cc}.
Elles définissent un indice d'hétérogénéité (\as{hi}) du champ de SST.
La valeur de cette indice est calculé en chaque pixel à partir de la distribution en température dans une fenêtre (glissante donc) autour de ce pixel.
La valeur de l'indice est la somme (pondérée par des coefficient de normalisation) de la variance de la distribution, de sa bimodalité (calculée différement que dans les méthodes ci-dessus), et de son asymétrie (afin de cibler les plus petits fronts).

On notera dans cette dernière étude une différence d'approche, essentiellement sémantique mais qui reste néanmoins d'intérêt.
C'est bien l'indice de l'\al{hi} qui apparaît comme au cœur de cette étude, qui finalement cherche plus à quantifier l'hétérogénéité spatiale de la SST pour chaque pixel, qu'à \guil{détecter les fronts}.
Par ailleurs, bien qu'elle reprenne l'utilisation d'une fenêtre glissante, cette méthode se démarque de celle de \as{cc} et ses variations, en donnant un résultat nuancé plutôt que purement dichotomique (front / background).
De plus, bien que nous les avons omises des descriptions ci-dessus, la plupart des méthodes incluent une étape algorithmique permettant d'obtenir en sortie uniquement la position des fronts, représentée par une ligne d'épaisseur nulle (ou 1~pixel).
En revanche, le positionement des fronts par \citeauthor{liu_2016} est réalisé par un simple\footnotemark{} seuil sur le HI.
\footnotetext{Bien que le choix du seuil soit fait de manière non-triviale par \textcite{liu_2016}, nous montrons dans notre implémentation qu'un seuil fixe suffit à notre étude.}
Par cette construction orientée vers une mesure graduelle des fronts, cette méthode de détection apparaît pertinente pour distinguer différents types de fronts par intensité.

\begin{note}
  La démocratisation des techniques de \eng{machine learning} s'est aussi étendu au domaine de la détection des fronts.
  Bien que nous les ayons pas considérées ou décrites ici, ces dernières années, diverses méthodes ont été proposée (voir la review de \cite{liu_2022}).
\end{note}

\subsection{Vers des fronts de densité}

Dans la section précédente, la majorité des méthodes opèrent sur les données satellite de températures, qui est utilisée comme proxy de la densité.
Cette dernière varie cependant aussi avec la salinité, qui n'est disponible par satellite qu'à basse résolution (entre \qtyrange[range-phrase={ et }]{35}{50}{\km} pour \as{smos} par exemple).
Il est donc possible que pour certains fronts thermique, la variation de température soit \emph{compensée} par une variation de salinité, donnant ainsi une densité constante.
Dans la région de l'Atlantique Nord, des transects sont réalisés régulièrement par un navire d'opportunité, l'Oleander.
Les résultats qui en découlent suggèrent que pour cette région, la salinité n'intervient que peu dans les variations de densité (\cite{flagg_2006}).

\textnote{Dans d'autres régions, il peut être nécessaire d'avoir recours à d'autres proxy de la densité, comme par exemple la hauteur d'eau (SSH) dans les régions polaires.}

\section{Objectifs}
\label{sec:problematique}

De multiples indices pointent vers le fait que les processus de fine échelle \OM(\qtyrange{1}{100}{\km}) ont une grande influence sur le phytoplancton (ainsi que de manière plus large sur les cycles biogéochimiques dans lesquels ce dernier s'inscrit).
Cependant la quantification précise de leurs effets sur le paysage biogéochimique général pose un certain nombre de difficultés non-résolues, et ainsi la place exacte de ces processus reste un sujet d'étude actif.
Un sujet d'autant plus important qu'il est étroitement lié au cycle du carbone et à la chaîne trophique de l'océan, et sur lequel l'impact du changement climatique n'est pas non plus connu avec certitude.
Parmi les efforts déployés pour cette quantification, les images satellites offrent la possibilité de couvrir de larges régions spatio-temporelles, tout en résolvant les échelles qui nous intéressent ici.

Nous nous penchons donc dans ce travail sur les moyens de quantifications par satellite des effets de ces processus sur le phytoplancton; en particulier ceux des \emph{fronts de submésoéchelle} sur la \emph{\al{chl}}, que nous utilisons comme proxy de la biomasse totale de ce dernier.
Pour développer et valider nos méthodes de quantification, nous nous limitons à la région autour du Gulf Stream ce qui facilite le travail d'exploration tout en gardant néanmoins une diversité dans les régimes biogéochimiques présents.
Nous cherchons à répondre aux questions suivantes:

\begin{tcolorbox}[
  enhanced,
  frame hidden, interior hidden,
  borderline={1.pt}{3pt}{black}, arc=2.mm,
  borderline north={1.2pt}{0pt}{black},
  borderline south={1.2pt}{0pt}{black},
  borderline west={1.2pt}{0pt}{black},
  borderline east={1.2pt}{0pt}{black},
  top=2\onelineskip,
  bottom=2\onelineskip,
  ]
  \begin{list}{}{
      \setlength{\labelsep}{0.5em}
      \setlength{\itemindent}{0pt}
      \setlength{\leftmargin}{2em}
      \setlength{\labelwidth}{0.5em}
      \setlength{\listparindent}{0pt}
      \setlength{\parsep}{\parskip}
      \setlength{\itemsep}{\onelineskip}
      \setlength{\topsep}{\onelineskip}
      \renewcommand*\makelabel[1]{\adfrightarrowhead}}
    \item Comment quantifier la réponse de la chl aux dynamiques frontales dans la région du Gulf Stream ?
    \item Comment est-ce que l'augmentation de chl est conditionné par l'intensité des fronts ?
    \item Peut-on détecter un bloom précoce dans les fronts depuis des données satellites ?
  \end{list}
\end{tcolorbox}

\section{Plan de thèse}
\label{sec:plan-de-these}

Cette thèse en organisée en 6~chapitres, en comptant l'introduction ci-dessus.
Dans le \nref[chapitre]{chp:methodes}, nous commencons par détailler les différents ensembles de données que nous avons considérés, puis les méthodes mises en places pour répondre aux problématiques posées plus haut.

Le \nref[chapitre]{chp:res-chl} détaille les résultats obtenus relatifs à l'impact des fronts sur le budget de \al{chl}.
La première partie de ce chapitre comprend l'entièreté d'un article soumis au journal \eng{\textit{Biogeosciences}}, dans sa version finale (en \engquote{galley proof}) et précédée d'un résumé.
Les points dans les parties d'introduction et de méthodes de l'article sont déjà abordés dans les chapitres correspondants, si bien qu'il est possible de passer ces sections.
L'article présente des résultats concernant le budget de \as{chl} et la phénologie du bloom, puis les discute, et conclus.
Le reste de ce chapitre présente des résultats complémentaires, omis dans l'article.

Le \nref[chapitre]{chp:res-phenologie} se penche sur les résultats obtenus concernant l'impact des fronts sur la phénologie du bloom.
Il résume les résultats présenté au chapitre précédent, puis y ajoute des résultats complémentaires.

Enfin, les chapitres~\ref{chp:perspectives} et~\ref{chp:conclusion} apportent des perspectives et une conclusion, respectivement, sur le travail effectué pour cette thèse.
