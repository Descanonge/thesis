% chktex-file 13

\chapter{Introduction}
\addChpLof
\label{chp:introduction}
\graphicspath{{resources/introduction}}

\minitoc%
\clearpage

\begin{figure}[!h]
  \centering
  {%
    \setlength{\fboxsep}{0pt}%
    \framebox[\figwidth]{\insertfig{gs_false_colors.jpg}}
  }%
  \captionT{Les couleurs du Gulf Stream par satellite}{%
    \small
    Image en fausses couleurs de l'océan Atlantique Nord, prises par \as{modis} \sur~Aqua le \frenchdate{2020}{02}{23}, retouchées par Norman Kuring (groupe \eng{Ocean Biology}, \abbrv{NASA}).
    Adapté de l'\eng{Image of the day} 2020-03-10 \textit{Hints of Spring in the Atlantic}, sur \eng{\glshref{ve-illustration} \& \glshref{eo-illustration}}.
  }
  \label{fig:oc-illustration}
\end{figure}

\vspace{1\baselineskip}

\section{Préambule}

L'essort de l'imagerie satellite appliquée à la couleur de l'océan \encadra{depuis la fin du \siecle{20}~siècle} a révélé la grande variabilité de la biologie aux échelles les plus fines de l'océan.
Cette variation de la couleur de l'océan, comme illustrée sur la \nref[figure]{fig:oc-illustration}, est due aux pigments du phytoplancton, une collection de micro"-organismes phytosynthétiques portés par les courants.
À la base de la chaîne trophique océanique, et partie centrale des cycles biogéochimiques océaniques \encadra{dont celui du carbone} la compréhension des phénomènes et facteurs régissant les évolutions du phytoplancton est cruciale.

Comme pour ses homologues terrestres, le phytoplancton a besoin de lumière et de nutriments.
Dans l'océan cependant, la répartition de ces deux composantes engendre un forcage particulier.
La lumière ne pénètre que dans une couche superficielle, profonde d'une centaine de mètres environ, la couche euphotique.
Les nutriments, rapidement consommés en surface, sont à l'inverse trouvés en profondeur.
Les échanges verticaux \encadra{de nutriments vers la couche euphotique, et de matière organique (ensuite reminéralisée) vers les profondeurs} sont ainsi nécessaires à la conservation d'un équilibre dans la pompe biologique ainsi décrite.

L'ensemble des courants de large échelle (\ab{cad} des basins océaniques \OM(\qty{1000}{\km})) définit une cartographie des caractéristiques biophysiques de l'environnement.
On peut distinguer par exemple les \guil*{deserts} que sont les gyres subtropicales, où les faibles échanges verticaux engendre un milieu très oligotrophe et peu productif.
À l'inverse, les zones d'\eng{upwelling} de bord est sont de véritables \guil*{forêts}, très productives, en raison des nutriments remontés avec les eaux profondes.
Une telle séparation à grande échelle entre deux biomes est par exemple bien visible sur la \nref[figure]{fig:oc-illustration}, où les eaux au sud du Gulf Stream sont peu productives (en bleu foncé), et celles au nord apparaissent beaucoup plus productives (en turquoise).
Toutefois, sur cette même image, il est également évident que des processus façonnent le paysage biologique à de plus petites échelles.
Par exemple, dans les méandres du Gulf Stream, et des tourbillons se forment.
Ici deux apparaissent en bleu foncé, ayant capturé des eaux du sud (chaudes et peu productives).

De tels tourbillons se forment aux \emph{méso"-échelles} (\qtyrange{10}{100}{\km}) et leurs effets sont multiples.
Comme on l'a vu dans l'exemple ci"-dessus, ils peuvent \guil*{capturer} des masses d'eaux et les transporter sur de larges distances (ici en transportant des masses d'eaux chaudes et oligotrophe au nord, \ab{cad} dans un milieu plus froid, productif, et riche en nutriments).
Ces tourbillons amplifient localement les échanges entre surface et profondeur, à la fois en générant des vitesses verticales, ainsi qu'en déplacant les isopycnes.
Néanmoins, ces échelles ne suffisent pas à décrire toute la variabilité biologique observée. De plus, leur contributions (estimées) à la pompe biologique ne permettent pas de boucler le budget de cette dernière <mal dit>.

En zoomant encore un peu plus sur la \nref[figure]{fig:oc-illustration}, nous pouvons distinguer des structures plus fine, dite de \emph{sub"-mésoéchelle} (\qtyrange{1}{10}{\km}).
Les champs de différentes variables (dont la densité) sont mélangés, étirés, par les courants des échelles supérieures (et les forcages atmosphériques), faisant apparaître des structures plus fines, et plus éphémères (entre le jour et la semaine).
Notamment, émergent des emplacements de fort gradient \encadra{autrement dit des fronts} de densité propices à la formations de circulations secondaires verticales, plus localisées mais aussi plus intenses que celles engendrées par les tourbillons de mésoéchelle.
Les fronts de sub-mésoéchelle, s'ils sont suffisamment marqués peuvent donc engendrer des circulations verticales suffisamment profondes pour remonter des nutriments dans la couche euphotique, et augmenter localement la productivité.

Une autre conséquence de ces gradients, due à leur tendance à applanir les isopycnes par instabilité barocline, est de restratifier les couches supérieures (toujours localement).
La restratification étant un facteur fort dans le démarrage du bloom printanier, ces structures de sub-mésoéchelles peuvent donc également conduire à des démarrages précoces (localement).

Enfin, bien que les effets de la sub-mésoéchelle soient locaux, ils peuvent néanmoins avoir une rétroaction sur les échelles supérieures, modifiant la circulation de grande échelle, ou la productivité d'un bassin océanique.

Les effets de la sub"-mésoéchelle sur les cycles biogéochimiques présentés ci"-dessus ne sont toutefois pas encore complétement élucidés, malgré les efforts des dernières décennies dans ce domaine.
En effet, l'étude des processus à ces échelles présente des difficultés.
Les diverses méthodologies usuelles peinent à donner une vue entière du problème.
Capturer les variations rapide de la biologie en travers d'un front lors d'une campagne in"-situ présente un défi technique certain.
De plus, la courte période de vie des structures d'intérêt rend la tâche d'autant plus ardue.
Il en va de même pour l'imagerie satellite, pour laquelle la couverture nuageuse rend difficile de suivre temporellement une structure.
Par ailleurs, il est compliqué pour les satellites d'accéder à toute la biodiversité du phytoplancton, ainsi qu'au delà des premiers mètres en surface.
Enfin, les simulations numériques peuvent résoudre ces fines échelles, mais à un coût calculatoire élevé et prohibitif sur de trop longues durées (des projections climatiques \ab{par-ex}).

Malgré les limites évoquées, l'imagerie satellite donne l'opportunité d'observer ces effets à une large échelle (spatiale et temporelle), et permettrait d'en fournir une quantification.
Une telle démarche a été entreprise par \textcite{liu_2016}, dans la gyre subtropicale du Pacifique Nord.
Les auteur·ices \encadra{en colocalisant les valeurs satellites de la \al{chl} avec les positions de fronts détectés à partir de la température de surface satellite} ont montré une augmentation des valeurs de \as{chl} dans les fronts par rapport au reste de la zone.
Nous étendons leur méthode, et l'appliquons à la région de l'Atlantique Nord, autour du Gulf Stream.

Cette zone comprend trois biomes, \al{cad} trois régimes biogéochimiques différents.
Au sud de notre zone, la gyre subtropicale est une zone oligotrophe, d'une productivité faible tout au long de l'année.
Au nord de celle-ci mais au sud du Gulf Stream, ce régime oligotrophe est mitigé par l'approfondissement de la couche de mélange en hiver, apportant des nutriments.
Enfin, au nord du Gulf Stream, on trouve des eaux riches et productives, berceau d'un fort bloom au printemps, et d'un second bloom en automne.

Ces trois régions sont également hétérogène en terme des structures qu'on peut y trouver.
Le front de densité (très marqué) associé au Gulf Stream est un réservoir d'énergie potentielle, qui est convertie en énergie cinétique par le biais des structures de fines échelles qui nous intéressent.
Nous nous attendons donc à trouver des fronts de densité plus stables, associés à de forts gradients, et donc à d'intenses vitesses verticales dont il est plus probable qu'elle puisse atteindre les nutriments en profondeur.
En revanche, loin de ce courant intense, dans la gyre subtropicale, nous pourrons étudier des fronts et gradients plus faibles.
Cette variété nous permet d'étudier la réponse biologique aux fronts, en fonction de leur intensité.

Afin de mieux définir nos objectifs, et avant de les présenter, nous dressons dans la section suivante un état de l'art des sujets concernés.
Nous exposerons ensuite le plan de ce manuscript.

\section{État de l'art}
\label{sec:etat-de-lart}

\subsection{Le phytoplancton dans le système terre}
\label{sec:phyto-ds-sys-terre}

La biologie marine, à sa base, repose sur les conversions d'un certain nombre d'éléments entre eux.
Schématiquement, le phytoplancton converti par photosynthèse le carbone et les nutriments dissouts dans l'eau, en dioxygène et en matière organique (pour sa croissance).
À la mort de ces organismes, ce stock de matière organique inerte plonge par gravité, et sera éventuellement reminéralisé plus en profondeur par d'autres organismes.
Le carbone et autres nutriments libérés pourront alors être remontés à la surface et réutilisés, complétant la pompe biologique marine.
\review{euh ?}

\subsubsection{Définitions générale}
\label{sec:phyto-def-gen}

Le phytoplancton étant au cœur de la problématique qui nous intéresse ici, je m'arrête ici brièvement sur ses caractéristiques.
Le plancton \guil*{\hbox{πλαγκτόσ}}: il \guil*{divague}.
C'est l'ensemble des organismes dont la motilité se lui permet guère plus que de se laisser dérivier au gré des courants.
Bien que certains de ces organismes puissent se déplacer de manière significative par leurs propres moyens (principalement sur la verticale), concentrons"-nous plutôt sur le sous"-ensemble (moins mobile) du phytoplancton.
Cet ensemble d'organismes est remarquablement divers, totalisant environ \num{20 000}~espèces, dont les tailles varient de la fraction de micromètre à la fraction de millimètre. Cette diversité se retrouve aussi bien dans leurs formes, au grand bonheur des zoologues et photographes, que dans leurs fonctions écosystèmiques.
Comme leur nom le suggère, les organismes phytoplanctoniques pratiquent la photosynthèse.
Ils transforment le dioxyde de carbone et les nutriments (Nitrate, Phosphate, Silice, Fer,\dots) dissous, en oxygène et en matière organique.
À l'instar des plantes terrestres, cette opération nécessite de capter de la lumière à l'aide de pigments, qui se trouvent être principalement la \al{chl}.
Ce sont ces pigments qui affectent la couleur de l'océan comme nous l'avons vu dans le préambule (\nref{fig:oc-illustration}), et sur lequel nous reviendrons plus loin.

Cette production de matière organique à partir de carbone in"-organique (du CO2 dissous), est dite primaire.
Le phytoplancton étant prédaté par le zooplancton, cette matière organique remonte éventuellement la chaîne (ou plutôt réseau) trophique: zooplancton, petits prédateurs, grands prédateurs, oiseaux marins, mamifères marins,\dots
Le phytoplancton occupe donc une place centrale dans l'écosystème marin.
Il est à la base de la chaîne alimentaire dans les océans, et il est estimé qu'il génère~\pct{50} de la productivité primaire mondiale.

La croissance du phytoplancton est favorisée, selon les espèces, par des conditions environnementales comme la température, la salinité ou l'acidité (ref).
Cependant les limitations principales à cette croissances restent la lumière et les nutriments.
La lumière n'est pas distribuée identiquement dans les océans: l'ensoleillement dépend de la saison, de la latitude, de la présence de glace de mer, et surtout: de la profondeur.
Elle ne pénètre en effet qu'en surface, dans la \emph{couche euphotique} profonde d'au maximum une centaine de mètres.
Les nutriments, à l'inverse, sont rapidement consommés par le phytoplancton en surface et sont donc trouvés plus en profondeur, sous la \emph{nutricline}.
Cette répartition des besoins induit dès lors une grande importance des échanges verticaux entre la surface ensoleillée, et les eaux profondes riches en nutriments.

Ces échanges verticaux sont conditionnés par les courants océaniques, à l'échelle des bassins, mais également à des échelles plus fines.
Cette interaction bio"-physique

\begin{figure}
  \centering
  \insertfig{pompe_biologique.pdf}
  \captionT{Schéma de la pompe biologique}{%
    Adapté de \textcite[fig.~1]{levy_2023}.
  }
  \label{fig:pompe-bio}
\end{figure}

\subsubsection{Mesurer le phytoplancton}
\label{sec:teledetection}

\begin{figure}
  \centering

  \captionT{Structures fines vue dans la couleur de l'océan}{%
    prendre une image depuis visibleearth
  }
  \label{fig:oc-fine-illustration}
\end{figure}

L'observation du phytoplancton comporte un certain nombre de défis que nous allons rapidement évoquer dans cette section.

Comme pour d'autres variables géophysiques, l'utilisation de l'imagerie satellite a permis de fournir depuis ces dernières décennies une vision synoptique du phytoplancton.
En effet cette dernière offre une couverture globale, journalière, et à haute résolution spatiale.
La possibilité d'observer des micro-organismes depuis l'espace n'apparaît pourtant pas comme évidente.
Rappelons alors que ces organismes, réalisant de la phytosynthèse, contiennent différents pigments leur permettant de récupérer l'énergie lumineuse du soleil.
Toutes les espèces de phytoplancton contiennent majoritairement de la \al{chl}~(\as{chl}) et, éventuellement, des pigments secondaires.
Ces pigments absorbent plus ou moins certaines longueurs d'onde du rayonnement pénétrant dans la couche de surface de l'océan, affectant ainsi la rétrodiffusion (\engquote{backscatter}).
En d'autres termes, la couleur de l'océan apparaît plus verte\footnote{%
  la chlorophylle, pigment majoritaire, absorbe dans le rouge et le bleu}
sur les zones où la concentration en phytoplancton est élevée.

Ce phénomène, compris depuis longtemps, fut testé avec succès par des mesures en avion: \ab{par-ex} par \textcite{clarke_1970} dans la région du Gulf Stream, au large de Cape Cod; puis par satellite par John Arvesen et Dr.\ Ellen Weaver\footnote{%
  Je conseille d'ailleurs l'article suivant qui s'attarde sur sa carrière scientifique passionnante (qui inclut un passage par le projet Manhattan !): \citetitle{marshall_2010}, \cite{marshall_2010}.},
en coopération avec le capitaine Cousteau à bord du Calypso\footnote{%
  Ce dernier commentera d'ailleurs: \engquote{The space-age satellites have opened a whole new dimension to ocean resources monitoring}, comme reporté par la presse locale de l'époque:
  \href{ https://news.google.com/newspapers?nid=1454&dat=19730320&id=Nmo0AAAAIBAJ&sjid=8QkEAAAAIBAJ&pg=897,4493586&hl=en}{\citetitle{macomber_1973}}, \cite{macomber_1973}.
}.
Ces premières expériences permettront le déploiement d'un premier capteur dédié à l'étude de la couleur de l'océan: \as{czcs} à bord de Nimbus-7, lancé en 1978 qui fonctionnera jusqu'en 1986.
Ce n'est que dix plus tard, en 1997, que le lancement de \as{seawifs} marquera le début d'une observation ininterrompue de la couleur de l'océan par un ensemble de capteurs et satellites.

Attardons nous brièvement sur le processus permettant d'obtenir une concentration en \al{chl} à partir des données de ces capteurs.
Ce derniers mesurent l'intensité lumineuse renvoyée, ou réflectance (\as{rrs}), pour différentes bandes de longueurs d'ondes (entre 6 et 16 dans le visible).
Le satellite mesure les réflectances en dehors de l'atmosphère, et non pas à la surface de l'eau; il est nécessaire d'apporter plusieurs corrections.
Les contributions de l'écume et du reflet direct du soleil sont retirées.
La diffusion de Rayleigh est corrigée assez simplement, mais l'absorption et diffusion par divers aérosols sont plus complexes, leurs répartitions (horizontale et verticale) étant variables (\cite{werdell_2018}).

Nous voulons maintenant relier ces réflectances au contenu de la surface de la colonne d'eau.
En toute rigueur, il faudrait relier les réflectances aux propriétés optiques de la colonne d'eau: c'est"-à"-dire l’absorption et la rétrodiffusion causées par le phytoplancton, mais aussi les particules non-algales (\as{nap}) et les matières organiques dissoute colorées (\as{cdom}) (\cite{werdell_2018}).
Certains algorithmes dits \emph{semi"-analytiques} s'attachent à résoudre ce problèmes inverse, ce qui nécessite un certain nombre de simplifications, mais permet en théorie de remonter à tous les composants optiques en jeux (\as{chl}, \as{nap}, \as{cdom}) (\cite{werdell_2019}).

Cependant, dans la majorité de l'océan ouvert il est raisonnable de considérer que le contenu optique de la colonne d'eau est le seul fait du phytoplancton\footnotemark, et qu'il est donc possible de directement relier les réflectances à la concentration en \as{chl} (\cite{bailey_2006, brewin_2015a}).
\footnotetext{C'est-à-dire qu'on fait l'hypothèse que les autres composantes (\as{nap} et \as{cdom}) sont des produits du phytoplancton, et co"-varient avec sa concentration.}
Ainsi, un certain nombres d'algorithmes dits \emph{empiriques} infèrent la \as{chl} en s'appuyant sur une simple relation vérifiée empiriquement par des mesures in"-situ.
Par exemple, les algorithmes \guil{OC} utilisent des polynômes de ratio entre les réflectances bleu et verte (voir plus bas).

Cette hypothèse ne tient plus dans certains cas (classifiés comme les \engquote{case II waters}, où \guil{optiquement complexes}), par exemples les régimes côtiers où la colonne d'eau est riche en sédiments (\cite{bailey_2006, brewin_2015a}), mais certains algorithmes sont tout de même disponibles pour ces cas plus complexes (\cite{gohin_2002}).
On gardera néanmoins en tête ces hypothèses, et le fait que ces relations empiriques (validées sur des données passées) pourraient perdre en validité face à des écosystèmes modifiés par le changement climatique (\cite{dierssen_2010}).
Toutefois, ces modèles empiriques sont de manière générale plus performant que les modèles semi"-analytiques (\cite{brewin_2015a}), et sont donc employés pour les données opérationelles.

\begin{technique}
Nous décrivons ici rapidement deux exemples d'algorithmes empiriques, utilisés de manière opérationelle par le groupe NASA \as{obpg}.
L'algorithme \abbrv{OC3/4} se base sur un polynôme d'ordre~4 du ratio des bandes verte~(\(V\)) et bleue~(\(B\)):
\begin{equation} \label{eq:ocx}
  \log_{10}\paren{\am{chl}} =
  \sum_{i=1}^{4} a_i \paren{\log_{10}\paren{
      \frac{ \am{rrs}^{max}\paren{\lambda_B} }
           { \am{rrs}\paren{\lambda_V} }
    }}^i ,
\end{equation}
où \(\am{rrs}^{max}\paren{\lambda_B}\) est la réflectance maximale parmis les bandes bleues, et \(a_i\) des coefficients empiriques adaptés à chaque capteur (\cite{oreilly_1998, oreilly_2000}).
Ces coefficients, régulièrement mis à jour par le groupe NASA \as{obpg}\footnote{voir l'historique \url{https://oceancolor.gsfc.nasa.gov/reprocessing/}}, sont dérivés par regression de la relation ci"-dessus avec les données in-situ \as{nomad} (\cite{werdell_2005}).
L'algorithme utilise au total 3~bandes pour \abbrv{OC3}, et 4 pour \abbrv{OC4}.

Cependant, \abbrv{OC3/4} montre des limites dans les eaux oligotrophes, où la concentration de \as{chl} est faible.
Dans ces cas là, l'algorithme CI (\eng{Color Index}) est utilisé (\cite{hu_2012}).
Il utilise la différence entre la bande verte~(\(V\)) et une combinaison des bandes rouge~(\(R\)) et bleue~(\(B\)):
\begin{equation}
  \label{eq:CI}
  \begin{split}
    CI &= \am{rrs}\paren{\lambda_V} -
         \paren{
          \am{rrs}\paren{\lambda_B}
          + \frac {\lambda_V - \lambda_B}
                  {\lambda_R - \lambda_B}
          \times \paren{\am{rrs}\paren{\lambda_R}
                      - \am{rrs}\paren{\lambda_B}}
         }
    \\
    \am{chl} &= 10^{\paren{a_0' + a_1' \times CI}},
  \end{split}
\end{equation}
avec ici aussi des coefficients empiriques \(a_i'\).

L'implémentation actuelle du groupe NASA \as{obpg} utilise \abbrv{OC3/4} pour les eaux où la \as{chl} est supérieure à~\qty{0.35}{\mgm}, et CI où elle est inférieure à~\qty{0.25}{\mgm}; entre ces deux valeurs, un mélange des résultats des deux algorithmes est utilisé assurant ainsi la continuité des valeurs de Chlorophylle (\cite{oreilly_2019}, \url{https://oceancolor.gsfc.nasa.gov/atbd/chlor_a/}).

À noter que de nombreux autres algorithmes ont été développés dans les deux dernières décennies, on pourra citer par exemple l'ajout d'une bande violette pour \abbrv{OC5/6} (\cite{oreilly_2019}), ou l'utilisation de 5~bandes et des tables de références afin d'améliorer l'estimation de \as{chl} en zone côtière (\cite{gohin_2002}).
\end{technique}

Comme nous venons de le voir, la dérivation de la concentration de \as{chl} est un processus compliqué, comprenant de multiple sources d'incertitudes dans les valeurs absolues inférées.
Ce processus comporte un certain nombre de limitations.
Nous avons évoqué la difficulté à travailler dans des zones où d'autres éléments que le phytoplancton dicte le comportement optique, comme des sédiments en suspension par exemple.
C'est souvent le cas près des côtes et des décharges fluviales.
La faible profondeur en régime côtier apporte également une difficulté, puisque dans le cas où la lumière incidente pénètre suffisamment profond, la réflectance mesurée comportera la lumière réfléchie par le fond marin (et re"-traversant la colonne d'eau).
Plus généralement, quelque soit la zone sondée, la profondeur jusqu'à laquelle pénètre la lumière est un point critique.
En effet, la réflectance mesurée ne peut comporter des informations que sur cette couche éclairée.
De plus, les modèles empiriques sont ajustés par des mesures in"-situ réalisées sur les premiers mètres.
Les données satellite manquent ainsi une biomasse plus en profondeur, potentiellement importante [ref].
Par ailleurs, nous nous sommes penchés sur la détection de la \al{chl}. Bien que ce soit le pigment majoritaire, ce n'est pas nécessairement un proxy parfait pour la biomasse présente dans la colonne d'eau [ref].
Et enfin, la couverture nuageuse bloque cette télédétection.

Pour toutes ces raisons, les mesures in"-situ constituent un outil nécessaire et complémentaire à la télédétection, à la fois pour l'observation opérationelle mais aussi pour mieux comprendre les processus en jeu.
En effet, au-delà de pouvoir accéder directement aux variables que les satellites doivent inférer, parfois à travers des proxys, ces mesures peuvent explorer des dimensions invisibles aux satellites.

D'abord, les mesures in"-situ ont accès à toute la colonne d'eau et non pas seulement la surface optiquement active.
D'autres part, ces mesures offrent une vue beaucoup plus large et précise de la biodiversité présente, et ce grâce à différentes techniques de mesures.
cytomètre,
hplc,
zooscan,
omique,

Par ailleurs, les mesures in"-situ offrent l'avantage de fournir une vision précise de la densité des masses d'eaux par mesure de la température et de la salinité.
Les satellites actuels n'ayant accès à la salinité qu'à relativement basse résolution (\qtyrange{35}{50}{\km} pour \as{smos} \ab{par-ex}), ils ne peuvent estimer les variations de densité aux fines échelles qu'avec la température.

Néanmoins, ces mesures présentent également un certain nombre de limitations, et de difficultés technique de mise en place.
Sonder à une grande profondeur prend du temps, ce qui se répercute sur la distance d'échantillonage.
Similairement, la fréquence d'échantillonage peut être dictée par certaines mesures biologiques longues (ref d'ogloli ?).
Quand bien même, il leur est impossible de couvrir une large zone spatio"-temporelle.
Il est usuel que les campagnes en mer ciblent une zone d'intérêt en particulier, par exemple un front ou un tourbillon, identifiée pendant la mission par images satellites.
Heureusement ces dernières années de telles campagnes se multiplient, devant la nécessité d'avoir tous ces paramètres dans des zones précises pour mieux comprendre ce qu'il se passe.

J'ai parlé que des obs mais des modèles numériques sont aussi dispo cependant.
Tous les modèles sont faux. Tous les modèles bio sont très faux.
La très grande variabilité des organismes est mal représentée (faute de puissance entre autre). Les petites échelles qui sont pourtant si importantes à la bio (subméso) sont très coûteuses à faire tourner et on est limité à de petites régions géographiques. Impossible de quantifier les effets convenablement avec des modèles.
On a pas de paramétrisation des processus bio. Est-ce que c'est seulement possible sachant qu'on peut avoir des effets non-locaux potentiellement ?
Les modèles climatiques sont, du coup, très incertains en ce qui concerne la bio.
Grosses barres d'erreurs dans les projections pour le prochain siècle.

Difficile de séparer les trois: processus dynamiques, mélange horizontal, et biologie, car ils ont les même temps caractéristiques.

\subsection{Interactions biophysiques}
\label{sec:interactions-biophys}

Définition par l'échelle spatiale (0.1-10km).

Ce sont les images satellite de couleur de l'océan qui ont révélé l'ubiquité des fines échelles dans l'océan (en surface tout du moins).

Importance de la SMS:
1) on observe que la variance de la biophysique est importante à ces échelles.
2) les processus dynamiques crée des échanges verticaux
3) Mal représenté dans les modèles climatiques, on doit mieux comprendre (et quantifier pour savoir à quel point c'est important)


\subsubsection{Mélange horizontal}
\label{sec:melange-horizontal}

Une partie des fines échelles observées correspondent à l'action de l'advection par les courants de méso-échelle. L'action passive du mélange horizontal fait apparaître de fins filaments.
Cela permet le mélange de communauté, car rapproche spatialement des parcelles d'eau de différentes origines, avec des caractéristiques physiques et des historique biologiques propres.

Approche lagrangienne très utile dans l'étude du phytoplancton, et dans son observation (méthode de sampling Lagrangiennes, ref ).


\subsubsection{Upwelling de nutriments par les circulations agéostrophiques}
\label{sec:upwelling-nutriments}

Comme dit plus haut, les échanges verticaux sont important pour la bio.
Or aux petites échelles on voit apparaître des vitesses verticales de grande magnitude.

à ces échelles (0.1-10km) émergent également des processus dynamiques nouveaux.
On est alors en dessous du rayon de déformation de Rossby (\(Ro < 1\)).
Forçage par l'atmosphère (hétérogène) et les courants méso qui génèrent des gradients de densité.
On décrit certains de ces processus et leur(s) impact(s) dans la suite.

advection selon les isopycnes qui peuvent être penchées.

importance des vents

aspect intermittent (local, petit spot)

winners and losers


\subsubsection{Modification de la phénologie du bloom}
\label{sec:modif-phenologie}

Les gradients de densité existant (créés par mélange par courants méso ou forçages atmos) sont des réservoirs d'énergie potentielle. À un front les isopycnes sont penchées et des circulations sub-méso se créent et transforment l'énergie potentielle en énergie cinétique (tourbillons), ce faisant ramenant les isopycnes à l'horizontale.
Ces tourbillons formés par l'instabilité de Mixed-Layer (?) s'étendent sur la hauteur de la ML. Ce sont les Mixed-Layer Eddies.
À travers ces instabilités la sub-mesoéchelle contribue fortement à re-stratifier la couche  de surface, et réduire le mélange.
C'est un processus local et on s'attend donc à un soulèvement local de la couche de mélange au niveau des fronts.


Expliquer lancement du bloom printanier par réduction mélange.

On s'attend à un départ du bloom d'abord dans les fronts.

exemples de détection précédentes (mahadevan 2020).


\subsection{Région d'étude: Extension du Gulf Stream}
\label{sec:region-detude}

Notre zone d'étude: 15°N-55°N, 82°W-40°W

\subsubsection{La physique}
\label{sec:gs-physique}

Gyre subtropicale Atlantique Nord.

Courant de bord ouest chaud et salé qui remonte des caraïbes le long de la Floride
Se détache à Cape Hatteras, quitte le plateau continentale et méandre vers l'est.
Plume énergétique autour de ces méandres (surtout au sud).

Au nord du jet, courant retour (slope current) avec notamment un jet sur le shelf break.
Entre le Gulf Stream North Wall et le shelf: slope seas. Très froid, plutôt fraîche (plus salé que sur le shelf néanmoins).
Ce courant froid plonge. Fait partie de la circulation d'overturning de l'atlantique.

\subsubsection{La biologie}
\label{sec:gs-biologie}

Gyre très pauvre en nutriment et productivité faible.
Pourquoi ?
Pas de circulation méso qui permette l'apport de nutriment en surface.
Importance de la sub-méso donc pour créer des échanges verticaux.

Eaux au nord du GS très productives.
Également importances à cause de pêcheries.
Important bloom (plus de détail, ref sur la phénologie, )


\subsection{Détection des fronts sur images satellites}
\label{sec:detection-fronts}

Afin de mesurer l'impact des fronts sur le phytoplancton, nous partons de l'idée de détecter les fronts de densité à partir de la SST afin de les colocaliser aux valeurs de \al{chl}.
Nous nous concentrons dans cette section sur les méthodes existantes cherchant à détecter les fronts sur des images satellites de SST.

Une première catégorie de méthodes s'appuie la dérivation spatiale du champ de SST, avec divers opérateurs: gradient (\cite{kazmin_1996, moore_1997, kostianoy_2004}), Sobel (\cite{sauter_1994}), ou Laplacien (\cite{holyer_1989}), etc.
Ces étapes de dérivation introduisent cependant du bruit qui impacte négativement la détection.
On notera d'ailleurs que la détection des fronts de fine échelle, qui nous intéresse ici, est plus sensible au bruit du fait de gradients plus faibles.
Un filtrage préalable de l'image permet d'en limiter l'impact.
C'est le cas de la méthode de détection de contours \emph{Canny} (\cite{canny_1986}), qui applique un filtre gaussien avant de calculer le gradient, et par ailleurs ajoute des traitements supplémentaire aux contours détectés.
Initialement développée pour de la détection de contours en traitement d'image \encadra{domaine où elle demeure un standard, et où ses implémentations sont amplement disponibles} elle a également été largement appliquée en océanographie à la détection de fronts en température, entre autres.
% (cite{93-101}).

Similairement, l'algorithme de Belkin--O'Reilly~(\cite{belkin_2009}) applique un filtre itératif contextuel capable de réduire le niveau de bruit de l'image d'entrée tout en préservant les forts gradients.
Un simple opérateur Sobel peut ensuite être utilisé.
Il a été développé afin de repérer les fronts aussi bien à partir de la SST que de la Chorophylle.

Il est également possible d'éviter l'utilisation du gradient, comme le fait la méthode de \al{cc}~(\as{cc}, \cite{cayula_1992}).

\begin{technique}
L'algorithme que nous utiliserons par la suite, décrit dans le \nref[chapitre]{chp:methodes} (\nref{sec:HI}), en est inspiré. Nous le décrivons donc succinctement ici.

Le principe de l'algorithme \as{cc} est le suivant: Un front sépare deux masses d'eau de température différentes; il est raisonnable que la température des pixels à proximité du front soit distribuées de manière bimodale, chaque mode se situant autour de la température d'une des masses d'eau.
Ainsi, afin de ne considérer les pixels que proche des fronts, la méthode utilise une fenêtre glissante dans laquelle on s'intéresse à l'histogramme des valeurs de SST.
Une température seuil est choisie pour séparer l'histogramme en deux modes, de façon à minimiser la variance intra"-mode\footnote{%
  La méthode de séparation des valeurs en deux modes est identique à la méthode d'\textcite{otsu_1979}, couramment utilisée en traitement d'image pour réaliser un \guil*{seuillage}, \as{cad} pour catégoriser les pixels d'une photo en deux.
}.
Pour cette séparation optimale, un critère évaluant la séparation des deux modes est calculé. Si ce dernier est supérieur à un certain seuil, la fenêtre est estimée contenir un front.

Une étape supplémentaire consiste à vérifier que la cohérence spatiale de chacune des masses d'eau dans la fenêtre est suffisante.
Cela permet d'éviter les faux positifs, notamment sur des images bruitée ou contaminée par des nuages.

% \begin{note}
  On notera que \textcite{cayula_1992} choisissent les seuils de cohérence spatiale de manière à ce que le cas le plus cohérent soit un front en ligne droite.
  Cela signifie qu'un front sinueux se vera attribuer une cohérence spatiale moindre, et pourrait être disqualifié.
  En fonction des structures à detecter, il sera nécessaire d'adapter les divers seuils de la méthode, éventuellement au détriment la spécificité de la méthode (éviter les faux positifs).
% \end{note}

\end{technique}

La méthode \as{cc} a largement été utilisée pour détecter des fronts de SST (voir les nombreux exemples dans la review de \cite{belkin_2021}), notamment à l'échelle globale (\cite{belkin_2009a, belkin_2007}).
Elle a également été appliquée avec succès sur des données de Chlorophylle (\cite{stegmann_2004, kahru_2012, bontempi_2004}).
Diverses dérivations en ont été proposées.
\textcite{cayula_1995} proposent une version travaillant sur plusieurs images consécutives.
Plutôt que d'utiliser des images à des instants différents, \textcite{nieto_2012} applique l'algorithme \as{cc} sur quatres fenêtre glissantes décalées en longitude et latitude et fusionnent les résultats des pixels se chevauchant.
\textcite{miller_2009} combine les résultats de l'algorithme \as{cc} pour des produits SST et \as{chl} afin d'augmenter la couverture disponible.

Une méthode au fonctionnement similaire est proposée par \textcite{vazquez_1999}, qui pour quantifier la séparation des deux distributions en température (de part et d'autre d'un éventuel front) utilise une mesure entropique: la divergence de Jensen--Shannon (\cite{barranco-lopez_1995}).
\textcite{shimada_2005} reprendra cette méthode et l'adjoindra d'un algorithme de morphologie mathématique (\cite{jiang_1997}).
Cette configuration sera réutilisée à plusieurs reprises (\cite{lan_2012} \ab{par-ex}).

Enfin, dans le but de quantifier les valeurs de \as{chl} dans les fronts de sub-mesoéchelle de la gyre subtropicale du Pacifique Nord, \textcite{liu_2016} propose une méthode inspirée de l'algorithme \as{cc}.
Elles définissent un indice d'hétérogénéité (\as{hi}) du champ de SST.
La valeur de cette indice est calculé en chaque pixel à partir de la distribution en température dans une fenêtre (glissante donc) autour de ce pixel.
La valeur de l'indice est la somme (pondérée par des coefficient de normalisation) de la variance de la distribution, de sa bimodalité (calculée différement que dans les méthodes ci"-dessus), et de son asymétrie (afin de cibler les plus petits fronts).

On notera dans cette dernière étude une différence d'approche, essentiellement sémantique mais qui reste néanmoins d'intérêt.
C'est bien l'indice de l'\al{hi} qui apparaît comme au cœur de cette étude, qui finalement cherche plus à quantifier l'hétérogénéité spatiale de la SST pour chaque pixel, qu'à \guil{détecter les fronts}.
Par ailleurs, bien qu'elle reprenne l'utilisation d'une fenêtre glissante, cette méthode se démarque de celle de \as{cc} et ses variations, en donnant un résultat nuancé plutôt que purement dichotomique (front / pas-front).
De plus, bien que nous les avons omises des descriptions ci"-dessus, la plupart des méthodes incluent une étape algorithmique permettant d'obtenir en sortie uniquement la position des fronts, représentée par une ligne d'épaisseur nulle (ou 1~pixel).
En revanche, le positionement des fronts par \citeauthor{liu_2016} est réalisé par un simple\footnotemark{} seuil sur le HI.
\footnotetext{Bien que le choix du seuil soit fait de manière non"-triviale par \textcite{liu_2016}, nous montrons dans notre implémentation qu'un seuil fixe suffit à notre étude.}
Par cette construction orientée vers une mesure graduelle des fronts, cette méthode de détection apparaît pertinente pour distinguer différents types de fronts par intensité.

\begin{note}
  La démocratisation des techniques de \eng{machine learning} s'est aussi étendu au domaine de la détection des fronts.
  Bien que nous les ayons pas considérées ou décrites ici, ces dernières années, diverses méthodes ont été proposée (voir la review de \cite{liu_2022}).
\end{note}

\subsubsection{Vers des fronts de densité}

Lien entre densité et température.
La salinité intervient aussi. La salinité par satellite est très basse résolution (SMOS) donc on a pas vraiment accès.

On ne peut qu'espérer que la salinité ne joue pas un rôle trop grand dans la densité, ie que les fronts ne soient pas trop compensés.
Pour vérifier cela on doit passer par les campagnes en mer.

Dans la région Nord-Atlantique des transects sont réalisé régulièrement par un navire d'opportunité, l'Oleander.
Des résultats suggèrent que la salinité ne joue pas beaucoup (Flagg 2006, succinct).

\section{Motivation et objectifs}
\label{sec:problematique}

\begingroup
\defaultlists
\begin{itemize}
        \setlength{\topsep}{\baselineskip}
        \setlength{\itemsep}{\baselineskip}
        \renewcommand*\labelitemi{\adfrightarrowhead}
  \item question uno
  \item question dos
  \item question tres
\end{itemize}
\endgroup

\section{Plan de thèse}
\label{sec:plan-de-these}

Cette thèse en organisée en 6~chapitres, en comptant l'introduction ci"-dessus.
Dans le \nref[chapitre]{chp:methodes}, nous commencons par détailler les différents ensembles de données que nous avons considérés, et expliciter le
