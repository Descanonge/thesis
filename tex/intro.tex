% chktex-file 13

\chapter{Introduction}
\addChpLof
\label{chp:introduction}
\graphicspath{{resources/introduction}}

\minitoc%
\clearpage

\begin{figure}[!h]
  \centering
  {%
    \setlength{\fboxsep}{0pt}%
    \framebox[\figwidth]{\insertfig{gs_false_colors.jpg}}
  }%
  \captionT{Les couleurs du Gulf Stream par satellite}{%
    \small
    Image en fausses couleurs de l'océan Atlantique Nord, prises par \as{modis} \slashfrac{s}{}~Aqua le \frenchdate{2020}{02}{23}, retouchées par Norman Kuring (groupe \eng{Ocean Biology}, \abbrv{NASA}).
    Adapté de l'\eng{Image of the day} 2020-03-10 \textit{Hints of Spring in the Atlantic}, sur \eng{\glshref{ve-illustration} \& \glshref{eo-illustration}}.
  }
  \label{fig:oc-illustration}
\end{figure}

\vspace{1\baselineskip}

\section{Préambule}

L'essort de l'imagerie satellite appliquée à la couleur de l'océan \encadra{depuis la fin du \siecle{20}~siècle} a révélé la grande variabilité de la biologie aux échelles les plus fines de l'océan.
Cette variation de la couleur de l'océan, comme illustrée sur la \nref[figure]{fig:oc-illustration}, est due aux pigments du phytoplancton, une collection de micro"-organismes phytosynthétiques portés par les courants.
À la base de la chaîne trophique océanique, et partie centrale des cycles biogéochimiques océaniques \encadra{dont celui du carbone} la compréhension des phénomènes et facteurs régissant les évolutions du phytoplancton est cruciale.

Comme pour ses homologues terrestres, le phytoplancton a besoin de lumière et de nutriments.
Dans l'océan cependant, la répartition de ces deux composantes engendre un forcage particulier.
La lumière ne pénètre que dans une couche superficielle, profonde d'une centaine de mètres environ, la couche euphotique.
Les nutriments, rapidement consommés en surface, sont à l'inverse trouvés en profondeur.
Les échanges verticaux \encadra{de nutriments vers la couche euphotique, et de matière organique (ensuite reminéralisée) vers les profondeurs} sont ainsi nécessaires à la conservation d'un équilibre dans la pompe biologique ainsi décrite.

L'ensemble des courants de large échelle (\ab{cad} des basins océaniques \OM(\qty{1000}{\km})) définit une cartographie des caractéristiques biophysiques de l'environnement.
On peut distinguer par exemple les \guil*{deserts} que sont les gyres subtropicales, où les faibles échanges verticaux engendre un milieu très oligotrophe et peu productif.
À l'inverse, les zones d'\eng{upwelling} de bord est sont de véritables \guil*{forêts}, très productives, en raison des nutriments remontés avec les eaux profondes.
Une telle séparation à grande échelle entre deux biomes est par exemple bien visible sur la \nref[figure]{fig:oc-illustration}, où les eaux au sud du Gulf Stream sont peu productives (en bleu foncé), et celles au nord apparaissent beaucoup plus productives (en turquoise).
Toutefois, sur cette même image, il est également évident que des processus façonnent le paysage biologique à de plus petites échelles.
Par exemple, dans les méandres du Gulf Stream, et des tourbillons se forment.
Ici deux apparaissent en bleu foncé, ayant capturé des eaux du sud (chaudes et peu productives).

De tels tourbillons se forment aux \emph{méso"-échelles} (\qtyrange{10}{100}{\km}) et leurs effets sont multiples.
Comme on l'a vu dans l'exemple ci"-dessus, ils peuvent \guil*{capturer} des masses d'eaux et les transporter sur de larges distances (ici en transportant des masses d'eaux chaudes et oligotrophe au nord, \ab{cad} dans un milieu plus froid, productif, et riche en nutriments).
Ces tourbillons amplifient localement les échanges entre surface et profondeur, à la fois en générant des vitesses verticales, ainsi qu'en déplacant les isopycnes.
Néanmoins, ces échelles ne suffisent pas à décrire toute la variabilité biologique observée. De plus, leur contributions (estimées) à la pompe biologique ne permettent pas de boucler le budget de cette dernière <mal dit>.

En zoomant encore un peu plus sur la \nref[figure]{fig:oc-illustration}, nous pouvons distinguer des structures plus fine, dite de \emph{sub"-mésoéchelle} (\qtyrange{1}{10}{\km}).
Les champs de différentes variables (dont la densité) sont mélangés, étirés, par les courants des échelles supérieures (et les forcages atmosphériques), faisant apparaître des structures plus fines, et plus éphémères (entre le jour et la semaine).
Notamment, émergent des emplacements de fort gradient \encadra{autrement dit des fronts} de densité propices à la formations de circulations secondaires verticales, plus localisées mais aussi plus intenses que celles engendrées par les tourbillons de mésoéchelle.
Les fronts de sub-mésoéchelle, s'ils sont suffisamment marqués peuvent donc engendrer des circulations verticales suffisamment profondes pour remonter des nutriments dans la couche euphotique, et augmenter localement la productivité.

Une autre conséquence de ces gradients, due à leur tendance à applanir les isopycnes par instabilité barocline, est de restratifier les couches supérieures (toujours localement).
La restratification étant un facteur fort dans le démarrage du bloom printanier, ces structures de sub-mésoéchelles peuvent donc également conduire à des démarrages précoces (localement).

Enfin, bien que les effets de la sub-mésoéchelle soient locaux, ils peuvent néanmoins avoir une rétroaction sur les échelles supérieures, modifiant la circulation de grande échelle, ou la productivité d'un bassin océanique.

Les effets de la sub"-mésoéchelle sur les cycles biogéochimiques présentés ci"-dessus ne sont toutefois pas encore complétement élucidés, malgré les efforts des dernières décennies dans ce domaine.
En effet, l'étude des processus à ces échelles présente des difficultés.
Les diverses méthodologies usuelles peinent à donner une vue entière du problème.
Capturer les variations rapide de la biologie en travers d'un front lors d'une campagne in"-situ présente un défi technique certain.
De plus, la courte période de vie des structures d'intérêt rend la tâche d'autant plus ardue.
Il en va de même pour l'imagerie satellite, pour laquelle la couverture nuageuse rend difficile de suivre temporellement une structure.
Par ailleurs, il est compliqué pour les satellites d'accéder à toute la biodiversité du phytoplancton, ainsi qu'au delà des premiers mètres en surface.
Enfin, les simulations numériques peuvent résoudre ces fines échelles, mais à un coût calculatoire élevé et prohibitif sur de trop longues durées (des projections climatiques \ab{par-ex}).

Malgré les limites évoquées, l'imagerie satellite donne l'opportunité d'observer ces effets à une large échelle (spatiale et temporelle), et permettrait d'en fournir une quantification.
Une telle démarche a été entreprise par \textcite{liu_2016}, dans la gyre subtropicale du Pacifique Nord.
Les auteur·ices \encadra{en colocalisant les valeurs satellites de la \al{chl} avec les positions de fronts détectés à partir de la température de surface satellite} ont montré une augmentation des valeurs de \as{chl} dans les fronts par rapport au reste de la zone.
Nous étendons leur méthode, et l'appliquons à la région de l'Atlantique Nord, autour du Gulf Stream.

Cette zone comprend trois biomes, \al{cad} trois régimes biogéochimiques différents.
Au sud de notre zone, la gyre subtropicale est une zone oligotrophe, d'une productivité faible tout au long de l'année.
Au nord de celle-ci mais au sud du Gulf Stream, ce régime oligotrophe est mitigé par l'approfondissement de la couche de mélange en hiver, apportant des nutriments.
Enfin, au nord du Gulf Stream, on trouve des eaux riches et productives, berceau d'un fort bloom au printemps, et d'un second bloom en automne.

Ces trois régions sont également hétérogène en terme des structures qu'on peut y trouver.
Le front de densité (très marqué) associé au Gulf Stream est un réservoir d'énergie potentielle, qui est convertie en énergie cinétique par le biais des structures de fines échelles qui nous intéressent.
Nous nous attendons donc à trouver des fronts de densité plus stables, associés à de forts gradients, et donc à d'intenses vitesses verticales dont il est plus probable qu'elle puisse atteindre les nutriments en profondeur.
En revanche, loin de ce courant intense, dans la gyre subtropicale, nous pourrons étudier des fronts et gradients plus faibles.
Cette variété nous permet d'étudier la réponse biologique aux fronts, en fonction de leur intensité.

Afin de mieux définir nos objectifs, et avant de les présenter, nous dressons dans la section suivante un état de l'art des sujets concernés.
Nous exposerons ensuite le plan de ce manuscript.

\section{État de l'art}
\label{sec:etat-de-lart}

\subsection{Le phytoplancton dans le système terre}
\label{sec:phyto-ds-sys-terre}

La biologie marine, à sa base, repose sur les conversions d'un certain nombre d'éléments entre eux.
Schématiquement, le phytoplancton converti par photosynthèse le carbone et les nutriments dissouts dans l'eau, en dioxygène et en matière organique (pour sa croissance).
À la mort de ces organismes, ce stock de matière organique inerte plonge par gravité, et sera éventuellement reminéralisé plus en profondeur par d'autres organismes.
Le carbone et autres nutriments libérés pourront alors être remontés à la surface et réutilisés, complétant la pompe biologique marine.
Le phytoplancton joue un rôle clef dans


\subsubsection{Définitions générale}
\label{sec:phyto-def-gen}

Le phytoplancton étant au cœur de la problématique qui nous intéresse ici, je m'arrête ici brièvement sur ses caractéristiques.
Le plancton \guil*{\hbox{πλαγκτόσ}}: il \guil*{divague}.
C'est l'ensemble des organismes dont la motilité se lui permet guère plus que de se laisser dérivier au gré des courants.
Bien que certains de ces organismes puissent se déplacer de manière significative par leurs propres moyens, concentrons"-nous plutôt sur le sous"-ensemble moins mobile du phytoplancton.
Cet ensemble d'organismes est remarquablement divers, totalisant environ \num{20 000}~espèces, dont les tailles varient de la fraction de micromètre à la fraction de millimètre. Cette diversité se retrouve aussi bien dans leurs formes, au grand bonheur des zoologues et photographes, que dans leurs fonctions écosystèmiques.
Comme leur nom le suggère, les espèces de phytoplancton pratiquent la photosynthèse.
Ils transforment le dioxyde de carbone et les nutriments (Nitrate, Phosphate, Silice, Fer,\dots) dissous, en oxygène et en matière organique.
À l'instar des plantes terrestres, cette opération nécessite de capter de la lumière à l'aide de pigments, qui se trouvent être principalement la \al{chl}.
Ce sont ces pigments qui affectent la couleur de l'océan comme nous l'avons vu dans le préambule (\nref{fig:oc-illustration}), mais nous reviendrons sur cet aspect plus loin.

Cette production de matière organique à partir de carbone in"-organique (du CO2 dissous), est dite primaire.
Le phytoplancton étant prédaté par le zooplancton, cette matière organique remonte éventuellement la chaîne (ou plutôt réseau) trophique: zooplancton, petits prédateurs, grands prédateurs, oiseaux marins, mamifères marins,\dots
Le phytoplancton occupe donc une place centrale dans l'écosystème marin.
Il est à la base de la chaîne alimentaire dans les océans, et il est estimé qu'il génère~\pct{50} de la productivité primaire mondiale.

Pour croître: ont besoin de lumière, de nutriments, de certaines conditions environnementales (température, salinité?, acidité). Dans le détail il y a des spécificités selon chaque espèce.
Sources de mortalité: vieillesse, virus, broutage par le phytoplancton
Quelques mots sur le phytoplancton Plus gros. Migration diurne. Ici aussi grande variété, préférence de broutage: complexité supplémentaire.

Les limitations majeures restent la lumière et les nutriments.
La lumière n'est pas présente partout identiquement: latitude, saisonnalité, et surtout profondeur. Importance de la couche euphotique, en surface.
Également limité en nutriments dans la quasi-totalité des eaux libres. Les eaux profondes (ou la lumière ne pénètre pas) contiennent des nutriments cependant. On peut définir une nutricline.

Cette dualité des sources de croissance (lumière en surface, nutriments en profondeur) rend les échanges verticaux très importants.
Or on considère généralement les courants océaniques à l'ordre 1 comme strictement horizontaux.
Les petites et moyennes échelles ((sub-)mésoéchelles) présentent des moyens de créer des échanges verticaux (comme on le verra plus tard).

\subsubsection{Télédétection}
\label{sec:teledetection}

Le phytoplancton est complexe à étudier.
Une observation synoptique est seulement possible par satellite.
Ces obs se font sur la couleur de l'océan. (quelques mots sur le principe, depuis quand ça existe, révélation des fines échelles dans les 1ères images).
Obs de la couleur présente des problèmes. Le lien avec le phytoplancton est indirect et imparfait (utilisation de la Chl-a comme proxy de biomasse). Limitation importante des observations par la couverture nuageuse. Et cela ne scanne que la composition de la surface (combien de m en vrai ?), pas accès à toute la colonne d'eau.

Expliquer la méthode pour obtenir Chl-a depuis satellite. Et méthode de Roy également (rapidement).

C'est pour cela que les observations in-situ toujours très utiles (pour combler les trous, ou compléter les obs sat). Vision de toute la colonne d'eau. Associé à une obs de la densité (temp + salinité qui n'est pas dispo en sat. à petite échelle).
Accès à la composition du phytoplancton (et zoo) grâce à différents outils (cytomètre, zooscan, HPLC, -omiques).
Mais aussi limitations. Toutes ces obs sont compliquées à mettre en place. Avoir une vue de toute la colonne d'eau pour tous ces paramètres (physiques, bio phyto + zoo) est compliqué techniquement. Une obs est limitée à une très petite zone spatio-temporelle. Obligé de cibler une zone d'intérêt (supposé).
Heureusement ces dernières années de telles campagnes se multiplient, devant la nécessité d'avoir tous ces paramètres dans des zones précises pour mieux comprendre ce qu'il se passe.

J'ai parlé que des obs mais des modèles numériques sont aussi dispo cependant.
Tous les modèles sont faux. Tous les modèles bio sont très faux.
La très grande variabilité des organismes est mal représentée (faute de puissance entre autre). Les petites échelles qui sont pourtant si importantes à la bio (subméso) sont très coûteuses à faire tourner et on est limité à de petites régions géographiques. Impossible de quantifier les effets convenablement avec des modèles.
On a pas de paramétrisation des processus bio. Est-ce que c'est seulement possible sachant qu'on peut avoir des effets non-locaux potentiellement ?
Les modèles climatiques sont, du coup, très incertains en ce qui concerne la bio.
Grosses barres d'erreurs dans les projections pour le prochain siècle.

Difficile de séparer les trois: processus dynamiques, mélange horizontal, et biologie, car ils ont les même temps caractéristiques.

\subsection{Interactions biophysiques}
\label{sec:interactions-biophys}

Définition par l'échelle spatiale (0.1-10km).

Ce sont les images satellite de couleur de l'océan qui ont révélé l'ubiquité des fines échelles dans l'océan (en surface tout du moins).

Importance de la SMS:
1) on observe que la variance de la biophysique est importante à ces échelles.
2) les processus dynamiques crée des échanges verticaux
3) Mal représenté dans les modèles climatiques, on doit mieux comprendre (et quantifier pour savoir à quel point c'est important)


\subsubsection{Mélange horizontal}
\label{sec:melange-horizontal}

Une partie des fines échelles observées correspondent à l'action de l'advection par les courants de méso-échelle. L'action passive du mélange horizontal fait apparaître de fins filaments.
Cela permet le mélange de communauté, car rapproche spatialement des parcelles d'eau de différentes origines, avec des caractéristiques physiques et des historique biologiques propres.

Approche lagrangienne très utile dans l'étude du phytoplancton, et dans son observation (méthode de sampling Lagrangiennes, ref ).


\subsubsection{Upwelling de nutriments par les circulations agéostrophiques}
\label{sec:upwelling-nutriments}

Comme dit plus haut, les échanges verticaux sont important pour la bio.
Or aux petites échelles on voit apparaître des vitesses verticales de grande magnitude.

à ces échelles (0.1-10km) émergent également des processus dynamiques nouveaux.
On est alors en dessous du rayon de déformation de Rossby (\(Ro < 1\)).
Forçage par l'atmosphère (hétérogène) et les courants méso qui génèrent des gradients de densité.
On décrit certains de ces processus et leur(s) impact(s) dans la suite.

advection selon les isopycnes qui peuvent être penchées.

importance des vents

aspect intermittent (local, petit spot)

winners and losers


\subsubsection{Modification de la phénologie du bloom}
\label{sec:modif-phenologie}

Les gradients de densité existant (créés par mélange par courants méso ou forçages atmos) sont des réservoirs d'énergie potentielle. À un front les isopycnes sont penchées et des circulations sub-méso se créent et transforment l'énergie potentielle en énergie cinétique (tourbillons), ce faisant ramenant les isopycnes à l'horizontale.
Ces tourbillons formés par l'instabilité de Mixed-Layer (?) s'étendent sur la hauteur de la ML. Ce sont les Mixed-Layer Eddies.
À travers ces instabilités la sub-mesoéchelle contribue fortement à re-stratifier la couche  de surface, et réduire le mélange.
C'est un processus local et on s'attend donc à un soulèvement local de la couche de mélange au niveau des fronts.


Expliquer lancement du bloom printanier par réduction mélange.

On s'attend à un départ du bloom d'abord dans les fronts.

exemples de détection précédentes (mahadevan 2020).


\subsection{Région d'étude: Extension du Gulf Stream}
\label{sec:region-detude}

Notre zone d'étude: 15°N-55°N, 82°W-40°W

\subsubsection{La physique}
\label{sec:gs-physique}

Gyre subtropicale Atlantique Nord.

Courant de bord ouest chaud et salé qui remonte des caraïbes le long de la Floride
Se détache à Cape Hatteras, quitte le plateau continentale et méandre vers l'est.
Plume énergétique autour de ces méandres (surtout au sud).

Au nord du jet, courant retour (slope current) avec notamment un jet sur le shelf break.
Entre le Gulf Stream North Wall et le shelf: slope seas. Très froid, plutôt fraîche (plus salé que sur le shelf néanmoins).
Ce courant froid plonge. Fait partie de la circulation d'overturning de l'atlantique.

\subsubsection{La biologie}
\label{sec:gs-biologie}

Gyre très pauvre en nutriment et productivité faible.
Pourquoi ?
Pas de circulation méso qui permette l'apport de nutriment en surface.
Importance de la sub-méso donc pour créer des échanges verticaux.

Eaux au nord du GS très productives.
Également importances à cause de pêcheries.
Important bloom (plus de détail, ref sur la phénologie, )


\subsection{Détection des fronts sur images satellites}
\label{sec:detection-fronts}

On se concentre sur la SST, mais on peut utiliser d'autres traceurs (Chloro par ex).

méthodes avec gradient

méthodes se basant sur histogrammes (Otsu, Cayula)
méthodes avec fenêtre glissante

méthodes utilisant variance, entropie

Quelques mots sur le suivi des fronts + reconstruction de fronts 'linéaire'
qu'on utilise pas on reste en 'vue pixelisée'

\subsubsection{Vers des fronts de densité}


Lien entre densité et température.
La salinité intervient aussi. La salinité par satellite est très basse résolution (SMOS) donc on a pas vraiment accès.

On ne peut qu'espérer que la salinité ne joue pas un rôle trop grand dans la densité, ie que les fronts ne soient pas trop compensés.
Pour vérifier cela on doit passer par les campagnes en mer.

Dans la région Nord-Atlantique des transects sont réalisé régulièrement par un navire d'opportunité, l'Oleander.
Des résultats suggèrent que la salinité ne joue pas beaucoup (Flagg 2006, succinct).

\section{Motivation et objectifs}
\label{sec:problematique}

\begingroup
\defaultlists
\begin{itemize}
        \setlength{\topsep}{\baselineskip}
        \setlength{\itemsep}{\baselineskip}
        \renewcommand*\labelitemi{\adfrightarrowhead}
  \item question uno
  \item question dos
  \item question tres
\end{itemize}
\endgroup

\section{Plan de thèse}
\label{sec:plan-de-these}

comment gérer l'article
