% chktex-file 13

\chapter{Introduction}
\addChpLof
\label{chp:introduction}

\minitoc%
\clearpage

\section{Préambule}

Quelques pages d'introduction générale pour donner les objectifs.

Le plancton objet intéressant.
Importance dans le cycle du carbone et la chaine alimentaire.
Complexe car grande variété et du fait le l'environement pas fixe.
Porté par les courants: impacté par des phénomènes physiques.

D'une part à grande échelle (basins, circulation océaniques) où sont définis les caractéristiques physique de l'environnement.
Mais plus récemment, notamment grâce à l'imagerie satellites, on se rend compte que la bio est forcée aux fines échelles (quelques jours, quelques kms). Échelles ou apparaissent non seulement de la filamentation par mélange horizontal, mais aussi des circulations agéostrophiques qui affectent les échanges verticaux (par des vitesses verticales et par modification du mélange vertical).
Expliquer l'importance des fronts dans ces 2 phénomènes.

paragraphe upwelling.

paragraphe restratification.

Pourtant il est d'importance de pouvoir mieux comprendre ces interactions biophysiques et de les quantifier.
D'une part parce que visiblement ça régit la bio à ces échelles, mais ça peut changer significativement l'image à grande échelle (preuve ?).
D'autre part parce que les modèles climatiques qui font les prévisions à l'aune du CC sont incapables (et le resteront pour un bout de temps) de résoudre ces échelles. Il faut donc arriver à donner une paramétrisation des ces effets pour obtenir des prévisions sensées (ce qui est également vrai pour la physique seule).
La quantification des interactions biophysique aux fines échelles est encore à faire (quelques exemples existent).

Il est difficile d'observer le plancton à ces échelles. Les campagnes in-situ doivent sampler précisement des structures fines, et on arrive à la limite de résolution des satellites (qui ne fournissent que la biomasse totale en surface).

Malgré ses limites, l'imagerie satellite permet une observation synoptique à large échelle (spatio-temporelle).
De plus de récentes avancés donnent également accès à la composition du phytoplancton.
Opportunité de quantifier ces effets sur une large zone (avant de passer à une étude globale).
On choisit pour cela la région Nord Atlantic autour du Gulf Stream. On a accès sur une zone relativement restrainte à différent biomes (olligotrophe, bloom important dans une zone subpolaire, bloom faible dans une zone olligotrophe mais avec apport de nutriments par ML profonde en hiver).

Objectifs:
\begin{itemize}
  \item
  \item
  \item
\end{itemize}


\section{État de l'art}
\label{sec:etat-de-lart}

\subsection{Le phytoplancton dans le système terre}
\label{sec:phyto-ds-sys-terre}

\subsubsection{Définition générale}
\label{sec:phyto-def-gen}

définition: plancton = organisme porté par les courants.
Phyto: ils pratiquent la photosynthèse, ils transforment le carbone dissous dans l'eau en matière organique.
Grande variété: de taille et de type et d'espèces.

50\% de la productivité primaire.
base de la chaîne alimentaire dans les océans.

Pour croître: ont besoin de lumière, de nutriments, de certaines conditions environnementales (température, salinité?, acidité). Dans le détail il y a des spécificités selon chaque espèce.
Sources de mortalité: vieillesse, virus, broutage par le phytoplancton
Quelques mots sur le phytoplancton Plus gros. Migration diurne. Ici aussi grande variété, préférence de broutage: complexité supplémentaire.

Les limitations majeures restent la lumière et les nutriments.
La lumière n'est pas présente partout identiquement: latitude, saisonnalité, et surtout profondeur. Importance de la couche euphotique, en surface.
Également limité en nutriments dans la quasi-totalité des eaux libres. Les eaux profondes (ou la lumière ne pénètre pas) contiennent des nutriments cependant. On peut définir une nutricline.

Cette dualité des sources de croissance (lumière en surface, nutriments en profondeur) rend les échanges verticaux très importants.
Or on considère généralement les courants océaniques à l'ordre 1 comme strictement horizontaux.
Les petites et moyennes échelles ((sub-)mésoéchelles) présentent des moyens de créer des échanges verticaux (comme on le verra plus tard).

\subsubsection{Télédétection}
\label{sec:teledetection}

Le phytoplancton est complexe à étudier.
Une observation synoptique est seulement possible par satellite.
Ces obs se font sur la couleur de l'océan. (quelques mots sur le principe, depuis quand ça existe, révélation des fines échelles dans les 1ères images).
Obs de la couleur présente des problèmes. Le lien avec le phytoplancton est indirect et imparfait (utilisation de la Chl-a comme proxy de biomasse). Limitation importante des observations par la couverture nuageuse. Et cela ne scanne que la composition de la surface (combien de m en vrai ?), pas accès à toute la colonne d'eau.

Expliquer la méthode pour obtenir Chl-a depuis satellite. Et méthode de Roy également (rapidement).

C'est pour cela que les observations in-situ toujours très utiles (pour combler les trous, ou compléter les obs sat). Vision de toute la colonne d'eau. Associé à une obs de la densité (temp + salinité qui n'est pas dispo en sat. à petite échelle).
Accès à la composition du phytoplancton (et zoo) grâce à différents outils (cytomètre, zooscan, HPLC, -omiques).
Mais aussi limitations. Toutes ces obs sont compliquées à mettre en place. Avoir une vue de toute la colonne d'eau pour tous ces paramètres (physiques, bio phyto + zoo) est compliqué techniquement. Une obs est limitée à une très petite zone spatio-temporelle. Obligé de cibler une zone d'intérêt (supposé).
Heureusement ces dernières années de telles campagnes se multiplient, devant la nécessité d'avoir tous ces paramètres dans des zones précises pour mieux comprendre ce qu'il se passe.

J'ai parlé que des obs mais des modèles numériques sont aussi dispo cependant.
Tous les modèles sont faux. Tous les modèles bio sont très faux.
La très grande variabilité des organismes est mal représentée (faute de puissance entre autre). Les petites échelles qui sont pourtant si importantes à la bio (subméso) sont très coûteuses à faire tourner et on est limité à de petites régions géographiques. Impossible de quantifier les effets convenablement avec des modèles.
On a pas de paramétrisation des processus bio. Est-ce que c'est seulement possible sachant qu'on peut avoir des effets non-locaux potentiellement ?
Les modèles climatiques sont, du coup, très incertains en ce qui concerne la bio.
Grosses barres d'erreurs dans les projections pour le prochain siècle.

Difficile de séparer les trois: processus dynamiques, mélange horizontal, et biologie, car ils ont les même temps caractéristiques.

\subsection{Interactions biophysiques}
\label{sec:interactions-biophys}

Définition par l'échelle spatiale (0.1-10km).

Ce sont les images satellite de couleur de l'océan qui ont révélé l'ubiquité des fines échelles dans l'océan (en surface tout du moins).

Importance de la SMS:
1) on observe que la variance de la biophysique est importante à ces échelles.
2) les processus dynamiques crée des échanges verticaux
3) Mal représenté dans les modèles climatiques, on doit mieux comprendre (et quantifier pour savoir à quel point c'est important)


\subsubsection{Mélange horizontal}
\label{sec:melange-horizontal}

Une partie des fines échelles observées correspondent à l'action de l'advection par les courants de méso-échelle. L'action passive du mélange horizontal fait apparaître de fins filaments.
Cela permet le mélange de communauté, car rapproche spatialement des parcelles d'eau de différentes origines, avec des caractéristiques physiques et des historique biologiques propres.

Approche lagrangienne très utile dans l'étude du phytoplancton, et dans son observation (méthode de sampling Lagrangiennes, ref ).


\subsubsection{Upwelling de nutriments par les circulations agéostrophiques}
\label{sec:upwelling-nutriments}

Comme dit plus haut, les échanges verticaux sont important pour la bio.
Or aux petites échelles on voit apparaître des vitesses verticales de grande magnitude.

à ces échelles (0.1-10km) émergent également des processus dynamiques nouveaux.
On est alors en dessous du rayon de déformation de Rossby (\(Ro < 1\)).
Forçage par l'atmosphère (hétérogène) et les courants méso qui génèrent des gradients de densité.
On décrit certains de ces processus et leur(s) impact(s) dans la suite.

advection selon les isopycnes qui peuvent être penchées.

importance des vents

aspect intermittent (local, petit spot)

winners and losers


\subsubsection{Modification de la phénologie du bloom}
\label{sec:modif-phenologie}

Les gradients de densité existant (créés par mélange par courants méso ou forçages atmos) sont des réservoirs d'énergie potentielle. À un front les isopycnes sont penchées et des circulations sub-méso se créent et transforment l'énergie potentielle en énergie cinétique (tourbillons), ce faisant ramenant les isopycnes à l'horizontale.
Ces tourbillons formés par l'instabilité de Mixed-Layer (?) s'étendent sur la hauteur de la ML. Ce sont les Mixed-Layer Eddies.
À travers ces instabilités la sub-mesoéchelle contribue fortement à re-stratifier la couche  de surface, et réduire le mélange.
C'est un processus local et on s'attend donc à un soulèvement local de la couche de mélange au niveau des fronts.


Expliquer lancement du bloom printanier par réduction mélange.

On s'attend à un départ du bloom d'abord dans les fronts.

exemples de détection précédentes (mahadevan 2020).


\subsection{Région d'étude: Extension du Gulf Stream}
\label{sec:region-detude}

Notre zone d'étude: 15°N-55°N, 82°W-40°W

\subsubsection{La physique}
\label{sec:gs-physique}

Gyre subtropicale Atlantique Nord.

Courant de bord ouest chaud et salé qui remonte des caraïbes le long de la Floride
Se détache à Cape Hatteras, quitte le plateau continentale et méandre vers l'est.
Plume énergétique autour de ces méandres (surtout au sud).

Au nord du jet, courant retour (slope current) avec notamment un jet sur le shelf break.
Entre le Gulf Stream North Wall et le shelf: slope seas. Très froid, plutôt fraîche (plus salé que sur le shelf néanmoins).
Ce courant froid plonge. Fait partie de la circulation d'overturning de l'atlantique.

\subsubsection{La biologie}
\label{sec:gs-biologie}

Gyre très pauvre en nutriment et productivité faible.
Pourquoi ?
Pas de circulation méso qui permette l'apport de nutriment en surface.
Importance de la sub-méso donc pour créer des échanges verticaux.

Eaux au nord du GS très productives.
Également importances à cause de pêcheries.
Important bloom (plus de détail, ref sur la phénologie, )


\subsection{Détection des fronts sur images satellites}
\label{sec:detection-fronts}

On se concentre sur la SST, mais on peut utiliser d'autres traceurs (Chloro par ex).

méthodes avec gradient

méthodes se basant sur histogrammes (Otsu, Cayula)
méthodes avec fenêtre glissante

méthodes utilisant variance, entropie

Quelques mots sur le suivi des fronts + reconstruction de fronts 'linéaire'
qu'on utilise pas on reste en 'vue pixelisée'

\subsubsection{Vers des fronts de densité}


Lien entre densité et température.
La salinité intervient aussi. La salinité par satellite est très basse résolution (SMOS) donc on a pas vraiment accès.

On ne peut qu'espérer que la salinité ne joue pas un rôle trop grand dans la densité, ie que les fronts ne soient pas trop compensés.
Pour vérifier cela on doit passer par les campagnes en mer.

Dans la région Nord-Atlantique des transects sont réalisé régulièrement par un navire d'opportunité, l'Oleander.
Des résultats suggèrent que la salinité ne joue pas beaucoup (Flagg 2006, succinct).

\section{Motivation et objectifs}
\label{sec:problematique}

\section{Plan de thèse}
\label{sec:plan-de-these}
