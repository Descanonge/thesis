\documentclass[index]{subfiles}
\begin{document}

\chapter{Méthodes}
\label{chp:methodes}

\tocsubfile

\section{Données}
\label{sec:donnees}

Dans cette section sont décrit les jeux de données utilisés pour ce projet.
Le choix de ces données s'est avéré être une question importante et difficile dès le départ du projet.
Nous savions qu'il allait être nécessaire de combiner plusieurs jeux de données (\gls{SST} avec \gls{chla}, puis avec des données de composition de la communauté de phytoplancton).
Cela a donc informé le choix des données, ainsi que la création des outils nécessaires à la gestion de plusieurs sets de données.

Plusieurs options ont été considérées, notamment pour la \gls{SST}.
Elles sont reportées ci-dessous, même si elles ne sont pas utilisées pour le reste des résultats.

\subsection{SST}
\label{sec:donnees-sst}

petite explication des difficultés lié au choix sst ?
bonne rés, exactitude des fronts ?

\subsubsection{MODIS-1km}
\label{sec:donnees-sst-modis}

Un des premiers jeu de données \gls{SST} à considérer est celui utilisé par \textcite{liu_2016}, dont l'étude consistait également à détecter des fronts de \gls{SST} et les colocaliser aux données de \gls{chla}.
Ces variables étant rarement distribuées sur des grilles à des résolutions kilométriques, il apparaît comme nécessaire de \enquote{re-grider} des produits L2 (données d'un capteur converties en variables géophysiques, disposées à la résolution de capture).

Pour le jeu de données intitulé \verb|MODIS-1km| dans la suite, il s'agit des données provenant du capteur \gls{MODIS}, à bord du satellite Aqua, développées par \textcite{sst_modis, kilpatrick_2015}.
Les données sont téléchargées depuis le \href{https://cmr.earthdata.nasa.gov/search/}{Common Metadata Repository (CMR)}, en sélectionnant les swaths intersectant notre région d'étude, collectés de jour, et pour les courtes longueurs d'onde infrarouges (\qty{11}{\um}).
Pour ensuite regridder les données sur une grille kilométrique globale, j'ai repris les outils utilisés par \textcite{liu_2016}, à savoir les programmes présents dans le paquet de l'\href{https://oceandata.sci.gsfc.nasa.gov/ocssw}{OceanColor Science Software (OCSSW)}, version v7.5, qui est maintenu par l'Ocean Biology Processing Group et notamment distribué à travers l'outil \href{https://seadas.gsfc.nasa.gov/}{SeaDAS}.
Les swaths, maintenant tous sur la même grille spatiale peuvent être moyennés par date.

Cela permet donc d'obtenir des données à la limite de la résolution des capteurs, mais présente un certain nombre de désavantages majeurs.
C'est tout d'abord un travail important à réaliser: d'abord de téléchargement les fichiers L2 étant particulièrement lourds et nombreux (jusqu'à une dizaine pour une journée); ensuite de regrillage, qui à une résolution kilométrique demande une certaine puissance de calcul.
De plus, toutes les étapes décrites jusqu'ici ne concernent qu'un seul capteur.
Des étapes et calculs additionnels seraient nécessaires pour y combiner les données d'autres capteurs, ce qui pose des problèmes supplémentaires et se ferait non sans difficultés.
Même pour un capteur identique (\gls{MODIS} à bord de Terra par exemple), il devient nécessaire de considérer des différences de calibrations, ainsi que de possibles artefacts lorsque l'on superpose plusieurs swaths (qui d'ailleurs ont été largement ignorés pour agréger les swaths sur une seule journée\dots).

Face à ces difficultés, nous avons donc décidé dans un premier temps d'en rester à un seul capteur malgré la couverture spatiale réduite, notamment à cause des nuages.
Il est également à noter qu'à ce niveau, peu d'actions ont été prises pour disqualifier les pixels nuageux.
Il en résulte des pixels aux valeurs visiblement erronées mais qui n'ont pas été masqués comme nuageux.
Cela se manifeste par exemple comme du bruit autour de nuages.
Il est difficile de les disqualifier correctement sans un bon algorithme de détection de nuage, ce qui nécessite encore un large travail.
Nous contournons le problème d'une manière similaire à celle de \textcite{liu_2016} en ne travaillant que sur des fenêtres de \qtyproduct{100 x 100}{\km} avec une faible couverture nuageuse (\textless\qty{70}{\percent}), ce qui élimine une partie de ces cas problématiques.

TODO: avantage modis il y a l'OC aussi !

Il est néanmoins possible de trouver d'autres jeux de données permettant notre étude et nécessitant moins de travail (et donc de possible erreurs), ces dernières années ayant vu apparaître des données distribuées à des résolutions élevées comme c'est le cas dans le paragraphe suivant.

\subsubsection{MUR}
\label{sec:donnees-sst-ghrsst}

Le produit suivant, désigné ici \gls{MUR} est dévelopé par le \gls{GHRSST} au JPL Physical Oceanography DAAC \parencite{sst_mur}.
Il présente les avantages d'être distribué à une résolution kilométrique et sans nuages grâce à une méthode d'interpolation par vaguelettes \parencite{chin_2017}.
Il intègre de nombreuses sources provenant de plusieurs capteurs infrarouges et micro-ondes, ainsi que de mesures in-situ.
La grande couverture spatiale de ce produit se fait donc par l'utilisation de mesures non-contraintes par la couverture nuageuse (c'est-à-dire micro-ondes et in-situ), mais qui présente une résolution bien plus faible.
Le champ de \gls{SST} se retrouve donc lissé à la fois par l'interpolation et par l'inclusion de ces mesures.

L'utilisation du produit suivant cherche à palier à ce problème.

\subsubsection{ESA SST CCI and C3S}
\label{sec:donnees-sst-esacci}

Ce produit est distribué conjointement par l'\gls{ESA} \gls{SST} \gls{CCI} et le \gls{C3S}.
Il agglomère les mesures de tous les capteurs infrarouges disponibles depuis 1981, ce qui comprend 11 \gls{AVHRR} à bord des satellites \gls{MetOp}[-A] et \gls{NOAA} (entre les itérations 6 et 19) et trois \gls{ATSR} à bord de \gls{ERS} 1 et 2, et Envisat.
Cela donne au moins deux capteurs en fonctionnement simultané, et au moins trois depuis 1992 .
Outre la bonne couverture obtenue, l'uniformité des capteurs permet d'obtenir un jeu de données stable dans le temps, pouvant servir à la détection de tendance climatiques \parencite{marchant_2019}.

Les données sont distribuées à différents niveaux:
\begin{itemize}
  \item L2P mono-capteur pré-traité,
  \item L3U mono-capteur grille régulière,
  \item L3C multi-capteur (regroupé par famille d'instrument),
  \item L4 multi-capteur analysé.
\end{itemize}
La grille utilisée a une résolution de \qty{0.05}{\degree} en latitude et longitude (\(\sim\qty{4}{\km}\)).

Le produit L2P peut être utilisé de la même manière que pour \cref{sec:donnees-sst-modis} afin d'obtenir une résolution la plus haute possible, mais avec l'avantage de disposer de données uniformes sur un grand nombre de capteurs.
Il est à noter que les outils à utiliser pour regriller pourrait ne pas directement fonctionner sur ces données.

Les produits L3U et L3C peuvent être utilisés dans l'étude de structures fines et où la présence de nuages n'est pas <incompatible>.

Enfin, le produit L4 regroupe l'ensemble des capteurs en utilisant le schéma d'intégration variationnelle \gls{NEMOVAR}, intégré dans le système \gls{OSTIA}.
Ce produit donne ainsi un champ de \gls{SST} sans nuages, estimé par une combinaison des observations satellites et de la prévision d'un modèle numèrique d'un jour sur l'autre.
L'absence de couverture nuageuse et la simplicité d'usage se fait au détriment du lissage inévitable des structures les plus fines.

L'estimation du lissage est compliqué (dépend du std de la solution à J-1).
On peut imaginer que moins de fronts sur zone nuageuse, ou en tout cas moins exact (cf réanalyses).
Donc dans la suite, on prend nos résultats que si Chl clear sky.

grille différente que celle de la chlorophylle.
explication regrillage (simple interpolation 2D)

montrer comparaison (même image plusieurs datasets)

+ Chl pas dispo à 1km
(cf \cref{sec:donnees-chl})

Sauf indication contraite, c'est le produit L4 que nous utiliserons dans toute la suite de cette thèse.
Il permet d'accéder à des données \gls{SST} simplement, avec une résolution <améliorable> mais convenable, et stable sur une longue durée (plusieurs décennies).

aieaieouie

\subsubsection{Réanalyses}
\label{sec:donnees-sst-reanalyses}

Montrer décalage

\subsection{Chlorophylle}
\label{sec:donnees-chl}

Globcolour processor
distribué par ESA/CCI

4km, L4 (capteurs merged, pas d'interpolation spatio-temp)

à 1km c'est difficile à trouver, seulement quand on regride MODIS
dataset: \parencite{chl_modis}
Même processing que pour 4km normalement


\subsection{Bathymétrie}
\label{sec:donnees-bathymetrie}


\section{Délimitations des sous-régions}
\label{sec:delimitations-regions}

Zone d'étude hétérogène, notamment en biologie.
Il apparait 3 régimes différents discuté en intro.
Pour limiter les comparaisons dans des zones homogènes on découpe la zone d'étude en 3 sous régions.

Une première zone correspondant à la gyre subtropicale de l'Atlantique Nord est délimitée comme étant au sud de 32°N.
Cela correspond à un saut des valeurs de \gls{chla} à cette latitude.
Cette séparation est en accord avec la limite entre les biomes présentés par \cite{sarmiento_2004}.
Cette zone correspond donc à la gyre stratifiée de façon permanente (GS-PS) [GLS ?].

On sépare ensuite la région restante au nord de 32°N en prenant comme limite le front nord du jet du Gulf Stream (le `North wall'). Cette séparation donne donc au sud, une zone peu productive, la gyre subtropicale stratifié de façon saisonière (GS-SS) [GLS ?]; et au nord des eaux plus froides, plus fraîches, et plus productives correspondan au régime de fort bloom (High-Chlorophyll-Bloom) \pcite{sarmiento_2004} ou à des eaux subpolaires \pcite{bock_2022}.

La délimitation entre ces deux zones suit le front nord du jet du Gulf Stream et est donc dynamique. Elle est déterminée pour chaque jour, à partir de l'image en température, de la manière suivante.
Après exploration des données au long de l'année, il apparaît que la SST est un marqueur relativement fiable et robuste, et que la limite qui nous intéresse ici est facilement repérable dans les valeurs de SST.
En effet, nous regardons la distribution des valeurs de SST au nord de 32°N, sur laquelle apparaît clairement un pic dans les valeurs élevées correspondant au jet.
Il est aisé de repérer le pic et le fitter par une Gaussienne. La largeur de pic obtenue est utilisée pour repérer un seuil de température en soustrayant à la température moyenne du pic deux fois son écart-type.
La valeur de seuil journalière obtenue est filtrée temporellement par un filtre médian d'une fenêtre de 8 jours afin d'éviter les anomalies de détection.
Enfin ce seuil en température est utilisé pour séparer la région en deux.

\begin{figure}
  \caption{Figure de Sarmiento 2004 ?}
  \label{fig:sarmiento}
\end{figure}

\begin{figure}
  Histograme de SST. Fit gaussien. Flèche qui dénote sigma (la largeur du pic).
  Barre verticale pour le seuil.
  Super-imposer en léger l'histograme <32°N.
  \caption{Détermination du seuil en température correspondant au front nord du Gulf Stream.}
  \label{fig:seuil-temp}
\end{figure}


\begin{figure}
  Superposition de tous les contours de seuil pour une année.
  Tracé moyen ?
  \caption{Variation de la délimitation entre les zones GS-SS et HCB.}
  \label{fig:var-delim}
\end{figure}


\begin{figure}
  % \includegraphics[width=\textwidth]{zones.pdf}
  \caption{Délimitations des zones.}
  \label{fig:zone-delimitation}
\end{figure}

Enlever les côtes pour éviter régime cotier.

\section{Heterogeneity Index}
\label{sec:HI}

\subsection{Définition et implémentation}
\label{sec:HI-definition}

On a adapté un outil \parencite{liu_2016}.
Composantes.
Détail du calcul.
Implémentation.

Coefficients de normalisation.

\begin{figure}
  \centering
  \includegraphics{méthodes/bimodality.pdf}
  \caption[Calcul de la bimodalité]{Calcul de la bimodalité à partir de la distribution des valeurs de températures.}
  \label{fig:bimodality}
\end{figure}

\subsection{Sensibilité aux paramètres}
\label{sec:HI-sensibilite}

Nécessité d'évaluer les incertitudes sur la méthode.
Pour voir si résultat significatif.
Taille de la fenêtre glissante. Coefs de normalisation.

\subsection{Exemples de fronts}
\label{sec:HI-exemples}

Sélectionner des images ou ça marche (ou pas).
Préférentiellement avec couverture MODIS mais pas obligé.

\section{Extraction des résultats}
\label{sec:extraction-res}

Une fois le HI calculé, il devient possible de catégoriser chaque pixel par la région à laquelle il appartient (GS-SS, GS-SP, HCB), ainsi que par sa valeur de HI.
On cherche donc à extraire des informations de ces ensembles de pixels ainsi consistués.
Le nombre total de pixels étant conséquent, et parce que l'établissement des ensembles est compliqué et coûteux (il faut superposer la SST, le HI, et la \gls{chla}), les ensembles de pixels sont chacuns réduits à des histogrammes des variables d'intérêt (SST, \gls{chla}, PFT).
Cela diminue les étapes de calcul ainsi que la quantité de données à traiter pour obtenir un diagnostique (on passe de \numproduct{1000 x 1000} pixels à \num{100} intervalles d'histogramme).

Les histogrammes peuvent être rendus représentatifs sans pour autant utiliser un nombre prohibitif d'intervalles, les données étant elles-mêmes stockées compressées et donc discrétisées. Par exemple pour la température, 450 intervalles permettre de couvrir toutes les valeurs (de \qty{-5}{\dC} à \qty{40}{\dC}), avec une largeur d'intervalle de \qty{0.1}{\dC} égale à l'intervalle de discrétisation des données, lui même équivalent à l'incertitude sur la mesure.
Les intervalles pour la Chlorophylle et les autres variables biologiques sont pris de largeur logarithmique afin de couvrir les plusieurs ordres de grandeur que peuvent prendre leurs valeurs.

Les histogrammes présentent également l'avantage de pouvoir être combiné entre eux.
En effet, tout les histogrammes calculés sont stockés non-normalisés, en nombre de pixel par intervalle donc. Ainsi plusieurs histogrammes peuvent être sommés entre eux avant d'être normalisés pour en extraire une valeur, comme la valeur médiane de la distribution résultant par exemple.
Ce procédé est notamment utilisé pour calculé des diagnostiques sur des périodes de temps autre que journalières, sans avoir besoin de refaire un calcul impliquant les pixels.

Une fois les histogrammes obtenus il est nécessaire de les normaliser afin d'obtenir une densité de probabilité. On prendra soin de considérer les tailles des intervalles dans les calculs.
On considère un histogramme qui compte \(h_{i}\) points pour le i\ieme intervalle \(\left[x_{i}; x_{i+1} \right]\).
La largeur des intervalles est donc \(w_{i} = x_{i+1}-x_{i}\). On transforme le nombre de points en probabilité par unité de valeur \(p_{i} = h_{i} / w_{i} \), avant de le normaliser de sorte à obtenir une intégralle égale à 1:
\begin{equation}
  f_{i} = \frac{p_{i}}{\sum_{j}{p_{i}*w_{i}}},
\end{equation}
pour ainsi obtenir une approximation de la densité de probabilité \(f_{i}\).

On laise scipy faire.

valeurs médianes.
valeur surplus.

vérification que les histogrammes sont well-behaved pour prendre des valeurs.

\sectiontoc{Références informatiques}
\label{sec:ref-info}

J'ai été amené à utiliser durant ma thèse une myriade d'outils informatiques sans lequels ce travail n'aurait pu aboutir, ou en tout cas avec assurément beaucoup plus de difficulté.
Il me semble important d'en citer au moins une partie ici.
La reproducibilité de mon travail étant déjà garantie par la mise à disposition de mes codes (accompagnés d'une courte documentation) sur un dépôt public (url gitlab) associé à un doi (doi zenodo article), il s'agit ici plutôt de [rendre hommage] aux contributeurs qui ont participé à l'élaboration de ces outils.

les plus-à-présenter: Python, IPython, numpy, scipy
les utiles en géosciences: Xarray, Dask.
les histogrammes: Xhistogram, dask-histogram, boost-histogram.
pour les plots: matplotlib, cartopy, cmocean, Paul Tol,

plus perso: Doom Emacas, zotero

\ifSubfilesClassLoaded{
  \emergencystretch=1em
  \sectiontoc{Bibliographie}
  \printbibliography[heading=none]
}{}

\end{document}
