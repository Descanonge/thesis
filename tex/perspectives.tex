% chktex-file 13

\chapter{Perspectives}
\addChpLof
\label{chp:perspectives}
\graphicspath{{resources/perspectives}}

{
  \hypersetup{hidelinks}
  \minitoc%
  \clearpage
}

% petite intro qui ref \cite{levy_2023}.

\section{Impact des fronts sur la composition du phytoplancton}
\label{sec:persp-pft}

% Est-ce que la somme des PFT en sortie de l'algo de Roy est contrainte à être égale à 1?

Les résultats que nous avons présentés se focalisent sur la concentration de \al{chl} comme indicateur de biomasse.
Mais \encadra{et nous l'avons souligné en introduction} le phytoplancton regroupe un ensemble d'organismes très variés.
Il est légitime de s'attendre à ce que la fine échelle ait un impact différent selon ces types d'organismes.
Par là même, les modifications de la composition de la communauté phytoplanctonique peuvent influencer l'export en carbone (\cite{treguer_2018,serra-pompei_2022}).

Plusieurs études montrent ainsi que les apports sporadiques de nutriments par les fronts bénéficient en grande partie aux diatomées, dont le taux de croissance peut être important (\cite{allen_2005,taylor_2012,treguer_2018,hernandez-carrasco_2020}).
En outre, des interactions biologiques complexes entre les différents groupes de cette communauté, ainsi qu'avec leurs prédateurs (\as{ie} le zooplancton), modulent encore les effets des fronts (\cite{mangolte_2022}).

La compréhension des interactions entre les fines échelles et la composition du phytoplancton présente évidemment la difficulté supplémentaire de disposer d'une quantification de cette composition.
Néanmoins, des groupes de phytoplancton utilisent des pigments auxiliaires diagnostiques (c'est-à-dire qui permettent leur identification de manière univoque) potentiellement identifiable par imagerie satellitaire, ce qui motive donc un certain nombre de méthodes visant à inférer leur concentration (\cite{uitz_2006,uitz_2010,uitz_2012,pan_2010,pan_2010,organelli_2013,chase_2013,chase_2017,bracher_2015a,xi_2020,xi_2021}).

Le travail présenté ici pourrait ainsi quantifier les impacts des fronts sur la composition du phytoplancton, à une échelle synoptique de la même manière que pour la biomasse.
Nous avons réalisé quelques étapes préliminaires dans ce sens, en utilisant la méthode développée par \textcite{elhourany_2019,elhourany_2019a}.
Succinctement, cette technique s'appuie sur un type de réseau neuronal, les \glsmargin[asp]{som}, entraînées sur des colocalisations entre des mesures satellites (réflectances pour 4~longueurs d'ondes, \as{chl}, et SST), et des mesures in situ (\as{hplc}).
Cette méthode permet l'obtention de cartes synoptiques de la concentration de 9~pigments diagnostiques, que nous associons à 7~groupes fonctionnels (\glsmargin{pft}) en suivant des relations empiriques (\nref{ax:pig2pft}, \cite{vidussi_2001,uitz_2006,brewin_2010,hirata_2011})

Les résultats préliminaires obtenus soulignent des réponses de magnitude différentes selon le type fonctionnel (\nref{fig:ts-pft}). Par exemple dans le biome subtropical permanent, le surplus causé par les fronts suit la même saisonalité mais avec des amplitudes différentes: les prymnésiophytes ont la réponse la plus importante (\pct{5}), puis les diatomées (\pct{4}), et enfin les procaryotes (\pct{1.8}, \nref{fig:ts-pft} g2"~h2"~i2).
Dans le biome subpolaire (a2"~b2"~c2), les procaryotes sont perdants dans les fronts, présentant ici un surplus négatif, contrairement aux deux autres groupes.
On retrouve ici le concept de \guil{gagnants} et \guil{perdants} décrit par \textcite{mangolte_2022}.

\begin{figure}
  \centering
  \insertfig{ts_pft.pdf}
  \captionT{Impact des fronts sur la composition du phytoplancton}{%
    Moyennes climatologiques des valeurs médiane de concentration (a1--i1) et du surplus (a2--i2) pour trois types fonctionnels (colonnes), pour les trois biomes (lignes), dans le background (rouge), les fronts faibles (bleu), et les fronts forts (vert).
  }
  \label{fig:ts-pft}
\end{figure}

\section{Vers une application à l'échelle globale}
\label{sec:appl-globale}

Dans les chapitres précédents, ainsi que dans l'article publié, nous nous sommes concentrés sur la région de l'Atlantique Nord autour du Gulf Stream.
Nous avons ainsi étendu les résultats de \textcite{liu_2016}, qui se concentraient sur la seule biorégion de la gyre subtropicale du Pacifique Nord, à deux biomes supplémentaires: un biome oligotrophe avec un mélange profond saisonnier, et un biome subpolaire comprenant un bloom printanier.
Cette première étude nous a permis de valider la méthode de détection des fronts et de colocalisation avec la \as{chl}, et de préciser la trajectoire à adopter dans l'optique d'une quantification de ces mêmes effets à l'échelle globale.

\subsection{Défis techniques}

% from 961x1008 -> 4320x8640 =
Nous avons pu remarquer pendant cette étude un ensemble de défis techniques, dont certains pourraient devenir contraignants dans une étude à l'échelle globale.
Ces défis sont essentiellement liés au volume de données conséquent qui serait requis, puisqu'en gardant la même résolution, et donc une grille globale de \numproduct{4320x8640}~pixels, nous multiplierions le volume des données par~40.
Les ressources de calcul constitueraient probablement alors le goulot d'étranglement dans l'exploration des résultats.

J'ai tenté, pour l'écriture des différents scripts, de mitiger cet éventuel problème notamment en veillant à permettre l'utilisation des fonctionnalités de \citesoft{dask}.
Cet outil automatise la parallélisation des calculs, et facilite leur application \engquote{out-of-core}\footnote{%
  En séparant les données en sous-parties (quasi-)indépendantes \citesoft{dask} permet d'effectuer un calcul sans charger en mémoire la totalité des données d'un coup.
}, accélérant les calculs et limitant les besoins en mémoire.
En pratique cependant, son utilisation n'est pas triviale et garantir son bon fonctionnement pour les calculs plus complexes demande une certaine expertise.

% Au-delà d'un simple confort, l'optimisation des calculs ainsi que la mise en place d'outils efficaces et flexibles ont permis de mener une étude exploratoire, avec de nombreuses itérations.

\subsection{Variations spatiales}

Nous avons montré précédemment qu'il existe des différences importantes entre les biorégions de notre zone d'étude, dans la prévalence des fronts selon leur intensité ainsi que dans la réponse du phytoplancton à ces fronts.
Des calculs préliminaires du HI à l'échelle globale confirment cette variabilité dans la répartition des fronts (\nref{fig:global-fronts}).
Les fronts forts (\(\am{hi}>10\)) semblent se confiner aux courants de mésoéchelle les plus intenses \encadra*{les courants de bord ouest que sont le Gulf Stream, le Kuroshio, le courant du Brésil, le courant des Aiguilles, et dans une moindre mesure le courant Est Australien}.
Les fronts faibles, bien que présents sur une plus grande surface, semblent tout de même survenir dans des zones énergétiques limitées.
On notera par exemple les recirculations du Gulf Stream et du Kuroshio, la zone de convergence intertropicale du Pacifique, ou le courant circumpolaire Antarctique.

\begin{figure}
  \centering
  \insertfig{global_fronts.pdf}
  \captionT{Instantané des fronts à l'échelle globale}{%
    Le HI est calculé à partir des mêmes données, et avec les mêmes coefficients de normalisations que ceux calculés dans la région du Gulf Stream (le \frenchdate{2018}{2}{15}).
  }
  \label{fig:global-fronts}
\end{figure}

Toutes ces régions sont susceptibles de présenter une hydrologie unique amenant à des fronts thermiques spécifiques.
Si notre méthode capture des fronts d'intensité variable, il n'est pas exclu que l'équivalence entre front forts (\(\am{hi}>10\)) et fronts permanents (ainsi qu'entre fronts faibles (\(5<\am{hi}<10\)) et fronts éphémères) ne puisse être appliquée dans d'autres régions.
De même, la normalisation des composantes du HI a été réalisée pour notre région d'étude, et pourrait ne pas s'appliquer aussi bien dans d'autres régions.
L'étude de sensibilité que nous avons réalisée indique néanmoins que les résultats \encadra{\emph{concernant la distribution de \as{chl} dans les fronts}} ne sont pas significativement affectés par les valeurs de ces coefficients.
La prochaine étape vers une étude globale apparaît donc être l'application de méthodes présentées dans d'autres régions que l'Atlantique Nord.

\subsection{Première application: courant de Californie}
\label{sec:cce}

La méthode du HI a d'ores et déjà été appliquée autour du courant de Californie dans le cadre de l'étude de \textcite{mangolte_2023} (\nref{ax:article-cce}), afin de délimiter sur des images de SST les fronts ciblés par un ensemble de campagnes in situ.

Cette étude analyse les données de 8~transects haute résolution (espacement entre \qtyrange[range-phrase={ et }]{1}{5}{\km}) ciblant des fronts au large de la côte californienne.
Ces transects combinent des données de température et de salinité, ainsi que 24~groupes de bactéries, phytoplancton, et zooplancton, en surface et en profondeur.

Les stations frontales sont identifiés à partir des mesures in situ, et le contexte régional menant à la formation du front est étudié avec des données satellitaires.
Les fronts sont aussi repérés sur les images de SST en calculant le HI (taille de fenêtre de \qty{20}{\km} et seuil à \num{10}).
Les \eng{backward} \glsmargin{fsle} (calculés à partir des champs de vitesses géostrophiques dérivés par altimétrie et des courants d'Ekman dérivés du vent) sont utilisés afin de repérer les structures cohérentes liées au mélange horizontal.

\textnote{Ici le seuil utilisé pour discriminer les fronts est élevé (\(\am{hi}>10\)), bien qu'il s'agisse de fronts de submésoéchelle éphémères.}

\Textcite{mangolte_2023} montrent que la majorité des fronts échantillonnés comportent un pic élevé de biomasse, en opposition à une simple séparation entre deux populations.
Le ciblage de fronts relativement stables et intenses par les campagnes peuvent biaiser ce résultat cependant.
Ces pics de biomasse sont généralement moins larges (\qtyrange{1}{5}{\km}) que les fronts (\qtyrange{20}{30}{\km}), et sont décalés par rapport au maximum de gradient de densité jusqu'à \qty{10}{\km}, majoritairement du côté froid.
Les espèces favorisées constituant les pics sont presque systématiquement les diatomées et \eng{pico-grazers}, avec néanmoins des variations plus fines à l'échelle du taxon.

Le travail réalisé pour cette étude \encadra{bien que cela ne soit pas son objectif premier} constitue un premier effort de validation dans une autre région, et avec des données biologiques in situ.
L'exploration des données de l'étude révèlent que le HI identifie correctement la position des fronts mais \emph{surestime leur largeur} par rapport aux données in situ (\nref{fig:comp-hi-fsle}).
À l'inverse, les valeurs élevées de \as{fsle} sont décalées par rapport aux fronts in situ, les données d'altimétrie et de vent ne permettant pas de représenter complètement la dynamique des fronts.

\begin{figure}
  \centering
  \insertfig{comp_hi_fsle.png}
  \captionT{Comparaison entre les fronts détectés par les transects in situ, le HI, et le FSLE}{%
    Carte de \as{fsle} (couleur), avec les contours de HI (\(\am{hi}=\num{5}\) en trait plein et \num{10} en pointillés).
    Les points noirs marquent les stations du transect, et les croix rouges celles identifiées comme frontales par le gradient de densité mesuré in situ.
  }
  \label{fig:comp-hi-fsle}
\end{figure}

\section{Améliorer la méthode}

La méthode de détection des fronts que nous avons utilisée \encadra{calculer l'index d'hétérogénéité HI et sélectionner les valeurs supérieures à un seuil} est conditionnée par un certain nombre de variables.
Attardons-nous ici sur les variables qu'il serait possible de changer.

La surestimation de la largeur des fronts par le HI, soulevée plus haut suite à l'étude de \textcite{mangolte_2023}, permet d'exposer deux leviers d'actions. D'une part le choix des produits satellitaires, d'autre part la méthode de calcul du HI et de ses composantes.

\subsection{Choix des produits satellitaires}

La largeur des fronts pourrait être surestimée avec le HI, simplement du fait de la résolution du produit de SST utilisé (\dataname{sst_esacci}, \qty{4}{\km}).
Nous l'avons mentionné en présentant les données utilisées, nous utilisons le produit \datasect{sst_esacci} au niveau \abbrv{L4}, pour lequel les données des capteurs sont interpolées spatialement avec le système \glsmargin{ostia}.
Il en résulte un champ de SST de résolution \qty{4}{\km} légèrement lissé par l'interpolation.

Les capteurs agrégés pour ce produit permettent cependant une résolution plus élevée (\qty{1}{\km}), à laquelle sont distribués les produits mono-capteurs \abbrv{L3}.
Si ces derniers sont recommandés pour l'étude des structures les plus fines (\cite{merchant_2019}), leur utilisation est cependant accompagnée d'une perte importante de couverture spatiale.
Pour éviter de perdre en couverture, il serait nécessaire de détecter les fronts sur la SST de chaque capteur, avant de réunir tous ces résultats.
Cette étape supplémentaire a déjà été accomplie dans des situations similaires (\cite{miller_2009,nieto_2012}), mais une méthode précise resterait à être choisie et implémentée.
Par ailleurs, le volume de données à traiter augmenterait, du fait de l'augmentation de la résolution bien entendu, mais aussi pour chaque capteur à traiter individuellement.
Enfin, soit le produit de \as{chl} devrait être de résolution kilométrique, soit les fronts obtenus devraient être sous-échantillonnés pour s'adapter à la résolution plus faible du produit de \as{chl}.

Un travail est nécessaire sur l'influence sur nos résultats de la résolution et du choix du produit (mono- ou multicapteur, avec ou sans nuages).
Il pourrait être réalisé dans une région limitée, afin de mitiger les difficultés évoquées.

\subsection{Choix sur la méthode de détection}

En second lieu, la largeur des fronts détectés peut être affectée par la méthode utilisée pour calculer les valeurs du HI, ainsi que le seuil à partir duquel on classifie un pixel comme \guil{front}.

Assez naturellement, le seuil choisi impacte la surface occupée par les fronts, puisqu'on va classifier plus ou moins de pixels comme frontaux.
D'une part, le seuil permet de sélectionner des fronts plus ou moins intenses, les valeurs du HI étant plus élevées sur les fronts plus marqués.
Nous avons ainsi fait la distinction entre fronts faibles, et fronts forts.
Mais d'autre part, abaisser le seuil aura également pour effet de sélectionner plus largement autour des fronts.
Les valeurs du HI étant (à peu près) les plus élevées au centre du front puis diminue lorsque l'on s'en éloigne.
Notre méthode comporte donc un couplage fort entre présence du front, intensité du front, et largeur du front.

Intéressons-nous donc plus en détail à la méthode de calcul des valeurs du HI.
En particulier, trois points sont d'intérêt: la taille de la fenêtre glissante, les coefficients de normalisation des composantes (\ab{ie} l'importance relative de chaque componsante), et les composantes à inclure.
Nous avons démontré l'influence relativement faible sur les distributions de \as{chl} du choix de la taille de fenêtre et des coefficients de normalisation (\nref{sec:sensibilite-parametres}, \nref{fig:ts-sensitivity}), mais nous n'avons pas discuté plus en détail le choix des composantes en soi.

La bimodalité a fait ses preuves à travers son utilisation pour la méthode de \al*{cc}~(\as*{cc}) (voir \nref{sec:detection-fronts}).
On utilise ici pour le HI une implémentation plus simple, principalement du fait de la taille de la fenêtre glissante limitant le nombre de pixels disponibles.
\textnote{\Textcite{cayula_1992} utilisent des données kilométriques et une fenêtre de taille entre \qtyrange[range-phrase={ et }]{20}{30}{\km}.}

L'écart-type des valeurs de SST dans la fenêtre est relié au gradient moyen dans cette fenêtre.
Le gradient est utilisé avec succès dans d'autres techniques de détection de fronts.
% Une des critiques qu'on peut émettre c'est que c'est trop proche du gradient justement, et qui peut prendre des valeurs très élevées si on a un gradient de grande échelle.

Enfin l'inclusion de l'asymétrie est justifiée par \textcite{liu_2016} par sa capacité à repérer les fins filaments de SST dont la distribution en température serait unimodale et asymétrique.
Nous pouvons nous interroger sur la contribution de ces filaments lorsque l'on travaille à des résolutions plus faibles (\citeauthor{liu_2016} utilisent des données kilométriques).
En outre, il apparaît que pour nos données et notre région, l'asymétrie montre essentiellement de forte valeurs \emph{autour} des fronts.
En effet, en se déplaçant en travers du front, l'asymétrie est maximale de chaque côté du front et de signe opposée, et passe donc par zéro au centre du front, lorsque la fenêtre glissante contient autant de valeurs chaudes que froides (\nref{fig:asymetrie}).
L'asymétrie semble donc principalement participer à élargir les fronts.

\begin{figure}
  \centering
  \insertfig[0.7]{asymétrie.pdf}
  \captionT{Illustration de l'asymétrie en travers des fronts}{%
    Ici la température (calculée à partir de la fonction d'erreur \(\operatorname{erf}\)) suit une évolution identique du côté chaud et du côté froid.
    L'asymétrie passe donc par zéro au point d'inflexion et de gradient maximum.
  }
  \label{fig:asymetrie}
\end{figure}

Nous avons montré l'impact potentiel de la surface de fronts détectés sur les distributions de \as{chl}.
En faisant varier les paramètres (coefficients de normalisation et taille de fenêtre), la surface impactée par les fronts varie légèrement et avec la distribution de \as{chl} dans les fronts (\nref{sec:sensibilite-parametres}, \nref{fig:sensibilite-surface}).
Pour une augmentation de la surface des fronts, on observe une diminution des valeurs de \as{chl}.
Il est probable que cela soit le fait d'un élargissement des fronts et donc une sélection de pixels plus éloignés du centre du front, aux valeurs de \as{chl} plus faible.
On assiste ainsi à une sorte de dilution des valeurs de \as{chl}.

Ce résultat souligne l'impact potentiel des choix dans la méthode sur les qualités des fronts détectés (positionnement, intensité, largeur), et donc sur les valeurs de \as{chl} sélectionnées.
Ces couplages pourraient être mieux qualifiés par des études approfondies sur l'influence de chaque composante, de manière individuelle, sur ces qualités.

\subsection{Tester d'autres méthodes existantes}

Il pourrait être intéressant de se pencher sur des méthodes ne reposant pas sur une variable dont il faudrait prendre le seuil, comme celle de \textcite{cayula_1995} par exemple.
L'intensité et la largeur des fronts serait ainsi à déterminer de manière indépendante, par des méthodes à définir.

Changer de méthode serait néanmoins loin d'être trivial.
Les algorithmes ne sont pas nécessairement distribués tel qu'il est usuel dans le domaine de l'open-source, et/ou pas nécessairement implémenté de manière à permettre une réutilisation facile et pour de grands volumes de données.
J'entends en fait ce qui a été fait pour cette thèse, à savoir un dépôt ouvert et communautaire pour le code et la documentation, et une implémentation en Python s'appuyant sur \citesoft{xarray} et \citesoft{dask}, dont les parties les plus intensives sont écrites en langage compilé\footnote{le calcul des composantes du HI à proprement parler est fait en Fortran, lancé directement depuis Python}.
Par exemple, l'algorithme Belkin--O'Reilly est distribué pour R (adapté d'un code IDL), et l'algorithme Cayula--Cornillon en Matlab\dots

On pourra noter un effort dans cette direction par le projet \citesoft{juno}, qui distribue les algorithmes de Canny, \as{cc}, et \as{boa}, implémentés en Python.
On déplorera néanmoins l'absence de version en language compilé, et un support seulement partiel pour Xarray (qui simplifierait pourtant le calcul \as{boa}! voir \nref[annexe]{ax:boa}).


% points clés
% 1km possible, mais rend l'étude bien plus difficile
% global possible

% \section{Article review}
% \label{sec:review}

% Voir \cref{ax:article-review}.
