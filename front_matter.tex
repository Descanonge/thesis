
\pagenumbering{Alph}
\newgeometry{left=1.5cm, right=1.5cm, top=2cm, bottom=3cm}

\begin{titlingpage}

\begin{center}
  \includegraphics[height=3em]{Logos/sorbonne.pdf}
  \hfill
  \includegraphics[height=3em]{Logos/locean.png}
  \hfill
  \includegraphics[height=3em]{Logos/ipsl.png}
  \hfill
  \includegraphics[height=3em]{Logos/chanel_enspsl.png}

  \vspace{1cm}

  {\LARGE Sorbonne Université}\\[2ex]
  École Doctorale 129 Sciences de l'Environnement\\
  \emph{Laboratoire d'Océanographie et du Climat: Expérimentations et Approches Numériques}

  \vspace{3cm}

  \par\noindent\rule[0.7em]{\textwidth}{2pt}
  {\bfseries\Large \Title}\\
  \par\noindent\rule{\textwidth}{2pt}

  \vspace{3cm}

  Thèse de doctorat\\
  Spécialité : Cycles biogéochimiques et changements environnementaux globaux

  \vspace{1cm}

  {\normalsize par Clément Haëck}

  \vspace{1cm}

  Dirigée par Marina Lévy et Laurent Bopp

  \vspace{2cm}
\end{center}

\par\noindent Présentée et soutenue publiquement le 31 février 2023,\\
devant un jury composé de:

\begin{center}
\begin{tabular}{llr<{\raggedleft}}
  Prénom NOM & Titre & Rapporteur \\
  Prénom NOM & Titre & Rapporteur \\
  Prénom NOM & Titre & Président \\
  Marina LÉVY & DR CNRS & Directrice de thèse \\
  Laurent BOPP & DR CNRS & Co-directeur de thèse \\
\end{tabular}
\end{center}

\end{titlingpage}
\restoregeometry

\frontmatter

\section{Résumé}
Résumé

\clearpage
\section*{Remerciements}
Remerciements \dots

\clearpage
\section{Publications et productions}

\begin{itemize}
        \item Article CHL
\end{itemize}
\medskip

Lors de mon travail de thèse, j'ai été amené à écrire des outils qu'il m'a parru utile de rendre publiques et accesibles. Tous les codes utilisés sont disponibles ici: \url{https://gitlab.in2p3.fr/clement.haeck/submeso-color}.
\medskip

Certains outils sont distribués à part:
\begin{itemize}
  \item \emph{FileFinder}: un paquet python qui permet entre autres de trouver des fichiers grâce à la structure de leur nom de fichier (\url{https://github.com/Descanonge/filefinder}).
  \item \emph{XArray-Histogram}: un paquet python qui permet de calculer des histogrammes depuins des données gérées par XArray (\url{https://github.com/Descanonge/xarray-histogram}).
  \item \emph{Tol-Colors}: un paquet python qui donne accès à des jeux de couleurs adaptés aux personnes daltoniennes. Les jeux de couleurs existaient déjà, je les ai seulement rendus accessibles sur Pypi (\url{https://github.com/Descanonge/tol_colors}).
  \item \emph{Dateloop}: un script bash permettant de générer des ensembles de dates (\url{https://github.com/Descanonge/dateloop}).
\end{itemize}
\medskip

J'ai également participé à un projet open-source visant à rendre accessible au paquet python \emph{XArray} les conventions de métadonnées CF (\url{https://github.com/xarray-contrib/cf-xarray}).
J'ai ajouté le support pour les variables drapeau utilisant un masque binaire.
