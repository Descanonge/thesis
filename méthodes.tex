
\chapter{Méthodes}

\section{Données}

\subsection{SST}

\subsubsection{MODIS}
1km daily
regriddé 'à la main' avec occsw depuis MODIS L2
difficile, prend du temps, demande grosse capacité de stockage
un seul capteur (à moins de faire soi-même un merge, ce qui est encore plus difficile), donc plus de nuages

\subsubsection{GHRSST}
bien (1km) mais inclue capteur microondes

\subsubsection{ESA CCI SST}
4km, seulement capteurs IR

\subsection{Chlorophylle}

\subsection{Bathymétrie}

\section{Délimitations des sous-régions}

On sépare en 3 zone (\fref{fig:zone-delimitation}).

\begin{figure}
  % \includegraphics[width=\textwidth]{zones.pdf}
  \caption{Délimitations des zones.}
  \label{fig:zone-delimitation}
\end{figure}

\section{Heterogeneity Index}
\subsection{Définition}

On a adapté un outil \parencite{liu_2016}.
Composantes.
Détail du calcul.
Implémentation.

Coefficients de normalisation.

\subsection{Sensibilité aux paramètres}

Taille de la fenêtre glissante. Coefs de normalisation.
Ajouter influence de la résolution ?

\section{Extraction des résultats}

concept des pools de pixels?
utilisation des histogrammes.

vérification que les histogrammes sont well-behaved pour prendre des valeurs.
valeurs médianes.
valeur surplus.
