
\makeglossaries

\setglossarystyle{long}

\makeatletter
% Automatically set glossary width to maximum available
% minus largest 'short' from all entries
\newcommand{\setglswidth}{%
  \glsfindwidesttoplevelname
  \newlength{\gls@widest}
  \settowidth{\gls@widest}{\glstreenamefmt{\@glswidestname\space}}
  \multiply\gls@widest by -1%
  \setlength{\glsdescwidth}{\textwidth}
  \addtolength{\glsdescwidth}{\gls@widest}
  \addtolength{\glsdescwidth}{-21pt} % We are overfull by 20.5pt otherwise, idk why
  \message{glsdescwidth: \the\glsdescwidth\newline}
}
\makeatother

%% Acronyms glossary
\setabbreviationstyle[acronym]{short-nolong}
\loadglsentries{glossaries/acronyms}

%% Symbols glossary
\newignoredglossary{symbol}
\setabbreviationstyle[symbol]{nolong-short}
\glssetcategoryattribute{symbol}{nohyper}{true}

\newcommand{\newsymbol}[4][]{%
  \newabbreviation[#1, category=symbol,
  sort={#2}, symbol={#3}
  ]{#2}{\(#3\)}{#4}
}
\newcommand{\am}[1]{\glssymbol{#1}}

\loadglsentries[symbol]{glossaries/symbols}

%% Links glossary
% This allows to define URLs that can be used in the document
\newignoredglossary{link}
\glssetcategoryattribute{link}{nohyper}{true}
\glssetcategoryattribute{link}{regular}{true}

% Add keys
\glsaddkey{url}{none}{\glsentryurl}{\Glsentryurl}%
{\glsurlorg}{\Glsurlorg}{\GLSurlorg}
\glsaddkey{doi}{none}{\glsentrydoi}{\Glsentrydoi}%
{\glsdoiorg}{\Glsdoiorg}{\GLSdoiorg}

\newcommand{\newlink}[4][]{%
  \newglossaryentry{#2}{type=link, category=link,%
    name={#3}, url={#4}, #1}%
}
\newcommand{\newdoi}[4][]{%
  \newglossaryentry{#2}{type=link, category=link,%
    name={#3}, url={https://doi.org/#4}, doi={#4}, #1}%
}

\newcommand{\glshref}[2][]{%
  \href{\glsentryurl{#2}}{%
    \ifstrempty{#1}{\glsentryname{#2}}{#1}%
  }%
}
\newcommand{\glsurl}[1]{%
  \ifglsfieldeq{#1}{doi}{none}%
  {%
    \url{\glsentryurl{#1}}%
  }{%
    \textsc{doi}:~\href{\glsentryurl{#1}}{\texttt{\glsentrydoi{#1}}}%
  }%
}

\loadglsentries[link]{glossaries/links}

%% Other abbreviations
\newignoredglossary{other}
\glssetcategoryattribute{other}{nohyper}{true}
\setabbreviationstyle[other]{long}

\newcommand{\newabbreviationOther}[4][]{%
  \newglossaryentry{#2}{type=other, category=other,%
    name={#3}, short={#3}, long={#4}, #1}%
}
\newcommand{\newalias}[2]{\newglossaryentry{#1}{name={#1}, alias=#2}}

\loadglsentries[other]{glossaries/other}
