
\makeglossaries%

\setglossarystyle{long}

\makeatletter
% Automatically set glossary width to maximum available
% minus largest 'short' from all entries
\newcommand{\setglswidth}{%
  \glsfindwidesttoplevelname%
  \newlength{\gls@widest}%
  \settowidth{\gls@widest}{\glstreenamefmt{\@glswidestname\space}}%
  \multiply\gls@widest by -1%
  \setlength{\glsdescwidth}{\textwidth}%
  \addtolength{\glsdescwidth}{\gls@widest}%
  \addtolength{\glsdescwidth}{-21pt} % We are overfull by 20.5pt otherwise, idk why
  \message{glsdescwidth: \the\glsdescwidth\newline}
}
\makeatother

%% Acronyms glossary
\setabbreviationstyle[acronym]{short-nolong}
\loadglsentries{glossaries/acronyms}


%% Symbols glossary
\newignoredglossary{symbol}

% Custom style for Math symbols
% When an abbreviation is added, we should have
%   short = \(<command>\)
%   long = description
%   symbol = <command>
% This is achieved with \newsymbol
% The style only display short (so put in math mode).
% To input the symbol when already in math mode use \am{<label>}
% If needed the full format display <long>~\(<command>\)
\newabbreviationstyle{short-symbol}%
{%
  \GlsXtrUseAbbrStyleSetup{nolong-short}%
  \renewcommand*{\CustomAbbreviationFields}{%
    name={\the\glslabeltok},%
    sort={\the\glslabeltok},%
    first={\the\glsshorttok},%
    firstplural={\the\glsshorttok},%
    text={\the\glsshorttok},%
    plural={\the\glsshorttok},%
    description={\the\glslongtok}}%
  \renewcommand*{\GlsXtrPostNewAbbreviation}{%
    \glssetattribute{\the\glslabeltok}{regular}{true}%
    \glssetattribute{\the\glslabeltok}{nohyper}{true}%
  }%
}
{%
  \GlsXtrUseAbbrStyleFmts{nolong-short}%
  \renewcommand*{\glsfirstlongfont}[1]{\glsfirstlongdefaultfont{##1}}%
  \renewcommand*{\glslongfont}[1]{\glslongdefaultfont{##1}}%
  \renewcommand*{\glsxtrinlinefullformat}[2]{%
    \protect\glsfirstlongfont{\glsaccesslong{##1}%
      \ifglsxtrinsertinside##2\fi}%
    \ifglsxtrinsertinside\else##2\fi~%
    \glsaccessshort{##1}%
  }%
  \renewcommand*{\glsxtrinlinefullplformat}[2]{%
    \protect\glsfirstlongfont{\glsaccesslongpl{##1}%
     \ifglsxtrinsertinside##2\fi}%
    \ifglsxtrinsertinside\else##2\fi~%
    \glsaccessshortpl{##1}%
  }%
  \renewcommand*{\Glsxtrinlinefullformat}[2]{%
    \protect\glsfirstlongfont{\Glsaccesslong{##1}%
      \ifglsxtrinsertinside##2\fi}%
    \ifglsxtrinsertinside\else##2\fi~%
    \glsaccessshort{##1}%
  }%
  \renewcommand*{\Glsxtrinlinefullplformat}[2]{%
    \protect\glsfirstlongfont{\Glsaccesslongpl{##1}%
       \ifglsxtrinsertinside##2\fi}%
     \ifglsxtrinsertinside\else##2\fi~%
    \glsaccessshort{##1}%
  }%
  \renewcommand*{\glsxtrfullformat}[2]{%
    \glsaccessshort{##1}\ifglsxtrinsertinside##2\fi%
    \ifglsxtrinsertinside\else##2\fi
  }%
  \renewcommand*{\glsxtrfullplformat}[2]{%
    \glsaccessshort{##1}\ifglsxtrinsertinside##2\fi%
    \ifglsxtrinsertinside\else##2\fi
  }%
  \renewcommand*{\Glsxtrfullformat}[2]{%
    \glsaccessshort{##1}\ifglsxtrinsertinside##2\fi%
    \ifglsxtrinsertinside\else##2\fi
  }%
  \renewcommand*{\Glsxtrfullplformat}[2]{%
    \glsaccessshort{##1}\ifglsxtrinsertinside##2\fi%
    \ifglsxtrinsertinside\else##2\fi
  }%
}

\setabbreviationstyle[symbol]{short-symbol}

% \newsymbol[<options>]{<label>}{<symbol>}{<long>}
\newcommand{\newsymbol}[4][]{%
  \newabbreviation[#1, category=symbol,
  sort={#2}, symbol={#3}
  ]{#2}{\(#3\)}{#4}
}
\newcommand{\am}[1]{\glsentrysymbol{#1}}

\loadglsentries[symbol]{glossaries/symbols}

%% Links glossary
% This allows to define URLs that can be used in the document
% Should be used with \glsurl, \glsentryurl et \glshref
\newignoredglossary{link}
\glssetcategoryattribute{link}{nohyper}{true}
\glssetcategoryattribute{link}{regular}{true}

% Add keys
\glsaddkey{url}{none}{\glsentryurl}{\Glsentryurl}%
{\glsurlorg}{\Glsurlorg}{\GLSurlorg}
\glsaddkey{doi}{none}{\glsentrydoi}{\Glsentrydoi}%
{\glsdoiorg}{\Glsdoiorg}{\GLSdoiorg}

\newcommand{\newlink}[4][]{%
  \newglossaryentry{#2}{type=link, category=link,%
    name={#3}, url={#4}, #1}%
}
\newcommand{\newdoi}[4][]{%
  \newglossaryentry{#2}{type=link, category=link,%
    name={#3}, url={https://doi.org/#4}, doi={#4}, #1}%
}

\newcommand{\glshref}[2][]{%
  \href{\glsentryurl{#2}}{%
    \ifstrempty{#1}{\glsentryname{#2}}{#1}%
  }%
}
\newcommand{\glsurl}[1]{%
  \ifglsfieldeq{#1}{doi}{none}%
  {%
    \url{\glsentryurl{#1}}%
  }{%
    \textsc{doi}:~\href{\glsentryurl{#1}}{\texttt{\glsentrydoi{#1}}}%
  }%
}

\loadglsentries[link]{glossaries/links}

%% Other abbreviations
\newignoredglossary{other}
\glssetcategoryattribute{other}{nohyper}{true}
\glssetcategoryattribute{initialism}{discardperiod}{true}
\setabbreviationstyle[other]{short}
\setabbreviationstyle[initialism]{short}
\setabbreviationstyle[biomes]{long}

\newcommand{\newabbreviationOther}[4][]{%
  \newglossaryentry{#2}{type=other, category=other,%
    name={#3}, short={#3}, long={#4}, #1}%
}
\newcommand{\newalias}[2]{\newglossaryentry{#1}{name={#1}, alias=#2}}

% Discard period if enddot key is defined
% Only need that to avoid double dots.
% Inter word/sentence are equal, we use frenchspacing
\glsxtrprovidestoragekey{enddot}{}{}
\renewcommand*\glsxtrifcustomdiscardperiod[2]{%
  \glsxtrifkeydefined{enddot}{#1}{#2}}

\loadglsentries[other]{glossaries/other}

\glsaddall%
